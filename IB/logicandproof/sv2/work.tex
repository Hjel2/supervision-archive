\newcommand{\svcourse}{CST Part IA: Software Engineering and Security}
\newcommand{\svnumber}{1}
\newcommand{\svvenue}{Microsoft Teams}
\newcommand{\svdate}{2022-05-11}
\newcommand{\svtime}{15:00}
\newcommand{\svuploadkey}{CBd13xmL7PC1zqhNIoLdTiYUBnxZhzRAtJxv/ytRdM1r7qIfwMsxeVwM/pPcIo8l}

\newcommand{\svrname}{Dr Sam Ainsworth}
\newcommand{\jkfside}{oneside}
\newcommand{\jkfhanded}{yes}

\newcommand{\studentname}{Harry Langford}
\newcommand{\studentemail}{hjel2@cam.ac.uk}


\documentclass[10pt,\jkfside,a4paper]{article}

% DO NOT add \usepackage commands here.  Place any custom commands
% into your SV work files.  Anything in the template directory is
% likely to be overwritten!

\usepackage{fancyhdr}

\usepackage{lastpage}       % ``n of m'' page numbering
\usepackage{lscape}         % Makes landscape easier

\usepackage{verbatim}       % Verbatim blocks
\usepackage{listings}       % Source code listings
\usepackage{graphicx}
\usepackage{float}
\usepackage{epsfig}         % Embed encapsulated postscript
\usepackage{array}          % Array environment
\usepackage{qrcode}         % QR codes
\usepackage{enumitem}       % Required by Tom Johnson's exam question header

\usepackage{hhline}         % Horizontal lines in tables
\usepackage{siunitx}        % Correct spacing of units
\usepackage{amsmath}        % American Mathematical Society
\usepackage{amssymb}        % Maths symbols
\usepackage{amsthm}         % Theorems

\usepackage{ifthen}         % Conditional processing in tex

\usepackage[top=3cm,
            bottom=3cm,
            inner=2cm,
            outer=5cm]{geometry}

% PDF metadata + URL formatting
\usepackage[
            pdfauthor={\studentname},
            pdftitle={\svcourse, SV \svnumber},
            pdfsubject={},
            pdfkeywords={9d2547b00aba40b58fa0378774f72ee6},
            pdfproducer={},
            pdfcreator={},
            hidelinks]{hyperref}

\renewcommand{\headrulewidth}{0.4pt}
\renewcommand{\footrulewidth}{0.4pt}
\fancyheadoffset[LO,LE,RO,RE]{0pt}
\fancyfootoffset[LO,LE,RO,RE]{0pt}
\pagestyle{fancy}
\fancyhead{}
\fancyhead[LO,RE]{{\bfseries \studentname}\\\studentemail}
\fancyhead[RO,LE]{{\bfseries \svcourse, SV~\svnumber}\\\svdate\ \svtime, \svvenue}
\fancyfoot{}
\fancyfoot[LO,RE]{For: \svrname}
\fancyfoot[RO,LE]{\today\hspace{1cm}\thepage\ / \pageref{LastPage}}
\fancyfoot[C]{\qrcode[height=0.8cm]{\svuploadkey}}
\setlength{\headheight}{22.55pt}


\ifthenelse{\equal{\jkfside}{oneside}}{

 \ifthenelse{\equal{\jkfhanded}{left}}{
  % 1. Left-handed marker, one-sided printing or e-marking, use oneside and...
  \evensidemargin=\oddsidemargin
  \oddsidemargin=73pt
  \setlength{\marginparwidth}{111pt}
  \setlength{\marginparsep}{-\marginparsep}
  \addtolength{\marginparsep}{-\textwidth}
  \addtolength{\marginparsep}{-\marginparwidth}
 }{
  % 2. Right-handed marker, one-sided printing or e-marking, use oneside.
  \setlength{\marginparwidth}{111pt}
 }

}{
 % 3. Alternating margins, two-sided printing, use twoside.
}


\setlength{\parindent}{0em}
\addtolength{\parskip}{1ex}

% Exam question headings, labels and sensible layout (courtesy of Tom Johnson)
\setlist{parsep=\parskip, listparindent=\parindent}
\newcommand{\examhead}[3]{\section{#1 Paper #2 Question #3}}
\newenvironment{examquestion}[3]{
\examhead{#1}{#2}{#3}\setlist[enumerate, 1]{label=(\alph*)}\setlist[enumerate, 2]{label=(\roman*)}
\marginpar{\href{https://www.cl.cam.ac.uk/teaching/exams/pastpapers/y#1p#2q#3.pdf}{\qrcode{https://www.cl.cam.ac.uk/teaching/exams/pastpapers/y#1p#2q#3.pdf}}}
\marginpar{\footnotesize \href{https://www.cl.cam.ac.uk/teaching/exams/pastpapers/y#1p#2q#3.pdf}{https://www.cl.cam.ac.uk/\\teaching/exams/pastpapers/\\y#1p#2q#3.pdf}}
}{}


\usepackage{ebproof}
\usepackage{stmaryrd}

\begin{document}

\begin{examquestion}{1999}{6}{10}

\begin{enumerate}

\item Describe the role of Herbrand models in mechanical theorem proving.
What may we infer when a set of clauses has no Herbrand model?

A Herbrand Model is an interpretation such that the clauses hold and the
domain of the interpretation is the Herbrand Universe. Formally, a Herbrand
Model is a pair $(\mathcal{H}, I)$ where $\mathcal{H}$ is the domain of the
variables and $I$ is a mapping from variables and function literals to
elements in the Herbrand Universe $\mathcal{H}$ or concrete functions.

The goal of many mechanical theorem provers (such as those using DPLL) is to
\textit{find} a Herbrand Model. If a set of clauses has no Herbrand Model;
the set of clauses is \textit{unsatisfiable}.

\item Convert the following problem into clause form. Justify each step you
take and explain in what respect the set of clauses is equivalent to the
original problem.

\[
\exists x \left[ P(x) \wedge Q(x) \right]
\longrightarrow
\exists x \left[ P(f(x, x)) \vee \forall y Q(y) \right]
\]
\begin{align*}
\intertext{To convert to clause form; we must Skolemize, then negate and convert to CNF.
The clauses are the disjunctions.}
\intertext{For clarity, I will rename the variables such
that the new expression is $\alpha$-equivalent and has no duplicate names}
\exists x \left[ P(x) \wedge Q(x) \right]
&\longrightarrow
\exists z \left[ P(f(z, z)) \vee \forall y Q(y) \right] \\
\intertext{Skolemize. We must firstly replace existentially bound variables
with Skolem Constants or Skolem Functions as appropriate. In this case, both
$x$ and $z$ become Skolem Constants}
& P(a) \wedge Q(a)
\longrightarrow
P(f(b, b)) \vee \forall y Q(y) \\
\intertext{Now remove all universal quantifiers}
& P(a) \wedge Q(a)
\longrightarrow
P(f(b, b)) \vee Q(y) \\
\intertext{Negate. Then use the equivalence $\neg(A \to B) \simeq A \wedge
\neg B$}
& P(a) \wedge Q(a)\wedge \neg (P(f(b, b)) \vee Q(y)) \\
\simeq & P(a) \wedge Q(a)\wedge \neg P(f(b, b)) \wedge \neg Q(y)) \\
\intertext{Now convert CNF to clauses by taking each disjunct as a clause}
& \{P(a)\}\ \{Q(a)\}\ \{\neg P(f(b, b))\}\ \{\neg Q(y))\}
\end{align*}

Skolemization does not preserve meaning or validity of a formula. However,
it retains inconsistency. The final set of clauses is inconsistent if
and only if the original formula was inconsistent.

\item Describe the Herbrand universe for your clauses.

The Herbrand universe $\mathcal{H}$ for these clauses is described
as follows:
\begin{align*}
H_0 &= \{c, d\} \\
H_{i+1} &= H_i \cup \{f(a, b) \ | \ a, b \in H_i\} \\
\mathcal{H} &= \bigcup_{i \geq 0} H_i
\end{align*}

The Herbrand Universe can also be defined recursively as follows:
\[
\mathcal{H} = \{f(x, y) \ | \ x, y \in \mathcal{H}\}
\]

\item Produce a resolution proof from your clauses or give reasons why none
exists.
\begin{align*}
\intertext{To prove the theorem, we must derive the empty clause $\boxempty$}
& \{P(a)\}\ \{Q(a)\}\ \{\neg P(f(b, b))\}\ \{\neg Q(y))\} \\
\intertext{Firstly, resolve $Q(a)$ with $\neg Q(y)$ using the unifier
$\sigma = [a/y]$}
\sim& \{P(a)\}\ \{Q(a)\}\ \{\neg P(f(b, b))\}\ \{\neg Q(a))\} \\
\sim& \{P(a)\}\ \{\neg P(f(b, b))\}\ \boxempty \\
\intertext{The empty clause been derived so the original theorem cannot be
proved}
\end{align*}
\item Exhibit a Herbrand model for your clauses or give reasons why none
exists.

Since the empty clause was shown to be inconsistent. Therefore the theorem
is unsatisfiable.

A Herbrand Model is an interpretation for which the clauses are satisfied.
By the proof above, the clauses are unsatisfiable and therefore there is no
Herbrand Model.

\end{enumerate}

\end{examquestion}

\begin{examquestion}{1998}{5}{10}

\begin{enumerate}

\setcounter{enumi}{1}

\item Attempt to prove the above formula using the sequent calculus until
either it is proved or the proof cannot be continued.

\begin{center}
\begin{prooftree}
\infer0{
P \to Q, R
\Rightarrow
R
}
\infer1[$(\neg l)$]{
P \to Q, \neg R, R
\Rightarrow
}
\hypo{
R
\Rightarrow
P, Q
}
\infer0{
Q, R
\Rightarrow
Q
}
\infer2[$(\to l)$]{
P \to Q, R
\Rightarrow
Q
}
\infer1[$(\neg l)$]{
P \to Q, \neg Q, R
\Rightarrow
}
\infer2[$(\vee l)$]{
P \to Q, \neg R \vee \neg Q, R
\Rightarrow
}
\infer1[$(\neg r)$]{
P \to Q, \neg R \vee \neg Q
\Rightarrow
\neg R
}
\infer1[$(\wedge l)$]{
(P \to Q) \wedge (\neg R \vee \neg Q)
\Rightarrow
\neg R
}
\infer1[$(\to r)$]{
\Rightarrow \left[ (P \to Q) \wedge (\neg R \vee \neg Q) \right] \to \neg R
}
\end{prooftree}
\end{center}

\item Design a method for determining whether a prepositional formula is
inconsistent. The method should work by examining the formula's disjunctive
normal form. Demonstrate your method by applying it to the formula:
\[
\neg \left[ (P \wedge Q) \vee (Q \to P) \right]
\]

Informally:
\begin{itemize}

\item Convert the formula to DNF

\item If the reduction to DNF yields $\mathbf{f}$ then the formula is
inconsistent and unsatisfiable.

\item If the reduction yields $\mathbf{t}$ then the formula is valid

\item If the DNF is a disjunction of multiple terms, then the formula is
satisfiable but not valid.

\end{itemize}

Formally:
\begin{itemize}

\item Start with a propositional formula $F$.

\item Replace all occurrences of $A \leftrightarrow B$ with $(A \to B) \wedge
(B \to A)$:
\[
A \leftrightarrow B \Rightarrow (A \to B) \wedge (B \to A)
\]

\item Replace all occurrences of $A \leftarrow B$ with $B \to A$
\[
A \leftarrow B \Rightarrow B \to A
\]

\item Replace all occurrences of $A \to B$ with $\neg A \vee B$:
\[
A \to B \Rightarrow \neg A \vee B
\]

\item While there are any negated formulae (ie negations which are not
literals); push in a negation using De Morgens Laws. IE while it can be done;
, do one one of the following operations.
\begin{align*}
\neg (A \vee B) &\Rightarrow \neg A \wedge \neg B \\
\neg (A \wedge B) &\Rightarrow \neg A \vee \neg B
\end{align*}

The formula is now in negation normal form

\item While there is a double negation, eliminate a double-negation:
\[
\neg \neg A \Rightarrow A
\]

\item While it can be done; apply distributivity to push all $\wedge$s in.
\[
A \wedge (B \vee C) \Rightarrow A \wedge B \vee A \wedge C
\]

\item Use associativity to remove all brackets

\item Remove all conjuncts of the form $A \wedge \dots \wedge \neg A$

\end{itemize}

After performing these operations, the formula is in DNF\@. If the formula
is non-empty then there exists one conjunct which was not removed -- this
conjunct is satisfiable. Since Skolemization preserves consistency, we can
conclude that the original formula was consistent in this case.

\end{enumerate}

\end{examquestion}

\begin{examquestion}{2005}{5}{9}

\begin{enumerate}

\item In order to prove the following formula by resolution, what set of
clauses should be submitted to the prover? Justify your answer briefly.
\[
\forall x \left[ P(x) \vee Q \to \neg R(x) \right] \wedge
\forall x \left[ \left( Q \to \neg S(x) \right) \to \left( P(x) \wedge R(x)
\right) \right] \to \forall x S(x)
\]

We should Skolemize the formula, then convert to DNF and negate (so the
negation is in CNF). The clauses we submit to the prover should be the
disjunctions.

In this particular case, this formula uses implication. Therefore, we can
simplify it quicker using the identity
$\neg (A \to B) \simeq A \wedge \neg B$.
\begin{align*}
& \forall x \left[ P(x) \vee Q \to \neg R(x) \right] \wedge
\forall x \left[ \left( Q \to \neg S(x) \right) \to \left( P(x \wedge R(x))
\right) \right] \to \forall x S(x) \\
\intertext{Skolemize the formula. Since there are no existential
quantifiers, this involves only dropping the universal quantifiers}
& \left[ P(x) \vee Q \to \neg R(x) \right] \wedge
\left[ \left( Q \to \neg S(x) \right) \to \left( P(x) \wedge R(x)
\right) \right] \to S(x) \\
\intertext{Next negate. I will use the identity $\neg(A \to B) \simeq A
\wedge \neg B$}
\simeq& \left[ P(x) \vee Q \to \neg R(x) \right] \wedge
\left[ \left( Q \to \neg S(x) \right) \to \left( P(x) \wedge R(x)
\right) \right] \wedge \neg S(x) \\
\simeq& \left[ \neg (P(x) \vee Q) \vee \neg R(x) \right] \wedge
\left[ \neg\left( Q \to \neg S(x) \right) \wedge \left( P(x) \wedge R(x)
\right) \right] \wedge \neg S(x) \\
\simeq& \left[ \neg P(x) \wedge Q \vee \neg R(x) \right] \wedge
\left[ \neg\left( \neg Q \vee \neg S(x) \right) \wedge P(x) \wedge R(x)
 \right] \wedge \neg S(x) \\
\simeq& \left[ \neg (P(x) \vee Q) \vee \neg R(x) \right] \wedge
\left[ \neg\left( Q \to \neg S(x) \right) \wedge \left( P(x) \wedge R(x)
\right) \right] \wedge \neg S(x) \\
\simeq& \left[ \neg P(x) \wedge \neg Q \vee \neg R(x) \right] \wedge
\left[ Q \wedge S(x) \wedge P(x) \wedge R(x)
 \right] \wedge \neg S(x) \\
\simeq& (\neg P(x) \vee \neg R(s)) \wedge (\neg Q \vee \neg R(x)) \wedge
Q \wedge S(x) \wedge P(x) \wedge R(x) \wedge \neg S(x) \\
\simeq& \{\neg P(x), \neg R(s)\}\ \{\neg Q, \neg R(x)\}\
\{Q\}\ \{S(x)\}\ \{P(x)\}\ \{R(x)\}\ \{\neg S(x)\}
\end{align*}

\item Derive the empty clause using resolution with the following set of
clauses, or give convincing reasons why it cannot be derived.

\begin{align*}
 & \{\neg P(x, x)\}\ \{P(x, f(x))\}\ \{\neg P(x, y), \neg P(y, z), P(x, z)\} \\
\intertext{Note that variables scope is \textit{only} in their clause.
For clarity, I will rename the variables}
=& \{\neg P(v, v)\}\ \{P(w, f(w))\}\ \{\neg P(x, y), \neg P(y, z), P(x, z)\} \\
\intertext{To apply resolution, we must unify $\neg P(v, v)$ and $P(x, z)$.
Using the most general unifier $\sigma = [v/x, v/z]$}
=& \{\neg P(v, v)\}\ \{P(w, f(w))\}\ \{\neg P(v, y), \neg P(y, v), P(v, v)\} \\
=& \{P(w, f(w))\}\ \{\neg P(v, y), \neg P(y, v)\} \\
\intertext{Notice the two terms in the RHS clause are the same under
$\alpha$-equivalence. Using idempotence, we can remove one.}
=& \{P(w, f(w))\}\ \{\neg P(y, v)\} \\
\intertext{Use the unifier $\sigma = [w/y, f(x)/v]$}
=& \{P(w, f(w))\}\ \{\neg P(w, f(w))\} \\
=& \boxempty
\end{align*}

\item Derive the empty clause using resolution with the following set of
clauses or give convincing reasons why it cannot be derived. (Note that $a$
and $b$ are constants.)

\begin{align*}
 & \{\neg P(a)\}\ \{Q(a)\}\ \{R(b)\}\ \{S(b)\}\ \{\neg Q(x), P(x), \neg R(y),
\neg Q(y)\}\ \{\neg S(x), \neg R(x), Q(x)\} \\
\intertext{Since variables are scoped only within their clause, I rename for
 clarity}
=& \{\neg P(a)\}\ \{Q(a)\}\ \{R(b)\}\ \{S(b)\}\ \{\neg Q(x), P(x), \neg R(y),
\neg Q(y)\}\ \{\neg S(z), \neg R(z), Q(z)\} \\
\intertext{Using the unifier $\sigma = [a/x]$}
=& \{R(b)\}\ \{S(b)\}\ \{\neg P(a)\}\ \{Q(a)\}\ \{\neg Q(a), P(a), \neg R(y),
\neg Q(y)\}\ \{\neg S(z), \neg R(z), Q(z)\} \\
=& \{R(b)\}\ \{S(b)\}\ \{\neg P(a)\}\ \{P(a), \neg R(y),
\neg Q(y)\}\ \{\neg S(z), \neg R(z), Q(z)\} \\
=& \{R(b)\}\ \{S(b)\}\ \{\neg R(y), \neg Q(y)\}\ \{\neg S(z), \neg R(z), Q
(z)\} \\
\intertext{Using the unifier $\sigma = [b/y, b/z]$}
=& \{R(b)\}\ \{S(b)\}\ \{\neg R(b), \neg Q(b)\}\ \{\neg S(b), \neg R(b), Q
(b)\} \\
=& \{R(b)\}\ \{\neg R(b), \neg Q(b)\}\ \{S(b)\}\ \{\neg S(b), \neg R(b), Q
(b)\} \\
\intertext{Using idempotence, we can duplicate the clause $\{R(b)\}$}
=& \{R(b)\}\ \{\neg R(b), \neg Q(b)\}\ \{R(b)\}\ \{S(b)\}\ \{\neg S(b), \neg
 R(b), Q(b)\} \\
=& \{\neg Q(b)\}\ \{R(b)\}\ \{\neg R(b), Q(b)\} \\
=& \{\neg Q(b)\}\ \{Q(b)\} \\
=& \boxempty
\end{align*}

\end{enumerate}

\end{examquestion}

\end{document}
