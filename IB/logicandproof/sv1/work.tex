\newcommand{\svcourse}{CST Part IA: Introduction to Probability}
\newcommand{\svnumber}{1}
\newcommand{\svvenue}{Churchill, Room TBD}
\newcommand{\svdate}{2022-05-14}
\newcommand{\svtime}{11:00}
\newcommand{\svuploadkey}{PO5ogKIM8KQA22FZS8IAf8gxA8XKi19jxIBVHIfFZ+3GCBXuNUXS9lVN6bNYjxM/}

\newcommand{\svrname}{Mr Matthew Ireland}
\newcommand{\jkfside}{twoside}
\newcommand{\jkfhanded}{right}

\newcommand{\studentname}{Harry Langford}
\newcommand{\studentemail}{hjel2@cam.ac.uk}


\documentclass[10pt,\jkfside,a4paper]{article}

% DO NOT add \usepackage commands here.  Place any custom commands
% into your SV work files.  Anything in the template directory is
% likely to be overwritten!

\usepackage{fancyhdr}

\usepackage{lastpage}       % ``n of m'' page numbering
\usepackage{lscape}         % Makes landscape easier

\usepackage{verbatim}       % Verbatim blocks
\usepackage{epsfig}         % Embed encapsulated postscript
\usepackage{array}          % Array environment
\usepackage[nolinks]{qrcode}         % QR codes
\usepackage{enumitem}       % Required by Tom Johnson's exam question header

\usepackage{hhline}         % Horizontal lines in tables
\usepackage{siunitx}        % Correct spacing of units
\usepackage{amsmath}        % American Mathematical Society
\usepackage{amssymb}        % Maths symbols
\usepackage{amsthm}         % Theorems

\usepackage{ifthen}         % Conditional processing in tex

\usepackage[top=3cm,
            bottom=3cm,
            inner=2cm,
            outer=5cm]{geometry}

% PDF metadata + URL formatting
\usepackage[
            pdfauthor={\studentname},
            pdftitle={\svcourse, SV \svnumber},
            pdfsubject={},
            pdfkeywords={9d2547b00aba40b58fa0378774f72ee6},
            pdfproducer={},
            pdfcreator={},
            hidelinks]{hyperref}

\renewcommand{\headrulewidth}{0.4pt}
\renewcommand{\footrulewidth}{0.4pt}
\fancyheadoffset[LO,LE,RO,RE]{0pt}
\fancyfootoffset[LO,LE,RO,RE]{0pt}
\pagestyle{fancy}
\fancyhead{}
\fancyhead[LO,RE]{{\bfseries \studentname}\\\studentemail}
\fancyhead[RO,LE]{{\bfseries \svcourse, SV~\svnumber}\\\svdate\ \svtime, \svvenue}
\fancyfoot{}
\fancyfoot[LO,RE]{For: \svrname}
\fancyfoot[RO,LE]{\today\hspace{1cm}\thepage\ / \pageref{LastPage}}
\fancyfoot[C]{\qrcode[height=0.8cm]{\svuploadkey}}
\setlength{\headheight}{22.55pt}

\ifthenelse{\equal{\jkfside}{oneside}}{

 \ifthenelse{\equal{\jkfhanded}{left}}{
  % 1. Left-handed marker, one-sided printing or e-marking, use oneside and...
  \evensidemargin=\oddsidemargin
  \oddsidemargin=73pt
  \setlength{\marginparwidth}{111pt}
  \setlength{\marginparsep}{-\marginparsep}
  \addtolength{\marginparsep}{-\textwidth}
  \addtolength{\marginparsep}{-\marginparwidth}
 }{
  % 2. Right-handed marker, one-sided printing or e-marking, use oneside.
  \setlength{\marginparwidth}{111pt}
 }

}{
 % 3. Alternating margins, two-sided printing, use twoside.
}

\setlength{\parindent}{0em}
\addtolength{\parskip}{1ex}

% Exam question headings, labels and sensible layout (courtesy of Tom Johnson)
\setlist{parsep=\parskip, listparindent=\parindent}
\newcommand{\examhead}[3]{\section{#1 Paper #2 Question #3}}
\newenvironment{examquestion}[3]{
    \examhead{#1}{#2}{#3}\setlist[enumerate, 1]{label=(\alph*)}\setlist[enumerate, 2]{label=(\roman*)}
    \marginpar{\qrcode{https://www.cl.cam.ac.uk/teaching/exams/pastpapers/y#1p#2q#3.pdf}}
    \marginpar{\footnotesize \url{https://www.cl.cam.ac.uk/teaching/exams/pastpapers/y#1p#2q#3.pdf}}
}{}



\begin{document}

\begin{examquestion}{2001}{5}{11}

\begin{enumerate}

\item Explain the meaning of the notation $A \vDash B$, where $A$ and $B$
denote formulae of propositional logic.

$A \vDash B$ is notation for ``$A$ entails $B$''. This means that under any
interpretation for which $A$ is valid, $B$ is also valid. Informally, ``for
every mapping from formulae / variables to truth values for which every
formula in $A$ is true; $B$ is also true''.

\item For each of the following equivalences, state whether it holds or not,
justifying each answer rigorously.

\begin{enumerate}

\item \[
\left( P \wedge \left( Q \to R \right) \right) \to S
\simeq
\left( \neg P \vee \neg Q \vee S \right) \wedge \left( \neg P \vee \neg R
\vee S \right)
\]
The equivalence does not hold. Consider the interpretation \\$I = \{P
\mapsto \mathbf{t}, Q \mapsto \mathbf{t}, R \mapsto \mathbf{f}, S \mapsto
\mathbf{f}\}$.
\begin{align*}
\left( P \wedge \left( Q \to R \right) \right) \to S
&= \left( \mathbf{t} \wedge \left( \mathbf{t} \to \mathbf{f} \right) \right)
\to \mathbf{f} \\
&= \left( \mathbf{t} \wedge \mathbf{f} \right) \to \mathbf{f} \\
&= \mathbf{f} \to \mathbf{f} \\
&= \mathbf{t} \\
\intertext{So under the interpretation $I$, the LHS formula holds}
\left( \neg P \vee \neg Q \vee S \right) \wedge \left( \neg P \vee \neg R
\vee S \right)
&= \left( \neg \mathbf{t} \vee \neg \mathbf{t} \vee \mathbf{f} \right) \wedge
\left( \neg \mathbf{t} \vee \neg \mathbf{f} \vee \mathbf{f} \right) \\
&= \left( \mathbf{f} \vee \mathbf{f} \vee \mathbf{f} \right) \wedge
\left( \mathbf{f} \vee \mathbf{t} \vee \mathbf{f} \right) \\
&= \mathbf{f} \wedge \mathbf{t} \\
&= \mathbf{f} \\
\intertext{Therefore there exists an interpretation $I$ which satisfies the
LHS formula but does not satisfy the RHS formula. So the LHS formula does
not entail the RHS formula. We can therefore conclude that the two formulae
are not equivalent.}
\left( P \wedge \left( Q \to R \right) \right) \to S
&\not\vDash
\left( \neg P \vee \neg Q \vee S \right) \wedge \left( \neg P \vee \neg R
\vee S \right) \Longrightarrow \\
\left( P \wedge \left( Q \to R \right) \right) \to S
&\not\simeq
\left( \neg P \vee \neg Q \vee S \right) \wedge \left( \neg P \vee \neg R
\vee S \right) \\
\end{align*}

\iffalse
\[
\begin{split}
& (P \wedge (Q \to R)) \to S \\
\simeq & \neg(P \wedge (Q \to R)) \vee S \\
\simeq & \neg (Q \to R) \vee \neg P \vee S\\
\simeq & \neg P \vee \neg (\neg Q \vee R) \vee S \\
\simeq & \neg P \vee Q \wedge \neg R \vee S \\
\simeq & (\neg P \vee Q \vee S) \wedge (\neg P \vee \neg R \vee S) \\
\not\simeq & (\neg P \vee \neg Q \vee S) \wedge (\neg P \vee \neg R \vee S) \\
\end{split}
\]
\fi

\item \[
\left( P \to Q \right) \to \left( Q \to P \right)
\simeq
\left( Q \to P \right)
\]
This equivalence holds. It can be proved by algebraic manipulation:
\begin{align*}
(P \to Q) \to (Q \to P)
&\simeq \neg(P \to Q) \vee (Q \to P) \\
&\simeq \neg(\neg P \vee Q) \vee \neg Q \vee P \\
&\simeq P \wedge \neg Q \vee \neg Q \vee P \\
&\simeq \neg Q \vee P \wedge (\mathbf{t} \vee \neg Q) \\
&\simeq \neg Q \vee P \\
&\simeq Q \to P
\end{align*}

\item \[
\forall xy \left( P(x) \vee \neg P(y) \right)
\simeq
\forall xy \left( P(x) \leftrightarrow P(y) \right)
\]
This equivalence holds. It can be proved algebraically:
\begin{align*}
\forall x y (P(x) \leftrightarrow \neg P(y))
&\simeq \forall x y (P(x) \wedge P(y) \vee \neg P(x) \wedge
\neg P(y)) \ &&\text{by definition of $\leftrightarrow$} \\
&\simeq \forall x y ((P(x) \vee \neg P(y)) \wedge (\neg P(x)
\vee P(y))) \ &&\text{using distributivity} \\
&\simeq \forall x y (P(x) \vee \neg P(y)) \wedge \forall x y
(\neg P(x) \vee P(y)) \ &&\text{using distributivity} \\
&\simeq \forall x y (P(x) \vee \neg P(y)) \wedge \forall x y
(P(x) \vee \neg P(y)) \ &&\text{by renaming variables} \\
&\simeq \forall x y (P(x) \vee \neg P(y)) \ &&\text{by idempotence}
\end{align*}

\end{enumerate}

\end{enumerate}

\end{examquestion}

\begin{examquestion}{2002}{5}{11}

\begin{enumerate}

\item For each of the following formulae, state (with justification) whether
it is satisfiable, valid or neither.

\begin{enumerate}

\item \[
\left( \left( Q \to R \right) \to Q \right) \wedge \neg Q
\]

This formula is satisfiable, but not valid. The formula holds under the
interpretation $\{Q \mapsto \mathbf{f}, R \mapsto \mathbf{f}\}$; so it is
satisfiable. However, under the interpretation $\{Q \mapsto \mathbf{t}, R
\mapsto \mathbf{f}\}$ the formula does not hold; so it is not valid.
Therefore, the formula is satisfiable but not valid.

\item \[
\left( \left( P \leftrightarrow Q \right) \leftrightarrow P \right)
\leftrightarrow Q
\]

This formula is also satisfiable, but not valid. It is true under the
interpretation $\{P \mapsto \mathbf{t}, Q \mapsto \mathbf{t}\}$ and so is
satisfiable; but false under the interpretation $\{P \mapsto \mathbf{f}, Q
\mapsto \mathbf{t}\}$ so is not valid. Therefore the formula is satisfiable
but not valid.

\item \[
\exists xy \left[ P(x, y) \to \forall xy \ P(x,y) \right]
\]

This formula is satisfiable but not valid. The formula holds under the
interpretation $\mathcal{I} = (\{2\}, \{P(x, y) \mapsto x = y\})$ --
so the formula is satisfiable. However, the formula does not hold under the
interpretation $\mathcal{I} = (\mathbb{N}, \{P(x, y) \mapsto x = y\})$.
Therefore, the formula is satisfiable but not valid.

\item \[
\left[ \forall x \left( P(x) \to Q(x) \right) \wedge \exists x P(x)
\right] \to \forall x \ Q(x)
\]

This formula is satisfiable but not valid. It holds under the interpretation\\
$\mathcal{I} = (\{2\}, \{P(x) \mapsto x = 2, Q(x) = x = 2\}$ so is
satisfiable, but does not hold $\mathcal{I} = (\mathbb{N}, \{P(x) \mapsto x =
2, Q(x) \mapsto x = 2\})$ so is not valid. Therefore, the formula is
satisfiable but not valid.

\end{enumerate}

\item Briefly outline the semantics of first-order logic, taking as an
example the formula $\forall xy \ f(x,y) = f(y, x)$

A formula in first-order logic is an element of a first-order language
$\mathcal{L}$. Formulas with an interpretation $\mathcal{I} = (D, I)$ are
either true or false. A formula is satisfiable if there is at least one
interpretation for which it evaluates to true. A formula is valid if it
is true under every interpretation. An interpretation $\mathcal{I}$ consists
of a pair of a domain (the set of values from which existentially or
universally bound quantifiers may be drawn from) and a mapping from
function and predicates to implementations.

Variables in formulae are either bound (by $\forall$ or $\exists$) or free.
A valuation $\mathcal{V}$ is a mapping from free variables to elements in $D$.

The example is satisfiable since there exist interpretations for which it is
true. However, it does not always hold and so it is not valid. The formula
is true under the interpretation $\mathcal{I} = (\mathbb{N}, \{f(x, y) \mapsto
x + y\})$. However, it does not hold under the interpretation $\mathcal{I} =
 (\mathbb{N}, \{f(x, y) \mapsto x - y\})$.

The given example has no free values, so any function works as a valuation.

\item Exhibit a model that satisfies both of the following formulae ($a$ is
a constant):
\begin{gather*}
    \forall x \ g(x) \neq a \\
    \forall xy \left[ g(x) = g(y) \to x = y \right]
\end{gather*}
\begin{align*}
\mathcal{I} = (\mathbb{N}, \{a \mapsto 0, g(x) \mapsto x + 1\})
\end{align*}

\end{enumerate}

\end{examquestion}

\end{document}
