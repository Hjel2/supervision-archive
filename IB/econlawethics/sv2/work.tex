\newcommand{\svcourse}{CST Part IA: Software Engineering and Security}
\newcommand{\svnumber}{1}
\newcommand{\svvenue}{Microsoft Teams}
\newcommand{\svdate}{2022-05-11}
\newcommand{\svtime}{15:00}
\newcommand{\svuploadkey}{CBd13xmL7PC1zqhNIoLdTiYUBnxZhzRAtJxv/ytRdM1r7qIfwMsxeVwM/pPcIo8l}

\newcommand{\svrname}{Dr Sam Ainsworth}
\newcommand{\jkfside}{oneside}
\newcommand{\jkfhanded}{yes}

\newcommand{\studentname}{Harry Langford}
\newcommand{\studentemail}{hjel2@cam.ac.uk}


\documentclass[10pt,\jkfside,a4paper]{article}

% DO NOT add \usepackage commands here.  Place any custom commands
% into your SV work files.  Anything in the template directory is
% likely to be overwritten!

\usepackage{fancyhdr}

\usepackage{lastpage}       % ``n of m'' page numbering
\usepackage{lscape}         % Makes landscape easier

\usepackage{verbatim}       % Verbatim blocks
\usepackage{listings}       % Source code listings
\usepackage{graphicx}
\usepackage{float}
\usepackage{epsfig}         % Embed encapsulated postscript
\usepackage{array}          % Array environment
\usepackage{qrcode}         % QR codes
\usepackage{enumitem}       % Required by Tom Johnson's exam question header

\usepackage{hhline}         % Horizontal lines in tables
\usepackage{siunitx}        % Correct spacing of units
\usepackage{amsmath}        % American Mathematical Society
\usepackage{amssymb}        % Maths symbols
\usepackage{amsthm}         % Theorems

\usepackage{ifthen}         % Conditional processing in tex

\usepackage[top=3cm,
            bottom=3cm,
            inner=2cm,
            outer=5cm]{geometry}

% PDF metadata + URL formatting
\usepackage[
            pdfauthor={\studentname},
            pdftitle={\svcourse, SV \svnumber},
            pdfsubject={},
            pdfkeywords={9d2547b00aba40b58fa0378774f72ee6},
            pdfproducer={},
            pdfcreator={},
            hidelinks]{hyperref}

\renewcommand{\headrulewidth}{0.4pt}
\renewcommand{\footrulewidth}{0.4pt}
\fancyheadoffset[LO,LE,RO,RE]{0pt}
\fancyfootoffset[LO,LE,RO,RE]{0pt}
\pagestyle{fancy}
\fancyhead{}
\fancyhead[LO,RE]{{\bfseries \studentname}\\\studentemail}
\fancyhead[RO,LE]{{\bfseries \svcourse, SV~\svnumber}\\\svdate\ \svtime, \svvenue}
\fancyfoot{}
\fancyfoot[LO,RE]{For: \svrname}
\fancyfoot[RO,LE]{\today\hspace{1cm}\thepage\ / \pageref{LastPage}}
\fancyfoot[C]{\qrcode[height=0.8cm]{\svuploadkey}}
\setlength{\headheight}{22.55pt}


\ifthenelse{\equal{\jkfside}{oneside}}{

 \ifthenelse{\equal{\jkfhanded}{left}}{
  % 1. Left-handed marker, one-sided printing or e-marking, use oneside and...
  \evensidemargin=\oddsidemargin
  \oddsidemargin=73pt
  \setlength{\marginparwidth}{111pt}
  \setlength{\marginparsep}{-\marginparsep}
  \addtolength{\marginparsep}{-\textwidth}
  \addtolength{\marginparsep}{-\marginparwidth}
 }{
  % 2. Right-handed marker, one-sided printing or e-marking, use oneside.
  \setlength{\marginparwidth}{111pt}
 }

}{
 % 3. Alternating margins, two-sided printing, use twoside.
}


\setlength{\parindent}{0em}
\addtolength{\parskip}{1ex}

% Exam question headings, labels and sensible layout (courtesy of Tom Johnson)
\setlist{parsep=\parskip, listparindent=\parindent}
\newcommand{\examhead}[3]{\section{#1 Paper #2 Question #3}}
\newenvironment{examquestion}[3]{
\examhead{#1}{#2}{#3}\setlist[enumerate, 1]{label=(\alph*)}\setlist[enumerate, 2]{label=(\roman*)}
\marginpar{\href{https://www.cl.cam.ac.uk/teaching/exams/pastpapers/y#1p#2q#3.pdf}{\qrcode{https://www.cl.cam.ac.uk/teaching/exams/pastpapers/y#1p#2q#3.pdf}}}
\marginpar{\footnotesize \href{https://www.cl.cam.ac.uk/teaching/exams/pastpapers/y#1p#2q#3.pdf}{https://www.cl.cam.ac.uk/\\teaching/exams/pastpapers/\\y#1p#2q#3.pdf}}
}{}


\begin{document}

\section{The Emotional App}

You are designing and are about to launch a mobile-only social media app
which will seek to understand the emotional condition of the user, using
multiple inputs such as motion sensing, facial expression recognition, voice
stress measurement, user interactions with the app's content, and the
sentimental analysis of user-inputted text.

Its declared purpose is to enable services to interact more empathetically
with users and serve more relevant content. You propose to monetize it by
serving ads at times when the user is more likely to buy.

A core team member with a Cambridge CS education that has previously studied
the ELE course claims that this monetization technique is predatory as it
will be exploiting users at times of emotional vulnerability.

As the capitalistic enterprising mastermind behind the genius app idea, you
were initially included to fire this colleague as their point-of-view would
result in less ad revenue. However, your investors have also raised a
concern that this app will be able to diagnose depression, and that in
consequence, you may be storing substantial amounts of sensitive personal
information.

Discuss this problem from the viewpoints of both data protection law and ethics.

\subsection{Draft}

\begin{itemize}

\item Introduction

A brief discussion of how outdated laws will allow an ethically questionable
app to be released.

\item GDPR 1

Discuss the provisions of the GDPR.

\item GDPR 2

Discuss the rights of the GDPR.

\item Other legal concerns

Discuss compliance with RIPA and the associated ethics.

\item Extremism and racism

Discuss other potential legal issues -- promoting political hate material and
racism for money. Exposing vulnerable people to this material will promote
division.

\item Exploitation

We will learn more about the emotional state of users of the app. This will
be correlated with peoples financial situation. Thus we can serve more
tailored adverts. This could lead to dangerous adverts such as scams being
served ``I made \$113948 in 3 weeks -- see how!''. Or serving
``energy-saving tips'' involving placing tealights under flowerpots which the
metropolitan police has told people not to do due to risk of fire to people
who are cold during energy crises.

\item Facebook Trial

Discuss Facebooks trial which found users responded more to negative content.
Since we know more about peoples emotional state, we will know more about
what they view as negative. We will therefore \textit{always} provide people
with material slightly more negative than how they are currently feeling.
Think providing depressed people with material promoting self-harm and
suicide. IE Molly's Law.

\item Professional standards of ethics

Discuss the professional standards of ethics: ACM or approval from ethical
committees.

\item Conclusion

A brief overview of all the topics covered and reiteration of the most
important points.

\end{itemize}

\subsection{The Essay}

It is well-known that laws surrounding technology are consistently a decade
behind reality. As a result of this, there are often cases where what is
legal is not ethically correct. The proposed app and its monetization
strategy is an example of this. Although the app can be made to comply with
the GDPR, it would remain grossly unethical; cause great harm to its users
and likely promote racism and political divisions. While there are proposals
such as ``Molly's Law'' which would give social media platforms a duty-of-care
to their users and would make the apps current model illegal; the app and its
monetization strategy would currently be legal.

The GDPR is EU regulation designed to give data subjects ownership of their
own data; limiting the amount of data that companies can collect, forcing
them to collect data only by legitimate routes and obligating companies only
to use data for the purpose the user consented. The GDPR does not regulate
the ethics of how data is used -- only force the company to make the user
aware of how it is used. The GDPR has a number of provisions
and rights. The six provisions of the GDPR are: data must be collected
fairly and lawfully; data may only be used for the original purpose; data
must be adequate, relevant and not excessive; data must be accurate and
kept up-to-date; identifiable data must not be kept for excessive periods
of time; and data must be processed securely and protected against loss or
damage. The app can be designed in such a way to comply with all of these.

The GDPR also gives ``data subjects'' a number of rights, three
of these are especially relevant to the app. The right of access gives all
data subjects the right to see all data which is held on them; the app
manufacturers will have to is designed the app such that this data can be
collected or face significant legal repercussions and a costly fee to
rebuild the app. The right to erasure permits users to request a total
deletion of all the data the company holds on them. The app must be able to
do this; they must design the app such that users can delete their accounts
and all associated data. The right to object allows users to object to the
use of their data in a particular way. This right is absolute when the use
is marketing. In order to comply, the app will have to provide the ability for
users to disable personalised adverts.

In 2016, the law was changed to legalise the interception and
surveillance that security agencies such as GCHQ were performing. The Act
implemented is called IPA (Investigatory Powers Act). The provision most
relevant to the app (inherited from RIPA [Regulation of Investigatory
Powers Act]) states that companies may be required to supply information
to government bodies (in an intelligible form) and not inform their clients
that they are doing so. If the app were to become popular, it's likely there
would be IPA requests. The app should create frameworks to enable the
collection of this data quickly.

The affect of social media on politics has gone under global spotlight and
scrutiny over the last decade. To maximise engagement, users are recommended
content which they are most likely to click on -- content which reinforces
their own beliefs. This creates cliques or cults such as QAnon where members
continually reinforce each others beliefs and continue to promote hate,
racism and extremism. The people who are most susceptible to being inducted
into extremist groups are the most vulnerable -- those who are suffering from
depression or are social outcasts. Our social media platform is likely to
pick up on this and recommend content which they are more likely to interact
with -- extremist content. Extremist adverts or those which promote hate are
most likely to be shown to the most vulnerable members of society. Our app
would be monetizing the spread of extremism, hate and political division.
This is highly unethical and likely to cause a backlash which would affect
revenue and public image of the company.

Many adverts are misleading or even scams. They peddle fake information in an
attempt to make a mundane or useless product or service ``clickbait'' enough
to ``beat'' recommendation algorithms. These adverts are designed to relate
to and pray on a very specific audience. With additional information about
peoples mental state, the app will be able to tailor these manipulative
adverts to those most likely to fall for them. For example, early in
December 2022 dangerous online advertisements almost caused a second Grenfell.
A tower block was evacuated after a resident suffering from the cold tried
an ``energy-saving trick'' which had been advertised to them online. The
police and Fire Brigades begged the public not to use this due to the high
risk of fire. As of late December 2022 this advert was still being run
by Google and targeted at that same demographic -- as they are the most
likely to engage with it. An app which has additional information about the
users emotional state will be able to supply these manipulative and
dangerous adverts to those most likely to fall for them at exactly the wrong
times. Our app would therefore be profiting from dangerous scams and fire
call-outs.

Attempting to maximise engagement with additional knowledge about a users
mental state will result in exposing users to negative emotions and
propagating depression and self-harm. In 2014, Facebook launched an
experiment on 700000 users to see what type of content users engaged with
the most. This experiment was not consented to by users and is widely
regarded as one of the most unethical experiments in modern history. The
researchers found that users engaged most with negative content. Knowing
more about a users mental state will enable the app to provide the most
engaging stories to them -- the negative content with which they are most
likely to engage. This will spread negative emotions: hate and depression.
If we know that users are already depressed, then negative content relative
to them will promote self-harm and suicide. Recommending content such as
this disregards any duty-of-care and is known to have caused suicides.
Therefore the app will survive only by damaging its users mental states and
in the worst case promoting self-harm and suicide. Many advertisers may be
unwilling to sponsor this: ``this post promoting self-harm is brought to you
by \ldots''.

Modern software development is not covered ethically by the law -- there
are many legal actions which are totally unethical. In order to counteract
this, ethics boards have been set up which look at plans ethically. Approval
from one of these boards could cement the app as ethical -- and if not,
could help transition it into one which would be. Professional ethical
guidelines also help bridge the gap between ethical and legal in the
technology industry. ACMs guidelines state that computing professionals
should contribute to society and human well-being. This app is likely to
contribute to human misery (by worsening depressive disorders and spreading
extremism); therefore is unethical by professional ethical standards.

Compliance with data protection laws is not problematic -- the app must
ensure users are aware of the collection of data and provide provision to
extract data from the system. However, there are many other legal issues
surrounding the app. Due to outdated laws, the app would be legal in the
UK\@. However, recent attempts to give social media companies a
duty-of-care would make the app illegal. The app is grossly unethical. It's
likely that the app would exploit vulnerable users, spread hate and
eventually alienate itself from investors over ethical concerns.

\begin{examquestion}{2019}{7}{3}

\begin{enumerate}[label=(\alph*)]

\item What do sections 1, 2 and 3 of the Computer Misuse Act 1990 prohibit?

\begin{enumerate}[label=Section \arabic*. ]

\item Unauthorised access to computer material.

This prohibits seeing data you are not allowed to see -- finding backdoors
or hacking to gain access to data you are not meant to see.

\item Unauthorised access to computer material with intent to commit or
facilitate further offenses.

This is a separate offense to allow for the more serious crime of accessing
data and intending to use it for something to have a higher maximum sentence.

\item Impairing the operation of a computer.

The way in which the computer has been impaired is broad.

This covers unauthorised modification of data -- the reliability of the
data has changed and so the computer no longer operates in the intended way.

This also covers denial-of-service attacks where a computer is no longer
able to operate due to the high load.

There are later subsections of part 3 which prohibit creating, making or
supplying articles intended to be used to break the Computer Misuse Act.

\end{enumerate}

In all sections of the CMA, the attacker must know that they are not
authorised to access the data; the attack does not have to be against any
specific computer or program or data. This means a large company which was
attacked does not have to find the particular bytes and the particular
computers which were attacked.

\item Eve is operating a DDoS-for-hire service and has recruited 100,000
CCTV cameras into a botnet. If Mallory pays Eve \$2 to take down a gaming
teamspeak server for five minutes, what offences, if any, are being
committed by Eve and Mallory?

Section 3 of the computer misuse act prohibits ``performing any act intended
to impair the operation of a computer''. Mallory has performed such an act:
paying Eve to DDoS the gaming teamspeak.

Eve can also be persecuted under section 3 of the computer misuse act --
Eve is controlling the botnet; so has made acts to impair the operation of
a computer. Namely the servers running the gaming teamspeak.

As of 2007, the Computer Misuse Act also prohibits creating, obtaining or
supplying artefacts used to break the Computer Misuse Act. Eve has supplied
an artefact (a DDoS-for-hire service) which is used to break the Computer
Misuse Act. Depending on the exact semantics, Eve has also either created or
obtained the artefact; depending on whether Eve wrote the code that built
the botnet.

\item How might the Wimbledon case (R v. Lennon 2005) apply to this case?

In the Wimbledon case, a 16-year old teenager (David Lennon) worked for a
company for three months until he was fired. A year later, he decided to DoS
them using the program Avalanche. This sent 5 million emails to the company
over a weekend and took down their email server and caused an estimated
\pounds 18000 damage.

Since Lennon was initially a minor when he committed the act, the course was
dealt with in the Youth Court. The defence argued that there was ``no case
to answer''. The persecution attempted to argue that sending so many email
modified the internal state of the email server and was therefore an
unauthorised modification under the CMA section 3. The defense countered
that a single email would do the same -- since the purpose of the email
server was to receive emails this was authorised. The defense then argued by
induction that there could be no threshold above which any amount of emails
would be unauthorised. The youth court agreed with the defense and dropped
the case.

This was taken up in a court of appeals a year later. The DPP ruled that the
consent should be similar to ``if you asked beforehand if you could send 5m
emails, would you be allowed to''. Since Lennon had used the program
Avalanche, and the company had never consented to receiving unsolicited
emails from DoS software, none of the emails were consented and therefore
even sending the first email from the Avalanche software was in breach of the
CMA section 3.

Lennon was then sentenced to 3 months of house arrest, where he was tagged.
This ended shortly before he started university.

This landmark ruling states that there is no threshold beyond which the CMA
was breached; any Denial of Service attack at all is a breach of the CMA\@.
Therefore the ``the site was down only for five minutes'' is no defense
against the law!

\end{enumerate}

\end{examquestion}

\begin{examquestion}{2017}{4}{7}

You are commissioned by a customer to design a toy robot that children will
be able to control using a smartphone app. This app will also enable them to
program the robot using a simple scripting language. To simplify the
networking, all communications between the app and the robot will flow over
wifi via your server.

\begin{enumerate}[label=(\alph*)]

\item Discuss the legal and ethical implications

Since the target audience of this toy is children, the company will be
unable to form a contract with them (UK law forbids making contracts with
children under the age of 13). Therefore the company will be unable to
disclaim liability and would be responsible for any injury caused by the toy.
Instead, there is separate legislation which the toy must conform to: the
Toy Safety Regulation. This regulation sets out a number of safety criteria
which toys must comply with to be sold in the UK\@. These are
broadly commonsense and intuitive -- if the company takes reasonable
measures and places warnings for any obvious risks (ie battery cover as a
choking hazard) then the robot can pass the Toy Safety Regulation.

The toy could be used for or cause harm in other ways. Consider the extreme
example of commands to the toy being used to detonate a bomb. The command to
detonate the bomb was sent through the company's servers. The company (and
the ISP) would be able to apply the ``mere conduit'' defence and would not
be responsible for the detonation.

The company may want default ownership of code written in its app for use in
tutorials or marketing similar. By default, copyright is owned by
the original producer of the material; and moral rights are permanently
owned by the original producer. However, since children cannot make
contracts, they cannot agree to terms and conditions which give ownership
of their code over to the company. Ethically, the children have ownership of
their own code; the company may wish ask for permission to use their code or
even purchase their code.

Although there is no legal obligation to do so, the company should attempt
to provide encrypted transmission between the Toy and their servers. Users
can put arbitrary code and inputs into the app; it's possible they may
input sensitive information. The company morally has a duty-of-care to the
children using it's app -- it would be unethical not to protect the
information they are sending. The company should therefore encrypt messages
sent between the server and the Toy. The easiest way to do this would be using
TLS\@.

E-waste is a growing problem: the lifetime if electronics is short and they
contain many toxic chemicals. Eventually, the children will grow out of the
Toy Robot and throw it away where it will end up either in landfill or
the ocean. The company should endeavour to make the robot as
environmentally-friendly as possible by using few toxic materials.

\item Your customer decides to incorporate a microphone so that the robot
can also recognise spoken commands. To save battery life, the speech
recognition will be done in the server. What effect does this have on the
ethical and legal situation.

Potentially personal audio data is now flowing through the company servers.
Legally, the company must process this in compliance with the GDPR\@.
\begin{itemize}
\item Data must be collected fairly and lawfully
\item Data must be used only for the intended purpose
\item Data must be adequate, relevant and not excessive
\item Personally identifiable information must not be kept for excessive
periods of time
\item Data must be kept accurate and up-to-date
\item Data must be processed securely and protected against loss or damage.
\end{itemize}

Many of the app's users will be under 13 and must acquire parental consent
to use the app -- young children are unable to form contracts and therefore
cannot agree to their voice data being processed.

Since the company is now no longer a ``mere conduit'', they are more likely
to be liable for damages caused by the robot: if the robot mis-hears a
command then the company is liable for the damage caused. This can be
alleviated with stronger warnings and a further requirement for parents to
consent that the robot's voice recognition may not be perfect.

Unfiltered information is now flowing into the company's servers. This
could also contain sensitive information which the robot was not meant to
overhear such as bank information or passwords. Storing records of all audio
the robot hears is excessive (in breach of the GDPR) and would contain
enough sensitive information to make the company a target for criminal
organisations who could run voice recognition AI searching for keywords
relating to banking or passwords. The company is required to protect the
data they collect; so communications would need to be well-encrypted
(perhaps using TLS). Collecting and storing thousands of hours of speech
data from thousands of households around the country is morally
unjustifiable! The company should therefore not store speech data or
transcripts.

The company has a microphone in thousands of households; this may be of
interest to GCHQ who can force the company to disclose information under the
IPA\@.

\item What practical advice can you give your customer about mitigating the
legal risks?

The company should have warnings, usage guidance and terms of service which
require parental consent. Furthermore, the company should appoint a data
protection officer to ensure full compliance with the GDPR (and take the
blame if something does go wrong). The company should also be clear that any
legal issues are to be resolved in a UK court and put firm caps on the
legal costs associated with any disputes to minimise risk. The company
should be transparent about the processing done to ensure any risks are
exposed early before they become scandals. Communication should all be
encrypted to avoid wiretappers finding personal information.

\item Your customer now wants to include a camera so that the robot can
recognise gestures as well. Does this create any further ethical or legal
risks, and if so, what might be done about them?

The company must not store any images. The robot is intended to be used by
children and as such may be stored in their bedrooms. It is likely that over
extended periods of use, it would see indecent images of many of it's users
(children) changing. It is not legal to possess indecent images of children
and so the company must not store any images. Furthermore, it must take even
greater care to encrypt data. I would suggest passing all images through an
encoder (trained to minimise error of gesture recognition with specific
effort taken to ensure that accurate reconstruction of the original image
is not possible) and only send this encoded data from the toy. This would
also alleviate ethical issues and issues regarding compliance with the
IPA\@ -- the company does not have video feeds of users houses and so
cannot send them to GCHQ\@. It would be strongly advisable to take extreme
precautions to guarantee no child pornography/child exploitation lawsuits:
the worst possible publicity for a Toy manufacturer. Further consent for
recordings is required in all cases.

By using gestures, the company opens itself up to racial bias lawsuits: many
companies which use facial recognition or gestures have faced lawsuits about
racial bias for systems which do not work properly for people with non-white
skin. For example, Uber was sued in 2021 over bias facial recognition
software which was unable to recognise black drivers and terminated
several black drivers accounts. The company should therefore take care to
ensure that the gestures are trained on people with all colours of skin.

If the company were to work only on encoded data, the vast majority of ethical
concerns would be removed: the gestures themselves are not sensitive
information. Many users would rightfully question the point of sending this
data to be processed on a server given that the Toy would already have to run
an encoder -- the majority of the work for gesture recognition.

If the images were not encoded, users would have significant ethical issues
regarding surveillance: the Toy would become a live camera in thousands of
houses. This could leak intimate images and ruin peoples lives.

\item How might the situation be affected by Brexit?

The GDPR is an EU regulation. Post-brexit the UK may introduce new
legislation which supercedes the GDPR. This could mean some of the
precautions necessary when processing speech data may not be required.
In reality, any company which processes EU citizens data must comply with
the GDPR\@. If the manufacturers want to sell the Toy in Europe (which is
rational) then they will have to comply with the GDPR even after Brexit.

The UK is currently affected by an EU law which states that you cannot
exclude liability for death or injury. On leaving the EU, the UK would be
able to change this should we want to. This could allow the company to write
off injuries completely and may be able to release less child-proofed toys
in the future.

The age at which children can make contracts is 13 in the UK\@. In many
other EU countries this is 16. Prior to Brexit, there was a chance that the
UK age could be influenced by our partners leading to precautions necessary
with under-13s being extended to under-16s. Post-Brexit, this will not
be a realistic issue.

The primary concern with image data is indecent images of children. While
some EU laws do pertain to child exploitation, it is unlikely that they will
be loosened post-Brexit.

\end{enumerate}

\end{examquestion}

\end{document}
