\documentclass[10pt, a4paper]{article}

\newcommand{\svcourse}{CST Part IA: Software Engineering and Security}
\newcommand{\svnumber}{1}
\newcommand{\svvenue}{Microsoft Teams}
\newcommand{\svdate}{2022-05-11}
\newcommand{\svtime}{15:00}
\newcommand{\svuploadkey}{CBd13xmL7PC1zqhNIoLdTiYUBnxZhzRAtJxv/ytRdM1r7qIfwMsxeVwM/pPcIo8l}

\newcommand{\svrname}{Dr Sam Ainsworth}
\newcommand{\jkfside}{oneside}
\newcommand{\jkfhanded}{yes}

\newcommand{\studentname}{Harry Langford}
\newcommand{\studentemail}{hjel2@cam.ac.uk}

% DO NOT add \usepackage commands here.  Place any custom commands
% into your SV work files.  Anything in the template directory is
% likely to be overwritten!

\usepackage{fancyhdr}

\usepackage{lastpage}       % ``n of m'' page numbering
\usepackage{lscape}         % Makes landscape easier

\usepackage{verbatim}       % Verbatim blocks
\usepackage{listings}       % Source code listings
\usepackage{graphicx}
\usepackage{float}
\usepackage{epsfig}         % Embed encapsulated postscript
\usepackage{array}          % Array environment
\usepackage{qrcode}         % QR codes
\usepackage{enumitem}       % Required by Tom Johnson's exam question header

\usepackage{hhline}         % Horizontal lines in tables
\usepackage{siunitx}        % Correct spacing of units
\usepackage{amsmath}        % American Mathematical Society
\usepackage{amssymb}        % Maths symbols
\usepackage{amsthm}         % Theorems

\usepackage{ifthen}         % Conditional processing in tex

\usepackage[top=3cm,
            bottom=3cm,
            inner=2cm,
            outer=5cm]{geometry}

% PDF metadata + URL formatting
\usepackage[
            pdfauthor={\studentname},
            pdftitle={\svcourse, SV \svnumber},
            pdfsubject={},
            pdfkeywords={9d2547b00aba40b58fa0378774f72ee6},
            pdfproducer={},
            pdfcreator={},
            hidelinks]{hyperref}

\renewcommand{\headrulewidth}{0.4pt}
\renewcommand{\footrulewidth}{0.4pt}
\fancyheadoffset[LO,LE,RO,RE]{0pt}
\fancyfootoffset[LO,LE,RO,RE]{0pt}
\pagestyle{fancy}
\fancyhead{}
\fancyhead[LO,RE]{{\bfseries \studentname}\\\studentemail}
\fancyhead[RO,LE]{{\bfseries \svcourse, SV~\svnumber}\\\svdate\ \svtime, \svvenue}
\fancyfoot{}
\fancyfoot[LO,RE]{For: \svrname}
\fancyfoot[RO,LE]{\today\hspace{1cm}\thepage\ / \pageref{LastPage}}
\fancyfoot[C]{\qrcode[height=0.8cm]{\svuploadkey}}
\setlength{\headheight}{22.55pt}


\ifthenelse{\equal{\jkfside}{oneside}}{

 \ifthenelse{\equal{\jkfhanded}{left}}{
  % 1. Left-handed marker, one-sided printing or e-marking, use oneside and...
  \evensidemargin=\oddsidemargin
  \oddsidemargin=73pt
  \setlength{\marginparwidth}{111pt}
  \setlength{\marginparsep}{-\marginparsep}
  \addtolength{\marginparsep}{-\textwidth}
  \addtolength{\marginparsep}{-\marginparwidth}
 }{
  % 2. Right-handed marker, one-sided printing or e-marking, use oneside.
  \setlength{\marginparwidth}{111pt}
 }

}{
 % 3. Alternating margins, two-sided printing, use twoside.
}


\setlength{\parindent}{0em}
\addtolength{\parskip}{1ex}

% Exam question headings, labels and sensible layout (courtesy of Tom Johnson)
\setlist{parsep=\parskip, listparindent=\parindent}
\newcommand{\examhead}[3]{\section{#1 Paper #2 Question #3}}
\newenvironment{examquestion}[3]{
\examhead{#1}{#2}{#3}\setlist[enumerate, 1]{label=(\alph*)}\setlist[enumerate, 2]{label=(\roman*)}
\marginpar{\href{https://www.cl.cam.ac.uk/teaching/exams/pastpapers/y#1p#2q#3.pdf}{\qrcode{https://www.cl.cam.ac.uk/teaching/exams/pastpapers/y#1p#2q#3.pdf}}}
\marginpar{\footnotesize \href{https://www.cl.cam.ac.uk/teaching/exams/pastpapers/y#1p#2q#3.pdf}{https://www.cl.cam.ac.uk/\\teaching/exams/pastpapers/\\y#1p#2q#3.pdf}}
}{}


\begin{document}

\section*{Rough Plan}

\section*{Introduction}

\begin{itemize}

\item The information goods and services sector has been one of the fastest
growing sectors in recent decades.

\item However, the sector has been deeply dependent on loans

\item The global financial crisis removing the availability of loans and
plunging almost every developed country into recession

\end{itemize}

\section*{More Affected}

\begin{itemize}

\item A reduction in availability of loans will make it harder for new
businesses to break into the sector, increasing the succeptability of the
sector to monopolise.

\item After the economic bubble of 2008 bursts, many people will recall the
dot com bubble bursting in the early 2000s. This will lead a lot of
investors to be wary of the information goods and services industry and the
reverberations of this will cause a large drop-out of investment. This will
stunt growth and development across the sector as a whole.

\item Information goods and services do not have a very good reputation as
secure after the dot com bubble bursting in 2002. In light of this,
investments in information goods and services are likely to be considered
subprime. In light of the events of 2008, many investors will notice that

\item The crisis has disproportionately affected the affluent. Many
information goods and services are targeted at affluent people -- people
with internet connectivity and office workers. This means that the target
clientele of information goods and services has dramatically decreased in
wealth.

\item During a recession, people like to have a higher liquidity than
outside of a recession. Therefore, even though the affluent clientele of
the technology companies may not significantly decrease in wealth, their
expenditure will significantly decrease. This will make growth of the
technology sector very difficult.

\item A lot of companies in the information goods and services sector follow
the ``startup'' business model, where a large amount is invested and the
company gradually loses that money over the following years before turning a
profit later. This business model will not be viable until the global
financial and lending market becomes more risk-taking. This will mean growth
in the sector will decrease.

\item Lack of availability could lead to an increase in piracy -- the piracy
rate is about 20\% before the crisis.

\end{itemize}

\section*{Less affected}

\begin{itemize}

\item Many businesses in the information sector have very high lock-in; this
means its very hard for businesses to stop using existing products. This
means there will not be large swathes of clients abandoning companies.

\item Information goods and services are characterised by high setup costs but
low running costs and low marginal costs.

\item The marginal costs per customer are fairly low. Additionally, the
fixed costs are relatively low. Most of the expenses of companies in the
information goods and services sector comes from employees
and development. Therefore if a company starts losing clients it's easier
for it to make temporary cuts than companies in other sectors. Consider a
car manufacturer. It cannot make cuts as each employee does an essential job
even if the total number of cars they want to make decreases. This is not
the same as an existing information goods and services company -- where the
company can cut jobs and just shift people over fairly easily.

\end{itemize}

\begin{itemize}

\item This question refers to the global financial crisis

\item A credit crunch is when banks reduce lending

\item This means they usually stop investing in risky investments -- as they
can't afford to lose money

\item Usually this reduces the available funds for small-medium sized
businesses

\item The information sector is very prone to monopoly as there are both
strong ``network effects'' and a high barrier to entry.

\item Therefore most of the revenue and work in the information sector is
very concentrated in a few large companies -- therefore information is more
likely to have access to funds than other sectors

\item However, the lack of availability of funds will also increase how
central business is in the information sector! As smaller firms will be
unable to get funding to challenge the larger firms.

\item The ``startup'' model where a company gets pumped full of cash by
optimistic lenders and loses money for several years before it makes
anything will stop temporarily as banks and lenders will be unwilling to
take the risk. The information goods and services sector has a higher
proportion of startups than other sectors.

\item Information goods and services have a very high lock-in -- this means
that many users will find it more expensive to change or stop using them
than to continue using them. This means information goods and services will
not be very affected.

\item They also have negligible marginal costs and fairly low fixed costs;
how much does it cost Microsoft to maintain office? It'll be very hard to
bankrupt existing information goods and services companies.

\item The global financial crisis more severely affected the affluent -- who
spend more on luxuries. Therefore expenditure on luxuries and expensive
items will decrease, which will adversely affect businesses who deal in that.
However, this will not affect information goods and services.

\item Often, switching to a digital system can save money. In an effort to
cost-cut, many businesses may make the switch, increasing revenue for
information goods and services companies.

\item However, with less disposable income the amount of money companies are
willing to spend on services will decrease. This means that companies
existing discrimination will not work -- therefore there will be a very
short period where no company will buy anything.

\item People want a certain level of liquidity. In a recession, liquidity
preference rises (people want more savings) and therefore people want to spend
less.

\item Innovation is expensive and requires a lot of investment. The
information goods and services sector requires innovation -- and for that it
needs belief and loans -- due to the credit crunch, those loans will vanish
and a lot of companies will struggle.

\item Prospect theory: people really dislike losing money but gains are
sublinear. Therefore banks will stop investing in potentially risky ventures
and pour all their money into sectors which are safe.

\end{itemize}

Sections:

\begin{itemize}

\item Introduction

\item More affected

\item Less affected

\item Conclusion

\end{itemize}

\end{document}