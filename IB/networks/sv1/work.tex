\newcommand{\svcourse}{CST Part IA: Software Engineering and Security}
\newcommand{\svnumber}{1}
\newcommand{\svvenue}{Microsoft Teams}
\newcommand{\svdate}{2022-05-11}
\newcommand{\svtime}{15:00}
\newcommand{\svuploadkey}{CBd13xmL7PC1zqhNIoLdTiYUBnxZhzRAtJxv/ytRdM1r7qIfwMsxeVwM/pPcIo8l}

\newcommand{\svrname}{Dr Sam Ainsworth}
\newcommand{\jkfside}{oneside}
\newcommand{\jkfhanded}{yes}

\newcommand{\studentname}{Harry Langford}
\newcommand{\studentemail}{hjel2@cam.ac.uk}


\documentclass[10pt,\jkfside,a4paper]{article}

% DO NOT add \usepackage commands here.  Place any custom commands
% into your SV work files.  Anything in the template directory is
% likely to be overwritten!

\usepackage{fancyhdr}

\usepackage{lastpage}       % ``n of m'' page numbering
\usepackage{lscape}         % Makes landscape easier

\usepackage{verbatim}       % Verbatim blocks
\usepackage{listings}       % Source code listings
\usepackage{graphicx}
\usepackage{float}
\usepackage{epsfig}         % Embed encapsulated postscript
\usepackage{array}          % Array environment
\usepackage{qrcode}         % QR codes
\usepackage{enumitem}       % Required by Tom Johnson's exam question header

\usepackage{hhline}         % Horizontal lines in tables
\usepackage{siunitx}        % Correct spacing of units
\usepackage{amsmath}        % American Mathematical Society
\usepackage{amssymb}        % Maths symbols
\usepackage{amsthm}         % Theorems

\usepackage{ifthen}         % Conditional processing in tex

\usepackage[top=3cm,
            bottom=3cm,
            inner=2cm,
            outer=5cm]{geometry}

% PDF metadata + URL formatting
\usepackage[
            pdfauthor={\studentname},
            pdftitle={\svcourse, SV \svnumber},
            pdfsubject={},
            pdfkeywords={9d2547b00aba40b58fa0378774f72ee6},
            pdfproducer={},
            pdfcreator={},
            hidelinks]{hyperref}

\renewcommand{\headrulewidth}{0.4pt}
\renewcommand{\footrulewidth}{0.4pt}
\fancyheadoffset[LO,LE,RO,RE]{0pt}
\fancyfootoffset[LO,LE,RO,RE]{0pt}
\pagestyle{fancy}
\fancyhead{}
\fancyhead[LO,RE]{{\bfseries \studentname}\\\studentemail}
\fancyhead[RO,LE]{{\bfseries \svcourse, SV~\svnumber}\\\svdate\ \svtime, \svvenue}
\fancyfoot{}
\fancyfoot[LO,RE]{For: \svrname}
\fancyfoot[RO,LE]{\today\hspace{1cm}\thepage\ / \pageref{LastPage}}
\fancyfoot[C]{\qrcode[height=0.8cm]{\svuploadkey}}
\setlength{\headheight}{22.55pt}


\ifthenelse{\equal{\jkfside}{oneside}}{

 \ifthenelse{\equal{\jkfhanded}{left}}{
  % 1. Left-handed marker, one-sided printing or e-marking, use oneside and...
  \evensidemargin=\oddsidemargin
  \oddsidemargin=73pt
  \setlength{\marginparwidth}{111pt}
  \setlength{\marginparsep}{-\marginparsep}
  \addtolength{\marginparsep}{-\textwidth}
  \addtolength{\marginparsep}{-\marginparwidth}
 }{
  % 2. Right-handed marker, one-sided printing or e-marking, use oneside.
  \setlength{\marginparwidth}{111pt}
 }

}{
 % 3. Alternating margins, two-sided printing, use twoside.
}


\setlength{\parindent}{0em}
\addtolength{\parskip}{1ex}

% Exam question headings, labels and sensible layout (courtesy of Tom Johnson)
\setlist{parsep=\parskip, listparindent=\parindent}
\newcommand{\examhead}[3]{\section{#1 Paper #2 Question #3}}
\newenvironment{examquestion}[3]{
\examhead{#1}{#2}{#3}\setlist[enumerate, 1]{label=(\alph*)}\setlist[enumerate, 2]{label=(\roman*)}
\marginpar{\href{https://www.cl.cam.ac.uk/teaching/exams/pastpapers/y#1p#2q#3.pdf}{\qrcode{https://www.cl.cam.ac.uk/teaching/exams/pastpapers/y#1p#2q#3.pdf}}}
\marginpar{\footnotesize \href{https://www.cl.cam.ac.uk/teaching/exams/pastpapers/y#1p#2q#3.pdf}{https://www.cl.cam.ac.uk/\\teaching/exams/pastpapers/\\y#1p#2q#3.pdf}}
}{}


\begin{document}

\begin{examquestion}{2001}{6}{3}

\begin{enumerate}

\item Define the terms \textit{circuit} and \textit{packet} in the context
of communications systems.

A packet is a small ($\sim$ 1kb) collection of bytes which are transmitted
together. Packets consist of headers and payload. There will be many headers
appended onto the packet -- one for each layer of the stack.

A circuit is an dedicated link from one node to another on a network. This
link is only used by the sender. If the sender does not send any data then
the link remains idle.

\item What sort of guarantee does circuit switching provide?

Circuit switching guarantees performance -- since the circuit is dedicated
to a given conversation, no other nodes will attempt to communicate over it.
So if a circuit capable of sending $n$ bytes per second is setup, then there
is guaranteed to be sufficient bandwidth to send $n$ bytes per second even
if other nodes are trying to send across the network. Furthermore, since the
circuit is dedicated to a given conversation, there will be no packet
collisions.

\item What advantages does packet switching provide over circuit switching?

Packet switching allows better bandwidth utilisation via statistical
multiplexing. Often, communication is bursty. Using circuit switching, nodes
must request their maximum possible usage (which is often significantly
above their average usage), while under packet switching the network makes
more realistic assumptions allowing for more than
$\frac{\text{bandwidth}}{\text{max node usage}}$ nodes to use the network.

Packet switching also better resilience to failure -- packets can take
multiple routes and easily be re-routed in the case of failure. However, if
a circuit fails then the sender and receiver must notice and establish a new
link before transmitting any more data.

Packet switching has no setup cost -- circuit switching requires the sender
and receiver to setup a circuit, which takes time. Packet switching does not
suffer from this initial overhead. However, note there is higher continued
overhead with routing packets.

\item Which of \textit{frequency division multiplexing}, \textit{time
division multiplexing} and \textit{code division multiplexing} lend
themselves to circuit switching? Which to packet switching? Explain why or
why not in each case.

\begin{itemize}

\item

\textbf{Frequency Division Multiplexing}

In FDM, signals are sent at different frequencies. Each conversation has a
frequency band on which messages are transmitted. If a conversation does not
transmit, then the frequency band (capacity) is wasted.

FDM lends itself to circuit switching -- it allows multiple conversations to
take place simultaneously without impeding each other. This is highly
beneficial. It complies perfectly to the circuit switching paradigm.

\item

\textbf{Time Division Multiplexing}

The bandwidth is split into timeslots. Each conversation is allocated a
specific timeslot which it can use. If a timeslot is unused, then that
capacity is wasted. Timeslots are allocated during circuit setup and
deallocated during circuit teardown.

TDM lends itself to circuit switching since it allows multiple
simultaneous conversations without overhead.

\item

\textbf{Code Division Multiplexing}

In Code Division Multiplexing, each sender/receiver pair is allocated a
unique code $n$-bit code. In this application, we view 0s as -1. The sender
then XORs their code with the bit they wish to transmit and adds this onto
the signal it is forwarding. The receiver takes the dot-product between its
code and the signal, then divides by $n$ to get the original bit. This
allows multiple senders to transmit over the same wire.

These codes can be pseudorandom, but are more commonly Walsh codes. Walsh
codes are the columns of the set of matrices generated inductively by
$H_1 = (0)$, $H_{2n} \begin{pmatrix}H_n & H_n \\ H_n & \overline{H_n} \end{pmatrix}$.

\textbf{Does CDMA waste bandwidth?}

Given $n$-bit codes and $n$-senders, we need $n+1$ states. To represent
$n+1$ possible states, we need $\lceil\lg(n+1)\rceil$ bits per state. So to
transmit $n$ bits, we have to send $n\lceil\lg(n+1)\rceil$ symbols.
If this is correct, CDMA would waste \\
$100 - \frac{100}{\lceil\lg(n+1)\rceil}\%$ of the bandwidth -- in
real applications this translates to around $90\%$ of the bandwidth. I'd think
this would be a massive drawback, however it's not mentioned anywhere. The
only brief mentions I've found are ambiguous as to whether they mean
``symbols per receiver'' is high or ``symbols per bit in transit'' is high.

CDM is bad for circuit switching -- for the reason explained above, it's highly
inefficient and wastes vast quantities of bandwidth. It is always beneficial
to use any other form of multiplexing, rather than one which burns 90\% of
bandwidth.

\end{itemize}

None of these multiplexing methods are useful for packet switching -- the
premise of packet switching is to allow more communication by exploiting
statistical multiplexing and the bursty nature of data transmission without
ever setting up a circuit. Therefore, there is no need to allow multiple
circuits to multiplex over the same wire -- the concept of a circuit does
not exist in packet switching.

\end{enumerate}

\end{examquestion}

\begin{examquestion}{2001}{5}{3}

Information is to be conveyed from $A$ to $B$ using automatic repeat request
(ARQ), forward error correction (FEC), and lossless compression.

\begin{enumerate}

\item Explain the terms ARQ, FEC and \textit{lossless compression}.

\begin{itemize}

\item

\textbf{Forward Error Correction}

Forward Error Correction is a type of error correction wherein an Error
Correcting Code (ECC) as appended to the end of each packet. The ECC is
``redundant'' data and a function (often a simple hash) of the rest of the
data. On receipt, the receiver will apply the function used to calculate
the ECC and compare the value to the ECC. If they are different, there has
been an error. In FEC, the receiver then tries to establish the most likely
error that occurred. This is often nontrivial since bit flips are unlikely
to be independent. If the recipient is unable to determine the error with high
certainty, it will send a NACK to the sender, requesting retransmission of
the corrupted packet.

\item

\textbf{Automatic Repeat Request}

Automatic Repeat Request is the strategy of using acknowledgements and
timeouts to implement reliable delivery. Sliding Window ARQ is one of the
most common implementations.

In the sliding window protocol:

\begin{itemize}

\item The sender maintains three integers: $sws$ (send window size -- a
constant), $lar$ (last acknowledgement received) and $lfs$ (last frame
sent). The invariant is maintained that $lfs - lar \leq sws$. The sender
will send frames $i$ for $lar < i \leq lar + sws$. If a frame times out
then it will retransmit it. On receiving an acknowledgement for frame $i$,
it sets $lar = i$ -- all frames with id $< i$ have been seen by the receiver.

\item The receiver maintains three integers: $laf$ (largest acceptable frame),
$lfr$ (last frame received) and $rws$ (receive window size). All frames
$\leq$ last frame received have been received and acknowledged. The receiver
buffers any frames $i$ such that $lfr < i \leq lfr + rws$. On receiving all
frames $\leq lfr + z$ for $1 \leq z \leq rws$, the receiver sends an
acknowledgement. If the receiver receives frames smaller than its last
acknowledged frame, it should retransmit acknowledgements for the last
acknowledged frame.

\end{itemize}

Frames in ARQ should have a sequence number -- this sequence number should
be $> \lceil \lg (rws + sws)\rceil$ bits to ensure the receiver does not mix
up the frames. This also aids with the order in which frames are delivered
-- unlike in C\&Ds, packets do not need monotonically increasing integers.

Furthermore, ARQ with a well-chosen sliding window size and timeout will
implement flow control. If the receiver gets overwhelmed, then it will not
acknowledge frames. So the senders window will not slide, the sender will
not be able to send new frames until the timeout for the original frame
triggers.

\item

\textbf{Lossless Compression}

Lossless compression is the strategy of changing the encoding of data into a
compressed format without losing any data. In good situations, lossless
compression can decrease the size of data transmitted by a factor of 10.
However, lossless compression adds overhead with encoding and decoding the
messages. This means that in many situations, best performance is not gained
by maximal compression.

The best type of Lossless compression to use is highly application-specific
-- text is well-suited to encodings such as huffman coding (which compresses
symbols to the entropy limit), code is well-suited to dictionary encoding
(which creates compressed special representations for common phrases),
while FAXs benefit from run-length-encoding (which records the number of
consecutive symbols -- FAXs have a lot of whitespace).

\end{itemize}

\item If we consider ech of these functions to be operating at different
protocol layers, what would be the most sensible ordering of the layers, and
why?

FEC is implemented on the physical layer. The types of errors which occur
(bit errors vs burst errors) and error rates depend on the physical medium.
It therefore makes the most sense to determine and implement error
correction on the physical layer, where appropriate ECCs can be chosen to
deal with the types of errors which occur.

ARQ is implemented on the Data-Link Layer. The Data-Link Layer transmits
data between nodes over the physical layer. ARQ resolves transmission issues
and manages flow control (flow control being a data-link protocol). It's
rational to implement ARQ on the Data-Link Layer.

Lossless compression is highly application-specific -- the right choice can
yield 10x compression, while the wrong choice can increase the size of the
message. It is therefore rational for lossless compression to be implemented
on the application layer.

\item Suppose:

\begin{itemize}

\item The underlying bit channel has a capacity of $B$, a delay $\tau$ and
error rate $\epsilon_0$.

\item The compression rate is $C < 1$

\item The FEC has rate $R < 1$ and given an error rate $\epsilon_0$ provides
an error rate $\epsilon_1$ (which is detected).

\item The ARQ protocol has a window size of $W$.

\end{itemize}

At what rate can information be conveyed?

I assume that the variant of ARQ we use is

The FEC rate is given by $\frac{\text{useful data per packet}}{\text{packet
size}}$.

The ARQ window size is the number of packets the sender will try to send at
once.

The compression rate is given by $\frac{\text{Compressed
size}}{\text{Uncompressed size}}$.

Therefore, the information rate is given by:
\[
\text{InfoRate} = \frac{R \cdot \text{DataRate}}{C}
\]

The data rate is the number of packets received per second.

For the purposes of this question, I assume that the delay before ARQ will
resend the message is $2\tau$ -- ARQ will only retransmit if the data has
been corrupted and will do so at the instant it would have expected to see
the data were it not corrupted.

The expected delay for a packet is given by:
\begin{align*}
\text{Delay} &= 2\tau \cdot (1 - \epsilon_1) + 4\tau \cdot (1 - \epsilon_1)
\epsilon_1 + \cdots \\
&= 2\tau \cdot (1 - \epsilon_1) \cdot \sum^{\infty}_{i=0} (i+1) \cdot
\epsilon^i \\
&= 2\tau \cdot (1 - \epsilon_1) \cdot \left(
\sum^{\infty}_{i=0} \epsilon_1^i +
\sum^{\infty}_{i=0} i \cdot
\epsilon^i \right) \\
&= 2\tau \cdot (1 - \epsilon_1) \cdot \left( \frac{1}{1 - \epsilon_1} + \frac{
\epsilon_1}{(1 - \epsilon_1)^2} \right) \\
&= \frac{2\tau}{1 - \epsilon_1} \\
\end{align*}
The number of packets which can be sent serially in one second is the
inverse of this:
\[
\frac{1 - \epsilon_1}{2\tau}
\]
Since the window size is $W$, we can send $W$ packets at once and so the
total throughput is $W$ times this.
\[
\frac{W(1 - \epsilon_1)}{2\tau}
\]
However, this is capped by the bandwidth of the channel. The number of
messages sent per packet is given by:
\begin{align*}
 & 1 + \epsilon_1 + \epsilon_1^2 \\
=& \frac{1}{1 - \epsilon_1}
\end{align*}
So the maximum number of packets which can be correctly sent along the
channel is given by:
\[
B\cdot(1 - \epsilon_1)
\]
Therefore the data rate is given by the minimum of these two quantities.
Substituting this into the equation for the information rate gives:
\[
\text{InfoRate} \approx \frac{R \cdot \min\left( B\cdot(1 - \epsilon_1),
\frac{W(1 - \epsilon_1)}{2\tau}\right)}{C}
\]

\end{enumerate}

\end{examquestion}

\end{document}
