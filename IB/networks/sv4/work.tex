\newcommand{\svcourse}{CST Part IA: Software Engineering and Security}
\newcommand{\svnumber}{1}
\newcommand{\svvenue}{Microsoft Teams}
\newcommand{\svdate}{2022-05-11}
\newcommand{\svtime}{15:00}
\newcommand{\svuploadkey}{CBd13xmL7PC1zqhNIoLdTiYUBnxZhzRAtJxv/ytRdM1r7qIfwMsxeVwM/pPcIo8l}

\newcommand{\svrname}{Dr Sam Ainsworth}
\newcommand{\jkfside}{oneside}
\newcommand{\jkfhanded}{yes}

\newcommand{\studentname}{Harry Langford}
\newcommand{\studentemail}{hjel2@cam.ac.uk}


\documentclass[10pt,\jkfside,a4paper]{article}
\usepackage{pythonhighlight}

% DO NOT add \usepackage commands here.  Place any custom commands
% into your SV work files.  Anything in the template directory is
% likely to be overwritten!

\usepackage{fancyhdr}

\usepackage{lastpage}       % ``n of m'' page numbering
\usepackage{lscape}         % Makes landscape easier

\usepackage{verbatim}       % Verbatim blocks
\usepackage{listings}       % Source code listings
\usepackage{graphicx}
\usepackage{float}
\usepackage{epsfig}         % Embed encapsulated postscript
\usepackage{array}          % Array environment
\usepackage{qrcode}         % QR codes
\usepackage{enumitem}       % Required by Tom Johnson's exam question header

\usepackage{hhline}         % Horizontal lines in tables
\usepackage{siunitx}        % Correct spacing of units
\usepackage{amsmath}        % American Mathematical Society
\usepackage{amssymb}        % Maths symbols
\usepackage{amsthm}         % Theorems

\usepackage{ifthen}         % Conditional processing in tex

\usepackage[top=3cm,
            bottom=3cm,
            inner=2cm,
            outer=5cm]{geometry}

% PDF metadata + URL formatting
\usepackage[
            pdfauthor={\studentname},
            pdftitle={\svcourse, SV \svnumber},
            pdfsubject={},
            pdfkeywords={9d2547b00aba40b58fa0378774f72ee6},
            pdfproducer={},
            pdfcreator={},
            hidelinks]{hyperref}

\renewcommand{\headrulewidth}{0.4pt}
\renewcommand{\footrulewidth}{0.4pt}
\fancyheadoffset[LO,LE,RO,RE]{0pt}
\fancyfootoffset[LO,LE,RO,RE]{0pt}
\pagestyle{fancy}
\fancyhead{}
\fancyhead[LO,RE]{{\bfseries \studentname}\\\studentemail}
\fancyhead[RO,LE]{{\bfseries \svcourse, SV~\svnumber}\\\svdate\ \svtime, \svvenue}
\fancyfoot{}
\fancyfoot[LO,RE]{For: \svrname}
\fancyfoot[RO,LE]{\today\hspace{1cm}\thepage\ / \pageref{LastPage}}
\fancyfoot[C]{\qrcode[height=0.8cm]{\svuploadkey}}
\setlength{\headheight}{22.55pt}


\ifthenelse{\equal{\jkfside}{oneside}}{

 \ifthenelse{\equal{\jkfhanded}{left}}{
  % 1. Left-handed marker, one-sided printing or e-marking, use oneside and...
  \evensidemargin=\oddsidemargin
  \oddsidemargin=73pt
  \setlength{\marginparwidth}{111pt}
  \setlength{\marginparsep}{-\marginparsep}
  \addtolength{\marginparsep}{-\textwidth}
  \addtolength{\marginparsep}{-\marginparwidth}
 }{
  % 2. Right-handed marker, one-sided printing or e-marking, use oneside.
  \setlength{\marginparwidth}{111pt}
 }

}{
 % 3. Alternating margins, two-sided printing, use twoside.
}


\setlength{\parindent}{0em}
\addtolength{\parskip}{1ex}

% Exam question headings, labels and sensible layout (courtesy of Tom Johnson)
\setlist{parsep=\parskip, listparindent=\parindent}
\newcommand{\examhead}[3]{\section{#1 Paper #2 Question #3}}
\newenvironment{examquestion}[3]{
\examhead{#1}{#2}{#3}\setlist[enumerate, 1]{label=(\alph*)}\setlist[enumerate, 2]{label=(\roman*)}
\marginpar{\href{https://www.cl.cam.ac.uk/teaching/exams/pastpapers/y#1p#2q#3.pdf}{\qrcode{https://www.cl.cam.ac.uk/teaching/exams/pastpapers/y#1p#2q#3.pdf}}}
\marginpar{\footnotesize \href{https://www.cl.cam.ac.uk/teaching/exams/pastpapers/y#1p#2q#3.pdf}{https://www.cl.cam.ac.uk/\\teaching/exams/pastpapers/\\y#1p#2q#3.pdf}}
}{}


\begin{document}

\begin{enumerate}

\item Distinguish between switching, forwarding and routing.

\begin{itemize}

\item Switching

Switching is the method Switches use to determine where to send frames.

The switching table is a mapping from MAC addresses to output ports on a
switch.

The algorithm for switching is as follows:
\begin{itemize}

\item On receipt of a packet from MAC address $x$ on port $y$ add an entry
into the switching table to send packets for MAC address $x$ out on port $y$.

\item On receipt of a packet destined for MAC address $x$, search for it in
the switching table. If there is an entry saying which port it should be
sent on, send it on that port.

Otherwise, flood the network asking if any nodes \textit{have} MAC address
$x$ or have an entry for $x$ in their switching table. If no response is
received, $x$ is not on the network so the packet is dropped. If a response
is received, the entry is input into the forwarding table and $x$ is sent
out on the correct link.

\end{itemize}

\item Routing

Routing is a network-wide operation which is done every time the network
changes. Routing is done in the control plane and is used to construct the
forwarding table.

\item Forwarding

Forwarding is a per-packet operation performed by Routers in the data plane.
The Forwarding Table is a mapping from ranges of IP addresses to output
ports. A copy of this is stored on the Input Linecard. For each packet, the
Input Linecard finds finds the longest prefix match (the most specific
destination) in the forwarding table and sends the packet to the
corresponding output port.

\end{itemize}

\item Describe the Link State and Distance Vector routing algorithms.

\begin{itemize}

\item Link State Routing

In Link State routing, each node is given total information about the whole
network then computes the shortest path to the destination and stores the
first-hop in the routing table. Initially, each node knows only its local
link state (information about links it is directly connected to). Each
node floods the network with this information (using eager reliable
broadcast). Every node on the network now has the global link state.

Each node now has a copy of the global link state. Every node then
independently uses Dijkstra's Algorithm to work out a least-cost path to
every router.

\item Distance Vector Routing

In Distance Vector Routing, each node maintains a Distance Vector $D$,
containing a triple of $(\text{address prefix}, \text{subnet mask},
\text{distance}, \text{router})$ where $(a, s, d, r) \in D$ means the
subnet $a/s$ can be reached in distance $d$ through router $r$.

Nodes continuously broadcast their distance vector to their neighbours.
They then update their distance vectors to the elementwise minimum.

\end{itemize}

\item What are RIP, OSPF and BGP?

RIP is ``Router Information Protocol'' -- the most common implementation of
Distance Vector Routing.

OSPF is ``Open Shortest Path First Protocol'' -- a common implementation of
link-state routing.

BGP is ``Border Gateway Protocol'' -- an inter-domain routing protocol
which uses a variant of of Distance Vector Routing and is designed to
allow border routers to hide what goes on inside their networks.

\end{enumerate}

\end{document}
