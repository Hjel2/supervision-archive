\newcommand{\svcourse}{CST Part IA: Software Engineering and Security}
\newcommand{\svnumber}{1}
\newcommand{\svvenue}{Microsoft Teams}
\newcommand{\svdate}{2022-05-11}
\newcommand{\svtime}{15:00}
\newcommand{\svuploadkey}{CBd13xmL7PC1zqhNIoLdTiYUBnxZhzRAtJxv/ytRdM1r7qIfwMsxeVwM/pPcIo8l}

\newcommand{\svrname}{Dr Sam Ainsworth}
\newcommand{\jkfside}{oneside}
\newcommand{\jkfhanded}{yes}

\newcommand{\studentname}{Harry Langford}
\newcommand{\studentemail}{hjel2@cam.ac.uk}


\documentclass[10pt,\jkfside,a4paper]{article}

\usepackage{mathtools}
\usepackage{ebproof}
\usepackage{stmaryrd}
\usepackage{textcomp}

% DO NOT add \usepackage commands here.  Place any custom commands
% into your SV work files.  Anything in the template directory is
% likely to be overwritten!

\usepackage{fancyhdr}

\usepackage{lastpage}       % ``n of m'' page numbering
\usepackage{lscape}         % Makes landscape easier

\usepackage{verbatim}       % Verbatim blocks
\usepackage{listings}       % Source code listings
\usepackage{graphicx}
\usepackage{float}
\usepackage{epsfig}         % Embed encapsulated postscript
\usepackage{array}          % Array environment
\usepackage{qrcode}         % QR codes
\usepackage{enumitem}       % Required by Tom Johnson's exam question header

\usepackage{hhline}         % Horizontal lines in tables
\usepackage{siunitx}        % Correct spacing of units
\usepackage{amsmath}        % American Mathematical Society
\usepackage{amssymb}        % Maths symbols
\usepackage{amsthm}         % Theorems

\usepackage{ifthen}         % Conditional processing in tex

\usepackage[top=3cm,
            bottom=3cm,
            inner=2cm,
            outer=5cm]{geometry}

% PDF metadata + URL formatting
\usepackage[
            pdfauthor={\studentname},
            pdftitle={\svcourse, SV \svnumber},
            pdfsubject={},
            pdfkeywords={9d2547b00aba40b58fa0378774f72ee6},
            pdfproducer={},
            pdfcreator={},
            hidelinks]{hyperref}

\renewcommand{\headrulewidth}{0.4pt}
\renewcommand{\footrulewidth}{0.4pt}
\fancyheadoffset[LO,LE,RO,RE]{0pt}
\fancyfootoffset[LO,LE,RO,RE]{0pt}
\pagestyle{fancy}
\fancyhead{}
\fancyhead[LO,RE]{{\bfseries \studentname}\\\studentemail}
\fancyhead[RO,LE]{{\bfseries \svcourse, SV~\svnumber}\\\svdate\ \svtime, \svvenue}
\fancyfoot{}
\fancyfoot[LO,RE]{For: \svrname}
\fancyfoot[RO,LE]{\today\hspace{1cm}\thepage\ / \pageref{LastPage}}
\fancyfoot[C]{\qrcode[height=0.8cm]{\svuploadkey}}
\setlength{\headheight}{22.55pt}


\ifthenelse{\equal{\jkfside}{oneside}}{

 \ifthenelse{\equal{\jkfhanded}{left}}{
  % 1. Left-handed marker, one-sided printing or e-marking, use oneside and...
  \evensidemargin=\oddsidemargin
  \oddsidemargin=73pt
  \setlength{\marginparwidth}{111pt}
  \setlength{\marginparsep}{-\marginparsep}
  \addtolength{\marginparsep}{-\textwidth}
  \addtolength{\marginparsep}{-\marginparwidth}
 }{
  % 2. Right-handed marker, one-sided printing or e-marking, use oneside.
  \setlength{\marginparwidth}{111pt}
 }

}{
 % 3. Alternating margins, two-sided printing, use twoside.
}


\setlength{\parindent}{0em}
\addtolength{\parskip}{1ex}

% Exam question headings, labels and sensible layout (courtesy of Tom Johnson)
\setlist{parsep=\parskip, listparindent=\parindent}
\newcommand{\examhead}[3]{\section{#1 Paper #2 Question #3}}
\newenvironment{examquestion}[3]{
\examhead{#1}{#2}{#3}\setlist[enumerate, 1]{label=(\alph*)}\setlist[enumerate, 2]{label=(\roman*)}
\marginpar{\href{https://www.cl.cam.ac.uk/teaching/exams/pastpapers/y#1p#2q#3.pdf}{\qrcode{https://www.cl.cam.ac.uk/teaching/exams/pastpapers/y#1p#2q#3.pdf}}}
\marginpar{\footnotesize \href{https://www.cl.cam.ac.uk/teaching/exams/pastpapers/y#1p#2q#3.pdf}{https://www.cl.cam.ac.uk/\\teaching/exams/pastpapers/\\y#1p#2q#3.pdf}}
}{}


\begin{document}

\begin{examquestion}{2014}{6}{9}

Consider the concurrent imperative language $L$ with syntax and conventional
operational semantics as below.

Say $p$, $m$ has a data race if there is a sequence of transitions $p$, $m
\stackrel{l_1}{\to}\dots\stackrel{l_n}{\to}\stackrel{l}{\to}\stackrel{l
'}{\to}$ were $l$ and $l'$ conflict: they are reads or writes t the same
location, at least one is a write and they are by different threads.

\begin{enumerate}

\item Give a $p$ for which $p$, $m_0$ has a data race.
\[
p = tid_1: x\coloneqq 0; \mathbf{skip} \ | \ tid_2: x \coloneqq 1;
\mathbf{skip}
\]

\item A \textit{vector clock} $c$ is a function from thread ids to natural
numbers, identifying the $c(tid)'$th transition of each thread $tid$. Modify
the semantics above to add a vector clock $c$ to each process thread
($tid_c:s$), each process label ($tid_c: l$), and each memory location (with
each $m(x)$ now being a pair $v_c$ of a value and vector clock). In your
semantics each vector clock should be computed so as to record the latest
transition number of all threads that have causally affected that point.
Explain your semantics, perhaps with some simple examples.

\newcommand{\letin}[1]{\mathbf{let} \ #1 \ \mathbf{in} \ }
\newcommand{\ielse}[1]{\mathbf{if} \ #1 \ \mathbf{else} \ }

\begin{align*}
&statement, s \Coloneqq \ldots \\
&expression, e \Coloneqq \ldots \\
&process, p \Coloneqq tid_c:s \ | \ p|p' \\
&label, l \Coloneqq \tau \ | \ \mathbf{W}x=v \ | \ \mathbf{R}x=v \ | \ tid_c:
\tau \ | \ tid_c: \mathbf{W}x=v \ | \ \ tid_c: \mathbf{R}x=v
\end{align*}
\begin{center}
\begin{prooftree}
\infer0[WR]{
x \coloneqq v; s \xrightarrow{\mathbf{W}x=v} s
}
\end{prooftree}
\hspace{0.5cm}
\begin{prooftree}
\infer0[RD]{
\letin{r = x}{s}
\xrightarrow{\mathbf{R}x=v} \{v/r\}s
}
\end{prooftree}

\begin{prooftree}
\infer0[IF1]{
\ielse{(v=v) \ s}{s'} \xrightarrow{\tau} s
}
\end{prooftree}
\hspace{0.5cm}
\begin{prooftree}
\hypo{v \neq v'}
\infer1[IF2]{
\mathbf{if} \ (v = v') \ s \ \mathbf{else}\ s'
\xrightarrow{\tau} s'
}
\end{prooftree}

\begin{prooftree}
\hypo{
v = \llbracket op\rrbracket (v_1, \dots, v_n)
}
\infer1[OP]{
\mathbf{let} \ r = op(v_1, \dots, v_n) \ \mathbf{in} \ s \xrightarrow{\tau}
\{v/r\}s
}
\end{prooftree}

\vspace{0.5cm}

\begin{prooftree}
\hypo{s \xrightarrow{l} s'}
\infer1[THREAD]{tid_c:s \xrightarrow{tid_c:l} tid_c:s'}
\end{prooftree}

\vspace{0.5cm}

\begin{prooftree}
\hypo{
tid_c: s \xrightarrow{tid_c: \tau} tid_c: s'
}
\hypo{
c' = c \oplus \{(tid, c(tid) + 1)\}
}
\infer2[STAU]{
tid_c: s, m \xrightarrow{tid_c: \tau} tid_{c'}: s', m
}
\end{prooftree}

\vspace{0.2cm}

\begin{prooftree}
\hypo{\begin{matrix}
tid_{c_0}:s \xrightarrow{tid_{c_0}:\mathbf{R}x=v} tid_{c_0}:s' \\
m(x) = (v, c_1) \\
c_2 =
\{
(t, i) | t \in (\mathop{\text{dom}}(c_0 \cup c_1) - \{tid\}) \wedge i = \max
(c_0(t), c_1(t))
\}
\oplus
\{
(tid, \max(c_0(tid), c_1(tid)) + 1)
\}
\end{matrix}
}
\infer1[SRD]{
tid_{c_0}: s, m
\xrightarrow{
tid_{c_0}:\mathbf{R}x=v
}
tid_{c_2}: s', m
}
\end{prooftree}

\vspace{0.5cm}

\begin{prooftree}
\hypo{\begin{matrix}
tid_{c_0}:s \xrightarrow{tid_{c_0}:\mathbf{W}x=v} tid_{c_0}:s' \\
m(x) = (\_, c_1) \\
c_2 =
\{
(t, i) | t \in (\mathop{\text{dom}}(c_0 \cup c_1) - \{tid\}) \wedge i = \max
(c_0(t), c_1(t))
\}
\oplus
\{
(tid, \max(c_0(tid), c_1(tid)) + 1)
\}
\end{matrix}
}
\infer1[SWR]{
tid_{c_0}: s, m
\xrightarrow{
tid_{c_0}:\mathbf{R}x=v
}
tid_{c_2}: s', m \oplus(x, (v, c_2))
}
\end{prooftree}

\vspace{0.2cm}

\begin{prooftree}
\hypo{
p_1, m \xrightarrow{l} p_1', m'
}
\infer1[PLEFT]{p_1 | p_2, m \xrightarrow{l} p_1'|p_2, m'}
\end{prooftree}

\vspace{0.2cm}

\begin{prooftree}
\hypo{
p_2, m \xrightarrow{l} p_2', m'
}
\infer1[PRIGHT]{p_1 | p_2, m \xrightarrow{l} p_1|p_2', m'}
\end{prooftree}

\end{center}

The operations WR, RD, IF1, IF2 and OP are unchanged -- these are thread
local semantics and so are not affected by changes regarding thread IDs.

Labels are also added unconditionally to reductions.

The rules STAU, SRD and SWR allow updates to the global store and keep the
vector clocks consistent.

The rules PLEFT and PRIGHT allow for interleaving of computation between
processes.

The main difference from the previous semantics is that parallelism is at a
process level rather than a thread level.

Consider the example below:
\[
0_{[1, 0]}:x\coloneqq 1; \mathbf{skip}| 1_{[0, 2]}:\mathbf{skip}, \{(x
\mapsto (0, [0, 2]))\}
\]
\begin{center}
\begin{prooftree}
\infer0[WR]{
x \coloneqq 1; \mathbf{skip} \xrightarrow{\mathbf{W}x=v} s
}
\infer1[THREAD]{
0_{[1,0]}: x \coloneqq 1; \mathbf{skip} \xrightarrow{0_{[1,0]}\mathbf{W}x=v}
0_{[1,0]}: \mathbf{skip}
}
\hypo{m(x)=(\_, [0, 2])}
\hypo{
c_2=[2, 2]
}
\infer3[SWR]{
0_{[1, 0]}: x \coloneqq 1; \mathbf{skip}, \{(x \mapsto (0, [0, 2])) \xrightarrow{0_{[1,0]}\mathbf{W}x=v}
0_{[2,2]}: \mathbf{skip}, \{(x \mapsto (1, [2, 2]))
}
\infer1[PRIGHT]{
0_{[1, 0]}: x \coloneqq 1; \mathbf{skip}|1_{[0, 2]}:\mathbf{skip},
\{(x \mapsto (0, [0, 2]))
\xrightarrow{0_{[1,0]}\mathbf{W}x=v}
0_{[2,2]}: \mathbf{skip}|1_{[0, 2]}:\mathbf{skip}, \{(x \mapsto (1, [2, 2]))
}
\end{prooftree}
\end{center}

\item Suppose that $p, m \xrightarrow{l}\xrightarrow{l_1}\dots
\xrightarrow{l_n}\xrightarrow{l'}$ in your vector clock semantics, where $l$
and $l'$ conflict but are separated by $l_1, \dots, l_n$. To implement a
dynamic race detector, we would like to find conditions on $l_1, \dots,
l_n$ under which there is some other execution with $l$ and $l'$ adjacent:
$p, m \xrightarrow{\hat{l_1}}\dots
\xrightarrow{\hat{l_{\hat{n}}}}\xrightarrow{\bar{l}}\xrightarrow{\bar{l'}}$
(where $\bar{l}$ and $\bar{l'}$ are like $l$ and $l'$ but perhaps with
different vector clocks). Give such a condition, as liberal as you can, and
explain why it has that property.

Unfortunately, I did not understand what the question was asking for.

\end{enumerate}

\end{examquestion}

\section{Notes Page 103}

\begin{enumerate}

\item[37.] Can you show all the conditions for \textit{O2PL} are necessary,
by giving for each an example that satisfies all the others and either is not
serialisable or deadlocks?

\begin{itemize}

\item If $a_i$ is $(\ell_j \coloneqq n)$ or $!\ell_j = n$ then for some $k <
i$ we have $a_k = \mathbf{lock} \ m_j$ without an intervening
$\mathbf{unlock} \ m_j$.

\[
\langle (\ell \coloneqq 1; \ell \coloneqq !\ell + 1)|(\ell \coloneqq 2), \{s
\mapsto 0\}, \emptyset \rangle
\]

\item For each $j$, the subsequence of $a_1$, $a_2$, $\dots$ with labels
$\mathbf{lock} \ m_j$ and $\mathbf{unlock} \ m_j$ is a prefix of
$((\mathbf{lock} \ m_j)(\mathbf{unlock} \ m_j))^*$. Moreover, if
$\neg(e_k \stackrel{a}{\longrightarrow})$ then the subsequence does not end
in a $\mathbf{lock} \ m_j$.

\[
\begin{split}
&\langle \\
&\mathbf{lock} \ M_{\ell}; \ell \coloneqq 0; \ell \coloneqq !\ell + 1;
\mathbf{unlock} \ M_\ell \\
&| \\
&\mathbf{unlock} \ M_{\ell} \\
&| \\
&\mathbf{lock} \ M_{\ell}; \ell \coloneqq 3; \mathbf{unlock} \ M_\ell, \\
&\{\ell \mapsto 0\}, \\
&\{M_\ell \mapsto \mathbf{false}\} \\
&\rangle \\
\end{split}
\]

\item If $a_i = \mathbf{lock} \ m_j$ and $a_{i'} = \mathbf{unlock} \ m_{j'}$
then $i < i'$

\[
\begin{split}
&\langle \\
&
\mathbf{lock} \ M_{\ell};
\ell \coloneqq 1;
\mathbf{unlock} \ M_{\ell};
\mathbf{lock} \ M_{\ell};
\ell \coloneqq !\ell + 1;
\mathbf{unlock} \ M_{\ell}; \\
&| \\
&
\mathbf{lock} \ M_{\ell};
\ell \coloneqq 0;
\mathbf{unlock} \ M_{\ell}
,\\
&\{\ell \mapsto 0\}, \\
&\{M_\ell \mapsto \mathbf{true}\} \\
&\rangle
\end{split}
\]

\item If $a_i = \mathbf{lock} \ m_j$ and $a_{i'} = \mathbf{lock} \ m_{j'}$
and $i < i'$ then $j < j'$.

\[
\begin{split}
&\langle \\
&\mathbf{lock} \ M_1; \mathbf{lock} \ M_2; \mathbf{unlock} \ M_2;
\mathbf{uklock} \ M_1 \\
&|\\
&\mathbf{lock} \ M_2; \mathbf{lock} \ M_1; \mathbf{unlock} \ M_2;
\mathbf{uklock} \ M_1 \\
&\emptyset \\
&\{M_{1} \mapsto \mathbf{true}, M_{2} \mapsto \mathbf{2}\} \\
&\rangle \\
\end{split}
\]

\end{itemize}

\item [39.] Write a semantics for an extension of L1 with threads that are
more like Unix threads (e.g.\ with thread ids, fork, etc\dots). Include some
of the various ways Unix threads can exchange information.
\begin{align*}
&Process, p \Coloneqq (p|p) | (e, s) \\
&Expression, e \Coloneqq \dots |\ fork()\ |\ exit(e)\ |\ join()
\end{align*}

\begin{center}
\begin{prooftree}
\infer0[FORK]{
\langle fork(), s \rangle \to \langle ((0, s)|(1, s)) \rangle
}
\end{prooftree}

\vspace{0.2cm}

\begin{prooftree}
\hypo{
\langle p_1 \rangle \to \langle p_1' \rangle
}
\infer1[PARL]{
\langle p_1|p_2 \rangle \to \langle p_1'|p_2 \rangle
}
\end{prooftree}

\vspace{0.2cm}

\begin{prooftree}
\hypo{
\langle p_1 \rangle \to \langle p_1' \rangle
}
\infer1[PARR]{
\langle p_1|p_2 \rangle \to \langle p_1'|p_2 \rangle
}
\end{prooftree}

\vspace{0.2cm}

\begin{prooftree}
\hypo{
\langle e, s \rangle \to \langle e', s' \rangle
}
\infer1[EXIT]{
\langle exit(e), s \rangle \to \langle exit(e'), s' \rangle
}
\end{prooftree}

\vspace{0.2cm}

\begin{prooftree}
\infer0[JOIN]{
\langle ((join(), s)|(exit(n), s')) \rangle
\to
\langle n, s \rangle
}
\end{prooftree}

\vspace{0.5cm}

\begin{prooftree}
\infer0[FORK\_T]{\Gamma \vdash fork(): \text{int}}
\end{prooftree}

\vspace{0.2cm}

\begin{prooftree}
\infer0[EXIT\_T]{\Gamma \vdash exit(e): unit}
\end{prooftree}

\vspace{0.2cm}

\begin{prooftree}
\infer0[JOIN\_T]{\Gamma \vdash join(): \text{int}}
\end{prooftree}

\end{center}

\end{enumerate}
\section{Notes Page 109}

\begin{enumerate}

\item [40.] Prove some of the other cases of the Congruence theorem for
equivalence in L1.

Assume that $e \simeq_\Gamma^T e'$

\begin{itemize}[label=\textbf{case}]

\item $C = \_ \ op \ e_2$

Therefore $C[e]$ is of the form $e_1 \ op \ e_2$

Recall the reduction rule $op1$
\[
\dfrac{
\langle e, s \rangle \to \langle e', s' \rangle'
}{
\langle e \ op \ e_2, s \rangle \to \langle e' \ op \ e_2, s' \rangle
}
\]

\begin{itemize}[label=\textbf{case}]

\item $e \to^\omega \wedge e' \to^\omega$

The reduction rule $op1$ will be applied until $e$ has been reduced to a
value. However, by assumption $e \to^\omega$ -- so will never be
reduced to a value. This implies that the $op1$ reduction rule can be
infinitely applied. Therefore $C[e] \to^\omega$.

Similarly, $C[e']\to^\omega$.

Therefore $C[e] \simeq^T_\Gamma C[e']$ and the congruence property holds in
this case.

\item $e \to^* v \wedge e' \to^* v$

The reduction rule $op1$ will be applied until $e$ has been reduced to a
value. By assumption, this happens in a finite number of steps. Therefore
$C[e] \to^* v \ op \ e_2$. Similarly, $C[e'] \to^* v \ op \ e_2$.

Since both expressions have reduced to the same expression, all further
reductions will be the same. We can therefore conclude that $C[e]
\simeq^T_\Gamma C[e']$ and the congruence property holds in this case.

\end{itemize}

Since the congruence property for $\simeq^T_\Gamma$ holds for both sub-cases,
we can conclude that it holds in this case.

\item $\mathbf{if} \ e_1 \ \mathbf{then} \ \_ \ \mathbf{else} \ e_3$

Recall the reduction rules for $if$. Invariant of the content of $e$ or
$e'$, both $C[e]$ and $C[e']$ will both evaluate the expression $e_1$ until
it is reduced to a value.

\begin{itemize}[label=\textbf{case}]

\item $e_1 \to^\omega$

In this case, the reduction rule $if1$ will be infinitely applied for both
$C[e]$ and $C[e']$. Therefore $C[e] \to^\omega$ and $C[e'] \to ^\omega$.
Since both $C[e]$ and $C[e']$ loop infinitely: $C[e] \simeq^T_\Gamma C[e']$.

\item $e_1 \to^* \mathbf{true}$

\begin{align*}
&e_1 \to^* \mathbf{true} \Longrightarrow \\
&C[e] \to^* \mathbf{if\ true\ then\ }e \mathbf{\ else\ }e_2
\end{align*}

In this case, the reduction rule $if2$ can be applied. Therefore $C[e] \to^*
e$. Similarly, $C[e'] \to^* e'$. Since $e \simeq_\Gamma^T e'$, we can
conclude that $C[e] \simeq^T_{\Gamma} C[e']$ -- therefore contextual
equivalence holds in this case.

\item $e_1 \to^* \mathbf{false}$

\begin{align*}
&e_1 \to^* \mathbf{false} \Longrightarrow \\
&C[e] \to^* \mathbf{if\ false\ then\ }e \mathbf{\ else\ }e_2
\end{align*}

In this case, the reduction rule $if3$ can be applied -- so $C[e] \to^* e_2$.
Similarly, $C[e']\to^* e_2$. Since both expressions have reduced to the same
expression, they are equivalent -- therefore $C[e]\simeq^T_\Gamma C[e']$ in
this case.

\end{itemize}

Since contextual equivalence is proved for all subcases, it must hold in
this case.

\end{itemize}

\end{enumerate}

\begin{examquestion}{2017}{6}{10}

Let $x$ range over a set $X$ of identifiers, $n$ range over the natural
numbers $\mathbb{N}$ and $s$ range over stores: total functions from $X$ to
$\mathbb{N}$.

Consider a language with the following abstract syntax.
\[
e \Coloneqq n \ | \ x \coloneqq e \ | \ !x \ | \ e_1; e_2
\]
\begin{enumerate}[label=(\alph*)]

\item Define a conventional deterministic small-step operational semantics
$\langle e, s \rangle \to \langle e', s' \rangle $ for the language. Comment
briefly on the choices you make.

The language does not contain $\mathbf{skip}$. It is therefore necessary for
sequences where the first expression is an integer to transition to the
second sequence. It also makes the most logical sense for assignment to
return the value. This permits ie chained assignment. The language uses
left-to-right evaluation.

\begin{center}
\begin{prooftree}
\infer0[(assign1)]{
\langle x \coloneqq n, s \rangle
\to
\langle n, s + \{x \mapsto n\} \rangle
}
\end{prooftree}
\end{center}

\begin{center}
\begin{prooftree}
\hypo{
\langle e, s \rangle
\to
\langle e', s' \rangle
}
\infer1[(assign2)]{
\langle x \coloneqq e, s \rangle
\to
\langle x \coloneqq e', s' \rangle
}
\end{prooftree}

\begin{prooftree}
\hypo{
s(x) = n
}
\infer1[(deref)]{
\langle !x, s \rangle
\to
\langle n, s \rangle
}
\end{prooftree}

\begin{prooftree}
\infer0[(seq1)]{
\langle n; e, s\rangle
\to
\langle e, s \rangle
}
\end{prooftree}

\begin{prooftree}
\hypo{
\langle e_1, s \rangle
\to
\langle e_1', s' \rangle
}
\infer1[(seq2)]{
\langle e_1; e_2, s \rangle
\to
\langle e_1'; e_2, s' \rangle
}
\end{prooftree}
\end{center}

\item If your language is deterministic and terminating, the operating
semantics implicitly defines a more abstract semantics: we can regard each
expression as a function over stores $\llbracket e \rrbracket$ that takes
store $s$ to the unique number $n$ and store $s'$ such that
\[
\langle e, s \rangle
\longrightarrow^*
\langle n, s' \rangle
\wedge
\nexists e'', s''. \langle n, s' \rangle
\longrightarrow
\langle e'', s'' \rangle
\]

This language is quite limited in expressiveness. Describe, as clearly and
precisely as you can, the set of functions from stores to (number, store)
pairs that are expressible as $ \llbracket e \rrbracket $ for some $e$.

The language has no ways of \textit{modifying} numbers or branching.
Informally, the value of every location in the final store after an
expression has evaluated is either a value in the initial store or is
hardcoded into the program.

Formally:
\begin{gather*}
\forall e, s, n, s'. \langle e, s \rangle \to^* \langle n, s' \rangle
\Longrightarrow \\
(\forall y \in \text{dom}(s'). \exists x \in \text{dom}(s). s
(x) = s'(y)) \vee
(\forall s''. \langle e, s'' \rangle \to^* \langle n', s''' \rangle
\Longrightarrow s'(y) = s'''(y)) \\ \wedge \\
(\exists x \in s. s (x) = n) \vee
(\forall s''. \langle e, s''
\rangle \to^* \langle n', s''' \rangle \Longrightarrow n = n')
\end{gather*}

\item the primitive contexts $C$ for this language are expressions with a
single hole:
\[
C \Coloneqq x \coloneqq \_ \ | \ e_1; \_ \ | \ \_; e_2
\]
Write $C[e]$ for the expression resulting from replacing the hole in $C$ by
$e$.

Say a binary relation $\sim$ over expressions is a congruence if $e \sim
e'$ implies $\forall C. C[e] \sim C[e']$.

Say a binary relation $\sim$ over expressions respects final values if $e
\sim e'$ implies $\forall s_0, n, n', s, s'. ( \langle e, s_0 \rangle
\longrightarrow\langle n, s \rangle \wedge \langle e', s_0 \rangle
\longrightarrow \langle n', s' \rangle ) \Longrightarrow n = n' $

Use your characterisation of part (b) to define an equivalence relation over
expressions that is a congruence and respects final values. Explain briefly
why it has those properties
\begin{gather*}
e \sim e' \Longleftrightarrow \\
\forall s_0. (\langle e, s_0 \rangle \longrightarrow^* \langle n, s \rangle \wedge
\langle e', s_0 \rangle \longrightarrow^* \langle n', s' \rangle
\Longrightarrow
n = n' \wedge s = s') \\
\end{gather*}
This equivalence respects final values. This language does not contain any
loops and so every valid expression is guaranteed to terminate.
When the expressions have reduced to an integer, this equivalence requires
that integer to be the same. This is the definition of respecting final
values. The equivalence also requires that any store the expressions
evaluate to are the same. So the side effects of any two expressions which
are equivalent are the same. So in any context: the return value and the
effects are the same. So they are both contextually equivalent and respect
final values.

\item Define a terminating algorithm that, for any expressions e and e',
computes whether $e \sim e'$ or not. Explain informally why it is correct.

Take two expressions. Consider the set of identifiers which occur in either
expression. Place a total order on these -- let $x_n$ be the $n^{\text{th}}$
identifier under this total order. Define two stores $s_e = \{(x_i, 2 \cdot i
)\}$ and $s_o = \{(x_i, 2 \cdot i + 1)\}$.

Evaluate each expression with each store:
\begin{align*}
\langle e, s_1 \rangle \longrightarrow^* \langle n_1, s_1' \rangle \\
\langle e', s_2 \rangle \longrightarrow^* \langle n_2, s_2' \rangle \\
\langle e, s_1 \rangle \longrightarrow^* \langle n_1', s_1'' \rangle \\
\langle e', s_2 \rangle \longrightarrow^* \langle n_2', s_2'' \rangle
\end{align*}
\begin{gather*}
n_1 = n_1' \wedge n_2 = n_2' \wedge s_1 = s_1' \wedge s_2 = s_2' \\
\Longleftrightarrow \\
e \sim e'
\end{gather*}
This algorithm evaluates expressions with an initial store with unique
values. If the results of the expressions are the same, it repeats with
another unique store to verify there is no situation where $e$ assigns a
constant value which happens to be the same as the value in the location
$e'$ assigned or vice-versa.

If this algorithm finds $e \nsim e'$ then the two expressions produced
different results in one of the tests. So there exists a counterexample to
$e \sim e'$. Therefore $e \nsim e'$.

The language has no integer manipulation. All the machine can do
is shuffle the contents of the state or assign them to constants. This is
also true for the return value. Since $\forall x. s(x)$ is unique and the
value of a location $s(x)$ after $e$ executes is $n$ then
either $e$ assigns $s(x)$ to the constant $n$ or there exists some $y$ such
that $e$ assigns $s(x) = s(y)$. By running the algorithm twice with
disjoint stores, we can tell for each location $\ell$ whether
the effect of an expression $e$ is to assign it to a constant or the value
of another location. We can also tell which constant or location this was. If
$e \nsim e'$, then either their effects on the return value will be
different or their effects on some store will be different. This test passes
if and only if their effects on both the store and the return value are the
same in all situations. Therefore, if $e \nsim e'$ then the test does not pass.

Combining this and the previous result:
\[
e \sim e' \Longleftrightarrow \text{ the test passes}
\]

\end{enumerate}

\end{examquestion}

\begin{examquestion}{2004}{6}{11}

L1 has the expression syntax
\[
\begin{split}
e \coloneqq & \ n \ | \ b \ | \ e_1 \ op \ e_2 \ | \ \mathbf{if} \ e_1 \
\mathbf{then} \ e_2 \ \mathbf{else} \ e_3 \\
&| \ell \coloneqq e \ | \ !\ell \ | \ \mathbf{skip} \ | \ e_1; e_2 \ | \
\mathbf{while} \ e_1 \ \mathbf{do} \ e_2
\end{split}
\]

\begin{enumerate}

\item Give the reduction rules for conditionals and while-loops.

\begin{center}
\begin{prooftree}
\hypo{
\langle e_1, s \rangle
\to
\langle e_1', s' \rangle
}
\infer1[(if1)]{
\langle \mathbf{if} \ e_1 \ \mathbf{then} \ e_2 \ \mathbf{else} \ e_3, s
\rangle
\to
\langle \mathbf{if} \ e_1' \ \mathbf{then} \ e_2 \ \mathbf{else} \ e_3, s'
\rangle
}
\end{prooftree}
\end{center}

\begin{center}
\begin{prooftree}
\infer0[(if2)]{
\langle \mathbf{if} \ \mathbf{true} \ \mathbf{then} \ e_2 \ \mathbf{else}
\ e_3, s
\rangle
\to
\langle e_2, s \rangle
}
\end{prooftree}
\end{center}

\begin{center}
\begin{prooftree}
\infer0[(if3)]{
\langle \mathbf{if} \ \mathbf{false} \ \mathbf{then} \ e_2 \ \mathbf{else}
\ e_3, s
\rangle
\to
\langle e_3, s \rangle
}
\end{prooftree}
\end{center}

\begin{center}
\begin{prooftree}
\infer0[(while)]{
\langle \mathbf{while} \ e \ \mathbf{do} \ e', s\rangle
\to
\langle \mathbf{if} \ e \ \mathbf{then} \ e'; \mathbf{while} \ e \
\mathbf{do} \ e' \ \mathbf{else} \ \mathbf{skip} , s \rangle
}
\end{prooftree}
\end{center}

\item Define semantic equivalence $e_1 \simeq^T_\Gamma e_2$ for L1.

Informally, either both $e_1$ and $e_2$ loop infinitely or they both
terminate and their final values and stores are the same.

\[
\begin{split}
e_1 \simeq^T_{\Gamma} e_2 \Longleftrightarrow \\
(\Gamma \vdash e_1 : T \wedge \Gamma \vdash e_2 : T \wedge \\
\forall s. \text{dom}(\Gamma) \subseteq \text{dom}(s) \Longrightarrow \\
\langle e_1, s \rangle \longrightarrow^{\omega} \\
\wedge \langle e_2, s \rangle \longrightarrow^{\omega} \\
\vee \langle e_1, s \rangle \longrightarrow^* \langle v, s' \rangle \\
\wedge \langle e_2, s \rangle \longrightarrow^* \langle v, s' \rangle \\
\end{split}
\]

\item For each of the following pairs, state whether they are semantically
equivalent; if not, state a nontrivial condition on the subexpressions $e$,
$e_1$, $e_2$, $e_3$ that makes them so, and explain informally why it suffices.

\begin{enumerate}

\item  $l \coloneqq 3; e \stackrel{?}{\simeq} e; l \coloneqq 3$

These are not equivalent, consider the case $e = l \coloneqq 0$.

A necessary and sufficient criteria is: for all stores $s$ such that $l\in
\text{dom}(s)$. $e$ assigns a value to $l$ before any reads and the final
access to $l$ is a write setting $l$ to 3. Note that this holds for no
accesses to the variable.

\item $e; (\mathbf{if} \ e_1 \ \mathbf{then} \ e_2 \ \mathbf{else} \ e_3)
\stackrel{?}{\simeq} \mathbf{if} \ e_1 \ \mathbf{then} \ e; e_2 \
\mathbf{else} \ e; e_3$

These two expressions are not equivalent. This is because $e$ is evaluated
before $e_1$ in the first expression and $e_1$ is evaluated before $e$ in
the second expression.

\[
e; \mathbf{if} \ e_1 \ \mathbf{then} \ \mathbf{skip} \ \mathbf{else} \
\mathbf{skip} \simeq^T_{\Gamma} \mathbf{if} \ e_1 \ \mathbf{then} \
\mathbf{skip} \ \mathbf{else} \ \mathbf{skip}; e
\]

\item $e; (\mathbf{if} \ e_1 \ \mathbf{then} \ e_2 \ \mathbf{else} \ e_3)
\stackrel{?}{\simeq} \mathbf{if} \ e; e_1 \ \mathbf{then} \ e_2 \
\mathbf{else} \
e_3$

These are semantically equivalent. Since in all cases, the order in which the
expressions is evaluated is the same.

\item $\mathbf{while} \ !l \geq 0 \ \mathbf{do} \ (e_2; e_3)
\stackrel{?}{\simeq} \mathbf{if} \ !l \geq 0 \ \mathbf{then} \ e_2;
(\mathbf{while} \ !l \geq 0 \ \mathbf{do} \ (e_3; e_2)); e_3 \ \mathbf{else}
 \ \mathbf{skip}$

These are both semantically equivalent. The predicate $l \geq 0$ is always
true since $!l \in \mathbb{N}$ and all natural numbers are $\geq 0$. So both
expressions a never-ending sequence of reductions. So they are semantically
equivalent.

Note that if the predicate $l \geq 0$ was not always true then the
expressions would not be semantically equivalent.

\end{enumerate}

\end{enumerate}

\end{examquestion}

\begin{examquestion}{2021}{4}{9}

\begin{enumerate}

\item Suppose that $l : \text{intref} \in \Gamma$. Now, consider the
following program equivalence for L1:
\[
(\text{if} \ !l \leq 0 \ \text{then} \ e_1 \ \text{else} \ e_2); e_3
\simeq^{\Gamma}_{\text{unit}} (\text{it} \ !l \leq 0 \ \text{then} \ e_1;
e_3 \ \text{else} \ e_2; e_3)
\]

\begin{enumerate}

\item Explain informally but carefully why this equivalence holds.

The order in which the subexpressions are evaluated is the same in both
programs.

\item Using the definition of semantic equivalence, prove that this
equivalence holds.

Split into cases on the truth of $!l \leq 0$:

\begin{itemize}

\item[\textbf{case}] $!l \leq 0$

For the LHS program:
\[
\begin{split}
\langle (\text{if} \ !l \leq 0 \ \text{then} \ e_1 \ \text{else} \ e_2);
e_3, s \rangle &\longrightarrow^2 \\
\langle (\text{if} \ \mathbf{true} \ \text{then} \ e_1 \ \text{else} \ e_2);
e_3, s \rangle &\longrightarrow \\
\langle e_1; e_3, s \rangle &\longrightarrow \\
\end{split}
\]

For the RHS program:
\[
\begin{split}
\langle (\text{if} \ !l \leq 0 \ \text{then} \ ((e_1; e_3)) \ \text{else} \
(e_2; e_3)), s \rangle &\longrightarrow^2\\
\langle (\text{if} \ \mathbf{true} \ \text{then} \ ((e_1; e_3)) \ \text{else} \
(e_2; e_3)), s \rangle &\longrightarrow \\
\langle e_1; e_3, s \rangle &\longrightarrow \\
\end{split}
\]

Since both programs reduce to the same program (and a program is trivially
semantically equivalent to itself), we can conclude that the two original
programs are semantically equivalent.

\item [\textbf{case}] $!l > 0$

For the LHS program:
\[
\begin{split}
\langle (\text{if} \ !l \leq 0 \ \text{then} \ e_1 \ \text{else} \ e_2);
e_3, s \rangle &\longrightarrow^2 \\
\langle (\text{if} \ \mathbf{false} \ \text{then} \ e_1 \ \text{else} \ e_2);
e_3, s \rangle &\longrightarrow \\
\langle e_2; e_3, s \rangle &\longrightarrow \\
\end{split}
\]

For the RHS program:
\[
\begin{split}
\langle (\text{if} \ !l \leq 0 \ \text{then} \ ((e_1; e_3)) \ \text{else} \
(e_2; e_3)), s \rangle &\longrightarrow^2\\
\langle (\text{if} \ \mathbf{false} \ \text{then} \ ((e_1; e_3)) \
\text{else} \
(e_2; e_3)), s \rangle &\longrightarrow \\
\langle e_2; e_3, s \rangle &\longrightarrow \\
\end{split}
\]

In this case, both programs reduce to the same program. A program is
trivially semantically equivalent to itself. Therefore the two programs are
semantically equivalent.

\end{itemize}

\end{enumerate}

\item Now, consider the following \textit{non}-equivalence

\[
e_3; (\text{if} \ !l \leq 0 \ \text{then} \ e_1 \ \text{else} \ e_2)
\simeq^{\Gamma}_{\text{unit}} (\text{if} \ !l \leq 0 \ \text{then} \ e_3;
e_1 \ \text{else} \ e_3; e_2)
\]

\begin{enumerate}

\item Give a well-typed example exhibiting a counterexample of this
equivalence.

\begin{align*}
e_1 &= l \coloneqq 1 &
e_2 &= l \coloneqq 2 &
e_3 &= l \coloneqq 3 &
s = \{l \mapsto 0\}
\end{align*}

In the LHS program, $e_3$ is executed. When the predicate ie evaluated, it is
false. So the else-branch is taken and $l$ is set to 2.

In the RHS program, the predicate is first evaluated and is decided to be
true. So the if-branch is taken, which sets $l$ to 1.

\item Give a statically decidable condition under which the transformation
is valid.

The transformation is valid if $e$ does not write to $l$. Let the predicate
$\Phi(e, l)$ mean ``$e$ cannot write to $l$''.

Define it inductively:
\begin{align*}
\Phi(v, l) &= \mathbf{true} \\
\Phi(l \coloneqq e, l) &= \mathbf{false} \\
\Phi(l' \coloneqq e, l) &= \Phi(e, l) \\
\Phi(\text{fn} \ x \Rightarrow e, l) &= \Phi(e, l) \\
\Phi(e \ e', l) &= \Phi(e, l) \wedge \Phi(e', l) \\
\Phi(\text{while} \ e \ \text{do} \ e', l) &= \Phi(e, l) \wedge \Phi(e', l) \\
\Phi(\text{if} \ e_1 \ \text{then} \ e_2 \ \text{else} \ e_3, l) &=
\Phi(e_1, l) \wedge \Phi(e_2, l) \wedge \Phi(e_3, l) \\
\Phi(e_1 \ \text{op} \ e_2, l) &= \Phi(e_1, l) \wedge \Phi(e_2, l) \\
\end{align*}
The transformation is valid if $\Phi(e, l)$.

\end{enumerate}

\end{enumerate}

\end{examquestion}

\end{document}
