\newcommand{\svcourse}{CST Part IA: Software Engineering and Security}
\newcommand{\svnumber}{1}
\newcommand{\svvenue}{Microsoft Teams}
\newcommand{\svdate}{2022-05-11}
\newcommand{\svtime}{15:00}
\newcommand{\svuploadkey}{CBd13xmL7PC1zqhNIoLdTiYUBnxZhzRAtJxv/ytRdM1r7qIfwMsxeVwM/pPcIo8l}

\newcommand{\svrname}{Dr Sam Ainsworth}
\newcommand{\jkfside}{oneside}
\newcommand{\jkfhanded}{yes}

\newcommand{\studentname}{Harry Langford}
\newcommand{\studentemail}{hjel2@cam.ac.uk}


\documentclass[10pt,\jkfside,a4paper]{article}

% DO NOT add \usepackage commands here.  Place any custom commands
% into your SV work files.  Anything in the template directory is
% likely to be overwritten!

\usepackage{fancyhdr}

\usepackage{lastpage}       % ``n of m'' page numbering
\usepackage{lscape}         % Makes landscape easier

\usepackage{verbatim}       % Verbatim blocks
\usepackage{listings}       % Source code listings
\usepackage{graphicx}
\usepackage{float}
\usepackage{epsfig}         % Embed encapsulated postscript
\usepackage{array}          % Array environment
\usepackage{qrcode}         % QR codes
\usepackage{enumitem}       % Required by Tom Johnson's exam question header

\usepackage{hhline}         % Horizontal lines in tables
\usepackage{siunitx}        % Correct spacing of units
\usepackage{amsmath}        % American Mathematical Society
\usepackage{amssymb}        % Maths symbols
\usepackage{amsthm}         % Theorems

\usepackage{ifthen}         % Conditional processing in tex

\usepackage[top=3cm,
            bottom=3cm,
            inner=2cm,
            outer=5cm]{geometry}

% PDF metadata + URL formatting
\usepackage[
            pdfauthor={\studentname},
            pdftitle={\svcourse, SV \svnumber},
            pdfsubject={},
            pdfkeywords={9d2547b00aba40b58fa0378774f72ee6},
            pdfproducer={},
            pdfcreator={},
            hidelinks]{hyperref}

\renewcommand{\headrulewidth}{0.4pt}
\renewcommand{\footrulewidth}{0.4pt}
\fancyheadoffset[LO,LE,RO,RE]{0pt}
\fancyfootoffset[LO,LE,RO,RE]{0pt}
\pagestyle{fancy}
\fancyhead{}
\fancyhead[LO,RE]{{\bfseries \studentname}\\\studentemail}
\fancyhead[RO,LE]{{\bfseries \svcourse, SV~\svnumber}\\\svdate\ \svtime, \svvenue}
\fancyfoot{}
\fancyfoot[LO,RE]{For: \svrname}
\fancyfoot[RO,LE]{\today\hspace{1cm}\thepage\ / \pageref{LastPage}}
\fancyfoot[C]{\qrcode[height=0.8cm]{\svuploadkey}}
\setlength{\headheight}{22.55pt}


\ifthenelse{\equal{\jkfside}{oneside}}{

 \ifthenelse{\equal{\jkfhanded}{left}}{
  % 1. Left-handed marker, one-sided printing or e-marking, use oneside and...
  \evensidemargin=\oddsidemargin
  \oddsidemargin=73pt
  \setlength{\marginparwidth}{111pt}
  \setlength{\marginparsep}{-\marginparsep}
  \addtolength{\marginparsep}{-\textwidth}
  \addtolength{\marginparsep}{-\marginparwidth}
 }{
  % 2. Right-handed marker, one-sided printing or e-marking, use oneside.
  \setlength{\marginparwidth}{111pt}
 }

}{
 % 3. Alternating margins, two-sided printing, use twoside.
}


\setlength{\parindent}{0em}
\addtolength{\parskip}{1ex}

% Exam question headings, labels and sensible layout (courtesy of Tom Johnson)
\setlist{parsep=\parskip, listparindent=\parindent}
\newcommand{\examhead}[3]{\section{#1 Paper #2 Question #3}}
\newenvironment{examquestion}[3]{
\examhead{#1}{#2}{#3}\setlist[enumerate, 1]{label=(\alph*)}\setlist[enumerate, 2]{label=(\roman*)}
\marginpar{\href{https://www.cl.cam.ac.uk/teaching/exams/pastpapers/y#1p#2q#3.pdf}{\qrcode{https://www.cl.cam.ac.uk/teaching/exams/pastpapers/y#1p#2q#3.pdf}}}
\marginpar{\footnotesize \href{https://www.cl.cam.ac.uk/teaching/exams/pastpapers/y#1p#2q#3.pdf}{https://www.cl.cam.ac.uk/\\teaching/exams/pastpapers/\\y#1p#2q#3.pdf}}
}{}


\usepackage{mathtools}
\usepackage{tikz}
\usetikzlibrary{arrows,shapes,automata,petri,positioning}

\begin{document}

\begin{examquestion}{2004}{4}{1}

\begin{enumerate}

\item A context-free grammar can be formally defined as a 4-tuple. Give a 
precise statement of what the components are
\[
G = (N, T, P, S)
\]

\begin{itemize}

\item $G$ is the grammar

\item $N$ is the set of nonterminals

A Nonterminal is an internal symbol. These represent concepts such as
expressions or statements.

\item $T$ is the set of terminals

A Terminal is a token passed to the parser by the lexer. These may
correspond to an individual literal or a sequence of literals. Terminals are
indivisible. The input to any PDA is a sequence of terminals.

\item $P \subseteq N \times (N \cup T)*$ is the set of productions

A production is of the form $A \to \alpha$ and says that it is legal for any
occurrence of $A$ to be replaced with $\alpha$ at any point.

\item $S \in N$ is the start symbol

\end{itemize}

\item Explain the difference between a grammar and the language it generates.

A grammar is a set of rules which is used to generate a language.

The language generated by a grammar is a set of strings.

Each grammar generates exactly one language, however a given language may be
generated by many languages.

\item Explain what makes a grammar ambiguous, with reference to the grammar 
which may commonly be expressed as a ``rule''
\[
E \Coloneqq 1 \ | \ 2 \ | \ X \ | \ E + E \ | \ E * E \ | \ - E
\]
where $X$ is an identifier

A grammar is ambiguous if there exists any string for which there are
multiple ways the grammar can be used to generate that string. Consider the
string $1 + 2 * X$ with the grammar above.

Under the grammar above, there are two possible parse trees for $1 + 2 * X$
and therefore the grammar is ambiguous.

\begin{center}
\begin{tikzpicture}
\node (start) {$E$};
\node (leftplus) [below left = of start] {$E$};
\node (times) [below = of start] {$*$};
\node (righte) [below right = of start] {$E$};
\path [->] (start) edge (leftplus);
\path [->] (start) edge (times);
\path [->] (start) edge (righte);
\node (l2E) [below left = of leftplus] {$E$};
\node (r2E) [below right = of leftplus] {$E$};
\node (plus) [below = of leftplus] {$+$};
\path [->] (leftplus) edge (l2E);
\path [->] (leftplus) edge (r2E);
\path [->] (leftplus) edge (plus);
\node (n1) [below = of l2E] {$1$};
\node (n2) [below = of r2E] {$2$};
\node (nx) [below = of righte] {$X$};
\path [->] (l2E) edge (n1);
\path [->] (r2E) edge (n2);
\path [->] (righte) edge (nx);
\end{tikzpicture}
\hspace{2cm}
\begin{tikzpicture}
\node (start) {$E$};
\node (leftplus) [below right = of start] {$E$};
\node (times) [below = of start] {$+$};
\node (righte) [below left = of start] {$E$};
\path [->] (start) edge (leftplus);
\path [->] (start) edge (times);
\path [->] (start) edge (righte);
\node (l2E) [below left = of leftplus] {$E$};
\node (r2E) [below right = of leftplus] {$E$};
\node (plus) [below = of leftplus] {$*$};
\path [->] (leftplus) edge (l2E);
\path [->] (leftplus) edge (r2E);
\path [->] (leftplus) edge (plus);
\node (n1) [below = of l2E] {$2$};
\node (n2) [below = of r2E] {$X$};
\node (nx) [below = of righte] {$1$};
\path [->] (l2E) edge (n1);
\path [->] (r2E) edge (n2);
\path [->] (righte) edge (nx);
\end{tikzpicture}
\end{center}

\item For the ``rule'' in part (c), give a formal grammar containing this
``rule'' and adhering to your definition in part (a).
\[
\begin{split}
G = (&\{E\}, \\
&\{1, 2, X\}, \\
&\{(E, 1), (E', 2), (E', X), (E, E + E), (E, E * E), (E, -E)\} \\
&E)
\end{split}
\]
\begin{align*}
E  &\Coloneqq T \ E' \\
E' &\Coloneqq +T \ E' \ | \ *T \ E' \ | \ \varepsilon \\
T  &\Coloneqq N \ | \ -N \\
N  &\Coloneqq 1 \ | \ 2 \ | \ X
\end{align*}

\item Give non-ambiguous grammars each generating the same language as your
grammar in part (d) for the cases:

\begin{enumerate}

\item ``$-$`` is most tightly binding and ``$+$'' and ``$*$'' have equal
binding power and associate to the left.
\[
\begin{split}
G_1 = (&\{E, E', N, T\}, \\
	   &\{1, 2, X\}, \\
	   &\{(E, TE'), (E', E'T+), (E', E'T*), (E', \varepsilon), (T, N), (T,
	   -N), (N, 1), (N, 2), (N, X)\} \\
	   &E)
\end{split}
\]
\begin{align*}
E  &\Coloneqq E' \ T \\
E' &\Coloneqq E' \ T+ \ | \ E' \ T* \ | \ \varepsilon \\
T  &\Coloneqq N \ | \ -N \\
N  &\Coloneqq 1 \ | \ 2 \ | \ X
\end{align*}

\item ``$-$'' is most tightly binding and ``$+$'' and ``$*$'' have equal
binding power and associate to the right.
\[
\begin{split}
G_2 = (&\{E, E', N, T\}, \\
	   &\{1, 2, X\}, \\
	   &\{(E, TE'), (E', +TE'), (E', *TE'), (E', \varepsilon), (T, N), (T,
	   -N), (N, 1), (N, 2), (N, X)\} \\
	   &E) \\
\end{split}
\]
\begin{align*}
E  &\Coloneqq T \ E' \\
E' &\Coloneqq +T \ E' \ | \ *T \ E' \ | \ \varepsilon \\
T  &\Coloneqq N \ | \ -N \\
N  &\Coloneqq 1 \ | \ 2 \ | \ X
\end{align*}

\item ``$-$'' binds more tightly than ``$+$'', but less tightly than
``$*$'', with ``$+$'' left-associative and ``$*$'' right-associative so that
``$-a + -b * c * c + d$'' is associated as ``$((-a) + (-(b * (c * d)))) + d$''.
\[
\begin{split}
G_3 = (&\{E, E', A, T, T', N\}, \\
	   &\{1, 2, X\}, \\
	   &\{(E, E'A), (E', E'A+), (E',\varepsilon), (A, T), (A, -T), (T, NT'),
	    (T', *NT'), (T', \varepsilon), (N, 1), (N, 2), (N, X)\} \\
	   &E) \\
\end{split}
\]
\begin{align*}
E  &\Coloneqq E' \ A \\
E' &\Coloneqq E' \ A + \ | \ \varepsilon \\
A  &\Coloneqq T \ | \ -T \\
T  &\Coloneqq N \ T' \\
T' &\Coloneqq *N \ T' \ | \ \varepsilon \\
N  &\Coloneqq 1 \ | \ 2 \ | \ X
\end{align*}

\end{enumerate}

\item Give a simple recursive descent parser for your grammar in part (e)(iii)
above which yields a value of type \texttt{ParseTree}. You may assume
operations \textit{mkplus}, \textit{mktimes}, \textit{mkneg} acting on type
\texttt{ParseTree}.

Firstly, note that the grammar (e)(iii) is by definition left-associative.
A gramamr is left-recursive if and only if it is left-associative. So there
exists no grammar which fulfils the criteria for (e)(iii) that is not
left-recursive. Left-recursive grammars cannot be parsed by a recursive
descent parser. My solution to this is to build a parse tree for the
language and then assume \textit{mkplus}, \textit{mktimes} and
\textit{mkneg} rotate parse trees into the correct shape. This algorithm
will build a valid parse tree for the grammar (e)(iii).

\begin{lstlisting}[language=Caml]
type n = E | E' | A | T | T' | N

type t = + | - | 1 | 2 | X | Epsilon

type parseTree = Branch of n * parseTree list | Leaf of t

let parse ts =
	let rec parse ts, n =
		match ts, n with
		| Plus::ts, E' ->
			let pt1, ts = parse ts, T in
			let pt2, ts = parse ts E' in
			(Branch n, [pt1; Leaf Plus; pt2]), ts
		| Times::ts, E' ->
			let pt1, ts = parse ts, T in
			let pt2, ts = parse ts E' in
			(Branch n, [pt1; Leaf Times; pt2]), ts
		| Minus::ts, E -> let pt, ts = parse (Minus::ts) T in
			(Branch n, [Leaf Minus; pt]), ts
		| Minus::ts, T -> let pt, ts = parse ts N in
			(T_P2 T, [Leaf Minus; pt]), ts
		| One::ts, N -> (Leaf One), ts
		| Two::ts, N -> (Leaf Two), ts
		| X::ts, N -> (Leaf X), ts
		| x::ts, T when x = 1 || x = 2 || x = X ->
			let pt, ts = parse (x::ts), N in
			(Branch n, [pt]), ts
		| x::ts, T when x = 1 || x = 2 || x = X ->
			let pt1, ts = parse ts, T in
			let pt2, ts = parse ts E' in
			(Branch n [pt1; pt2]), ts
		| _, E' -> (Leaf Epsilon), ts
		| _ -> raise ParseException
	in
	match parse ts E with
	| pt, [] -> mkplus (mkminus (mktimes pt))
	| _ -> raise ParseException
\end{lstlisting}

\iffalse

\begin{lstlisting}[language=Caml, mathescape=||]

(* I assume mkplus rotates the tree such that plus is left-associative *)
(* I assume mkneg pulls all -'s up until they are at the correct precedence *)

type n = E | E' | A | T | T' | N

type t = + | - | 1 | 2 | X

type parseTree = E_P1 of parseTree * parseTree
		| E'_P1 of t * parseTree * parseTree
		| E'_P2 of Epsilon
		| ...

let parse ts =
	let rec parse ts, n =
		match ts with
		| [] ->
			if n != E' and n != T' then
				raise ParseException
		| Plus::ts ->
			match n with
			| E' ->
				let pt1, ts = parse ts T in
				let pt2, ts = parse ts E' in
				(E'_P1 Plus pt1 pt2), ts
			| _ ->
				raise ParseException
		| Minus::ts ->
			match n with
			| E ->
				let pt, ts = parse ts T in
				(E_P1 Minus pt), ts
			| T ->
				let pt, ts = parse ts N in
				(T_P2 Minus pt), ts
			| _ ->
				raise ParseException
		| x::ts when x = 1 || x = 2 || x = X ->
			match n with
			| N -> N_P1, ts
			| T -> (T_P1 N_P1), ts
			| E ->
				let pt1, ts = parse ts, T in
				let pt2, ts = parse ts, E' in
				(E_P1 x pt1 pt2), ts
			| _ -> raise ParseException
	in
	match parse ts E in with
	| pt, [] -> mkplus (mkminus (mktimes pt))
	| _ -> raise ParseException
;;


\end{lstlisting}

\fi

\end{enumerate}

\end{examquestion}

\begin{examquestion}{2002}{4}{2}

The specification for a pocket-calculator-style programming language is as 
follows:

\begin{itemize}

\item Valid inputs consist either of an Expression followed by the
\framebox{enter} button of of an Expression followed by \framebox{store}
Identifier \framebox{enter};

\item Expressions consist of Numbers and Identifiers connected with the
binary operators \framebox{$+$}, \framebox{$\times$} and
\framebox{$\uparrow$} (in increasing binding power), with the nary operators
\framebox{$-$} and \framebox{abs}, and possibly grouped with parentheses.
Unary operators bind more strongly than \framebox{$+$} but weaker than
\framebox{$\times$} so that $-a + b$ means $(-a) + b$ but $-a \times b$
means $-(a \times b)$.

\item Numbers consist of a sequence of at least one digit, possibly
interspersed with exactly one decimal point, and possibly followed by an
exponential marker ``$e$'' followed by a signed integer, e.g. $6.023e+22$.
Identifiers are sequences of lower-case letters.

\end{itemize}

\begin{enumerate}

\newcommand{\Start}{\mathbf{Start}}
\newcommand{\Unary}{\mathbf{Unary}}
\newcommand{\OptExpression}{\mathbf{OptExpression}}
\newcommand{\Expression}{\mathbf{Expression}}
\newcommand{\Times}{\mathbf{Times}}
\newcommand{\OptTimes}{\mathbf{OptTimes}}
\newcommand{\Arrow}{\mathbf{Arrow}}
\newcommand{\OptArrow}{\mathbf{OptArrow}}
\newcommand{\Value}{\mathbf{Value}}
\newcommand{\Identifier}{\mathbf{Identifier}}
\newcommand{\OptIdentifier}{\mathbf{OptIdentifier}}
\newcommand{\Letter}{\mathbf{Letter}}
\newcommand{\Number}{\mathbf{Number}}
\newcommand{\Int}{\mathbf{Int}}
\newcommand{\OptInt}{\mathbf{OptInt}}
\newcommand{\OptDecimal}{\mathbf{OptDecimal}}
\newcommand{\OptSuffix}{\mathbf{OptSuffix}}
\newcommand{\Sign}{\mathbf{Sign}}

\item Give a Context-Free Grammar for the set of valid input sequences using
names beginning with an upper-case letter for non-terminals. It should be
complete in that you should go as far as to define e.g.
\[
\mathbf{Letter} \Coloneqq \mathbf{a} \ | \ \mathbf{b} \ | \ \mathbf{c} \ |
\ \dots \ | \ \mathbf{z}
\]
\begin{align*}
\Start &\Coloneqq \Expression \ \framebox{enter} \ | \ \Expression \
\framebox{store} \ \Identifier \ \framebox{enter} \\
\Expression &\Coloneqq \Unary \ \OptExpression \\
\OptExpression &\Coloneqq \framebox{$+$} \ \Unary \ \OptExpression \ | \
\varepsilon \\
\Unary &\Coloneqq \Times \ | \ \framebox{$-$} \ \Times \ | \ \framebox{abs} \
\Times \\
\Times &\Coloneqq \Arrow \ \OptTimes \\
\OptTimes &\Coloneqq \framebox{$\times$} \ \Arrow \ \OptTimes \\
\Arrow &\Coloneqq \Value \ \OptArrow \\
\OptArrow &\Coloneqq \framebox{$\uparrow$} \ \Value \ \OptArrow \\
\Value &\Coloneqq \Identifier \ | \ \Number \\
\Identifier &\Coloneqq \Letter \ \OptIdentifier \\
\OptIdentifier &\Coloneqq \Letter \OptIdentifier \ | \ \varepsilon \\
\Letter &\Coloneqq \mathbf{a} \ | \ \mathbf{b} \ | \ \mathbf{c} \ | \ \dots \
 | \ \mathbf{z} \\
\Number &\Coloneqq \Int \ \OptInt \ \OptDecimal \ \OptSuffix \ | \
\framebox{$.$} \ \Int \ \OptInt \ \OptSuffix \\
\Int &\Coloneqq 0 \ | \ 1 \ | \ \dots \ | \ 9 \\
\OptInt &\Coloneqq \Int \ \OptInt \ | \ \varepsilon \\
\OptDecimal &\Coloneqq \framebox{$.$} \ \OptInt \ | \ \varepsilon \\
\OptSuffix &\Coloneqq \mathbf{e} \ \Sign \ \Int \ \OptInt \\
\Sign &\Coloneqq \framebox{$+$} \ | \ \framebox{$-$} \\
\end{align*}

\item Indicate, giving brief reasoning, which non-terminals are appropriate
to be processed using lexical analysis and for which using syntax analysis
is proper.

It's appropriate to process $\Value$, $\Identifier$, $\OptIdentifier$,
$\Letter$, $\Number$, $\Int$, $\OptInt$, $\OptDecimal$, $\OptSuffix$ and
$\Sign$ in lexical analysis. This is because the langauge which these
non-terminals can match is regular and there is no binding tightness to
consider. Therefore, it's appropriate to process them during lexing.

\item Give yacc or CUP input describing those elements deemed in part (b) to
be suitable for syntax analysis. You need not give ``semantic actions''.

\begin{lstlisting}[language=C, escapeinside=()]

%token Start Expression OptExpression Unary Times OptTimes Arrow OptArrow

%%

Start 		: Expression 'enter'
		| Expression 'store' Identifier (\framebox{enter})

Expression	: Unary OptExpression

OptExpression	: '($+$)' Unary OptExpression
		| /* ($\varepsilon$) */

Unary		: Times
		| '($-$)' Times
		| 'abs' Times

Times		: Arrow OptTimes

OptTimes	: '($\times$)' Arrow OptTimes

Arrow		: Value OptArrow

OptArrow	: '($\uparrow$)' Value OptArrow
		| /* ($\varepsilon$) */
\end{lstlisting}

\end{enumerate}

\end{examquestion}

\end{document}
