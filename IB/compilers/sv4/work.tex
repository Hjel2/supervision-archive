\newcommand{\svcourse}{CST Part IA: Software Engineering and Security}
\newcommand{\svnumber}{1}
\newcommand{\svvenue}{Microsoft Teams}
\newcommand{\svdate}{2022-05-11}
\newcommand{\svtime}{15:00}
\newcommand{\svuploadkey}{CBd13xmL7PC1zqhNIoLdTiYUBnxZhzRAtJxv/ytRdM1r7qIfwMsxeVwM/pPcIo8l}

\newcommand{\svrname}{Dr Sam Ainsworth}
\newcommand{\jkfside}{oneside}
\newcommand{\jkfhanded}{yes}

\newcommand{\studentname}{Harry Langford}
\newcommand{\studentemail}{hjel2@cam.ac.uk}


\documentclass[10pt,\jkfside,a4paper]{article}
\usepackage{tikz}
\usetikzlibrary{positioning}

% DO NOT add \usepackage commands here.  Place any custom commands
% into your SV work files.  Anything in the template directory is
% likely to be overwritten!

\usepackage{fancyhdr}

\usepackage{lastpage}       % ``n of m'' page numbering
\usepackage{lscape}         % Makes landscape easier

\usepackage{verbatim}       % Verbatim blocks
\usepackage{listings}       % Source code listings
\usepackage{graphicx}
\usepackage{float}
\usepackage{epsfig}         % Embed encapsulated postscript
\usepackage{array}          % Array environment
\usepackage{qrcode}         % QR codes
\usepackage{enumitem}       % Required by Tom Johnson's exam question header

\usepackage{hhline}         % Horizontal lines in tables
\usepackage{siunitx}        % Correct spacing of units
\usepackage{amsmath}        % American Mathematical Society
\usepackage{amssymb}        % Maths symbols
\usepackage{amsthm}         % Theorems

\usepackage{ifthen}         % Conditional processing in tex

\usepackage[top=3cm,
            bottom=3cm,
            inner=2cm,
            outer=5cm]{geometry}

% PDF metadata + URL formatting
\usepackage[
            pdfauthor={\studentname},
            pdftitle={\svcourse, SV \svnumber},
            pdfsubject={},
            pdfkeywords={9d2547b00aba40b58fa0378774f72ee6},
            pdfproducer={},
            pdfcreator={},
            hidelinks]{hyperref}

\renewcommand{\headrulewidth}{0.4pt}
\renewcommand{\footrulewidth}{0.4pt}
\fancyheadoffset[LO,LE,RO,RE]{0pt}
\fancyfootoffset[LO,LE,RO,RE]{0pt}
\pagestyle{fancy}
\fancyhead{}
\fancyhead[LO,RE]{{\bfseries \studentname}\\\studentemail}
\fancyhead[RO,LE]{{\bfseries \svcourse, SV~\svnumber}\\\svdate\ \svtime, \svvenue}
\fancyfoot{}
\fancyfoot[LO,RE]{For: \svrname}
\fancyfoot[RO,LE]{\today\hspace{1cm}\thepage\ / \pageref{LastPage}}
\fancyfoot[C]{\qrcode[height=0.8cm]{\svuploadkey}}
\setlength{\headheight}{22.55pt}


\ifthenelse{\equal{\jkfside}{oneside}}{

 \ifthenelse{\equal{\jkfhanded}{left}}{
  % 1. Left-handed marker, one-sided printing or e-marking, use oneside and...
  \evensidemargin=\oddsidemargin
  \oddsidemargin=73pt
  \setlength{\marginparwidth}{111pt}
  \setlength{\marginparsep}{-\marginparsep}
  \addtolength{\marginparsep}{-\textwidth}
  \addtolength{\marginparsep}{-\marginparwidth}
 }{
  % 2. Right-handed marker, one-sided printing or e-marking, use oneside.
  \setlength{\marginparwidth}{111pt}
 }

}{
 % 3. Alternating margins, two-sided printing, use twoside.
}


\setlength{\parindent}{0em}
\addtolength{\parskip}{1ex}

% Exam question headings, labels and sensible layout (courtesy of Tom Johnson)
\setlist{parsep=\parskip, listparindent=\parindent}
\newcommand{\examhead}[3]{\section{#1 Paper #2 Question #3}}
\newenvironment{examquestion}[3]{
\examhead{#1}{#2}{#3}\setlist[enumerate, 1]{label=(\alph*)}\setlist[enumerate, 2]{label=(\roman*)}
\marginpar{\href{https://www.cl.cam.ac.uk/teaching/exams/pastpapers/y#1p#2q#3.pdf}{\qrcode{https://www.cl.cam.ac.uk/teaching/exams/pastpapers/y#1p#2q#3.pdf}}}
\marginpar{\footnotesize \href{https://www.cl.cam.ac.uk/teaching/exams/pastpapers/y#1p#2q#3.pdf}{https://www.cl.cam.ac.uk/\\teaching/exams/pastpapers/\\y#1p#2q#3.pdf}}
}{}


\begin{document}

\begin{examquestion}{2005}{5}{6}

\begin{enumerate}

\item Explain how a parse tree representing an expression can $(i)$ be
converted into stack-oriented intermediate code and then $(ii)$ be
translated into simple machine code for a register-oriented architecture
(e.g.\@ ARM or IA32) on an instruction-by-instruction basis. Also indicate
how this code might be improved to remove push-pop pairs introduced by
$(ii)$. Your answer need only consider expression forms encountered in the
expression:
\begin{center}
\texttt{h(a, g(b), c) * 3 + d}
\end{center}

\begin{enumerate}

\item A parse tree can be converted into stack-oriented intermediate code by
performing a depth-first-search and appending to the linear code when
different expressions are seen.

In OCaml-esque pseudocode:
\begin{lstlisting}[language=Caml]
type parsetree =
	| Add of parsetree * parsetree
	| Mul of parsetree * parsetree
	| Var of int ref
	| Call of ('a -> 'b) * parsetree
	| Args of parsetree * parsetree
	| Integer of int

let rec gen_code parsetree file parsetree =
  match parsetree with
  | Add(x, y) -> mk_code file x; mk_code file y; file.append(Add)
  | Mul(x, y) -> mk_code file x; mk_code file y; file.append(Mul)
  | Var(x) -> file.append(Push x); file.append(Deref x)
  | Call(f, args) -> mk_code file args; file.append(Apply(f))
  | Args(hd, args) -> mk_code file hd; mk_code file args
  | Integer(i) -> file.append(Push i)
\end{lstlisting}

\item \iffalse
We could convert a stack-oriented intermediate language into a
register-oriented intermediate language by implementing a stack in memory.
The program will have a stack pointer in execution. When the program starts
running, allocate a large block of memory. Next, initialise the code
pointer to the start of this block. Replace all \texttt{Push x} with
\texttt{*sp = x; sp += 4;}. Replace \texttt{Pop} with \texttt{sp -= 4;}.

A simple solution would be to consider the stack a finite size and if this
initially allocated size runs out then the program has run out of memory

This can be optimised by using registers when variables are accessed and
immediately re-used, only storing on the stack in memory when calling a
function (to provide ac consistent calling convention) or when the machine
runs out of registers (the second case would be machine dependent and
should therefore not be represented in the intermediate code). I assume as
part of its semantics,  ``call'' pushes all registers in use, the stack
pointer and the code pointer onto the stack (such that the function can
return).

\fi

The simplest way to convert a stack-oriented intermediate langauge into a
register-oriented intermediate language is to simulate a stack. Programs
would allocate a block of memory onto the heap and then use this as a stack.

This could be implemented by using one of the registers as a stack pointer.
Every time data is pushed, increment the stack pointer and decrement when
it is popped.

However, this doesn't exploit temporal locality and would be inefficient in
a real implementation. A better way would be to use one register as the
stack pointer, but to use the rest of the registers to simulate the top of
the stack. The intermediate code would simply assume an infinite number of
registers and when compiled, would move registers which aren't in use out
onto the stack (and when functions are called). The optimised semantics
would be as follows:

\begin{itemize}

\item \texttt{ldimm ri, n}

Set the contents of register \texttt{ri} to \texttt{n}

\item \texttt{call ri, f}

Call the function f with register \texttt{ri} as argument and return value
-- note arguments with multiple arguments can be implemented by $ri$ being a
pointer to objects on the heap or by taking a closure.

\item \texttt{ld ri, x}

Dereference \texttt{x} and load its contents into register \texttt{ri}

\item \texttt{add ri, rj, rk}

Add the contents of \texttt{rj} and \texttt{rk} and store them in \texttt{ri}

\item \texttt{set ri, x}

Store the contents of register \texttt{ri} in the variable \texttt{x}

\item \texttt{neg ri, rj}

Negate the value stored in register \texttt{rj} and store its value in the
register \texttt{ri}.

\item \texttt{sub ri, rj, rk}

Subtract the value stored in register \texttt{rk} from the value stored in
register \texttt{rj} and store the value in register \texttt{ri}.

\end{itemize}

\end{enumerate}

\item In Java, expressions are evaluated strictly left-t-right. Consider
compiling the function \texttt{f} in the following Java definition.
\begin{lstlisting}[language=Java]
class A{
	static int a, b;
	void f() { ... <<C>> ... }
	int g(int x) { ... a++; ... }
}
\end{lstlisting}

Indicate what \textit{both} the intermediate code \textit{and} (improved as
above) target code might be for \texttt{<<C>>} for the cases where
\texttt{<<C>>} is:

\begin{enumerate}

\item \texttt{b = g(7) + a;}

In the stack-oriented intermediate language:
\begin{lstlisting}
Push 7
Apply g
Push a
Deref a
Add
Assign b
\end{lstlisting}

In the register-oriented intermediate language:
\begin{lstlisting}
ldimm r0, 7
call r0, g
ld r1, a
add r0, r0, r1
set r0, b
\end{lstlisting}

\item \texttt{b = a + g(7);}

In the stack-oriented intermediate language:
\begin{lstlisting}
Push a
Deref a
Push 7
Apply g
Add
Assign b
\end{lstlisting}

In the register-oriented intermediate language:
\begin{lstlisting}
ld r0, a
ldimm r1, 7
call r1, g
add r0, r0, r1
set r0, b
\end{lstlisting}

\item \texttt{b = (-g(7)) + a;}

In the stack-oriented intermediate language:
\begin{lstlisting}
Push 7
Apply g
Neg
Push a
Deref a
Add
Assign b
\end{lstlisting}

In the register-oriented intermediate language:
\begin{lstlisting}
ldimm r0, 7
call r0, g
neg r0, r0
ld r1, a
add r0, r0, r1
set r0, b
\end{lstlisting}

\item \texttt{b = a - g(7);}

In the stack-oriented intermediate language:
\begin{lstlisting}
Push a
Deref a
Push 7
Apply g
Sub
Assign b
\end{lstlisting}

In the register-oriented intermediate language:
\begin{lstlisting}
ld r0, a
ldimm r1, 7
call r1, g
sub r0, r0, r1
set r0, b
\end{lstlisting}

\end{enumerate}

Comment on any inherent differences in efficiency at both the intermediate
code and target code levels.

\end{enumerate}

\end{examquestion}

\begin{examquestion}{2001}{6}{6}

\begin{enumerate}

\item Describe one possible structure (e.g ELF) of an object file.
Illustrate your answer by considering the form of object file which might
result from the following C program.
\begin{lstlisting}[language=C]
int a = 1, b = -1;
extern int g(int);
extern int c;
int f() { return g(a-b) + c; }
\end{lstlisting}

It is not necessary to consider the exact instruction sequence, just issues
concerning its interaction with the object file format.

ELF consists of segments and sections. Sections are used for linking and
during compilation but are not included in the final executable. Segments
are included in the final executable.

The ELF file contains:
\begin{itemize}

\item \texttt{.text}

This segment contains machine instructions. However, in the object file
these are not directly executable -- we need to link to variables using
\texttt{.data}.

In the above program, this would contain the machine code instructions for
\texttt{f} -- however none of the variables would refer to real addresses
-- they'd be null with the \texttt{.rel.text} containing sufficient
information to relocate.

\item \texttt{.data}

This segment contains the initial values for all (writeable global)
variables or static variables.

In the above program, \texttt{.data} would be 8 bytes large and contain 1,
-1 -- the values of \texttt{a} and \texttt{b}.

\item \texttt{.rodata}

This segment contains the initial values for all read-only global or
read-only static variables.

The above program has no read-only data of any sort so \texttt{.rodata}
would be empty.

\item \texttt{.bss}

This contains uninitialised global/static variables. However, since they're
uninitialised it only contains their sizes in bytes. The compiler will
initialised this much memory to zero.\footnote{Whether or not these
variables in \texttt{.bss} are initialised to zero is apparently
controversial, Windows and Unix-like systems do it; and I was not able to
find a convincing example of \texttt{.bss} \textit{not} being initialised
to zero -- the unconvincing examples were from the 1960s.}

The above program has no uninitialised global/static variables and so
\texttt{.bss} would be empty.

\item \texttt{.rel.text}

This section contains the relocation data required for the \texttt{.text}
segment to be converted into proper machine code. It is a list of pairs $i$,
$j$ where $(i, j) \in \texttt{.rel.text}$ means ``the first free operand
of instruction $i$ points to the variable stored at offset $j$ in the
\texttt{.data} segment''.

In the above program, the \texttt{.rel.text} segment would contain two
entries, relating the occurrences of $a$ and $b$ in $f$ to their offsets in
the data segment.

\textbf{What about c?} There isn't an address in the \texttt{.data} segment
for \texttt{.rel.text} to point to\ldots Same dilemma for \texttt{.sym}.

\item \texttt{.rel.data}

Some entries on the \texttt{.data} segment are initialised to pointers --
since memory is not allocated in the object file this is not known when the
object file is created. The \texttt{.rel.data} segment contains pairs $i,
j$ where $(i, j) \in \texttt{.rel.data}$ means ``the variable at
offset $i$ \texttt{.data} is a pointer to the variable referred to at index
$j$ in \texttt{.sym}''.

No variables in the above program contain pointers, so \texttt{.rel.data}
would be empty.

\item \texttt{.rel.rodata}

This performs the same function as \texttt{.rel.data} except for \texttt{
.rodata}.

No read-only variables in the above program contain pointers, so
\texttt{.rel.rodata} would be empty.

\item \texttt{.sym}

This is a mapping from offset in \texttt{.data} of variables to offsets in
the \texttt{.strtab} -- a mapping from variable names to locations.

This would contain two pairs: $(a, \ell_a), (b, \ell_b)$ where $\ell_i$ is
the offset of variable $i$ in the \texttt{.data} segment.

\item \texttt{.strtab}

This is a list of the names of global variables in the \texttt{.data}
segment -- a \texttt{/0} separated string list.

For the above program this would be ``a/0b/0c/0''.

\item \texttt{.line}

This contains a mapping from source code line to machine code instructions.
This is useful when throwing exceptions and for debugging.

\item \texttt{.debug}

This contains other debugging information.

\end{itemize}

Additionally, ELF files contain an import list (containing all the
variables which the program needs to run) and an export list (containing
all the variables the program allows other programs to import).

\item Describe \textit{how} a linker takes a sequence of such programs and
produces an executable file.

Linkers concatenate the segments to each other and resolve offsets.
This produces a second object file. If this object file is closed, it can
be converted into an executable.

Resolving offsets is simple. If there are two object files $o_1$ and
$o_2$, where the size of \texttt{.data} in $o_1$ is $n$ bytes, we would
resolve offsets by increasing every offset in the \texttt{.data} segment
in $o_2$ by $n$. Similarly for the \texttt{.text},
\texttt{.rodata}, \texttt{.sym} and \texttt{.strtab} segments.

Note that care must be taken to rename local variables and functions to avoid
name conflicts.

Linking the object files together forms a ``closed'' object file (which I
define as an object file which imports no variables). Closed object files
can be compiled into executable files.

\begin{itemize}

\item Firstly, allocate memory for the \texttt{.text} segment. This is
stored in the lower memory addresses. Next, allocate the, \texttt{.rodata}
\texttt{.bss} (initialise the specified number of bytes to zero) then
\texttt{.data} segments on top of the \texttt{.text} segment.

\item Next, we must use the relocation data stored in
\texttt{.rel.text}, \texttt{.rel.data} and \texttt{.rel.rodata} to fill in
values which were not known pre-compile time. This includes pointer
addresses, variable addresses and more -- for example we don't know at
compile time whether 64-byte ints must be 4-byte aligned or 8-byte aligned.

Use the sections as specified earlier in the description of ELF\@.

Iterate through \texttt{.rel.rodata} and \texttt{.rel.data} and for each
pair $(i, j)$, initialise the value of the variable stored at offset $i$
into the respective segments with the variable pointed to at index $j$ in
\texttt{.sym}.

Iterate through \texttt{.rel.text} and for each pair $(i, j)$, replace the
first uninitialised variable in instruction $i$ with a pointer to the
variable stored at index $j$ in \texttt{.sym}.

\end{itemize}

\item Compare and contrast \textit{static} and \textit{dynamic} linking in a
system using your object file format.

Static Linking:

\begin{itemize}

\item Code used from other libraries is statically linked at compile time.

\item This means linking is not done at runtime

\item However, code size can be substantially larger since every file needs
a copy of every library it uses. Although modern static linkers are smart
and only import the functions the code could use.

\item Updates to libraries will not affect existing code -- which can both
be good or bad. For example bug fixes will not affect existing code and can
lead to code with known bugs being left in deployment. Conversely,
backwards-incompatible changes to libraries will not break existing code.

\end{itemize}

Dynamic Linking:

\begin{itemize}

\item Code from libraries is dynamically linked at runtime.

\item This means only one copy of each library is needed on each computer --
code is shared.

\item This adds overhead at runtime.

\item This can use less RAM -- if another program is also using the library
then they can share and don't have to load in separate copies.

\end{itemize}

Many modern programs use a mixture of static and dynamic linking -- it's
more efficient to statically load small libraries and avoid page faults; while
it's significantly more efficient to dynamically load larger libraries
(ie pytorch is $\sim 512GB$ -- statically linking this into every program which
used it would rapidly use up disk space).

\end{enumerate}

\end{examquestion}

\end{document}
