\newcommand{\svcourse}{CST Part IA: Software Engineering and Security}
\newcommand{\svnumber}{1}
\newcommand{\svvenue}{Microsoft Teams}
\newcommand{\svdate}{2022-05-11}
\newcommand{\svtime}{15:00}
\newcommand{\svuploadkey}{CBd13xmL7PC1zqhNIoLdTiYUBnxZhzRAtJxv/ytRdM1r7qIfwMsxeVwM/pPcIo8l}

\newcommand{\svrname}{Dr Sam Ainsworth}
\newcommand{\jkfside}{oneside}
\newcommand{\jkfhanded}{yes}

\newcommand{\studentname}{Harry Langford}
\newcommand{\studentemail}{hjel2@cam.ac.uk}


\documentclass[10pt,\jkfside,a4paper]{article}

% DO NOT add \usepackage commands here.  Place any custom commands
% into your SV work files.  Anything in the template directory is
% likely to be overwritten!

\usepackage{fancyhdr}

\usepackage{lastpage}       % ``n of m'' page numbering
\usepackage{lscape}         % Makes landscape easier

\usepackage{verbatim}       % Verbatim blocks
\usepackage{listings}       % Source code listings
\usepackage{graphicx}
\usepackage{float}
\usepackage{epsfig}         % Embed encapsulated postscript
\usepackage{array}          % Array environment
\usepackage{qrcode}         % QR codes
\usepackage{enumitem}       % Required by Tom Johnson's exam question header

\usepackage{hhline}         % Horizontal lines in tables
\usepackage{siunitx}        % Correct spacing of units
\usepackage{amsmath}        % American Mathematical Society
\usepackage{amssymb}        % Maths symbols
\usepackage{amsthm}         % Theorems

\usepackage{ifthen}         % Conditional processing in tex

\usepackage[top=3cm,
            bottom=3cm,
            inner=2cm,
            outer=5cm]{geometry}

% PDF metadata + URL formatting
\usepackage[
            pdfauthor={\studentname},
            pdftitle={\svcourse, SV \svnumber},
            pdfsubject={},
            pdfkeywords={9d2547b00aba40b58fa0378774f72ee6},
            pdfproducer={},
            pdfcreator={},
            hidelinks]{hyperref}

\renewcommand{\headrulewidth}{0.4pt}
\renewcommand{\footrulewidth}{0.4pt}
\fancyheadoffset[LO,LE,RO,RE]{0pt}
\fancyfootoffset[LO,LE,RO,RE]{0pt}
\pagestyle{fancy}
\fancyhead{}
\fancyhead[LO,RE]{{\bfseries \studentname}\\\studentemail}
\fancyhead[RO,LE]{{\bfseries \svcourse, SV~\svnumber}\\\svdate\ \svtime, \svvenue}
\fancyfoot{}
\fancyfoot[LO,RE]{For: \svrname}
\fancyfoot[RO,LE]{\today\hspace{1cm}\thepage\ / \pageref{LastPage}}
\fancyfoot[C]{\qrcode[height=0.8cm]{\svuploadkey}}
\setlength{\headheight}{22.55pt}


\ifthenelse{\equal{\jkfside}{oneside}}{

 \ifthenelse{\equal{\jkfhanded}{left}}{
  % 1. Left-handed marker, one-sided printing or e-marking, use oneside and...
  \evensidemargin=\oddsidemargin
  \oddsidemargin=73pt
  \setlength{\marginparwidth}{111pt}
  \setlength{\marginparsep}{-\marginparsep}
  \addtolength{\marginparsep}{-\textwidth}
  \addtolength{\marginparsep}{-\marginparwidth}
 }{
  % 2. Right-handed marker, one-sided printing or e-marking, use oneside.
  \setlength{\marginparwidth}{111pt}
 }

}{
 % 3. Alternating margins, two-sided printing, use twoside.
}


\setlength{\parindent}{0em}
\addtolength{\parskip}{1ex}

% Exam question headings, labels and sensible layout (courtesy of Tom Johnson)
\setlist{parsep=\parskip, listparindent=\parindent}
\newcommand{\examhead}[3]{\section{#1 Paper #2 Question #3}}
\newenvironment{examquestion}[3]{
\examhead{#1}{#2}{#3}\setlist[enumerate, 1]{label=(\alph*)}\setlist[enumerate, 2]{label=(\roman*)}
\marginpar{\href{https://www.cl.cam.ac.uk/teaching/exams/pastpapers/y#1p#2q#3.pdf}{\qrcode{https://www.cl.cam.ac.uk/teaching/exams/pastpapers/y#1p#2q#3.pdf}}}
\marginpar{\footnotesize \href{https://www.cl.cam.ac.uk/teaching/exams/pastpapers/y#1p#2q#3.pdf}{https://www.cl.cam.ac.uk/\\teaching/exams/pastpapers/\\y#1p#2q#3.pdf}}
}{}


\usepackage{physics}
\usepackage{amsfonts}
\usepackage{color}

\begin{document}

\section*{Further Graphics Exercise Set I}

\begin{enumerate}

\setcounter{enumi}{1}

\item What geometric shape does the following curve describe?

\[
\left( x(t), y(t) \right) = \left( e^t\cos(t), e^t \sin(t) \right)
\]

This planar curve describes an anticlockwise spiral.

\item

\begin{enumerate}

\item How do you compute the surface normal $\mathbf{n}$ for a parametric
surface?

For a parametric surface with equation $f(u, v) = (x, y, z)$, the
surface normal $\mathbf{n}$ is given by:
\[
\pdv{f}{u}\times\pdv{f}{v}
\]

\item Below you find the parameterisation of a \textit{torus}. Compute the
surface normal $\mathbf{n}$ for $u = \frac{1}{3}\pi$, $v = \frac{3}{4}\pi$.
\begin{gather*}
s(u, v) = \left(\left(3 + \sqrt{2} \cos(v)\right)\cos(u), \left(3 + \sqrt{2} \cos(v)\right)\sin(u), \sqrt{2} \sin(v) \right) \\\\
\pdv{s}{u} =
\left(-\left(3 + \sqrt{2} \cos(v)\right)\sin(u), \left(3 + \sqrt{2} \cos(v)\right)\cos(u), 0 \right) \\\\
\pdv{s}{v} =
\left(-\sqrt{2} \sin(v)\cos(u), -\sqrt{2} \sin(v)\sin(u), \sqrt{2} \cos(v)\right)
\end{gather*}

{\color{blue} Substitute the values in earlier rather than cross-producting
the $\cos$, $\sin$ terms}

\[
\begin{split}
\mathbf{n}
&= \pdv{s}{u} \times \pdv{s}{u} \\
&= \begin{pmatrix}
-\left(3 + \sqrt{2}\cos(v)\right)\sin(u)\\
\left(3 + \sqrt{2}\cos(v)\right)\cos(u)\\
0\\
\end{pmatrix}
\times
\begin{pmatrix}
-\sqrt{2}\sin(v)\cos(u)\\
-\sqrt{2}\sin(v)\sin(u)\\
\sqrt{2}\cos(v)\\
\end{pmatrix}\\
&=
\begin{pmatrix}
\sqrt{2}\left( 3 + \sqrt{2}\cos(v) \right)\cos(u)\cos(v)\\
\sqrt{2}\left( 3 + \sqrt{2}\cos(v) \right)\sin(u)\cos(v) \\
\sqrt{2}\left( 3 + \sqrt{2}\cos(v) \right)\left(\sin^2(u) + \cos^2(u)\right)
\sin(v) \\
\end{pmatrix}\\
&=
\begin{pmatrix}
\sqrt{2}\left( 3 + \sqrt{2}\cos(v) \right)\cos(u)\cos(v)\\
\sqrt{2}\left( 3 + \sqrt{2}\cos(v) \right)\sin(u)\cos(v) \\
\sqrt{2}\left( 3 + \sqrt{2}\cos(v) \right)\sin(v) \\
\end{pmatrix}\\
\end{split}
\]

Substituting in the values for $(u, v)$ gives:

\[
\begin{split}
\mathbf{n}\left( \frac{\pi}{3}, \frac{3\pi}{4} \right)
&=
\begin{pmatrix}
\sqrt{2}\left( 3 + \sqrt{2}\cos\left( \frac{3\pi}{4} \right) \right)
\cos\left( \frac{\pi}{3} \right)
\cos\left( \frac{3\pi}{4} \right)\\
\sqrt{2}\left( 3 + \sqrt{2}\cos\left( \frac{3\pi}{4} \right) \right)\sin\left(
\frac{\pi}{3} \right)\cos\left( \frac{3\pi}{4} \right) \\
\sqrt{2}\left( 3 + \sqrt{2}\cos\left( \frac{3\pi}{4} \right) \right)\sin\left( \frac{3\pi}{4} \right) \\
\end{pmatrix}\\
&=
\begin{pmatrix}
-\sqrt{2}\left( 3 - 1 \right)
\frac{1}{2}\frac{\sqrt {2}}{2}\\
\sqrt{2}\left( 3 - 1 \right)\frac{\sqrt{3}}{2}\frac{\sqrt {2}}{2} \\
\sqrt{2}\left( 3 - 1 \right)\frac{1}{2} \\
\end{pmatrix}\\
&=
\begin{pmatrix}
-1\\
1 \\
\sqrt{2} \\
\end{pmatrix}\\
\end{split}
\]


\end{enumerate}

\setcounter{enumi}{4}

\item If we define a curve in $\mathbb{R}^2$ with the implicit function $f
(x, y) = x^2 y^2$, what problem could there arise when computing the
surface normal?

The surface normal for a function which is implicitly defined is:
\[
\mathbf{n} = \begin{pmatrix}
\pdv{f}{x} \\
\pdv{f}{x} \\
\end{pmatrix}
\]

Which in this case is:
\[
\begin{pmatrix}
2x y^2 \\
2x^2 y \\
\end{pmatrix}
=
2xy\begin{pmatrix}
y\\
x\\
\end{pmatrix}
\]

By definition, at all points on the surface of the curve: $x^2 y^2 = 0$.
Therefore $x$ or $y$ is zero. So for all points on the curve, $xy = 0$. So,
the normal for any point on the curve is:
\[
0\cdot
\begin{pmatrix}
y\\
x\\
\end{pmatrix}
= \begin{pmatrix}
0\\
0\\
\end{pmatrix}
\]

$ \begin{pmatrix}
0\\0\\
\end{pmatrix} $ is not a valid normal!

\item

\begin{enumerate}

\item What data structure is used to represent triangle meshes?

{\color{blue} Attributes such as position, u-v maps, normals and colour are
stored with each of the vertices}

Triangle meshes can be represented as a graph embedded in 3D space where $V$
is the set of vertices; $E \subseteq V \times V$ is the set of edges; $F
\subseteq V \times V \times V$ is the set of faces; and $P \subseteq V \times
\mathbb{R}^n$ is a mapping from each vertex to a point in n-dimensional
space (where $n$ is 3 in most applications).

\item What parts of this data structure pertain to the topology and which to
the geometry of a surface?

The mapping from the vertices to the points in 3D pertains to the topology
-- it varies when the shape is transformed. The edges and faces
pertain to the geometry; they are invariant of transformations.

{\color{blue} This is more like a parametric representation, we order the set
and then consider this as a function from indices to points}

\end{enumerate}

\item Compared to the triangle mesh, a point set lacks a vital piece of
information in that it no longer identifies the neighbouring vertices of any
given vertex. Why is such information necessary or useful in the first place
and how do point set surfaces work without it?

The adjacency matrix is essential to know the geometry of the polygon we are
using to  approximate the surface.

Point-set surfaces instead approximate the underlying surface. In
practice most surfaces are smooth and therefore the approximation-based
methods get very similar results to polygon-based surface representations.

A simple approximation method selects a query point and a fixed distance --
all points closer to the query point than this fixed distance are in the
``neighbourhood''. We then assume that the surface is the plane which best
fits all the points in the neighbourhood.

{\color{blue} In reality we train a NN to fit the surface from the point-set
representation and then query the NN}

\setcounter{enumi}{13}

\item

\begin{enumerate}

\item What is rigging? Briefly describe what kind of structure and data are
added to the geometry/surface model.

{\color{blue} The bones are a tree -- there is usually a root which we
define to be ``a central point which we will use to move the whole object''.
 Sometimes we just have a root bone which is not connected to anything.}

Rigging is the process of attaching points on the surface of an object to the
``bones'' of the object. This is an essential stage in the process of
animation.

We create bones, often storing them in a tree such that a transformation on
one will move the others below it in the tree -- making animation easier.

We add a (sparse) matrix to the surface model. This matrix
contains the weights which define the impact of a transformation on a
given bone on a point on the surface.

\item Assume we have a cylinder with an axis aligned to the $x$-axis. The
cylinder extends from $-1$ to $1$ on the $x$-axis and has a base radius of 1.
We embed two bones inside the cylinder along the $x$-axis: one extends from
-1 to 0, the other from 0 to 1. We define the influence of each bone on each
point on the cylinder as the inverse distance of the point to the bone.
Assume we transform the second bone with the new transformation $T$.
Determine the concrete weights and give a formula of how the new position
$x'$ would be computed from the weights $w_1(x)$, $w_2(x)$, $x$ and $T$ for
a point on the surface (a) in the middle and (b) on either end.

{\color{blue} They meant a point on the surface of a hollow cylinder. All
weights must sum to 1.}

Let bone 1 be the bone from $-1$ to $0$ and let bone 2 be the bone from $0$
to $1$.

\begin{enumerate}

\item In the middle

\begin{align*}
w_1(x) &= 0.5 & w_2(x) &= 0.5
\end{align*}

\[
x' = 0.5 \cdot x + 0.5\cdot\mathbf{T}x
\]

\item On either end.

For $x = (-1, 0)$:

\begin{align}
w_1(x) &= 1 & w_2(x) &= 0
\end{align}

\[
x' = 1 \cdot x + 0\cdot\mathbf{T}x = x
\]

For $x = (1, 0)$:

\begin{align*}
w_1(x) &= 0 & w_2(x) &= 1
\end{align*}

\[
x' = 0 \cdot x + 1 \cdot \mathbf{T}x = \mathbf{T}x
\]

\end{enumerate}

\end{enumerate}

\item Thinking about fundamental properties

\begin{enumerate}

\item What is the advantage of representing rigid transformations with dual
quaternions for blending.

Dual quaternions can be used to perform shortest-path transformations. This
allows us to blend transformations while ensuring the resulting
transformation is still valid; without distortions.

\item Briefly explain one fundamental disadvantage of using quaternion based
shortest-path blending for rotations as compared to linear blend skinning.

Linear-blend skinning on linearly interpolates points. Therefore,
transforming nearby points with the same transformations is guaranteed to
leave the transformed points close to each other. Therefore a continuous
surface will remain continuous after transformation.

However, quaternions find the shortest path. This path will change suddenly.
Therefore there exists weights such that a tiny change in weight will make
the same quaternion interpolate onto a completely different path.

{\color{blue} Imagine we interpolate between two rotations, $\theta_1$ and
$\theta_2$. The shortest-path interpolation is discontinuous at $\pi$. The
shortest-path at $\pi$ radians is wildly divergent.}

{\color{blue} Additionally, quaternions preserve volume and therefore cannot
represent volume changes}

\end{enumerate}

\item Derivations and deeper understanding

Below we list all required properties of dual quaternions for the rest of
the exercise.

A dual quaterion $\hat{\mathbf{q}}$ can be written in the form
$\hat{\mathbf{q}} = \mathbf{q}_0 + \epsilon\mathbf{q}_\epsilon $, where
$\mathbf{q}_0$ and $\mathbf{q}_\epsilon$ are quaternions and $\epsilon$ is
the dual unit with the property $\epsilon^2 = 0$. The norm of
$\hat{\mathbf{q}}$ is then given by:

\[
\|\hat{\mathbf{q}}\| = \|\mathbf{q}_0\| + \epsilon\frac{\langle
\mathbf{q}_0, \mathbf{q}_\epsilon \rangle}{\|\mathbf{q}_0\|}
\]

Dual quaternions representing rigid transformations can be written in the
following form:

\[
\hat{\mathbf{q}} = \cos\left( \frac{\hat{\theta}}{2} \right) +
\hat{\mathbf{s}}\sin\left( \frac{\hat{\theta}}{2} \right)
\]

where $\hat{\theta} = \theta_0 + \epsilon\theta_\epsilon$ and
$\hat{\mathbf{s}} = \mathbf{s}_0 + \epsilon\mathbf{s}_\epsilon$. Here,
$\mathbf{s}_0$ is the axis of rotation, $\theta_0$ is the rotation angle and
$\theta_\epsilon$ is the amount of translation along $\mathbf{s}_0$. since
this is a unit dual quaternion it can be shown that $ \langle \mathbf{s}_0,
\mathbf{s}_\epsilon \rangle = 0 $ and $ \langle \mathbf{s}_0, \mathbf{s}_0
\rangle = 1$.

The power of a dual quaternion is defined by

\begin{align*}
e^{\hat{\mathbf{q}}} &= \cos\left( \|\hat{\mathbf{q}}\| \right) +
\frac{\hat{\mathbf{q}}}{\|\hat{\mathbf{q}}\|}\sin\left( \|\hat{\mathbf{q}}\| \right)
&
\ln\left( \cos\left( \frac{\hat{\theta}}{2} \right) +
\hat{\mathbf{s}}\sin\left( \frac{\hat{\theta}}{2} \right) \right) &=
\hat{\mathbf{s}}\frac{\hat{\theta}}{2}
\end{align*}

\begin{enumerate}

\setcounter{enumi}{2}

\item Utilizing the properties above, for a dual quaternion 
$\hat{\mathbf{q}} = \cos\left( \frac{\hat{\theta}}{2} \right) + 
\hat{\mathbf{s}} \sin \left( \frac{\hat{\theta}}{2} \right) $, prove that:

\[
\hat{\mathbf{q}}^t = \cos\left( \frac{t\hat{\theta}}{2} \right) + 
\hat{\mathbf{s}}\sin\left( \frac{t\hat{\theta}}{2} \right)
\]

\[
\begin{split}
\hat{\mathbf{s}}\frac{\hat{\theta}}{2} &=
\ln \left( \cos\left( \frac{\hat{\theta}}{2} \right) +
\hat{\mathbf{s}}\sin\left( \frac{\hat{\theta}}{2} \right) \right) \\
e^{\hat{\mathbf{s}}\frac{\hat{\theta}}{2}} &=
\cos\left( \frac{\hat{\theta}}{2} \right) +
\hat{\mathbf{s}}\sin\left( \frac{\hat{\theta}}{2} \right) \\
e^{\hat{\mathbf{s}}\frac{t\hat{\theta}}{2}} &=
\cos\left( \frac{t\hat{\theta}}{2} \right) +
\hat{\mathbf{s}}\sin\left( \frac{t\hat{\theta}}{2} \right) \\
e^{t\left(\hat{\mathbf{s}}\frac{\hat{\theta}}{2}\right)} &=
\cos\left( \frac{t\hat{\theta}}{2} \right) +
\hat{\mathbf{s}}\sin\left( \frac{t\hat{\theta}}{2} \right) \\
e^{t\ln \hat{\mathbf{q}}} &=
\cos\left( \frac{t\hat{\theta}}{2} \right) +
\hat{\mathbf{s}}\sin\left( \frac{t\hat{\theta}}{2} \right) \\
\hat{\mathbf{q}}^t &=
\cos\left( \frac{t\hat{\theta}}{2} \right) +
\hat{\mathbf{s}}\sin\left( \frac{t\hat{\theta}}{2} \right) \\
\end{split}
\]

\item How can a single dual quaternion with represent rotations and
transformations in the $xy$-plane?

Chalses' Theorem states that any rigid displacement is equivalent to some
screw displacement.

In this case, the dual quaternion would have a screw axis perpendicular to
the $xy$-plane and passing through the midpoint between the center of the
original shape and the center of the target with rotation $\pi$ and
displacement 0.

{\color{blue} We have a center of rotation and rotate around the center. We
just choose the center carefully and then rotate. We choose a vector around
which we rotate -- so a rotation of $\pi$ around a vector
$\underline{\mathbf{0}} + \begin{pmatrix} 1 \\ 0 \\ 0 \\\end{pmatrix}$ is
the same as rotating the object with coordinates $\begin{pmatrix}
\hat{x} \\ \hat{y} \\ \hat{z} \\\end{pmatrix}$ by $\pi$ and
translating it by $\begin{pmatrix} -2\hat{x} \\ 0 \\ 0 \\ \end{pmatrix}$.}

\end{enumerate}

\end{enumerate}

\end{document}