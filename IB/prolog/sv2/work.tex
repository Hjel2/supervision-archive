\newcommand{\svcourse}{CST Part IA: Introduction to Probability}
\newcommand{\svnumber}{1}
\newcommand{\svvenue}{Churchill, Room TBD}
\newcommand{\svdate}{2022-05-14}
\newcommand{\svtime}{11:00}
\newcommand{\svuploadkey}{PO5ogKIM8KQA22FZS8IAf8gxA8XKi19jxIBVHIfFZ+3GCBXuNUXS9lVN6bNYjxM/}

\newcommand{\svrname}{Mr Matthew Ireland}
\newcommand{\jkfside}{twoside}
\newcommand{\jkfhanded}{right}

\newcommand{\studentname}{Harry Langford}
\newcommand{\studentemail}{hjel2@cam.ac.uk}


\documentclass[10pt,\jkfside,a4paper]{article}

% DO NOT add \usepackage commands here.  Place any custom commands
% into your SV work files.  Anything in the template directory is
% likely to be overwritten!

\usepackage{fancyhdr}

\usepackage{lastpage}       % ``n of m'' page numbering
\usepackage{lscape}         % Makes landscape easier

\usepackage{verbatim}       % Verbatim blocks
\usepackage{epsfig}         % Embed encapsulated postscript
\usepackage{array}          % Array environment
\usepackage[nolinks]{qrcode}         % QR codes
\usepackage{enumitem}       % Required by Tom Johnson's exam question header

\usepackage{hhline}         % Horizontal lines in tables
\usepackage{siunitx}        % Correct spacing of units
\usepackage{amsmath}        % American Mathematical Society
\usepackage{amssymb}        % Maths symbols
\usepackage{amsthm}         % Theorems

\usepackage{ifthen}         % Conditional processing in tex

\usepackage[top=3cm,
            bottom=3cm,
            inner=2cm,
            outer=5cm]{geometry}

% PDF metadata + URL formatting
\usepackage[
            pdfauthor={\studentname},
            pdftitle={\svcourse, SV \svnumber},
            pdfsubject={},
            pdfkeywords={9d2547b00aba40b58fa0378774f72ee6},
            pdfproducer={},
            pdfcreator={},
            hidelinks]{hyperref}

\renewcommand{\headrulewidth}{0.4pt}
\renewcommand{\footrulewidth}{0.4pt}
\fancyheadoffset[LO,LE,RO,RE]{0pt}
\fancyfootoffset[LO,LE,RO,RE]{0pt}
\pagestyle{fancy}
\fancyhead{}
\fancyhead[LO,RE]{{\bfseries \studentname}\\\studentemail}
\fancyhead[RO,LE]{{\bfseries \svcourse, SV~\svnumber}\\\svdate\ \svtime, \svvenue}
\fancyfoot{}
\fancyfoot[LO,RE]{For: \svrname}
\fancyfoot[RO,LE]{\today\hspace{1cm}\thepage\ / \pageref{LastPage}}
\fancyfoot[C]{\qrcode[height=0.8cm]{\svuploadkey}}
\setlength{\headheight}{22.55pt}

\ifthenelse{\equal{\jkfside}{oneside}}{

 \ifthenelse{\equal{\jkfhanded}{left}}{
  % 1. Left-handed marker, one-sided printing or e-marking, use oneside and...
  \evensidemargin=\oddsidemargin
  \oddsidemargin=73pt
  \setlength{\marginparwidth}{111pt}
  \setlength{\marginparsep}{-\marginparsep}
  \addtolength{\marginparsep}{-\textwidth}
  \addtolength{\marginparsep}{-\marginparwidth}
 }{
  % 2. Right-handed marker, one-sided printing or e-marking, use oneside.
  \setlength{\marginparwidth}{111pt}
 }

}{
 % 3. Alternating margins, two-sided printing, use twoside.
}

\setlength{\parindent}{0em}
\addtolength{\parskip}{1ex}

% Exam question headings, labels and sensible layout (courtesy of Tom Johnson)
\setlist{parsep=\parskip, listparindent=\parindent}
\newcommand{\examhead}[3]{\section{#1 Paper #2 Question #3}}
\newenvironment{examquestion}[3]{
    \examhead{#1}{#2}{#3}\setlist[enumerate, 1]{label=(\alph*)}\setlist[enumerate, 2]{label=(\roman*)}
    \marginpar{\qrcode{https://www.cl.cam.ac.uk/teaching/exams/pastpapers/y#1p#2q#3.pdf}}
    \marginpar{\footnotesize \url{https://www.cl.cam.ac.uk/teaching/exams/pastpapers/y#1p#2q#3.pdf}}
}{}



\input{../prologstyle}

\begin{document}

\begin{examquestion}{2001}{5}{7}

Consider the following Prolog program, which is intended to define the
third argument to be the maximum value of the first two numeric arguments:
\begin{lstlisting}[style=pstyle]
max(X, Y, X) :- X >= Y, !.
max(X, Y, Y).
\end{lstlisting}

\begin{enumerate}

\item Provide an appropriate query to show that the above program can give an
incorrect result.

The query below returns true -- when it's clearly false.

\begin{lstlisting}[style=pstyle]
max(2, 1, 1).
\end{lstlisting}

\item Explain the cause of the error.

The second clause has no guard. So if the first condition is not satisfied
then the second will always be satisfied -- even if it should not be. So any
false condition of the form \texttt{max(X, Y, Y)} will return true.

\item Suggest a correction.

\begin{lstlisting}[style=pstyle]
max2(X, Y, X) :- X >= Y.
max2(X, Y, Y) :- X < Y.
\end{lstlisting}

\item Write a Prolog program to find the maximum of a list of numbers.

\begin{lstlisting}[style=pstyle]
maxlist([H], H).
maxlist([H|T], X) :- maxlist(T, Y), max2(H, Y, X).
\end{lstlisting}

\end{enumerate}

\end{examquestion}

\begin{examquestion}{1996}{6}{7}

\begin{enumerate}

\item Describe how lists that are represented by difference lists may be
concatenated (or ``appended'') in constant time.

Difference lists are lists with exposed pointers at the end. Empty
difference lists are represented by
\begin{lstlisting}[style=pstyle]
A-A.
\end{lstlisting}

Appending to difference lists can be done by the single fact
\begin{lstlisting}[style=pstyle]
append(A-B, B-C, A-C).
\end{lstlisting}

\item Define a procedure \texttt{rotate(X, Y)} where both \texttt{X} and
\texttt{Y} are represented by difference lists, and \texttt{Y} is formed by
rotating \texttt{X} to the left by one element.

\begin{lstlisting}[style=pstyle]
% rotate(+[H|T]-[H|X], ?T-X)
% succeeds if the second argument is the first but rotated
rotate([H|T]-[H|X], T-X).
\end{lstlisting}

\end{enumerate}

\end{examquestion}

\begin{examquestion}{1997}{6}{7}

A binary tree is constructed from binary compound terms \texttt{n(a, b)}
called \textit{nodes}, where components \texttt{a} and \texttt{b} are
either nodes or integers. Suppose integer components are restricted to the
values 0 and 1.

Write a Prolog program to return a list of all the 0's and a list of all
the 1's in a given tree. For example, the goal \texttt{enum(n(n(0,1),1),X,Y)}
should instantiate \texttt{X} to \texttt{[0]} and \texttt{Y} to \texttt{[1,
1]}. The program is required to use difference lists.

\begin{lstlisting}[style=pstyle]
enum(T, X, Y) :- enum2(T, X-[], Y-[]).
enum2(0, [0|X]-X, Y-Y).
enum2(1, X-X, [1|Y]-Y).
enum2(n(A, B), X1-X3, Y1-Y3) :- enum2(A, X1-X2, Y1-Y2), enum2(B, X2-X3, Y2-Y3).
\end{lstlisting}

\end{examquestion}

\end{document}
