\newcommand{\svcourse}{CST Part IA: Introduction to Probability}
\newcommand{\svnumber}{1}
\newcommand{\svvenue}{Churchill, Room TBD}
\newcommand{\svdate}{2022-05-14}
\newcommand{\svtime}{11:00}
\newcommand{\svuploadkey}{PO5ogKIM8KQA22FZS8IAf8gxA8XKi19jxIBVHIfFZ+3GCBXuNUXS9lVN6bNYjxM/}

\newcommand{\svrname}{Mr Matthew Ireland}
\newcommand{\jkfside}{twoside}
\newcommand{\jkfhanded}{right}

\newcommand{\studentname}{Harry Langford}
\newcommand{\studentemail}{hjel2@cam.ac.uk}


\documentclass[10pt,\jkfside,a4paper]{article}

% DO NOT add \usepackage commands here.  Place any custom commands
% into your SV work files.  Anything in the template directory is
% likely to be overwritten!

\usepackage{fancyhdr}

\usepackage{lastpage}       % ``n of m'' page numbering
\usepackage{lscape}         % Makes landscape easier

\usepackage{verbatim}       % Verbatim blocks
\usepackage{epsfig}         % Embed encapsulated postscript
\usepackage{array}          % Array environment
\usepackage[nolinks]{qrcode}         % QR codes
\usepackage{enumitem}       % Required by Tom Johnson's exam question header

\usepackage{hhline}         % Horizontal lines in tables
\usepackage{siunitx}        % Correct spacing of units
\usepackage{amsmath}        % American Mathematical Society
\usepackage{amssymb}        % Maths symbols
\usepackage{amsthm}         % Theorems

\usepackage{ifthen}         % Conditional processing in tex

\usepackage[top=3cm,
            bottom=3cm,
            inner=2cm,
            outer=5cm]{geometry}

% PDF metadata + URL formatting
\usepackage[
            pdfauthor={\studentname},
            pdftitle={\svcourse, SV \svnumber},
            pdfsubject={},
            pdfkeywords={9d2547b00aba40b58fa0378774f72ee6},
            pdfproducer={},
            pdfcreator={},
            hidelinks]{hyperref}

\renewcommand{\headrulewidth}{0.4pt}
\renewcommand{\footrulewidth}{0.4pt}
\fancyheadoffset[LO,LE,RO,RE]{0pt}
\fancyfootoffset[LO,LE,RO,RE]{0pt}
\pagestyle{fancy}
\fancyhead{}
\fancyhead[LO,RE]{{\bfseries \studentname}\\\studentemail}
\fancyhead[RO,LE]{{\bfseries \svcourse, SV~\svnumber}\\\svdate\ \svtime, \svvenue}
\fancyfoot{}
\fancyfoot[LO,RE]{For: \svrname}
\fancyfoot[RO,LE]{\today\hspace{1cm}\thepage\ / \pageref{LastPage}}
\fancyfoot[C]{\qrcode[height=0.8cm]{\svuploadkey}}
\setlength{\headheight}{22.55pt}

\ifthenelse{\equal{\jkfside}{oneside}}{

 \ifthenelse{\equal{\jkfhanded}{left}}{
  % 1. Left-handed marker, one-sided printing or e-marking, use oneside and...
  \evensidemargin=\oddsidemargin
  \oddsidemargin=73pt
  \setlength{\marginparwidth}{111pt}
  \setlength{\marginparsep}{-\marginparsep}
  \addtolength{\marginparsep}{-\textwidth}
  \addtolength{\marginparsep}{-\marginparwidth}
 }{
  % 2. Right-handed marker, one-sided printing or e-marking, use oneside.
  \setlength{\marginparwidth}{111pt}
 }

}{
 % 3. Alternating margins, two-sided printing, use twoside.
}

\setlength{\parindent}{0em}
\addtolength{\parskip}{1ex}

% Exam question headings, labels and sensible layout (courtesy of Tom Johnson)
\setlist{parsep=\parskip, listparindent=\parindent}
\newcommand{\examhead}[3]{\section{#1 Paper #2 Question #3}}
\newenvironment{examquestion}[3]{
    \examhead{#1}{#2}{#3}\setlist[enumerate, 1]{label=(\alph*)}\setlist[enumerate, 2]{label=(\roman*)}
    \marginpar{\qrcode{https://www.cl.cam.ac.uk/teaching/exams/pastpapers/y#1p#2q#3.pdf}}
    \marginpar{\footnotesize \url{https://www.cl.cam.ac.uk/teaching/exams/pastpapers/y#1p#2q#3.pdf}}
}{}



\begin{document}

\begin{examquestion}{2008}{7}{5}

\begin{enumerate}[label=(\alph*)]

\item Why are multi-level caches often used in preference to a single larger
cache. How might the parameters of an L1 and L2 data cache typically differ?

The access time of a larger cache grows roughly proportional to $\sqrt{n}$.
Given that smaller caches can already have incredibly high hit rates
(order of 99\% for set-associative L1 caches), it does not make sense to double
the access time to increase this from 99\% to 99.9\%. Even if we were to do
this, it would make sense to add another higher-level cache to reduce the
miss-penalty to RAM. This would be similar to increasing the size of the L1
cache and introducing another L2 -- which reverts to a multi-level cache.

\begin{itemize}

\item L1 cache is smaller than L2 cache

\item L1 cache has a lower latency than L2 cache

\item L1 cache has a lower miss penalty than L2

\item L1 cache is physically closer to the core than L2 cache

\item L1 cache is private while L2 cache is often shared

\item There are often separate L1 data and instruction caches while there is
almost always a combined data-instruction L2 cache.

\item L2 cache has a lower compulsory miss rate than L1 cache

\item L2 cache is usually set-associative with a higher number of sets than
L1 cache.

\end{itemize}

\iffalse

\begin{itemize}

\item The efficiency of a cache grows proportionally to $\sqrt{n}$

\item It's not possible to construct a single large cache which would meet
the speed requirements for modern computers

\end{itemize}

\fi

\setcounter{enumi}{1}

\item A processor's multi-level cache hierarchy consists of L1 and L2 caches
with the following characteristics: L1 miss rate is 2\%, L1 hit time is 2
cycles, local L2 miss rate is 20\%, L2 hit time is 10 cycles, L2 miss
penalty is 200 cycles. It is suggested that reducing the size of the L1
cache will improve overall performance by reducing the L1 cache's hit time
to a single cycle. The reduction in L1 cache size will increase the L1's
miss rate to 3\%. Will average memory access time actually be improved?
Clearly state any formulae you use and show your calculations.

With $p_{L1}$, $p_{L2}$ as the L1 and L2 hit rates; $t_{L1}$, $t_{L2}$ as
the L1 and L2 hit times and $t_{RAM}$ as the L3 miss penalty.

The expected memory access time $\mathbb{E}(t)$ is given by:
\[
\mathbb{E}(t) = t_{L1} + \left(1 - p_{L1}\right)\left(
t_{L2}  + \left( 1 - p_{L2} \right)t_{RAM}\right)
\]

Under the original memory hierarchy, this is equal to:
\[
\begin{split}
\mathbb{E}(t) &= t_{L1} + \left(1 - p_{L1}\right)\left(
t_{L2}  + \left( 1 - p_{L2} \right)t_{RAM}\right) \\
&= 2 + 0.02 \cdot \left( 10 + 0.2 \cdot 200 \right) \\
&= 2 + 0.02 \cdot (10 + 40) \\
&= 2 + 0.02 \cdot 50 \\
&= 2 + 1 \\
&= 3 \\
\end{split}
\]

Under the proposed memory hierarchy, this is equal to:
\[
\begin{split}
\mathbb{E}(t) &= t_{L1} + \left(1 - p_{L1}\right)\left(
t_{L2}  + \left( 1 - p_{L2} \right)t_{RAM}\right) \\
&= 1 + 0.03 \cdot \left( 10 + 0.2 \cdot 200 \right) \\
&= 1 + 0.03 \cdot 50 \\
&= 1 + 1.5 \\
&= 2.5 \\
\end{split}
\]

Therefore the expected memory access time on the proposed memory hierarchy is
improved.

\end{enumerate}

\end{examquestion}

\begin{examquestion}{2014}{7}{5}

\begin{enumerate}[label=(\alph*)]

\item A 4KB, blocking, private L1 cache with 16B lines sees the follwoing
sequence of accesses from its core:

\begin{tabular}{rl}
0x00001000 & Load \\
0x00001010 & Store \\
0x00002000 & Load \\
0x00001010 & Load \\
0x00003000 & Load \\
0x00001010 & Store \\
0x00001010 & Store \\
0x00002000 & Load \\
0x00001000 & Load \\
0x00002000 & Load \\
\end{tabular}

Assuming a write-allocate cache that is empty at first and implements the
least-recently-used (LRU) replacement algorithm, what is the hit rate if the
cache is:

\begin{enumerate}[label=(\roman*)]

\item direct-mapped?

The cache is sized such that the rightmost hex character is the position in
the cacheline, and 2--3 rightmost hex characters are which cacheline
the address maps to. Therefore the addresses 0x00001000, 0x00002000 and
0x00003000 map to the same cacheline, while the address 0x00001010 is cached
properly.

Using a direct-mapped cache, there are 3 hits from 0x00001010 and 1 hit on
0x00002000 at the end. Since there were 10 total memory addresses, the hit
rate is therefore 0.4.

\item fully-associative?

Under a fully associative cache, we can choose which cacheline to evict.
Since initially the cache is empty and is larger than 4 cachelines, the cache
will not fill up and so each time we load in a new cacheline we can evict an
invalid entry. So in this situation, a fully associative cache will only have
compulsory misses.

We use 4 different addresses and therefore have 4 compulsory misses out of
10 accesses. So the hit rate is $\frac{10 - 4}{10} = 0.6$.

\item 2-way set-associative?

In a 2-way set associative cache, 0x00001010 maps to a unique cacheline, but
0x00001000, 0x00002000, 0x00003000 map to the same cacheline. Since the
cache is 2-way set associative, we can only store 2 of them. Following the LRU
algorithm on the cacheline 00, we will miss 0x00001000 and cache it,
then miss 0x00002000 and cache it, then miss 0x00003000 so evict 0x00001000
to cache 0x00003000. We will then hit 0x00002000, evict 0x00003000 to cache
0x00001000 and then hit 0x00002000.

So we get 3 cache hits on 0x00001010 and hit 0x00002000 twice. Therefore we
will have a cache hit rate of $\frac{5}{10} = 0.5$.

\end{enumerate}

\setcounter{enumi}{3}

\item Assume that this core and cache are part of a chip-multiprocessor,
with the cache connected to a shared L2 via a bus that maintains coherence
through a snooping MESI protocol. What sequence of steps would be taken if
another core wanted to load from 0x00001010 after the given sequence had
finished?

After the given sequence had finished, 0x00001010 is held in cache and has
been written to and is therefore in the $M$ state. This means the core which
performed the instruction sequence has the only copy of 0x00001010. So in
order to get it, the new core will request it on the data bus. The original
core will be snooping on the data bus and look up 0x00001010 in its own
cache. This will return a result. Therefore the new core will push it out to
the shared L2 cache and mark the entry as shared in both own cache and mark
the entry in the L2 cache as being shared (assuming the other core issued a
read request -- if the request was a write then it would write is as
modified in the L2 cache and invalid in it's own cache). The new core can
then read the value from L2 cache.

This ensures that no two cores can have the cacheline in the modified state
at once while enabling both to share it (have it in read-only).

\end{enumerate}

\end{examquestion}

\begin{examquestion}{2010}{7}{7}

\begin{enumerate}[label=(\alph*)]

\setcounter{enumi}{1}

\item In what situation might a shared second-level cache offer a
performance advantage over a memory hierarchy for a chip multiprocessor with
private L2 caches?

If the cores were sharing many variables between them then having a shared
L2 cache would mean sharing data would only require a L2 write rather than a
write to main memory -- this would be significantly faster.

Additionally, if the cores were frequently performing the same tasks then
the second cache could still have good spatial and temporal locality for
both cores.

However, if the cores were performing different tasks then each core would
be trashing the other cores L2 cache.

Building on this, if the cores frequently had poor spatial and
temporal locality then having a shared L2 cache may provide performance
benefits -- if they read from RAM then they have to check that the other
core doesn't have the value they are looking for which requires looking in
the other cores L2 cache -- this is awkward to implement. With a shared L2
cache we don't have to do that -- under an inclusive policy we can only
check our own L2 and under an exclusive or non-inclusive-non-exclusive
policy we check the other cores L1 cache -- which is faster than checking
their L2.

Having a shared L2 cache means that L2 access times are higher as the L2
cache has to be physically further from each core. This would decrease
performance.

\setcounter{enumi}{3}

\item How does adopting an inclusion policy simplify the implementatino of a
cache coherence mechanism in a chip multiprocessor wiht private L1 and L2
caches.

Consider using the MSI protocol.

On a system with an inclusive cache policy, we don't have to check the other
cores cache to ensure we have the only copy. We can check the copy in memory
-- if it is in the $M$ state then we know that the other core has modified
it and can request it from the other core. If it is in the $S$ state then
the requesting core knows it can load it into cache without conflicting with
the other core (and potentially move the cacheline into $M$).

Having an inclusive cache policy removes the need for snoopy buses which
greatly simplifies the system and scales better for larger numbers of cores.

Although, caches still need a way of signalling to other cores that they
wish to access a cacheline that is in the modified state; but the other core
will only have to address this if they have the cacheline -- they won't have
to look up their L2 cache every time the other core attempts to read from
memory.

\end{enumerate}

\end{examquestion}

\end{document}
