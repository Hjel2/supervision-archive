\newcommand{\svcourse}{CST Part IA: Introduction to Probability}
\newcommand{\svnumber}{1}
\newcommand{\svvenue}{Churchill, Room TBD}
\newcommand{\svdate}{2022-05-14}
\newcommand{\svtime}{11:00}
\newcommand{\svuploadkey}{PO5ogKIM8KQA22FZS8IAf8gxA8XKi19jxIBVHIfFZ+3GCBXuNUXS9lVN6bNYjxM/}

\newcommand{\svrname}{Mr Matthew Ireland}
\newcommand{\jkfside}{twoside}
\newcommand{\jkfhanded}{right}

\newcommand{\studentname}{Harry Langford}
\newcommand{\studentemail}{hjel2@cam.ac.uk}


\documentclass[10pt,\jkfside,a4paper]{article}

% DO NOT add \usepackage commands here.  Place any custom commands
% into your SV work files.  Anything in the template directory is
% likely to be overwritten!

\usepackage{fancyhdr}

\usepackage{lastpage}       % ``n of m'' page numbering
\usepackage{lscape}         % Makes landscape easier

\usepackage{verbatim}       % Verbatim blocks
\usepackage{epsfig}         % Embed encapsulated postscript
\usepackage{array}          % Array environment
\usepackage[nolinks]{qrcode}         % QR codes
\usepackage{enumitem}       % Required by Tom Johnson's exam question header

\usepackage{hhline}         % Horizontal lines in tables
\usepackage{siunitx}        % Correct spacing of units
\usepackage{amsmath}        % American Mathematical Society
\usepackage{amssymb}        % Maths symbols
\usepackage{amsthm}         % Theorems

\usepackage{ifthen}         % Conditional processing in tex

\usepackage[top=3cm,
            bottom=3cm,
            inner=2cm,
            outer=5cm]{geometry}

% PDF metadata + URL formatting
\usepackage[
            pdfauthor={\studentname},
            pdftitle={\svcourse, SV \svnumber},
            pdfsubject={},
            pdfkeywords={9d2547b00aba40b58fa0378774f72ee6},
            pdfproducer={},
            pdfcreator={},
            hidelinks]{hyperref}

\renewcommand{\headrulewidth}{0.4pt}
\renewcommand{\footrulewidth}{0.4pt}
\fancyheadoffset[LO,LE,RO,RE]{0pt}
\fancyfootoffset[LO,LE,RO,RE]{0pt}
\pagestyle{fancy}
\fancyhead{}
\fancyhead[LO,RE]{{\bfseries \studentname}\\\studentemail}
\fancyhead[RO,LE]{{\bfseries \svcourse, SV~\svnumber}\\\svdate\ \svtime, \svvenue}
\fancyfoot{}
\fancyfoot[LO,RE]{For: \svrname}
\fancyfoot[RO,LE]{\today\hspace{1cm}\thepage\ / \pageref{LastPage}}
\fancyfoot[C]{\qrcode[height=0.8cm]{\svuploadkey}}
\setlength{\headheight}{22.55pt}

\ifthenelse{\equal{\jkfside}{oneside}}{

 \ifthenelse{\equal{\jkfhanded}{left}}{
  % 1. Left-handed marker, one-sided printing or e-marking, use oneside and...
  \evensidemargin=\oddsidemargin
  \oddsidemargin=73pt
  \setlength{\marginparwidth}{111pt}
  \setlength{\marginparsep}{-\marginparsep}
  \addtolength{\marginparsep}{-\textwidth}
  \addtolength{\marginparsep}{-\marginparwidth}
 }{
  % 2. Right-handed marker, one-sided printing or e-marking, use oneside.
  \setlength{\marginparwidth}{111pt}
 }

}{
 % 3. Alternating margins, two-sided printing, use twoside.
}

\setlength{\parindent}{0em}
\addtolength{\parskip}{1ex}

% Exam question headings, labels and sensible layout (courtesy of Tom Johnson)
\setlist{parsep=\parskip, listparindent=\parindent}
\newcommand{\examhead}[3]{\section{#1 Paper #2 Question #3}}
\newenvironment{examquestion}[3]{
    \examhead{#1}{#2}{#3}\setlist[enumerate, 1]{label=(\alph*)}\setlist[enumerate, 2]{label=(\roman*)}
    \marginpar{\qrcode{https://www.cl.cam.ac.uk/teaching/exams/pastpapers/y#1p#2q#3.pdf}}
    \marginpar{\footnotesize \url{https://www.cl.cam.ac.uk/teaching/exams/pastpapers/y#1p#2q#3.pdf}}
}{}



\begin{document}

\section{C++ Templates}

\begin{enumerate}

\setcounter{enumi}{9}

\item Using metaprogramming, write a templated class \texttt{prime}, which
evaluates whether a literal integer constant is prime or not at compile time.

\begin{lstlisting}[language=C++]
template<int p, int i> struct factor{
    enum{factorless = (i <= 1) || (p % i) && factor<p, i-1>::factorless};
};

template<int p> struct factor<p, 1>{
    enum{factorless = true};
};
template<int p> struct prime{
    enum {is_prime = !factor<p, p-1>::factorless};
};
\end{lstlisting}

\item How can you be sure that your implementation of class \texttt{prime}
has been evaluated at compile time?

Values held in enums must be known at compile time. Since the value we wish
to compute is held in an enum, we know it must therefore have been evaluated
it compile time.

Secondly when I checked; the assembly generated for this program (with a main
function that initialised a prime and output the value) was
bit-for-bit identical to programs which output \texttt{true} or
\texttt{false} -- therefore the condition must have been evaluated at
compile time.

\end{enumerate}

\section{Diamond Problem}

What is the diamond problem? How does C++ deal with it? Illustrate
your answer with code.

The diamond inheritance problem is a byproduct of multiple inheritance. It
arises when a class inherits from different classes which themselves inherit
from the same superclass. To prevent superclasses breaking each others 
invariants, C++ has multiple copies of the attributes from this 
doubly-inherited superclass. However if we request for an attribute which 
the subclass has multiple copies of then it is ambiguous which copy we meant.
C++ requires the programmer to explicitly disambiguate using \texttt{::} (the
scope resolution operator).

\begin{lstlisting}[language=C++]
#include <iostream>
using namespace std;

struct X{
    int x;
    X(): x(0){}
};

struct Y: X{
    int y;
    Y(): y(0){}
};

struct Z: X{
    int z;
    Z(): z(0){}
};

struct C: Y, Z{
    C()= default;
    void setyx(int param){
    // set the x attribute inherited through Y
        Y::x = param;
    }
    int getyx(){
    // return the x attribute inherited through Y
        return Y::x;
    }
    void setzx(int param){
    // set the x attribute inherited through Z
        Z::x = param;
    }
    int getzx(){
    // return the x attribute inherited through Z
        return Z::x;
    }
};

int main(){
    C c;
    c.setyx(10);
    c.setzx(5);
    cout<<c.getyx()<<endl; // outputs 10
    cout<<c.getzx()<<endl; // outputs 5
    return 0;
}

\end{lstlisting}

\section{\texttt{virtual} keyword}

\begin{enumerate}

\item Give an example where failure to make a destructor \texttt{virtual}
causes a memory leak.

If a destructor is not virtual then which destructor is called will be based
on the static type of the object which is being deleted. If the static type
is different from the runtime type -- and the runtime type has additional
attributes on the heap, then they will not be deallocated; causing a memory
leak. Declaring a destructor as virtual means the destructor is determined
at runtime and is therefore guaranteed to be the runtime type rather than
the static type.

\begin{lstlisting}[language=C++]
#include <set>
using namespace std;

struct Base{};

struct Derived: public Base{
    set<int> *set_x;
    Derived(){
    // allocate a set<int> on the heap
        set_x = new set<int>();
    }
    ~Derived(){
    // deallocate the set<int> to prevent a memory leak
        delete set_x;
    }
};

int main(){
	// create a Base * that points to a Derived allocated on the heap
    Base *b = new Derived();

    /* this calls the Base destructor -- not Derived destructor
     * so set_x is never deallocated from the heap -- a memory leak! */
    delete b;
    return 0;
}
\end{lstlisting}

\item Why should destructors in an abstract class always be declared virtual?

An Abstract class in C++ is a class with at least one pure virtual function.
Abstract classes allow polymorphism by providing an interface specifying
what functions should be supported without specifying implementation details.
Functions can take abstract classes as arguments or return abstract classes.

The point of an abstract class is that functionality has not been fully
implemented -- and that subclasses should be able to implement this
functionality in whichever way they wish. Subclasses should therefore be
able to have other attributes. However, if an object has the static type of
the abstract class and is deleted then the destructor run will be that of the
abstract class. This would leak any of the attributes that the subclass added.
This can be solved by resolving the destructor functions by dynamic dispatch
and calling the subclasses destructor. In C++ we do this by declaring the
destructor virtual.

Failure to declare the destructor virtual would prevent subclasses from
adding new attributes; seriously restricting the way the subclass are
implemented and making the abstract class almost unusable.

\end{enumerate}

\section{C++ Concepts}

Explain what the following C++ concepts do and when one should use them:

\begin{enumerate}[label=(\alph*)]

\item \texttt{volatile} keyword

The volatile keyword is used on variables which may be changed by multiple
or processes. It ensures the compiler does not overly-optimise resulting in
incorrect performance.

We should use the volatile keyword when writing multithreaded
code with shared memory.

\item \texttt{friend} keyword

A class can declare another class or a function as ``friend'' and allow the
it to access its private methods and attributes.

This allows us to get the same effect as nested classes in Java. For example
we can make complicated data structures in C++ by allowing nodes to be
accessed by the structure; whilst also retaining private fields so that
other methods cannot break invariants.

\item pure virtual function

A pure virtual function is a function where the class provides no
implementation details: the function is declared using the following syntax
\texttt{A f(...) = 0;}. Any class with any pure virtual functions is known
as an abstract class and cannot be directly instantiated.

We should use pure virtual functions to define functionality that a type of
class should support in order to enable polymorphism.

\end{enumerate}

\section{Overloading}

C++ allows function overloading. Give an example of it. How and when is the
executed function chosen?

Overloading means declaring multiple functions with the same name and return
types but different argument types or lengths. There are then multiple
definitions which the compiler will decide between.

Since C++ is a statically typed language, overload resolution (choosing
which function to execute) can be performed at compile time. The compiler
will choose the best match for the type signature. If there is a perfect
match then this function will be called. Otherwise if the function arguments
can be cast to \textit{exactly one} function then this function will be
called. Otherwise which function we wish to call is ambiguous and a compiler
error will be thrown.

The function \texttt{f} is overloaded:

\begin{lstlisting}[language=C++]
int f(int x){
	return x;
}

int f(char x){
	return x + 1;
}

int g(int x){
	return x;
}

int g(int x[]){
	return 1;
}

int main(){
	f(3); // 3
	f((char) 3); // 4
	f(3LL); // error!
	/* long long is not a perfect match but
	 * can be cast to multiple declarations */

	g((char) 3); // 3
	/* which declaration of g we want is unambiguous
	 * as char cannot be cast to int[] */
	return 0;
}

\end{lstlisting}

\section{Pointers and References}

Give 2 similarities and 2 differences between pointers and references.
Support it with code if needed.

Similarities:

\begin{itemize}

\item Both pointers and references allow objects to be passed without copying.

\item Both pointers and references can be cast to different types.

\end{itemize}

Differences:

\begin{itemize}

\item Pointers can point to uninitialised objects; while references can only
point to initialised objects.

\item We can perform pointer arithmetic on pointers -- accessing data at
offsets; while we cannot do comparable operations with references.

\end{itemize}

\end{document}
