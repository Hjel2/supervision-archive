\newcommand{\svcourse}{CST Part IA: Introduction to Probability}
\newcommand{\svnumber}{1}
\newcommand{\svvenue}{Churchill, Room TBD}
\newcommand{\svdate}{2022-05-14}
\newcommand{\svtime}{11:00}
\newcommand{\svuploadkey}{PO5ogKIM8KQA22FZS8IAf8gxA8XKi19jxIBVHIfFZ+3GCBXuNUXS9lVN6bNYjxM/}

\newcommand{\svrname}{Mr Matthew Ireland}
\newcommand{\jkfside}{twoside}
\newcommand{\jkfhanded}{right}

\newcommand{\studentname}{Harry Langford}
\newcommand{\studentemail}{hjel2@cam.ac.uk}


\documentclass[10pt,\jkfside,a4paper]{article}

% DO NOT add \usepackage commands here.  Place any custom commands
% into your SV work files.  Anything in the template directory is
% likely to be overwritten!

\usepackage{fancyhdr}

\usepackage{lastpage}       % ``n of m'' page numbering
\usepackage{lscape}         % Makes landscape easier

\usepackage{verbatim}       % Verbatim blocks
\usepackage{epsfig}         % Embed encapsulated postscript
\usepackage{array}          % Array environment
\usepackage[nolinks]{qrcode}         % QR codes
\usepackage{enumitem}       % Required by Tom Johnson's exam question header

\usepackage{hhline}         % Horizontal lines in tables
\usepackage{siunitx}        % Correct spacing of units
\usepackage{amsmath}        % American Mathematical Society
\usepackage{amssymb}        % Maths symbols
\usepackage{amsthm}         % Theorems

\usepackage{ifthen}         % Conditional processing in tex

\usepackage[top=3cm,
            bottom=3cm,
            inner=2cm,
            outer=5cm]{geometry}

% PDF metadata + URL formatting
\usepackage[
            pdfauthor={\studentname},
            pdftitle={\svcourse, SV \svnumber},
            pdfsubject={},
            pdfkeywords={9d2547b00aba40b58fa0378774f72ee6},
            pdfproducer={},
            pdfcreator={},
            hidelinks]{hyperref}

\renewcommand{\headrulewidth}{0.4pt}
\renewcommand{\footrulewidth}{0.4pt}
\fancyheadoffset[LO,LE,RO,RE]{0pt}
\fancyfootoffset[LO,LE,RO,RE]{0pt}
\pagestyle{fancy}
\fancyhead{}
\fancyhead[LO,RE]{{\bfseries \studentname}\\\studentemail}
\fancyhead[RO,LE]{{\bfseries \svcourse, SV~\svnumber}\\\svdate\ \svtime, \svvenue}
\fancyfoot{}
\fancyfoot[LO,RE]{For: \svrname}
\fancyfoot[RO,LE]{\today\hspace{1cm}\thepage\ / \pageref{LastPage}}
\fancyfoot[C]{\qrcode[height=0.8cm]{\svuploadkey}}
\setlength{\headheight}{22.55pt}

\ifthenelse{\equal{\jkfside}{oneside}}{

 \ifthenelse{\equal{\jkfhanded}{left}}{
  % 1. Left-handed marker, one-sided printing or e-marking, use oneside and...
  \evensidemargin=\oddsidemargin
  \oddsidemargin=73pt
  \setlength{\marginparwidth}{111pt}
  \setlength{\marginparsep}{-\marginparsep}
  \addtolength{\marginparsep}{-\textwidth}
  \addtolength{\marginparsep}{-\marginparwidth}
 }{
  % 2. Right-handed marker, one-sided printing or e-marking, use oneside.
  \setlength{\marginparwidth}{111pt}
 }

}{
 % 3. Alternating margins, two-sided printing, use twoside.
}

\setlength{\parindent}{0em}
\addtolength{\parskip}{1ex}

% Exam question headings, labels and sensible layout (courtesy of Tom Johnson)
\setlist{parsep=\parskip, listparindent=\parindent}
\newcommand{\examhead}[3]{\section{#1 Paper #2 Question #3}}
\newenvironment{examquestion}[3]{
    \examhead{#1}{#2}{#3}\setlist[enumerate, 1]{label=(\alph*)}\setlist[enumerate, 2]{label=(\roman*)}
    \marginpar{\qrcode{https://www.cl.cam.ac.uk/teaching/exams/pastpapers/y#1p#2q#3.pdf}}
    \marginpar{\footnotesize \url{https://www.cl.cam.ac.uk/teaching/exams/pastpapers/y#1p#2q#3.pdf}}
}{}



\begin{document}

\begin{examquestion}{2008}{3}{3}

A hardware engineer stores a FIFO queue of bits in an int on a platform with
32-bit ints and 8-bit chars using the following C++ class:

\begin{lstlisting}[language=C++]

class BitQueue{
	int valid_bits;	// the number of valid bits held in the queue
	int queue;	// least significant bit is most recent bit added
public:
	BitQueue(): valid_bits(0), queue(0) {}
	void push(int val, int bsize);
	int pop(int bsize);
	int size();
};

\end{lstlisting}

\begin{enumerate}[label=(\alph*)]

\item Write an implementation of \texttt{BitQueue::size}, which should return
the number of bits currently held in queue.

\begin{lstlisting}[language=C++]

int BitQueue::size(){
	return valid_bits;
}

\end{lstlisting}

\item Write an implementation of \texttt{BitQueue::push}, which places the bsize
least significant bits from val onto queue and updates valid.bits. An
exception should be thrown in cases where data would otherwise be lost.

\begin{lstlisting}[language=C++]

#include <exception>
#include <string>
using namespace std;

struct BitQueueFullException : exception{
    int overflow;
    explicit BitQueueFullException(int i) : overflow(i){}
    const char* what() const throw(){
    	return "BitQueue overflowed by " + to_string(i) + "bits";
    }
};

void BitQueue::push(int val, int bsize){
    if (valid_bits + bsize > 32){
        throw BitQueueFullException(bsize+size() - 32);
    }
    valid_bits += bsize;
    queue <<= bsize;
    if (bsize == 32){
    	queue |= val & -1;
    }
    else{
    	queue |= val & (((1 << bsize - 1) - 1 << 1) + 1);
    }
}

\end{lstlisting}

\item Write an implementation of \texttt{BitQueue::pop}, which takes
\texttt{bsize} bits from queue, provides them as the bsize least significant
bits in the return value, and updates valid.bits. An exception should be
thrown when any requested data is unavailable.

\begin{lstlisting}[language=C++]

/* Using the standard exception library is not necessary but it's
 * good practice. We can throw any object. However, if the object
 * derives exception then general purpose code can handle the exception
 * better -- and this scales better for large projects rather than
 * throwing various unrelated objects from every different function */
#include <exception>
#include <string>
using namespace std;

struct BitQueueEmptyException : exception{
	const char* what() const throw(){
		return "BitQueue was popped from when empty";
	}
};

int BitQueue::pop(int bsize){
    if (bsize > valid_bits){
        throw BitQueueEmptyException();
    }
    if (!bsize){
        return 0;
    }
    valid_bits -= bsize;
    if (bsize == 32){
    	return queue&-1;
    }
    else{
    	return (queue&((((1<<bsize-1)-1<<1)+1)<<valid_bits))>>valid_bits;
    }
}

\end{lstlisting}

\item The hardware engineer has built a communication device together with a
C++ library function send to transmit data with the following declaration.

\begin{lstlisting}[language=C++]

void send(char);

\end{lstlisting}

Use the BitQueue class to write a C++ definition for:

\begin{lstlisting}[language=C++]

void sendmsg(const char* msg);

\end{lstlisting}

Each of the characters in msg should be encoded, in index order, using the
following binary codes: \texttt{'a'=0}, \texttt{'b'=10}, \texttt{'c'=1100} and
\texttt{'d'=1101}. All other characters should be ignored. Successive binary codes
should be bit-packed together and the code 111 should be used to denote the
end of the message. Chunks of 8-bits should be sent using the send function
and any remaining bits at the end of a message should be padded with zeros.
For example, executing \texttt{sendmsg("abcd")} should call the send
function twice, with the binary values 01011001 followed by 10111100.

\begin{lstlisting}[language=C++]

struct UnsupportedCharException : exception{};

void sendmsg(const char* msg){
    BitQueue bq;
    while (msg[0]){
        switch (msg[0]){
            case 'a':
                bq.push(0b0, 1);
                break;
            case 'b':
                bq.push(0b10, 2);
                break;
            case 'c':
                bq.push(0b1100, 4);
                break;
            case 'd':
                bq.push(0b1101, 4);
                break;
            default:
                throw UnsupportedCharException();
        }
        if (bq.size() >= 8){
            send(bq.pop(8));
        }
        msg++;
    }
    bq.push(7, 3);
    if (bq.size() > 8){
        send(bq.pop(8));
    }
    bq.push(0, 8);
    send(bq.pop(8));
}

\end{lstlisting}

\end{enumerate}

\end{examquestion}

\begin{examquestion}{2009}{3}{1}

Explain all of the following C or C++ features. You may use a short
fragment of code to complement your explanation.

\begin{enumerate}[label=(\alph*)]

\item The declaration of a C++ class illustrating constructor, variable and
method.

Declaring a C++ class declares its existence. This means it can be
instantiated (if it is not abstract), ``friended'' and subclassed. A
constructor is called when we create a new instance of the class. This
allows us to instantiate the class into a valid state. Methods operate on
the state of the class. Method declarations can be made inside the class
while the method is defined elsewhere as showed below.

\begin{lstlisting}[language=C++]
class C{
	int x;
	const int y;
public:
	C(int xp, int yp) : x(xp), y(yp){};
	void inc();
};

void C::inc(){
	x++;
}
\end{lstlisting}

\item The use of a virtual destructor

Destructors are called when an object goes out of scope. When an object goes
out of scope, the objects destructor and all the destructors of all
superclasses will be called (in order). This means that attributes are
deallocated properly. However, if the static type of an object is a
superclass, then the destructor which is called is the superclasses
destructor. This may cause a memory leak or other errors (for example if
the subclass allocated memory on heap). However, if the destructor is
declared as virtual then the destructor that is called will be the runtime
type (the subclass). This will avoid memory leakage as intended. However, in
order to determine at runtime which destructor to call, we will need a
vtable. This will create additional time and space overhead.

\begin{lstlisting}[language=C++]
#include <vector>
using namespace std;

struct A{
    vector<int> *vec1 = new vector<int>(10000, 0);
    A()= default;
    virtual ~A() {
        cout<<"A"<<endl;
        delete vec1;
    }
};

struct B : A{
    B()= default;
    vector<int> *vec2 = new vector<int>(10000, 0);
    ~B() override{
        cout<<"B"<<endl;
        delete vec2;
    }
};

void f(){
	A *a = new B(); // static type of A (A) != runtime type (B)
	delete a; // without a virtual function this will leak vec2
}
\end{lstlisting}

\item The difference between \texttt{malloc()} and \texttt{free()}; and new and delete

\texttt{malloc(i)} is a function in the \texttt{stdlib.h} header
which allocates \texttt{i} untyped bytes onto the heap and returns a
pointer to the first one. We can then interpret these bytes as whatever
type we wish.

\texttt{new} can be placed before an object instantiation and will allocate
the object on the heap and return a typed pointer to it. In contrast to
\texttt{malloc()}, the pointer is typed and the objects constructor will be
run.

The main differences between \texttt{new} and \texttt{malloc()} are that
\texttt{malloc} returns an untyped pointer while \texttt{new} returns a
typed pointer; and \texttt{malloc} does not initialise memory while
\texttt{new} will run the objects constructor.

\texttt{free(p)} takes a pointer \texttt{p} to bytes that were allocated on
the heap and deallocates them. It does not do this recursively and there is
no way to overload \texttt{free} to perform different deallocation for
different datatypes. If the datatype we wish to deallocate itself contains
pointers to other data on the heap we wish to deallocate, then we must
remember to free that ourselves. We must therefore know the runtime type of
any object we wish to deallocate and remember to deallocate it and all
attributes properly.

\texttt{delete} takes a pointer to an object \texttt{p}, runs the objects
destructor (and the destructor of all superclasses) and deallocates the
memory where \texttt{p} is held. The destructors can be changed to
deallocate attributes or recurse through a data structure.

The main difference between \texttt{delete} and \texttt{free()} is that
\texttt{delete} runs the destructor and deallocates while \texttt{free()}
only deallocates. \texttt{delete} is therefore higher level than
\texttt{free()}.

\begin{lstlisting}[language=C]
struct list{
	struct list *next;
}

void free_list(struct list *lst){
	while (*lst){
		struct list *tmp = lst;
		lst = lst->next;
		free(tmp);
	}
}

int main(void){
	// allocates 8 bytes on the heap and returns an untyped pointer.
	void *p = malloc(10000);
	// we can then interpret these bytes however we want
	int *i = p;
	struct list *l = p;
	l->next = malloc(8);
	/* there is no way to define a destructor in C with free
	 * so free(l) will cause a memory leak
	 * we have to define our own function to free memory and remember
	 * the actual type and call the correct function to deallocate the
	 * data */
	free_list(l);
}
\end{lstlisting}

\begin{lstlisting}[language=C++]
#include <vector>
using namespace std;

struct A{
	vector<int> *vec = new vector<int>(10000, 0);
	A()= default;
	~A(){
		delete vec;
	}
}
int main(){
	A *a = new A(); // the return type is typed and initialised
	delete a; // the destructor is run so vec is also deallocated
	return 0;
}
\end{lstlisting}

\item Overloading an operator

Operator overloading allows us to program in more concise and intuitive
ways, without calling many different verbose functions for basic things.
However, it is very easy to create inconsistencies. For example in the
below example \texttt{==} is defined but \texttt{!=} is not defined -- which
could cause usability issues.

\begin{lstlisting}[language=C++]
#include <iostream>
using namespace std;

class Vec2{
	const int x;
	const int y;
public:
	Vec2(int xp, int yp) : x(x2), y(y2){}
	bool operator==(const Vec2d &other) const{
		return x == other.x && y == other.y;
	}
}

int main(){
	Vec2 v1(1, 1);
	Vec2 v2(1, 2);
	if (v1 == v2){ // this is true
		cout<<"equal!"<<endl;
	}
	if (!(v1 != v2)){ // this fails compilation
		cout<<"equal!"<<endl;
	}
	return 0;
}

\end{lstlisting}

\item Pointer arithmetic

Pointer arithmetic allows us to increment and decrement pointers. This is
particularly useful when iterating through an array or a string in C .
However, pointer arithmetic is very low level and notoriously prone to bugs.
It is less relevant in C++ as we often use higher-level types such as
\textit{vector<>} or \textit{string}. Classic misuses of pointer arithmetic
are iterating through memory in the heap -- usually consecutively allocated
memory is allocated adjacent in the heap, however this is not guaranteed and
so this is not allowed.

\begin{lstlisting}[language=C++]
bool contains(const char *str, const char c){
	while (*str){ // while not "\0"
		if (*(str++) == c){
			return true;
		}
	}
	return false;
}
\end{lstlisting}

However, in C++ it would be more common to write code using higher level
objects which do not need pointer arithmetic.

\begin{lstlisting}[language=C++]
bool contains(string str, const char c){
	for (int i = 0; i < str.length(); i++){
		if (str[i] == c){
			return true;
		}
	}
	return false;
}
\end{lstlisting}

\item Catching and throwing exceptions including the passing of a
user-defined structure

Exceptions are designed for control flow in exceptional circumstances. Some
methods can fail and may do so recoverably. In these cases, an exception is
thrown which will unwind the stack until the exception is caught. Unwinding
the stack involves running the destructors of all objects which go out of
scope. If a destructor throws an exception, then the program will terminate.

\begin{lstlisting}[language=C++]
#include <exception>
#include <string>
using namespace std;

struct IndexOutOfBoundsException{
	IndexOutOfBoundsException(int iarg, narg) : i(iarg), n(narg){}
	const char *what() const throw(){
		return "Index " + to_string(i) + "is out of bounds for" +
			"BoundsCheckedArray of length " + to_string(n);
	}
}

class BoundsCheckedArray{
	const int n;
public:
	BoundsCheckedArray(int i) : n(i) {}
	int &operator[](int i){
		if (i < 0 || i >= n){
			throw IndexOutOfBoundsException(i, n);
		}
	}
};

\end{lstlisting}

\item The meaning of the keywords \texttt{static} and \texttt{const}

A \texttt{static} variable is initialised when the program starts -- there
is one shared copy of a static variable. \texttt{static} variables in
functions are preserved between function calls. \texttt{static} variables in
classes or structs are shared by all instances of that class or struct.

\texttt{static} variables and functions are only accessible from inside the
file in which they were declared.

\begin{lstlisting}[language=C++]
int f(){
	static int i = 0;
	return i++;
}

int main(){
	f(); // returns 0
	f(); // returns 1
	f(); // returns 2
	...
	return 0;
}
\end{lstlisting}

\begin{lstlisting}[language=C++]
class C(){
	static int x;
public:
	C(int i) : x(i);
	int getX(){return x;}
};

int main(){
	C c1(1);
	c.getX();  // 1
	C c2(10);
	c.getX();  // 10
	c2.getX(); // 10
	return 0;
}

\end{lstlisting}

A variable \texttt{x} declared with \texttt{const} is immutable and cannot be
changed after it is initialised. If it is declared inside a class then it
must be initialised before the main constructor body starts -- either at the
same time as its definition or immediately after the constructor
declaration using \texttt{x(i)} for some argument i.

\end{enumerate}

\end{examquestion}

\begin{examquestion}{2012}{3}{3}

In this question, where appropriate, you may use a short fragment of code to
complement your explanation.

\begin{enumerate}[label=(\alph*)]

\item

\begin{enumerate}[label=(\roman*)]

\item What is the difference between a local and global variable in C?
(Consider variable scope, storage and initialisation)

Global variables are declared and defined in the global scope. Therefore they
can be accessed by any function in the program (and in other programs if
declared extern). While Local variables are defined inside functions (in the
local scope) and are therefore only visible inside functions.

Local variables are stored on the stack while global variables are stored in
the data segment.

Global variables are initialised when the program starts while local
variables are initialised only when the function declares them. Local
variables go out of scope when the block they were declared in ends -- at
this point they are deallocated. Global variables are only deallocated when
the program ends. Static local variables are an exception to these rules --
they are initialised at startup and are stored on the data segment. Static
local variables can be viewed as global variables which are only visible
from one function.

\begin{lstlisting}[language=C]
int x = 0;
void f(void){
	int y = 0;
	x; // valid!
	y; // valid!
}
x; // valid!
y; // invalid!
\end{lstlisting}

\item What are the properties of a static member variable in a C++ class?

Static member variables are shared between all instances of a class. They
are stored on the data segment and are initialised when the program
first starts.

\begin{lstlisting}[language=C++]
class A{
	static int x = 0;
	A()=default;
public:
	int getX(){return x++;}
}

int main(){
	A a1;
	A a2;
	A a3;
	a1.getX(); // 0
	a2.getX(); // 1
	a1.getX(); // 2
	a3.getX(); // 3
	return 0;
}
\end{lstlisting}

\end{enumerate}

\item

\begin{enumerate}[label=(\roman*)]

\item Briefly explain pointer arithmetic in C. Give an example code snippet
involving pointers in which it would be \textit{inappropriate} to use
pointer arithmetic, and explain why.

In C, we can add offsets to pointers and access the memory saved there. This
is infamous for causing critical bugs, security exploits and segmentation
faults. Without great care, pointer arithmetic is very likely to lead to
issues.

A common assumption that causes bugs is that consecutively malloced memory
will be adjacent. The following code attempts to implement a dynamically
sized string:

\begin{lstlisting}[language=C]
void addChar(char *start, int *len, char c){
	malloc(1);
	*len++;
	*(start + len) = c; // equivalent to start[len] = c;
}
\end{lstlisting}

However, the C specification makes no guarantees that memory will be
allocated contiguously on the heap. Therefore this code will eventually
fail (and hopefully cause a segmentation fault) -- even if it works for a
short while before it does.

\iffalse

Appropriate when iterating through a string or an object with known finite
length where we know the size of each object -- we know the end and so won't
iterate off the end of the array. It's inappropriate to use it with heap
memory (often heap memory is allocated contiguously but we cannot rely on
that) or when the size of a structure is not known.

\fi

\item Explain how in some respects pointers are equivalent to arrays, and
give one respect in which they differ.

Both arrays and pointers can be used to store contiguous blocks of memory of
user-specified size. Both arrays and pointers are passed as a reference to
the first element in this contiguous block meaning that only one
pointer-size is passed as argument. Additionally, neither array nor pointers
are bounds-checked -- therefore it is equally easy to perform an illegal
access on an array as on a pointer.

Arrays are strongly typed while pointers have no type constraints. You
cannot have an array of type \texttt{void []} or typecast arrays to arrays
of other types; while you can have pointers of type \texttt{void *} and
can typecast pointers.

\end{enumerate}

\item Explain why a function might be declared virtual in a C++ superclass.

Declaring a function virtual in a C++ superclass means the implementation
of the function called is decided at runtime based on the dynamic type of the
object rather than the objects static type. This allows java-style
polymorphism. However, it also increases the runtime overhead associated with
function calls as we have to look up which function to call in the vtable
rather than jump to an address known at compile time.

\item

\begin{enumerate}[label=(\roman*)]

\item How does the use of \texttt{void *} pointer in C allow a form of
polymorphism? Give an example function declaration using the \texttt{void *}
pointer.

\texttt{void *} is an untyped pointer. This allows us to operate on
different datatypes without restriction. However, it also means we don't
know the size of the datatype we are operating on or the size of any
elements in the datatype or what operations we can do on the datatype.
Therefore for all nontrivial operations, we have to pass the size of the
datatype, the size of elements in the datatype and any functions we wish to
apply. This style of programming enables polymorphism but is very awkward to
program with and does not typecheck. For example in the example below, there
is no guarantee that the comparison function compares the same type as the
array we are sorting.

\begin{lstlisting}[language=C]
void sort(void *start, void *end, int esize, int cmp(void *, void *));
\end{lstlisting}

\item What is the main problem with the use of \texttt{void *}, and how does
C++ improve on this? Give the improved function declaration in C++ for your
example function in part (d)(i).

If we use \texttt{void *} we do not know different occurrences of
\texttt{void *} have the same type. For example we could call
\texttt{quicksort} with arguments of type \texttt{long long} but a comparison
function which compared \texttt{int}s. This would give invalid results
but would compile correctly and would not fail at runtime -- it will
just interpret each long long as being two integers and give an unintended
result.

C++ addresses this by using templates and polymorphism. Polymorphism allows
us to cast any object to any of its superclasses safely and therefore allows
us to pass any subclass as argument to any function expecting a superclass.

Templates allow us to define implementations of functions which take
specific types (or specific values). These can ensure that the types passed
are the same or have a particular relation. For example the declaration
below ensures that all arguments have the same type. C++ templates can be
used to implement compile-time metaprogramming -- a Turing Powerful language.

\begin{lstlisting}[language=C++]
template <class T> void quicksort(T *start, T *end, int cmp(T, T));
\end{lstlisting}

\end{enumerate}

\item

\begin{enumerate}[label=(\roman*)]

\item Why might it be useful to define a copy constructor for a C++ class?
Give an example of a copy constructor for a simple class.

Copy constructors allow us to initialise a new object with the same values
as an old object -- whatever that may be. This allows us to copy an existing
object in the correctly and initialise an object to a state without running
the same set of operations as we did on the original object. If we want to
share some attributes then we can and this can be more space efficient.

For example consider the following struct \texttt{Vec} which uses reference
counting. When we create a new \texttt{Vec}, we do not want to copy the
reference count. Copy constructors allow us to copy other fields without
copying the reference count.

\begin{lstlisting}[language=C++]
struct Vec{
	int refs;
	int x, y;
	Vec(int i, int j): x(i), y(j), refs(1){}
	Vec(const Vec& other): x(other.x), y(other.y), refs(1){}
};
\end{lstlisting}

\item Why might it be useful to explicitly define the assignment operator
(=) for a C++ class? Give an example definition of the assignment operator
for a simple class.

Copy constructors can only be used when declaring a new object -- copy
constructors always allocate new memory. If we have an existing object that
we wish to copy another object into, then we must explicitly define the
assignment operator.

Consider the reference counting example. If we were to assign to this class
then we would not want to copy reference counts. Explicitly defining =
allows us to do this.

\begin{lstlisting}[language=C++]
Vec::operator=(const Vec& other){
	x = other.x;
	y = other.y;
}
\end{lstlisting}

\end{enumerate}

\end{enumerate}

\end{examquestion}

\end{document}
