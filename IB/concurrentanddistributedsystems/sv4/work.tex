\newcommand{\svcourse}{CST Part IA: Software Engineering and Security}
\newcommand{\svnumber}{1}
\newcommand{\svvenue}{Microsoft Teams}
\newcommand{\svdate}{2022-05-11}
\newcommand{\svtime}{15:00}
\newcommand{\svuploadkey}{CBd13xmL7PC1zqhNIoLdTiYUBnxZhzRAtJxv/ytRdM1r7qIfwMsxeVwM/pPcIo8l}

\newcommand{\svrname}{Dr Sam Ainsworth}
\newcommand{\jkfside}{oneside}
\newcommand{\jkfhanded}{yes}

\newcommand{\studentname}{Harry Langford}
\newcommand{\studentemail}{hjel2@cam.ac.uk}


\documentclass[10pt,\jkfside,a4paper]{article}

\usepackage{float}

% DO NOT add \usepackage commands here.  Place any custom commands
% into your SV work files.  Anything in the template directory is
% likely to be overwritten!

\usepackage{fancyhdr}

\usepackage{lastpage}       % ``n of m'' page numbering
\usepackage{lscape}         % Makes landscape easier

\usepackage{verbatim}       % Verbatim blocks
\usepackage{listings}       % Source code listings
\usepackage{graphicx}
\usepackage{float}
\usepackage{epsfig}         % Embed encapsulated postscript
\usepackage{array}          % Array environment
\usepackage{qrcode}         % QR codes
\usepackage{enumitem}       % Required by Tom Johnson's exam question header

\usepackage{hhline}         % Horizontal lines in tables
\usepackage{siunitx}        % Correct spacing of units
\usepackage{amsmath}        % American Mathematical Society
\usepackage{amssymb}        % Maths symbols
\usepackage{amsthm}         % Theorems

\usepackage{ifthen}         % Conditional processing in tex

\usepackage[top=3cm,
            bottom=3cm,
            inner=2cm,
            outer=5cm]{geometry}

% PDF metadata + URL formatting
\usepackage[
            pdfauthor={\studentname},
            pdftitle={\svcourse, SV \svnumber},
            pdfsubject={},
            pdfkeywords={9d2547b00aba40b58fa0378774f72ee6},
            pdfproducer={},
            pdfcreator={},
            hidelinks]{hyperref}

\renewcommand{\headrulewidth}{0.4pt}
\renewcommand{\footrulewidth}{0.4pt}
\fancyheadoffset[LO,LE,RO,RE]{0pt}
\fancyfootoffset[LO,LE,RO,RE]{0pt}
\pagestyle{fancy}
\fancyhead{}
\fancyhead[LO,RE]{{\bfseries \studentname}\\\studentemail}
\fancyhead[RO,LE]{{\bfseries \svcourse, SV~\svnumber}\\\svdate\ \svtime, \svvenue}
\fancyfoot{}
\fancyfoot[LO,RE]{For: \svrname}
\fancyfoot[RO,LE]{\today\hspace{1cm}\thepage\ / \pageref{LastPage}}
\fancyfoot[C]{\qrcode[height=0.8cm]{\svuploadkey}}
\setlength{\headheight}{22.55pt}


\ifthenelse{\equal{\jkfside}{oneside}}{

 \ifthenelse{\equal{\jkfhanded}{left}}{
  % 1. Left-handed marker, one-sided printing or e-marking, use oneside and...
  \evensidemargin=\oddsidemargin
  \oddsidemargin=73pt
  \setlength{\marginparwidth}{111pt}
  \setlength{\marginparsep}{-\marginparsep}
  \addtolength{\marginparsep}{-\textwidth}
  \addtolength{\marginparsep}{-\marginparwidth}
 }{
  % 2. Right-handed marker, one-sided printing or e-marking, use oneside.
  \setlength{\marginparwidth}{111pt}
 }

}{
 % 3. Alternating margins, two-sided printing, use twoside.
}


\setlength{\parindent}{0em}
\addtolength{\parskip}{1ex}

% Exam question headings, labels and sensible layout (courtesy of Tom Johnson)
\setlist{parsep=\parskip, listparindent=\parindent}
\newcommand{\examhead}[3]{\section{#1 Paper #2 Question #3}}
\newenvironment{examquestion}[3]{
\examhead{#1}{#2}{#3}\setlist[enumerate, 1]{label=(\alph*)}\setlist[enumerate, 2]{label=(\roman*)}
\marginpar{\href{https://www.cl.cam.ac.uk/teaching/exams/pastpapers/y#1p#2q#3.pdf}{\qrcode{https://www.cl.cam.ac.uk/teaching/exams/pastpapers/y#1p#2q#3.pdf}}}
\marginpar{\footnotesize \href{https://www.cl.cam.ac.uk/teaching/exams/pastpapers/y#1p#2q#3.pdf}{https://www.cl.cam.ac.uk/\\teaching/exams/pastpapers/\\y#1p#2q#3.pdf}}
}{}


\begin{document}

\begin{examquestion}{1998}{12}{1}

For a large-scale distributed system or application:

\begin{enumerate}[label=(\alph*)]

\item Describe alternative approaches to creating unique names. Include a
discussion of the information conveyed by a name.

A name is a unique identifier paired with information about the node.
The information paired with the node can be arbitrary but should
be immutable and generally relevant. For example information about the
hardware or software the node is running or a ``priority''. If we assume a
fail-stop model then the time the node was created would also be reasonable.
Critically, the information in the nodes name can only be provided by the
node which named that node. So if a node named itself then it may not know
its time of creation or its own priority.

There are a three main ways to generate a unique identifier:
\begin{itemize}

\item Delegated by by a naming service such as a router or leader (RAFT).

This unique identifier can simply be a counter. If given by a ``superior''
then it should use a unique ``superior ID'' as a prefix.

\item A function of something known to be unique.

For example the MAC address or the ID of the user who is working on the node
paired with the wall time at which they logged in.

\item A random number

This is not guaranteed to be unique; but can be designed such that the
probability of uniqueness is astonishingly high. For most systems, a 64-bit
number would be sufficient and could even be paired with the current wall
time to bring the probability of uniqueness even higher and stop the
probability of ever having a collision in a given second increasing.

For example, if we used a random 128-bit integer paired with the wall time
(ms-precision) then the probability of the Internet having a name collision
in the next million years is $\approx 1.4 \times 10^{-14}$. This assumes
that the seed to the random number generator is \textit{not} the wall-time.

\end{itemize}

\item Discuss name to location binding under the headings

\begin{enumerate}[label=(\roman*)]

\item availability of the binding service

To get high availability of the binding service, the binding service should
be local. In this case either a random number or a function of an
existing unique identifier is preferable. If names are given by a leader
then the node may have to wait to be given a name.

\item consistency of the naming data

To ensure data put in the name is consistent, the data should all come from
teh same source: a coordinator. In this case it is preferable to use
strategy one -- where a leader or superior delegates names.

\item mobility of named objects

If a leader delegates a name, then it may not be unique on other networks.
This is therefore unusable. Additionally, if network-specific information
(such as time of joining or priority) is included in the name then the named
object is not portable. Names generated under strategy 2 or strategy 3 are
globally unique.

\end{enumerate}

\end{enumerate}

\end{examquestion}

\begin{examquestion}{1999}{9}{1}

Describe one of the following architectures for supporting distributed
programming:

\begin{table}[H]
\begin{tabular}{lll}
\textbf{Either} & \ \ \ & Remote Procedure Call \\
\\
\textbf{or} & & Object Request Brokerage \\
\end{tabular}
\end{table}

RPC is a method of providing location transparency which involves wrapping a
function call to another node inside a ``stub'' which has the same
signature and arguments as the function. From the users perspective, they
make a local function call to the stub.

When the stub is called, instead of performing the function, it marshalls
the arguments into a transmittable format (ie JSON) then sends them across the
network to the node on which the function is to be performed. The target
node then unmarshalls the data and performs the function. Dependent on the
exact semantics, a response or return value may be sent or the original
sender may be polled.

Many emerging applications require timely responses to events. Discuss the
following approaches to achieving this:

\begin{enumerate}[label=(\alph*)]

\item polling

Polling is the process of regularly checking if an event has happened. In
the context of distributed systems, this would mean sending a message to the
node on which the RPC was performed every $\tau$s. This is a very
inefficient strategy and will greatly increase network traffic.

\item synchronous call back

In synchronous call-back, the thread which made the  RPC blocks until the
call-back returns at which point it is immediately woken. This is analagous
to a blocking system call in an Operating System.

\item asynchronous notification

In asynchronous notification, the thread which makes the RPC can continue
executing and will receive an asynchronous notification when the RPC
completes. This is analagous to an asynchronous IO operation in an
Operating System.

\end{enumerate}

\end{examquestion}

\end{document}
