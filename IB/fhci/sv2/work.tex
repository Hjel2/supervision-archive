\newcommand{\svcourse}{CST Part IA: Introduction to Probability}
\newcommand{\svnumber}{1}
\newcommand{\svvenue}{Churchill, Room TBD}
\newcommand{\svdate}{2022-05-14}
\newcommand{\svtime}{11:00}
\newcommand{\svuploadkey}{PO5ogKIM8KQA22FZS8IAf8gxA8XKi19jxIBVHIfFZ+3GCBXuNUXS9lVN6bNYjxM/}

\newcommand{\svrname}{Mr Matthew Ireland}
\newcommand{\jkfside}{twoside}
\newcommand{\jkfhanded}{right}

\newcommand{\studentname}{Harry Langford}
\newcommand{\studentemail}{hjel2@cam.ac.uk}


\documentclass[10pt,\jkfside,a4paper]{article}

% DO NOT add \usepackage commands here.  Place any custom commands
% into your SV work files.  Anything in the template directory is
% likely to be overwritten!

\usepackage{fancyhdr}

\usepackage{lastpage}       % ``n of m'' page numbering
\usepackage{lscape}         % Makes landscape easier

\usepackage{verbatim}       % Verbatim blocks
\usepackage{epsfig}         % Embed encapsulated postscript
\usepackage{array}          % Array environment
\usepackage[nolinks]{qrcode}         % QR codes
\usepackage{enumitem}       % Required by Tom Johnson's exam question header

\usepackage{hhline}         % Horizontal lines in tables
\usepackage{siunitx}        % Correct spacing of units
\usepackage{amsmath}        % American Mathematical Society
\usepackage{amssymb}        % Maths symbols
\usepackage{amsthm}         % Theorems

\usepackage{ifthen}         % Conditional processing in tex

\usepackage[top=3cm,
            bottom=3cm,
            inner=2cm,
            outer=5cm]{geometry}

% PDF metadata + URL formatting
\usepackage[
            pdfauthor={\studentname},
            pdftitle={\svcourse, SV \svnumber},
            pdfsubject={},
            pdfkeywords={9d2547b00aba40b58fa0378774f72ee6},
            pdfproducer={},
            pdfcreator={},
            hidelinks]{hyperref}

\renewcommand{\headrulewidth}{0.4pt}
\renewcommand{\footrulewidth}{0.4pt}
\fancyheadoffset[LO,LE,RO,RE]{0pt}
\fancyfootoffset[LO,LE,RO,RE]{0pt}
\pagestyle{fancy}
\fancyhead{}
\fancyhead[LO,RE]{{\bfseries \studentname}\\\studentemail}
\fancyhead[RO,LE]{{\bfseries \svcourse, SV~\svnumber}\\\svdate\ \svtime, \svvenue}
\fancyfoot{}
\fancyfoot[LO,RE]{For: \svrname}
\fancyfoot[RO,LE]{\today\hspace{1cm}\thepage\ / \pageref{LastPage}}
\fancyfoot[C]{\qrcode[height=0.8cm]{\svuploadkey}}
\setlength{\headheight}{22.55pt}

\ifthenelse{\equal{\jkfside}{oneside}}{

 \ifthenelse{\equal{\jkfhanded}{left}}{
  % 1. Left-handed marker, one-sided printing or e-marking, use oneside and...
  \evensidemargin=\oddsidemargin
  \oddsidemargin=73pt
  \setlength{\marginparwidth}{111pt}
  \setlength{\marginparsep}{-\marginparsep}
  \addtolength{\marginparsep}{-\textwidth}
  \addtolength{\marginparsep}{-\marginparwidth}
 }{
  % 2. Right-handed marker, one-sided printing or e-marking, use oneside.
  \setlength{\marginparwidth}{111pt}
 }

}{
 % 3. Alternating margins, two-sided printing, use twoside.
}

\setlength{\parindent}{0em}
\addtolength{\parskip}{1ex}

% Exam question headings, labels and sensible layout (courtesy of Tom Johnson)
\setlist{parsep=\parskip, listparindent=\parindent}
\newcommand{\examhead}[3]{\section{#1 Paper #2 Question #3}}
\newenvironment{examquestion}[3]{
    \examhead{#1}{#2}{#3}\setlist[enumerate, 1]{label=(\alph*)}\setlist[enumerate, 2]{label=(\roman*)}
    \marginpar{\qrcode{https://www.cl.cam.ac.uk/teaching/exams/pastpapers/y#1p#2q#3.pdf}}
    \marginpar{\footnotesize \url{https://www.cl.cam.ac.uk/teaching/exams/pastpapers/y#1p#2q#3.pdf}}
}{}



\begin{document}

\begin{examquestion}{2019}{7}{8}

In this question you will be asked to reflect on a project you have been
involved in or observed, in which a design evolved, or could have evolved,
through applying a theory of user behaviour. You may refer to a Part IB
group project, practical work from IA/IB Interaction Design or a project
outside the Computer Science Tripos. You are advised to read the whole
question before choosing a project to describe.

\begin{enumerate}

\item Describe the project in one or two sentences.

The IA Interaction Design project: designing a weather app for usage by
Rowers in Cambridge. The app would function as a replacement weather app,
but providing rowers with specific information tailored to rowing -- with a
focus on the CUBC flag (which indicates what skill is required to go out at
this specific time).

\item Describe the intended users of this system, and the benefits that they
would obtain through using this system.

The users of this system would be Cambridge rowers -- they would be able to
quickly check and get tailored information to whether they were allowed to
go rowing. The current website is archaic, and rowers who wake up at 4/5am
aren't mentally prepared to check the weather on such an outdated website.

\item Describe a theory of user-behaviour that is relevant to the project,
explaining why it is relevant.

Goal-oriented search. The way in which users search for information on an
interface is via a goal-oriented search. The stages of a goal-oriented
search are:

\begin{itemize}

\item Goal

The user determines what their goal is.

\item Availability

The user searches for something which they believe may help them find
their goal.

\item Match

The user finds and selects the thing which they believe may help them
achieve their goal.

\item Feedback

The user observes the change caused by their action and determines whether
their goal is achieved and if not, whether the option they chose was
correct or helped achieve their goal.

\end{itemize}

\item Would application of this theory be formative or summative? Explain why.

Application of this theory would be formative. The application of
goal-oriented search in user-centered design is via Cognitive Walkthrough.
The researchers establish a task which a user may want to perform and
attempt to perform it, noting down any issues they encounter.

Since Cognitive Walkthrough finds specific issues with an interface,
researchers can use this information to determine how to improve an
interface. Therefore the evaluation would be formative.

\item How are the opportunities for design evolution different, when either
summative or formative evaluation methods are applied? Your answer should
refer to the roles for divergence and convergence in a design process.

Summative evaluation is an evaluation which tells you whether a product is
good or not. These do not explicitly help discover how to \textit{improve}
a product. For example A/B tests give evidence as to the size and direction
of an effect -- however do not tell us how to increase the effect size.

Formative evaluation tells us how to improve a system. For example, in
Design Ethnography we observe people using the system and query them about
things they do. This gives large amounts of quantitative information which
can be used to improve a product.

Formative evaluation is most useful for the divergent phase in the
iterative design process -- we're given new ideas about how to improve
systems so can add these new systems. While summative evaluation is better
for the convergent phase -- it tells us whether a system is good and which
systems are better -- we can use this to eliminate the worst interfaces.

\item Explain what kind of evidence would be required when applying this
theory in your project, and how you would obtain it, noting whether this
evidence would involve quantitative or qualitative data.

Goal-oriented search provides a concrete theory for how humans attempt to
achieve their goals on an interface with which they are unfamiliar. This
generates qualitative data. This would generate qualitative evidence that
the designers found with their own system.

\item Describe how your project team applied, or could have applied, a
method that would improve the reliability of \textit{quantitative} data.

We could have used Design Ethnography to observe usage of the product and
determine whether real user problems aligned with the problems we observed
during Cognitive Walkthrough. If they did not, we would need to investigate
why.

\item Describe how your project team applied, or could have applied, a
method that would improve the reliability of \textit{qualitative} data.

Get more of it. Using a larger data sample decreases the variance. After
the product was complete, we could have used an A/B test to perform
formative evaluation and determine whether rowers found our app to be
better than the alternatives on the market. A/B tests generate quantitative
data.

The way of increasing reliability of data is to minimise variance. This can
be done by taking a larger sample.

\end{enumerate}

\textbf{What did they mean by ``Theory''\ldots} under my understanding, the
question confused lots of different topics -- theories of user behaviour
with evaluation methods with empirical evaluation methods.

\end{examquestion}

\begin{examquestion}{2019}{7}{9}

\begin{enumerate}

\item If HCI methods were applied to the design of a programming language
and tools, what research questions might be explored, according to the
concerns of first wave, second wave and third wave HCI respectively?

\begin{itemize}

\item First Wave

First-wave HCI was primarily concerned with efficiency of usage by
professionals. In the context of a programming langauge, ``efficiency''
relates to productivity of users and quality of source code generated.

Research questions that might be explored under the concerns of first-wave
HCI may include:

\begin{itemize}

\item How can we design the language to minimise the error rate of
expert programmers

What is the optimal tradeoff between type annotations and type inference?

Should there be restrictions on the format of variable / function names so
they're not confused.

\item How can we design the language to maximise the rate at which
expert programmers can generated code

\item What language features or restrictions maximise the quality of code
that programmers write

\item What langauge features can we include/exclude to maximise the quality
of compiled code.

\end{itemize}

\item Second-wave HCI

Second-wave HCI was characterised by designing usable and learnable systems.
These systems often included personalisation options -- each user would
have their own system.

\begin{itemize}

\item What language features do users find hard-to-learn. Can we make them
easier to learn?

\item How can we allow users to tailor their programming to their own
personal taste without making code unreadable to other programmers.

For example, how much liberty should we give programmers in code style --
how many ways of doing things should there be? Consider list comprehension
-- it tailors programming style but can be unreadable for some people.

\end{itemize}

\item Third-wave

Third-wave HCI was characterised by interfaces designed to maximise user
experience. Due to the arrival of ubiquitous computing, usage of interfaces
became discretionary so interfaces had to become prettier to retain users.
Users now use many interfaces; all serious programmers know many
languages.

Converting these principles to programming languages leads to research
questions such as the following:

\begin{itemize}

\item What language features do users \textit{like} using? Why?

\item How can we write a language that users \textit{understand} and
\textit{enjoy}?

\end{itemize}

\end{itemize}

\iffalse

first wave would ask about efficiency, many-to-one -- lots of people using
the same workstation.

``how can we make this faster to use''

``how can we make the generated code as fast as possible''

building in as many features as possible to make it fast to use -- fast to
compile and make efficient code.

Write a brutally efficient language which compiles to efficient code IE C++



second wave is about realising it fits into peoples lives. they'd ask about
how to make it work for people -- how do these language features fit into
peoples lives.

``how can we increase usability, learnability and personalisation''

design a language with simple syntax, many programming paradigms,
weak[-ish] typing. IE python



third wave would be about making it accessible and making it fun to use --
for example adding features with little practical value but a lot of
entertainment value like C++ templates for metaprogramming

``how can we write a programming langauge people like using''

Write a language people find super cool and genuinely enjoy programming in
ie Haskell

\fi

\item What empirical methods might be appropriate for studying programming
activity from the perspective of each of these three waves.

\begin{itemize}

\item A/B tests

Implementing language features and testing whether they increase or
decrease efficiency or compiled code quality.

\item Design Ethnography

Observing workers to see how they use language features, what they use
right and what they use wrong.

\item Grounded Theory -- looking at data online to see how people feel
about different features, what do they find annoying and what do they like?

\end{itemize}

\iffalse

lab analysis (sit down with a stopwatch and see if its faster)

ethnography

questionnaires -- whatever Alan was talking about when he said ``don't do NLP''

\fi

\item Consider a programming language that has been proposed for the
specification of firework displays. Suggest an analytical method that would
be appropriate for evaluating and refining the usability of this language
and associated tools.

Cognitive Dimensions of Notations

\item In terms of the analytical evaluation method proposed in part $(c)$,
define the target user, the nature of their task and several specific
usability requirements that would result from that task.

The target user is an amateur fireworks enthusiast who is organising a
mid-size display for their local community. They want to make it perfect,
however the display shall take a long time so will not have time to
fine-tune everything.

\begin{itemize}

\item Premature commitment

The language should have negligible premature commitment. Designing
art is an iterative process and involves trial-and-error. Commitment to
launching or styling a display in a particular way would limit the
creativity and greatly increase the time commitment.

\item Progressive evaluation

The display should have very high progressive evaluation. The
display is aesthetic and the user should be able to check that the display
\textit{will} look good before the display itself. So they should be able
to simulate the display. Furthermore, they should be able to start this
simulation from any point in the display (ie they shouldn't have to resume
a 30 minute display from the start every time they modify the final\'e)

\item Viscosity

The display should minimise viscosity. As stated before, art is an
iterative process and so requires frequently editing previous work. This
should be supported and taken into account in the programming language design.

\item Error-proneness

The language should be designed to minimise common errors. For example,
the language should support subtyping such that the user can create
specific types of firework and operate on those rather than specifying the
parameters or each firework (which they're likely to get wrong eventually).

\item Abstraction

The language should have some abstraction -- while the target user is not a
professional programmer (and as such is unlikely to understand complex
typing relations); it should allow some abstraction such that users can
declare their own types of fireworks such that they don't have to
repeatedly create fireworks with the same specification.

IE if the user has 50 fireworks of the same type, they shouldn't have to
have 50 lines of \\
\texttt{Firework f = new Firework(250, 30, "red", 6);}).

\end{itemize}

\item Choose one requirement identified in part $(d)$ and describe in detail
an empirical approach that you would take to evaluating whether this
requirement has been met.

I choose error-proneness.

I would use design ethnography to determine whether this requirement has
been met.

This would involve going and watching people use the programming language,
and observing the types of error they make. If people consistently make the
same types of error then the programming language clearly encourages them
in some way to make this type of error. This would also involve talking to
people to find out the \textit{types} of error they \textit{think} they're
making. If this error differs to the types of errors they're actually
making, we may investigate \textit{why} this is the case. Is this because
they spend so much time thinking about this error that they no longer make
it? Perhaps the errors they actually make are small but the errors they
think they make are significant when they does happen and we should
instead focus efforts in ensuring the language doesn't encourage them.

\end{enumerate}

\end{examquestion}

\begin{examquestion}{2020}{7}{9}

You are designing a new syntax for a programming language like Java, with
the intention of making it more approachable to students by using English
words instead of punctuation symbols.

\begin{enumerate}

\item How does a HCI designer use a theory of human behaviour in their
design practice?

Theories of human behaviour can be used to model how humans will interact
with an interface and thereby predict the efficiency of it and discover
usability problems without the need for expensive human trials.

\item What is an empirical method? Give an example and short description of
that method.

An empirical method is an evaluation method which is based around the
observation and measurement of users. They should be used in the late
stages of development -- observation of users is summative -- we discover
whether a product works or not but aren't often told how to improve it.

Randomised Control Trials (RCTs) are a form of empirical evaluation. They
are used to compare two products (interfaces). I describe A/B tests (a
subtype of RCTs).

\begin{itemize}

\item Researchers first think of a task which they feel well-represents the
question they wish to answer.

\item Next, find an unbiased group of representative users.

\item Randomly partition them into two groups.

\item Each group performs the task on each of the interfaces.

\item Finally, collect data about how long it took users, how many
mistakes they made etc.

\item Significance tests can then be performed to determine whether one of
the interfaces was better than the other.

\end{itemize}

Many large companies use RCTs -- they run hundreds or thousands of
experiments simultaneously and determine whether or not the change has a
positive or negative impact. However, RCTs cannot determine how to make an
observed effect bigger (they are only useful for summative evaluation; not
formative evaluation).

\item Describe in terms of the Cognitive Dimensions two trade-offs involved
between the existing Java syntax and the new proposed syntax.

\begin{itemize}

\item Replacing symbols with words will increase the closeness of mapping
and role expressiveness -- by replacing symbols with english words which
explicitly state the functionality, it becomes very clear to the user what
a particular thing does.

However, this change will increase diffuseness. English words are
inherently longer than single symbols. This will increase the time required
by experts to write basic programs. Similarly, the consistency is increased
-- users who are used to other programming languages would find it hard to
transition since the syntax is so different.

\item the programmers must decide how far to segment symbols -- for example
\texttt{<} has different uses (in generics and bitwise operations). Simply
replacing symbols with their names will have no advantage. Rather, the
designers must decide to replace a symbol with a set of words in given
situations. The decision of how much to segment given symbols leads to some
questions.

Increasing the segmentation of symbols increases cognitive load --
the programmer must remember more notation. However, it will also decrease
the hidden dependencies. If we were to replace brackets with different
words for each application of a bracket, the programmer could immedately
see what a particular bracket related to -- ie $)$ vs
\texttt{close\_function\_call}.

\end{itemize}

\item How does the programming environment relate to this analysis?

Cognitive dimensions of notations are used for analysis of a particular
interface in a \textbf{specific use case} by a \textbf{specific user}. So
if the programming environment is different, the whole analysis would change.

For example, if the use case were for professional programmers, they may
find the increased diffuseness an insurmountable downside.

\item A manager makes a proposal to try different permutations of syntaxes
and measure students' performance using each permutation. How would you
measure and compare the students' performance?

I would suggest running an A/B test. I would first select a number of
representative tasks -- these tasks should reflect what students would
actually have to use. Since the target audience for this language is
students, these could be extracted from real syllabus (ie ticks).

Next, we would give students several lectures in the language (this is for
use by students so we wish for it to be similar to the real use-case)
before setting them a number of tasks. We would then evaluate the time
taken and errors made by the students on each of the syntaxes.

This could be tested for statistical significance and we could use the
syntax on which the students performed the best.

\item What are the likely strengths and weaknesses of the approach in Part
$(e)$?

This is not reflective of the case which students would actually use the
language. Students use languages for prolonged periods of time.

We're unlikely to get statistically significant results. Subtle changes in
syntax are likely to lead to <1\% changes in efficiency. We would need tens
or hundreds of thousands of students to run this trial to achieve any
statistically significant result. There is a very high variance in the
ability of students at programming -- orders of magnitude. This would
dominate the time taken -- some students are better programmers than others.

If the tests were done online, the students may use the internet and the
results would not test their ability -- rather the difficulty of converting
code online into this language.

If the tests were done in-person, the experimenters would affect the test.
Students could ask for hints or help when stuck -- which would skew the
results.

The students may not be fully engaged with the test, creating an
unrealistic environment. Personally, I would not put effort into learning a
language I did not expect would be used anywhere. Many students may share
the same sentiment and therefore the test would be invalid.

\end{enumerate}

\end{examquestion}

\end{document}
