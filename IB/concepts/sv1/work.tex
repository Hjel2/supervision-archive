\newcommand{\svcourse}{CST Part IA: Software Engineering and Security}
\newcommand{\svnumber}{1}
\newcommand{\svvenue}{Microsoft Teams}
\newcommand{\svdate}{2022-05-11}
\newcommand{\svtime}{15:00}
\newcommand{\svuploadkey}{CBd13xmL7PC1zqhNIoLdTiYUBnxZhzRAtJxv/ytRdM1r7qIfwMsxeVwM/pPcIo8l}

\newcommand{\svrname}{Dr Sam Ainsworth}
\newcommand{\jkfside}{oneside}
\newcommand{\jkfhanded}{yes}

\newcommand{\studentname}{Harry Langford}
\newcommand{\studentemail}{hjel2@cam.ac.uk}


\documentclass[10pt,\jkfside,a4paper]{article}

\usepackage{mathtools}

% DO NOT add \usepackage commands here.  Place any custom commands
% into your SV work files.  Anything in the template directory is
% likely to be overwritten!

\usepackage{fancyhdr}

\usepackage{lastpage}       % ``n of m'' page numbering
\usepackage{lscape}         % Makes landscape easier

\usepackage{verbatim}       % Verbatim blocks
\usepackage{listings}       % Source code listings
\usepackage{graphicx}
\usepackage{float}
\usepackage{epsfig}         % Embed encapsulated postscript
\usepackage{array}          % Array environment
\usepackage{qrcode}         % QR codes
\usepackage{enumitem}       % Required by Tom Johnson's exam question header

\usepackage{hhline}         % Horizontal lines in tables
\usepackage{siunitx}        % Correct spacing of units
\usepackage{amsmath}        % American Mathematical Society
\usepackage{amssymb}        % Maths symbols
\usepackage{amsthm}         % Theorems

\usepackage{ifthen}         % Conditional processing in tex

\usepackage[top=3cm,
            bottom=3cm,
            inner=2cm,
            outer=5cm]{geometry}

% PDF metadata + URL formatting
\usepackage[
            pdfauthor={\studentname},
            pdftitle={\svcourse, SV \svnumber},
            pdfsubject={},
            pdfkeywords={9d2547b00aba40b58fa0378774f72ee6},
            pdfproducer={},
            pdfcreator={},
            hidelinks]{hyperref}

\renewcommand{\headrulewidth}{0.4pt}
\renewcommand{\footrulewidth}{0.4pt}
\fancyheadoffset[LO,LE,RO,RE]{0pt}
\fancyfootoffset[LO,LE,RO,RE]{0pt}
\pagestyle{fancy}
\fancyhead{}
\fancyhead[LO,RE]{{\bfseries \studentname}\\\studentemail}
\fancyhead[RO,LE]{{\bfseries \svcourse, SV~\svnumber}\\\svdate\ \svtime, \svvenue}
\fancyfoot{}
\fancyfoot[LO,RE]{For: \svrname}
\fancyfoot[RO,LE]{\today\hspace{1cm}\thepage\ / \pageref{LastPage}}
\fancyfoot[C]{\qrcode[height=0.8cm]{\svuploadkey}}
\setlength{\headheight}{22.55pt}


\ifthenelse{\equal{\jkfside}{oneside}}{

 \ifthenelse{\equal{\jkfhanded}{left}}{
  % 1. Left-handed marker, one-sided printing or e-marking, use oneside and...
  \evensidemargin=\oddsidemargin
  \oddsidemargin=73pt
  \setlength{\marginparwidth}{111pt}
  \setlength{\marginparsep}{-\marginparsep}
  \addtolength{\marginparsep}{-\textwidth}
  \addtolength{\marginparsep}{-\marginparwidth}
 }{
  % 2. Right-handed marker, one-sided printing or e-marking, use oneside.
  \setlength{\marginparwidth}{111pt}
 }

}{
 % 3. Alternating margins, two-sided printing, use twoside.
}


\setlength{\parindent}{0em}
\addtolength{\parskip}{1ex}

% Exam question headings, labels and sensible layout (courtesy of Tom Johnson)
\setlist{parsep=\parskip, listparindent=\parindent}
\newcommand{\examhead}[3]{\section{#1 Paper #2 Question #3}}
\newenvironment{examquestion}[3]{
\examhead{#1}{#2}{#3}\setlist[enumerate, 1]{label=(\alph*)}\setlist[enumerate, 2]{label=(\roman*)}
\marginpar{\href{https://www.cl.cam.ac.uk/teaching/exams/pastpapers/y#1p#2q#3.pdf}{\qrcode{https://www.cl.cam.ac.uk/teaching/exams/pastpapers/y#1p#2q#3.pdf}}}
\marginpar{\footnotesize \href{https://www.cl.cam.ac.uk/teaching/exams/pastpapers/y#1p#2q#3.pdf}{https://www.cl.cam.ac.uk/\\teaching/exams/pastpapers/\\y#1p#2q#3.pdf}}
}{}


\begin{document}

\section*{Supervision 1}

\begin{enumerate}

\item An author writes:

``Most successful language design efforts share three important
characteristics:

\begin{itemize}

\item Motivating Application: The language was designed so that a specific
kind of program could be written more easily.

\item \dots

\item \dots

\end{itemize}

I discuss the merits and/or shortcomings of the statement that
successful language design efforts share a Motivating Application.

While many Theoreticians would like for programming languages to be adopted
due to beauty and elegance, it's undeniable that any successful programming
language must (by definition) have \textit{people who use it}. Programmers
will only adopt a programming language if there is a reason to do so. Why
would I learn a new language and swap my codebase into a new language
without a convincing reason. Any successful language must do something
\textit{so much better} than any existing language such that programmers are
willing to put money into transferring.

For example, Rust is an up-and-coming language which attempts to exploit a
``gap in the market'' of concurrent, high-performance, high-level programming
languages. For the last 10 years, CPUs have had multicore and multithreaded
parallelism which has been largely unexploited by modern programming
languages. The mainstream languages which are able to properly exploit
parallelism require explicit programmer input and management. This is
analogous to manging which variables are stored in registers; highly
impractical and it's impossible to determine the optimal choices without
thorough investigation. Rust attempts to bridge this gap by providing memory
safety, concurrency and performance by default. As a result, it has been
gaining popularity despite having no strong theoretical underpinnings.

Python is an example of a language with a strong motivating application; but
no theoretical underpinnings or abstract machine. Python was designed as a
general purpose programming language which was as logically close to human
intuition as possible. It was initially targeted at mathematicians and data
scientists. However, the simplicity and total abstraction from implementation
(even at the cost of efficiency) was found to be popular so the target
audience was broadened. Despite having no abstract machine and very few
theoretical underpinnings, Python has become one of the most popular
languages.

Javascript is the classical example of a language with \textit{only} a
motivating application: writing web programs. Previously, programs on the
web were only supported through Java applets. These had security holes and
major browsers were hesitant to support them. Javascript was famously
designed and written in 10 days; and its success has been to great
frustration of theoretical computer scientists around the world. It has a
weak type system and ugly syntax. However, at the time it was created there
was a great requirement for programming on the web -- a requirement which
Javascript fit.

\item Give two operational semantics for the following simple language, one
with dynamic scoping, and one with lexical scoping:

\[
E \Coloneqq x \ | \ \mathbf{fn}\ x \rightarrow E \ | \ E E
\]

You may assume an encoding for numbers, booleans, and syntax for
\texttt{let...in....end} and \texttt{if...then...else} without providing one.

For both, I define a value, $v$ as follows:
\[
v \Coloneqq \mathbf{int}\ |\ \mathbf{bool}\ |\
\mathbf{fn}\ x \to E \text{ if } \mathbf{fv}(E) \subseteq \{x\}
\]

\begin{itemize}

\newcommand{\env}{\text{Env}}

\item Lexical (static) Scoping

\begin{gather*}
\dfrac{
E_1 \to E_1'
}{
E_1\ E_2 \to E_1'\ E_2
}
\qquad
\dfrac{
E_2 \to E_2'
}{
(\mathbf{fn}\ x \to E)E_2
\to
(\mathbf{fn}\ x \to E)E_2'
}\\\\
\dfrac{
}{
(\mathbf{fn}\ x \to E)v \to \{v/x\}E
}\\\\
\dfrac{}{
\{v/x\}x = v
}
\qquad
\dfrac{
}{
\{v/x\}y = y
}\\\\
\dfrac{
}{
\{v/x\}(\mathbf{fn}\ y \to E) =
\mathbf{fn}\ y \to \{v/x\}E
}\\\\
\dfrac{
}{
\{v/x\}(\mathbf{fn}\ x \to E) =
\mathbf{fn}\ x \to E
}\\\\
\dfrac{}{
\{v/x\}(E\ E') = \{v/x\}E\ \{v/x\}E'
}
\end{gather*}

\item Dynamic Scoping

\begin{gather*}
\dfrac{}{
\langle
(\mathbf{fn}\ x \to E)v, \text{Env}
\rangle
\to
\langle
E, (x, v)\dblcolon\text{Env}
\rangle
}\\\\
\dfrac{
\text{Env} = (y_1, v_1)
\dblcolon \dots (y_n, v_n) \dblcolon (x, v) \dblcolon \text{Env}'
\qquad
x \notin \{y_1, \dots, y_n\}
}{
\langle
x, \text{Env}
\rangle
\to
\langle
v, \text{Env}
\rangle
}\\\\
\dfrac{
\langle
E_1, \text{Env}
\rangle
\to
\langle
E_1', \text{Env}'
\rangle
}{
\langle
E_1\ v, \text{Env}
\rangle
\to
\langle
E_1'\ v, \text{Env}'
\rangle
}\\\\
\dfrac{
\langle
E_2, \text{Env}
\rangle
\to^*
\langle
v, \text{Env}'
\rangle
}{
\langle
E_1\ E_2, \text{Env}
\rangle
\to^*
\langle
E_1\ v, \text{Env}
\rangle
}
\end{gather*}

\textbf{How does typing work with dynamic scoping?}

\end{itemize}

\item Outline the key features that a language must have to be called
object-oriented. Further, briefly discuss to what extent the programming
languages Simula, Smalltalk, C++, and Java have them.

The four main language concepts for object-oriented languages are:
\begin{itemize}

\item Dynamic lookup

Dynamic lookup means that which method to evaluate is selected dynamically
at runtime based on the actual type of an object rather than its static
compile-time type.

\item Abstraction

Abstraction means that programs are written in such a way that the
programmer interacts with an abstract interface whose
\textit{implementation details} are hidden from the programmer. This allows
the user to use the interface without concern for the underlying
implementation, and allows the creator of the class to change the
implementation without affecting usage.

\item Subtyping

A subtype relation $T <: T'$ is a relation which specifies a hierarchy of
types such that a subtype can be passed anywhere where a supertype is
expected. This means the same code (i.e functions) can be reused with many
different types.

\item Inheritance

There are two types of inheritance: type inheritance and code inheritance.
Code inheritance is the ability for a new type to inherit methods and
implementation details from another class. Type inheritance is the ability
for a new class to be a subtype of an existing class.

\end{itemize}

I now discuss the support which each language provides for the various
features of object-oriented languages:
\begin{itemize}

\item Simula

As the original \textit{object-oriented language}, Simula has support for
most of the features. As is to be expected, not all the main features were
deemed useful or thought of. Therefore, Simula does not support some
critical features.

Simula has dynamic lookup, subtyping and inheritance. However, abstraction
is not supported. Any code can access all attributes and functions of any
objects.

\item Smalltalk

Smalltalk was a language designed around Simula which attempted to overcome
its shortcomings. Smalltalk had support for all four of the main
features of object-oriented languages.

\item C++

C++ has optional dynamic lookup in the form of virtual functions. This gives
the programmer a choice between highly efficient implementations and
complicated language features (which require a runtime lookup in a vtable to
establish which implementation should be called). In my personal opinion, this
is the best of both worlds (although if I was designing a language,
lookups would be dynamic by default).

Like almost all modern programming languages, C++ does support abstraction
-- objects can have private attributes and present an interface which
completely abstracts away from all implementation details. C++ has a unique
feature which allows violation of abstraction -- the \texttt{friend} keyword
allows a specified class or function to see the private and protected
variables.

C++ has support for complex subtype relationships which can form a directed
acyclic graph -- in contrast to many other languages which restrict
subtyping to a tree.

C++ supports multiple inheritance -- a powerful (if confusing) form of
inheritance at its purest.

\item Java

Java has compulsory dynamic lookup. Java supports a restricted form of
inheritance and subtyping. Java makes no distinction between code
inheritance and type inheritance. Objects may only directly inherit from a
single class. However, inheritance (both type and code) is transitive. This
means java inheritance hierarchies form a tree. This is sufficient for
most applications and makes code easier to understand.

Java has strong support for abstraction which is forced on the programmer --
unlike C++ Java has no concept of a friend class.

\end{itemize}

\item Define the following parameter-passing mechanisms: pass-by-value,
pass-by-reference, pass-by-value/result, and pass-by-name. Briefly comment
on their merits and drawbacks.

\begin{itemize}

\item Pass-by-value

Parameters passed into functions are fresh copies of the parameter the
callee passed. Any changes to the parameter inside the function will not be
reflected outside the function.

This is easy to reason about -- although when passing large structures can
become inefficient.

\item Pass-by-reference

Aliases to parameters are passed into the function. Any changes to the
parameter inside the function are immediately reflected outside the function.

This means any variables and function calls only require copies of pointers
-- leading to greater efficiency. However, this can be illogical for the
programmer. Call by reference requires a pointer indirection per parameter
access and has a lower cache hit rate than other methods -- since
parameters are not all stored in the stack frame of the function being called.

\item Pass-by-value/result

In pass-by value/result, a copy of the parameter is passed into the function.
The function then copies this value and operates on it locally. When the
function returns, this result is copied into the parameter.

Pass-by-value/result will not affect variables if it crashes. This means
erroneous programs will not corrupt parameters as happens in
call-by-reference.

However, pass-by-value requires two copies of every parameter. This makes it
very inefficient to pass large objects.

\item Pass-by-name

Parameters are passed into arguments as they appear at the calling site.

Parameters are only evaluated if they are actually used.

However, they may be evaluated many times. This can be inefficient and lead
to confusing semantics if evaluation of parameters has side effects.

\end{itemize}

\item What is aliasing in the context of programming languages? Explain the
contexts in which it arises and provide examples of the phenomenon. In what
kinds of languages is it generally considered bad, and in what kinds of
languages is it considered useful?

Aliasing is where multiple variables refer to the same object. Thus editing
one of them will affect the value of the other. It's considered very useful
in pure functional programming languages -- objects are immutable so if
multiple variables refer to the same object then memory is saved.

However, aliasing is often bad in imperative programming languages since
it leads to confusing semantics.

\item Recall that Algol 60 has a primitive static type system. In 
particular, in Algol 60, the type of a procedure parameter does not include
the types of its parameters. Thus, for instance, in the Algol 60 code
\begin{lstlisting}[language=Algol]
procedure P(procedure F)
begin F(true) end ;
\end{lstlisting}
the types of the parameters of the procedure parameter \texttt{F} are not
specified. Explain why this piece of code is statically type correct.
Explain also why a call \texttt{P(Q)} may produce a run-time type error,
and exemplify your answer by exhibiting a declaration for \texttt{Q} with
this effect. Why does this problem not arise in Standard ML?

In Algol 60, the type of a parameter is dependent only on its return type.
This means that the arguments to parameters are not statically type-checked.
As a result, many programs which statically type-check will fail at runtime.
For example, it would pass type-checking to pass a function \texttt{Q} with
type \texttt{int $\to$ int} to $P$.

\item Consider the Simula declarations
\begin{lstlisting}[language=Simula]
CLASS A
A CLASS B;
\end{lstlisting}
which has the effect of producing the subtype relation \texttt{B<:A}, and
\begin{lstlisting}[language=Simula]
REF(A) a
REF(B) b;
\end{lstlisting}
Recall that Simula uses the semantically incorrect principle that if
\texttt{B<:A} then \texttt{REF(B)<:REF(A)} and consider now the following
Simula code
\begin{lstlisting}[language=Simula]
PROCEDURE ASSIGNa(REF(A) x)
BEGIN x :- a END ;
ASSIGNa(b);
\end{lstlisting}
Does it statically type check? If so, will it cause a run-time type error?
Justify your answers.

This program \textbf{does} statically type-check. However, it \textbf{will}
cause a runtime type error.

The program determines that \texttt{REF(B) <: REF(A)} -- therefore passing
an object of type \texttt{REF(B)} to the procedure \texttt{ASSIGNa} is legal
and passes static type-checking.

At runtime, \texttt{ASSIGNa} takes \texttt{b} as argument then tries to
assign \texttt{a} in a location expecting type \texttt{REF(B)}. This will
(ideally) fail dynamic type-checking or cause an error when \texttt{b} is
read from.

\item What is the problem with making the tag optional in Pascal's variant
records?

Variant types are disjoint sums. However, the tag is optional. If the tag is
not present then runtime checks aren't possible -- the record has no type
information so the program cannot check whether the value being extracted is
of the same type as the value which was inserted.

\item What is the type of the expression
\begin{lstlisting}[language=ML]
fn a => fn b => fn c => (a b) (b c)
\end{lstlisting}
given by the SML interpreter? Explain how this is inferred.

\[
((\alpha \to \beta) \to \beta \to \gamma) \to (\alpha \to \beta) \to \alpha
\to \gamma
\]
The SML interpreter will unify types, inferring function types however
keeping everything as general as possible.

Initially, it will infer the types:
\[
\alpha \to \beta \to \gamma \to \delta
\]
Next, the  $\alpha$ is unified with $\beta \to \epsilon$
\[
(\beta \to \epsilon) \to \beta \to \gamma \to \delta
\]
Next, $\beta$ is unified with $\gamma \to \zeta$:
\[
((\gamma \to \zeta) \to \epsilon) \to (\gamma \to \zeta) \to \gamma \to \delta
\]
Next, $\epsilon$ is unified with $\zeta \to \eta$
\[
((\gamma \to \zeta) \to (\zeta \to \eta)) \to (\gamma \to \zeta) \to \gamma
\to \delta
\]
Next $\eta$ is unified with $\delta$:
\[
((\gamma \to \zeta) \to (\zeta \to \eta)) \to (\gamma \to \zeta) \to \gamma
\to \eta
\]
Now, I rewrite to $\alpha$-equivalence for clarity:
\[
((\alpha \to \beta) \to (\beta \to \gamma)) \to (\alpha \to \beta) \to \alpha
\to \gamma
\]
Because of the order of precedence, we can remove some bracketing:
\[
((\alpha \to \beta) \to \beta \to \gamma) \to (\alpha \to \beta) \to \alpha
\to \gamma
\]

\item For each of the following ML declarations, either justify their
ability to be soundly used or give a program using them which would violate
type safety:

\begin{itemize}

\item
\begin{lstlisting}[language=ML]
exception poly of ’a;
\end{lstlisting}

This cannot be soundly used. For example, the following program would break
type safety:
\begin{lstlisting}[language=ML]
try (
	raise poly true
) with poly x -> x + 1;
\end{lstlisting}

\item
\begin{lstlisting}[language=ML]
val ml = ref [];
\end{lstlisting}

This cannot be soundly used. It's type is not known when it is instantiated.
Furthermore, it cannot be used polymorphically since after the first usage
we may be inserting multiple types into the same list. OCaml bypasses this by
``weak'' types -- types which are not yet known but will become bound after
their first usage.

\end{itemize}

\end{enumerate}

\end{document}
