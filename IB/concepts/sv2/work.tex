\newcommand{\svcourse}{CST Part IA: Software Engineering and Security}
\newcommand{\svnumber}{1}
\newcommand{\svvenue}{Microsoft Teams}
\newcommand{\svdate}{2022-05-11}
\newcommand{\svtime}{15:00}
\newcommand{\svuploadkey}{CBd13xmL7PC1zqhNIoLdTiYUBnxZhzRAtJxv/ytRdM1r7qIfwMsxeVwM/pPcIo8l}

\newcommand{\svrname}{Dr Sam Ainsworth}
\newcommand{\jkfside}{oneside}
\newcommand{\jkfhanded}{yes}

\newcommand{\studentname}{Harry Langford}
\newcommand{\studentemail}{hjel2@cam.ac.uk}


\documentclass[10pt,\jkfside,a4paper]{article}

% DO NOT add \usepackage commands here.  Place any custom commands
% into your SV work files.  Anything in the template directory is
% likely to be overwritten!

\usepackage{fancyhdr}

\usepackage{lastpage}       % ``n of m'' page numbering
\usepackage{lscape}         % Makes landscape easier

\usepackage{verbatim}       % Verbatim blocks
\usepackage{listings}       % Source code listings
\usepackage{graphicx}
\usepackage{float}
\usepackage{epsfig}         % Embed encapsulated postscript
\usepackage{array}          % Array environment
\usepackage{qrcode}         % QR codes
\usepackage{enumitem}       % Required by Tom Johnson's exam question header

\usepackage{hhline}         % Horizontal lines in tables
\usepackage{siunitx}        % Correct spacing of units
\usepackage{amsmath}        % American Mathematical Society
\usepackage{amssymb}        % Maths symbols
\usepackage{amsthm}         % Theorems

\usepackage{ifthen}         % Conditional processing in tex

\usepackage[top=3cm,
            bottom=3cm,
            inner=2cm,
            outer=5cm]{geometry}

% PDF metadata + URL formatting
\usepackage[
            pdfauthor={\studentname},
            pdftitle={\svcourse, SV \svnumber},
            pdfsubject={},
            pdfkeywords={9d2547b00aba40b58fa0378774f72ee6},
            pdfproducer={},
            pdfcreator={},
            hidelinks]{hyperref}

\renewcommand{\headrulewidth}{0.4pt}
\renewcommand{\footrulewidth}{0.4pt}
\fancyheadoffset[LO,LE,RO,RE]{0pt}
\fancyfootoffset[LO,LE,RO,RE]{0pt}
\pagestyle{fancy}
\fancyhead{}
\fancyhead[LO,RE]{{\bfseries \studentname}\\\studentemail}
\fancyhead[RO,LE]{{\bfseries \svcourse, SV~\svnumber}\\\svdate\ \svtime, \svvenue}
\fancyfoot{}
\fancyfoot[LO,RE]{For: \svrname}
\fancyfoot[RO,LE]{\today\hspace{1cm}\thepage\ / \pageref{LastPage}}
\fancyfoot[C]{\qrcode[height=0.8cm]{\svuploadkey}}
\setlength{\headheight}{22.55pt}


\ifthenelse{\equal{\jkfside}{oneside}}{

 \ifthenelse{\equal{\jkfhanded}{left}}{
  % 1. Left-handed marker, one-sided printing or e-marking, use oneside and...
  \evensidemargin=\oddsidemargin
  \oddsidemargin=73pt
  \setlength{\marginparwidth}{111pt}
  \setlength{\marginparsep}{-\marginparsep}
  \addtolength{\marginparsep}{-\textwidth}
  \addtolength{\marginparsep}{-\marginparwidth}
 }{
  % 2. Right-handed marker, one-sided printing or e-marking, use oneside.
  \setlength{\marginparwidth}{111pt}
 }

}{
 % 3. Alternating margins, two-sided printing, use twoside.
}


\setlength{\parindent}{0em}
\addtolength{\parskip}{1ex}

% Exam question headings, labels and sensible layout (courtesy of Tom Johnson)
\setlist{parsep=\parskip, listparindent=\parindent}
\newcommand{\examhead}[3]{\section{#1 Paper #2 Question #3}}
\newenvironment{examquestion}[3]{
\examhead{#1}{#2}{#3}\setlist[enumerate, 1]{label=(\alph*)}\setlist[enumerate, 2]{label=(\roman*)}
\marginpar{\href{https://www.cl.cam.ac.uk/teaching/exams/pastpapers/y#1p#2q#3.pdf}{\qrcode{https://www.cl.cam.ac.uk/teaching/exams/pastpapers/y#1p#2q#3.pdf}}}
\marginpar{\footnotesize \href{https://www.cl.cam.ac.uk/teaching/exams/pastpapers/y#1p#2q#3.pdf}{https://www.cl.cam.ac.uk/\\teaching/exams/pastpapers/\\y#1p#2q#3.pdf}}
}{}


\begin{document}

\begin{enumerate}

\item You manage two junior programmers and overhear the following
conversation:

A: ``I don't know why anyone needs a language other than Java, it provides
clean thread-based parallel programming''

B: ``Maybe, but I write my parallel programs in a functional programming
language because they are then embarrassingly parallel''

Discuss the correctness of these statements and the extent to which they
cover the range of languages for parallel programming.

Java only provides green-threads. While these provide concurrency, they do
not provide parallelism (user-level threading). Programs written in a
functional programming language \textit{are} embarassingly parallel. However,
this sets the limits of parallelism -- and modern functional languages are
not even close to these limits. In many cases, running small functions on
different cores or threads is more overhead than running it serially.
Therefore, many implementations do not exploit this massive possibility for
parallelism.

OCaml 5 has added multicore support by default. However, the speedup
observed when compared to OCaml 4 is small. While OCaml 5 \textit{is} faster
on many workloads, the difference in speed is not significant on most
workloads.

\item What is the difference between internal and external iteration?

External iteration is normal iterating over a loop. Internal iteration is
nested loops where each internal loop is executed on a different core. For
example:

External Iteration:
\begin{lstlisting}[language=C]
for (int i = 0; i < n; i++){
	// body
}
\end{lstlisting}

Internal Iteration:
\begin{lstlisting}[language=C]
for (int core = 0; core < NUMCORES; core++){
	for (int i = 0; i < n / NUMCORES; i++){
		// body
	}
}
\end{lstlisting}

\item Scripting languages and dynamically typed languages are identical;
discuss.

Scripting Languages are a type of languages which are intended to be used
for writing short programs which are used once (or a small number of times)
and subsequently thrown away. While dynamically typed languages are
languages which perform tyepchecking at runtime.

Scripting languages are often dynamically typed, there are dynamically
typed languages (such as python) which are not scripting languages.

\item Discuss the notion of ``class'' in relation to JavaScript.

% TODO

\item Explain what is meant by a monad in a programming language, giving
the two fundamental operations of a monad along with their types.

% TODO

\item Consider the use of a monad for input-output. For the purposes of this
question, take the IO monad as including two operations \texttt{readint} and
\texttt{writeint} which respectively read integers from \texttt{stdin} and
write integers to \texttt{stdout}. Give the types of these operators.

\begin{align}
\texttt{readint}:& \texttt{int}\ \texttt{IO}\\
\texttt{writeint}:& \texttt{int} \to \texttt{unit}\ \texttt{IO}
\end{align}

\item Assume \texttt{MLreadint} and \texttt{MLwriteint} are primitives with
side effects for input/output and consider the ML expression \texttt{add1}
of type \texttt{int}:

\begin{lstlisting}[language=ml]
let val x = MLreadint() in MLwriteint(x+1); x end
\end{lstlisting}

Give an equivalent expression which uses the IO monad instead of
side-effects, and state its type.

% TODO

Give a function \texttt{run2diff} which can be applied to the previous
answer. When so applied it should give a value in the IO monad which
corresponds to ML code that runs add1 twice and returns the difference
between the values read.

% TODO

\item State what happens when attempting to compile and execute the
following Java fragment (explaining the origin of any error messages or
exceptions which might arise).

\begin{lstlisting}[language=java]
Object n = new Integer(42), o = new String("Whoops");
Object [] v;
Integer [] w = new Integer[10];
v = w;
v[4] = n;
v[5] = o;
\end{lstlisting}

% TODO

\item Consider the Java code:
\begin{lstlisting}[language=java]
Object n = new Integer(42);
ArrayList<? extends Object> v1;
ArrayList<Object> v2;
ArrayList<Integer> w = new ArrayList<>(10);
\end{lstlisting}
Explain any differences in behaviour between assignments \texttt{v1 = w} and
\texttt{v2 = w} and also between method calls \texttt{v1.set(4,n)} and
\texttt{v2.set(4,n)}.

% TODO

\item In the programming language Scala, a generic class like the following
one
\begin{lstlisting}[language=scala]
abstract class Stack[A] {
  def push(x : A) : Stack[A];
  def top : A;
  def pop : Stack[A];
}
\end{lstlisting}

is non-variant by default. Why? Modify it to make it co-variant.

% TODO

\item Consider the declarations
\begin{lstlisting}[language=ml]
structure Z = struct   type t = int; val z = 0 end;
structure A = Z :  sig type t;       val z : t end;
structure B = Z :> sig type t = int; val z : t end;
structure C = Z :> sig type t;       val z : t end;
\end{lstlisting}
in the SML Modules language. Explain the behaviour of the SML interpreter
on inputting each of the expressions.

% TODO

\begin{lstlisting}[language=ml]
Z.z = A.z;
Z.z = B.z;
Z.z = C.z;
\end{lstlisting}

\item What are the similarities and differences between Haskell type
classes, overloading, and dynamic dispatch used in object-oriented languages?

% TODO

\end{enumerate}

\end{document}
