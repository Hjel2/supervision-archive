\newcommand{\svcourse}{CST Part IA: Introduction to Probability}
\newcommand{\svnumber}{1}
\newcommand{\svvenue}{Churchill, Room TBD}
\newcommand{\svdate}{2022-05-14}
\newcommand{\svtime}{11:00}
\newcommand{\svuploadkey}{PO5ogKIM8KQA22FZS8IAf8gxA8XKi19jxIBVHIfFZ+3GCBXuNUXS9lVN6bNYjxM/}

\newcommand{\svrname}{Mr Matthew Ireland}
\newcommand{\jkfside}{twoside}
\newcommand{\jkfhanded}{right}

\newcommand{\studentname}{Harry Langford}
\newcommand{\studentemail}{hjel2@cam.ac.uk}


\documentclass[10pt,\jkfside,a4paper]{article}

% DO NOT add \usepackage commands here.  Place any custom commands
% into your SV work files.  Anything in the template directory is
% likely to be overwritten!

\usepackage{fancyhdr}

\usepackage{lastpage}       % ``n of m'' page numbering
\usepackage{lscape}         % Makes landscape easier

\usepackage{verbatim}       % Verbatim blocks
\usepackage{epsfig}         % Embed encapsulated postscript
\usepackage{array}          % Array environment
\usepackage[nolinks]{qrcode}         % QR codes
\usepackage{enumitem}       % Required by Tom Johnson's exam question header

\usepackage{hhline}         % Horizontal lines in tables
\usepackage{siunitx}        % Correct spacing of units
\usepackage{amsmath}        % American Mathematical Society
\usepackage{amssymb}        % Maths symbols
\usepackage{amsthm}         % Theorems

\usepackage{ifthen}         % Conditional processing in tex

\usepackage[top=3cm,
            bottom=3cm,
            inner=2cm,
            outer=5cm]{geometry}

% PDF metadata + URL formatting
\usepackage[
            pdfauthor={\studentname},
            pdftitle={\svcourse, SV \svnumber},
            pdfsubject={},
            pdfkeywords={9d2547b00aba40b58fa0378774f72ee6},
            pdfproducer={},
            pdfcreator={},
            hidelinks]{hyperref}

\renewcommand{\headrulewidth}{0.4pt}
\renewcommand{\footrulewidth}{0.4pt}
\fancyheadoffset[LO,LE,RO,RE]{0pt}
\fancyfootoffset[LO,LE,RO,RE]{0pt}
\pagestyle{fancy}
\fancyhead{}
\fancyhead[LO,RE]{{\bfseries \studentname}\\\studentemail}
\fancyhead[RO,LE]{{\bfseries \svcourse, SV~\svnumber}\\\svdate\ \svtime, \svvenue}
\fancyfoot{}
\fancyfoot[LO,RE]{For: \svrname}
\fancyfoot[RO,LE]{\today\hspace{1cm}\thepage\ / \pageref{LastPage}}
\fancyfoot[C]{\qrcode[height=0.8cm]{\svuploadkey}}
\setlength{\headheight}{22.55pt}

\ifthenelse{\equal{\jkfside}{oneside}}{

 \ifthenelse{\equal{\jkfhanded}{left}}{
  % 1. Left-handed marker, one-sided printing or e-marking, use oneside and...
  \evensidemargin=\oddsidemargin
  \oddsidemargin=73pt
  \setlength{\marginparwidth}{111pt}
  \setlength{\marginparsep}{-\marginparsep}
  \addtolength{\marginparsep}{-\textwidth}
  \addtolength{\marginparsep}{-\marginparwidth}
 }{
  % 2. Right-handed marker, one-sided printing or e-marking, use oneside.
  \setlength{\marginparwidth}{111pt}
 }

}{
 % 3. Alternating margins, two-sided printing, use twoside.
}

\setlength{\parindent}{0em}
\addtolength{\parskip}{1ex}

% Exam question headings, labels and sensible layout (courtesy of Tom Johnson)
\setlist{parsep=\parskip, listparindent=\parindent}
\newcommand{\examhead}[3]{\section{#1 Paper #2 Question #3}}
\newenvironment{examquestion}[3]{
    \examhead{#1}{#2}{#3}\setlist[enumerate, 1]{label=(\alph*)}\setlist[enumerate, 2]{label=(\roman*)}
    \marginpar{\qrcode{https://www.cl.cam.ac.uk/teaching/exams/pastpapers/y#1p#2q#3.pdf}}
    \marginpar{\footnotesize \url{https://www.cl.cam.ac.uk/teaching/exams/pastpapers/y#1p#2q#3.pdf}}
}{}



\usepackage{ebproof}
\usepackage{relsize}

\begin{document}

\section{Question 1}
\begin{enumerate}[label=(\alph*)]

\item Let $M$ be the $\lambda$-term $\lambda xy.x(\lambda z.zu)y$. What is
the $\beta$-normal form of the term $N = M(\lambda vw.v(wb))(\lambda xy.yaz)$ ?

\begin{align*}
N =& M(\lambda vw.v(wb))(\lambda xy.yaz) \\
=& (\lambda xy.x(\lambda z.zu)y)(\lambda vw.v(wb))(\lambda xy.yaz) \\
\twoheadrightarrow& (\lambda y.(\lambda vw.v(wb))(\lambda z.zu)y)(\lambda
xy.yaz) \\
\twoheadrightarrow& (\lambda y.(\lambda w.v(wb))[\lambda z.zu/v]y)
(\lambda xy.yaz) \\
=& (\lambda y.(\lambda w.(\lambda z.zu)(wb))y)(\lambda xy.yaz) \\
\twoheadrightarrow& (\lambda y.(\lambda w.wbu)y)(\lambda xy.yaz) \\
\twoheadrightarrow& (\lambda w.wbu)(\lambda xy.yaz) \\
\twoheadrightarrow& (\lambda xy.yaz)bu \\
\twoheadrightarrow& (\lambda y. yaz)u \\
\twoheadrightarrow& uaz \\
\not\twoheadrightarrow&
\end{align*}
So the $\beta-NF$ of $N$ is $uaz$
\item Apply the simultaneous substitution $\sigma = [x/y, (\lambda xy.zy)/u]$
to $M$ and $N$ and find the $\beta$-normal form of $N[\sigma]$

\begin{align*}
M[\sigma]
&= (\lambda xy.x(\lambda z.zu)y)[x/y, (\lambda xy.zy)/u] \\
&= (\lambda xy.x(\lambda w.wu)y)[(\lambda xy.zy)/u] \\
&= (\lambda xy.x(\lambda w.w(\lambda xy.zy))y)\\
N[\sigma]
&= uaz[x/y, (\lambda xy.zy)/u] \\
&= (\lambda xy.zy)az \\
&\twoheadrightarrow (zy)[a/x. z/y] \\
&= zz
\end{align*}

\item Give 2 terms $\alpha$-equivalent to $M$. Give 2 other terms
$\beta$-equivalent to $M$.

\begin{align*}
M &
=_{\alpha} \lambda yz.y(\lambda z.zu)z
=_{\alpha} \lambda xy.x(\lambda x. xu)y \\
M &
=_{\beta} (\lambda fxy. xfy)(\lambda z. zu)
=_{\beta} (\lambda x. x)(\lambda xy. x(\lambda z. zu)y)
\end{align*}

\item We define $\eta$-equivalence as: $M=_\eta \lambda x.Mx$ for any
$\lambda$-term $M$. Give a shorter and a longer term $\eta$-equivalent to M\@.
\begin{align*}
M &=_{\eta} M \\
M &=_{\eta} \lambda xyz. ((M x)y)z
\end{align*}

\item What is the use of $\eta$-equivalence in functional programming?

This enables compiler optimisations. At any point, we can replace an
expression with any expression which is $\eta$-equivalent to it.

\end{enumerate}

\section{Question 2}

\begin{enumerate}[label=(\alph*)]

\item Give a complete proof of the correctness of Church addition from
Slide 119.
\[
\mathit{Plus}(m, n) =_\beta \langle m + n \rangle
\]

\newcommand{\un}[1]{\ensuremath{\underline{#1}}}

We are required to prove that for all church numerals $\un{n}$, $\un{m}$
the following holds:
\[
\un{m + n} =_\beta \lambda\ f\ x. \un{m}\ f\ (\un{n}\ f\ x)
\]
I will prove this by assuming arbitrary $n$ and perform mathematical
induction over $m$. Since the proof assumes $n$ is arbitrary, it holds for
all $n$. By induction over $m$, we will prove it holds for all $m$.

\begin{itemize}

\item Case $m = 0$
\begin{align*}
& \lambda\ f\ x.\ \underline{0}\ f\ (\underline{n}\ f\ x) \\
=_{\beta}& \lambda\ f\ x.\ (\lambda\ f\ x.\ x)\ f\ (\underline{n}\ f\ x) \\
=_{\beta}& \lambda\ f\ x.\ \underline{n}\ f\ x \\
=_{\beta}& \underline{n} \\
=_{\beta}& \underline{n + 0} \\
\end{align*}
Therefore, the proof holds in the case $m = 0$.

\item The inductive step.

Assume the lemma holds for $m = k$ and prove it
holds for $m = k + 1$:
\begin{align*}
&
\lambda\ f\ x.\ \underline{k + 1}\ f\ (\underline{n}\ f\ x) \\
=_{\beta}&
\lambda\ f\ x.\ \underline{k + 1}\ f\ (f^n\ x) \\
=_{\beta}&
\lambda\ f\ x.\ f(f^k(f^n\ x)) \\
=_{\beta}& \lambda\ f\ x.\ f(\underline{n + k}\ f\  x) \\
=_{\beta}& \mathop{Succ}\ \underline{n + k} \\
=_{\beta}& \underline{n + k + 1} \\
\end{align*}

Therefore, if the lemma holds for $m = k$ then it also holds for $m = k + 1$.
Since it held for $m = 0$, by mathematical induction, we have proved it
holds for all $m \in \mathbb{N}$. Since $n$ was arbitrary in the proof,
this statement holds for all $n$. Therefore:
\[
\forall n\forall m. \un{m + n} =_\beta \lambda\ f\ x. \un{m}\ f\ (\un{n}\ f\ x)
\]
As required.

\end{itemize}

\item Define the $\lambda$-terms Times and Exp representing multiplication
and exponentiation of Church numerals respectively. Prove the correctness
of your definitions.
\begin{align*}
\mathit{Times}(m, n) &=_\beta \langle m * n \rangle \\
\mathit{Exp}(m, n) &=_\beta \langle m^n \rangle
\end{align*}
\begin{align*}
\mathop{Times}(m, n) &= n\ (\mathop{Add}\ m)\ 0 \\
\mathop{Exp}(m, n) &= n\ (\mathop{Times}\ m)\ 1
\end{align*}
\begin{align*}
\mathop{Times}(m, n) &= m\ (\mathop{Add}\ n)\ 0 \\
&= \underbrace{m + \dots + m}_{n\ \text{times}} + 0 \text{\ using correctness
 of $\mathop{Add}$} \\
&= \langle m \times n \rangle \text{\ as required}\\\\
\mathop{Exp}(m, n) &= m\ (\mathop{Times}\ n)\ 1 \\
&= \underbrace{m \times \dots \times m}_{n\ \text{times}} \times 1 \text{\
using correctness of $\mathop{Times}$} \\
&= \langle m^n \rangle \text{\ as required}
\end{align*}

\end{enumerate}

\section{Question 3}

\begin{enumerate}[label=\arabic*.]

\setcounter{enumii}{10}

\item If you are \textit{still} not fed up with Ackermann's function
$\mathop{ack} \in \mathbb{N}^2 \to \mathbb{N}$, show that the $\lambda$-term
$\mathop{ack} \triangleq \lambda x. x(\lambda f y. y f(f \underline{1}))
\mathop{Succ}$ represents $\mathop{ack}$ (where $\mathop{Succ}$ is as on slide 123).

The Ackermann function has three rules:
\begin{align*}
\mathop{ack}(0, x_2) &\triangleq x_2 + 1 \\
\mathop{ack}(x_1 + 1, 0) &\triangleq \mathop{ack}(x_1, 1) \\
\mathop{ack}(x_1 + 1, x_2 + 1) &\triangleq
\mathop{ack}(x_1, \mathop{ack}(x_1 + 1, x_2))
\end{align*}

To prove that the $\lambda$-term above is computes Ackermann's function, I
will perform case analysis.

\begin{itemize}

\item Case $\mathop{ack}(0, x_2)$
\begin{align*}
(\lambda x. x(\lambda f y. y f(f \underline{1}))
\mathop{Succ})\underline{0}\ \underline{x_2}
&=_{\beta}
(\underline{0}(\lambda f y. y f(f \underline{1}))
\mathop{Succ})\underline{x_2} \\
&=_{\beta}
\mathop{Succ}\  \underline{x_2} \\
&=_{\beta}
\underline{x_2 + 1} \\
&=_{\beta} \mathop{ack}(0, x_2)
\end{align*}

\item Case $\mathop{ack}(x_1 + 1, 0)$
I will prove this by induction. Start by proving the base case $x_1 = 0$:
\begin{align*}
& (\lambda x. x(\lambda f y. y f(f \underline{1}))
\mathop{Succ})\underline{1}\ \underline{0} \\
\twoheadrightarrow &
(\underline{1}(\lambda f y. y f(f \underline{1}))
\mathop{Succ})\underline{0} \\
\twoheadrightarrow &
(\lambda f y. y f(f \underline{1})
\mathop{Succ})\underline{0} \\
\twoheadrightarrow &
(\lambda y. y \mathop{Succ} (\mathop{Succ} \underline{1}))\underline{0} \\
\twoheadrightarrow &
\underline{0} \mathop{Succ} (\mathop{Succ} \underline{1}) \\
\twoheadrightarrow &
\mathop{Succ} \underline{1} \\
\twoheadrightarrow &
\underline{2} \\
\end{align*}

Since $Ack(1, 0) = 2$, this proves that the $\lambda$-term $\mathop{ack}$
computes Ackermann's function in the case $x_1 = 0$, $x_2=0$.

Consider now the inductive step. Assume that the $\lambda$-term
$\mathop{ack}$ computes Ackermann's function for $x_1 = k$ and prove it
computes the function for $k + 1$:
\begin{align*}
& (\lambda x. x(\lambda f y. y f(f \underline{1}))
\mathop{Succ})\underline{k+1}\ \underline{0} \\
\twoheadrightarrow &
\underline{k+1}(\lambda f y. y f(f \underline{1}))
\mathop{Succ}\underline{0} \\
\twoheadrightarrow &
(\lambda f y. y f(f \underline{1}))(\underline{k}(\lambda f y. y f(f \underline{1}))
\mathop{Succ})\underline{0} \\
\twoheadrightarrow &
\underline{0} (\underline{k}(\lambda f y. y f(f \underline{1}))
\mathop{Succ})((\underline{k}(\lambda f y. y f(f \underline{1}))
\mathop{Succ}) \underline{1}) \\
\twoheadrightarrow &
((\underline{k}(\lambda f y. y f(f \underline{1}))
\mathop{Succ}) \underline{1}) \\
\end{align*}

\item Case $\mathop{ack}(x_1 + 1, x_2 + 1)$
\begin{align*}
&(\lambda x. x(\lambda f y. y f(f \underline{1}))\mathop{Succ})\
\underline{x_1 + 1}\ \underline{x_2 + 1} \\
=_{\beta}& \underline{x_1 + 1}(\lambda f y. y f(f \underline{1}))
\mathop{Succ}\ \underline{x_2 + 1} \\
=_{\beta}& (\lambda f y. y f(f \underline{1}))
(\underline{x_1}(\lambda f y. y f(f \underline{1}))
\mathop{Succ})\ \underline{x_2 + 1} \\
=_{\beta}&
\underline{x_2 + 1}\ (\underline{x_1}(\lambda f y. y f(f \underline{1}))
\mathop{Succ})\ ((\underline{x_1}(\lambda f y. y f(f \underline{1}))
\mathop{Succ}) \underline{1}) \\
=_{\beta}&
(\underline{x_1}(\lambda f y. y f(f \underline{1}))
\mathop{Succ})
(\underline{x_2}\ (\underline{x_1}(\lambda f y. y f(f
\underline{1}))
\mathop{Succ})\ ((\underline{x_1}(\lambda f y. y f(f \underline{1}))
\mathop{Succ}) \underline{1})) \\
=_{\beta}&
(\underline{x_1}(\lambda f y. y f(f \underline{1}))
\mathop{Succ})
\underbrace{(\lambda fy. y f(f\underline{1})\underline{x_1 + 1}\
\underline{x_2})}_{\mathop{ack}(x_1 + 1, x_2)} \\
=_{\beta}&
(\underline{x_1}(\lambda f y. y f(f \underline{1}))
\mathop{Succ})
\mathop{ack}(x_1 + 1, x_2) \\
=_{\beta}&
(\lambda yf. yf(f\underline{1}))\underline{x_1}
\ \mathop{ack}(x_1 + 1, x_2) \\
&=_{\beta} \mathop{ack}(x_1, \mathop{ack}(x_1 + 1, x_2))
\end{align*}
Therefore, the $\lambda$-term computes Ackermann's function for

\end{itemize}

\item Let $I$ be the $\lambda$-term $\lambda x. x$. Show that
$\underline{n}\ I =_\beta I$ holds for every Church numeral $\underline{n}$.
\begin{align*}
\underline{n}\ I &=
(\lambda fx. f^n x) I \\
&= \lambda x. I^n x \\
&= \lambda x. x \\
&= I
\end{align*}
Now consider
\[
B \triangleq \lambda\ f\ g\ x.\ g\ x\ I\ (f\ (g\ x))
\]
Assuming the fact about normal order reduction mentioned on slide 115, show
that if partial functions $f, g \in \mathbb{N} \rightharpoonup \mathbb{N} $
are represented by closed $\lambda$-terms $F$ and $G$ respectively, then
their composition $(f \circ g)(x) \equiv f(g(x))$ is represented by
$B\ F\ G$.

$B$ is required to ensure that $g$ is defined. I provide a proof by case
analysis below:

\begin{itemize}

\item Case $g(x)\uparrow$
\begin{align*}
(B\ F\ G)x &\triangleq (\lambda\ f\ g\ x.\ g\ x\ I\ (f\ (g\ x)))\ F\ G\ x \\
&\twoheadrightarrow G\ x\ I\ (F\ (G\ x))
\end{align*}
By assumption, $g(x)\uparrow$. So the $\lambda$-term $G\ x$ contains an
infinite reduction sequence and has no normal form. Using the fact on slide
115, since $G\ x$ is the left-most $\lambda$-term we can conclude that the
$\lambda$-term contains no normal form. Therefore, in this case of $(B\ F\
G)\ x$ undefined -- as required.

\item Case $g(x)\downarrow$
\begin{align*}
(B\ F\ G)x &\triangleq (\lambda\ f\ g\ x.\ g\ x\ I\ (f\ (g\ x)))\ F\ G\ x && \\
&\twoheadrightarrow G\ x\ I\ (F\ (G\ x)) &&\\
&\twoheadrightarrow \underline{g(x)}\ I\ (F\ (G\ x)) &&\text{ since $g
\subseteq \mathbb{N} \times \mathbb{N}$} \\
&\twoheadrightarrow (F\ (G\ x)) &&\text{ using }\ \underline{n}\ I =_\beta
I \\
&\twoheadrightarrow F\ \underline{g(x)} && \\
&\twoheadrightarrow \underline{f(g(x))} &&
\end{align*}

\end{itemize}

\end{enumerate}

\section{Question 4}

In this question you may use all of the $\lambda$-definable functions
presented in the notes, as well as the terms you define as part of this 
exercise. You should explain your answers (possibly using some examples), 
but don't need to prove their correctness.\\

\begin{enumerate}[label=(\alph*)]

\item ``Not'', i.e boolean negation
\[
\mathop{Not}(f) \triangleq \lambda\ x\ y. f\ y\ x
\]

Not inverts the result -- in $\lambda$ calculus this is achieved by swapping
the order in which the $\lambda$-terms are passed to the boolean.

\item ``And'' and ``Or'', i.e boolean conjunction and disjunction
\begin{align*}
\mathop{And}(f, g) &\triangleq \lambda\ x\ y. f\ (g\ x\ y)\ y \\
\mathop{Or}(f, g) &\triangleq \lambda\ x\ y. f\ x\ (g\ x\ y)
\end{align*}

\item ``Minus'', i.e truncated subtraction
\[
\mathop{Minus} \triangleq \lambda\ n\ m.\ m\ (\mathop{Pred})\ n
\]

$\mathop{Pred}$ performs truncated subtraction for 1. So applying
$\mathop{Pred}$ $m$ times has the same effect as truncated subtraction of
$m$ from $n$.

\item Numeric comparison operators $=, \neq, <, \leq , >, \geq$. You can
define them in any order you find most convenient.
\begin{align*}
\geq &\triangleq \lambda\ m\ n. \mathop{Eq_0} (\mathop{Minus}\ n\ m) \\
< &\triangleq \lambda\ m\ n. \mathop{Not}(\geq m\ n)\\
\leq &\triangleq \lambda\ m\ n. \mathop{Eq_0} (\mathop{Minus}\ m\ n) \\
> &\triangleq \lambda\ m\ n. \mathop{Not}(\leq m\ n)\\
= &\triangleq \lambda\ m\ n. \mathop{And} (\geq m\ n) (\leq m\ n) \\
\neq &\triangleq \lambda\ m\ n. \mathop{Not} (= m\ n) \\
\end{align*}

$m \geq n$ is the same as testing whether the truncated subtraction of $m$
from $n$ is equal to zero. So we do this.

$m < n$ is equivalent to $\neg(m \geq n)$, so I use this equivalence.

Similar logic applies to $\leq$ and $>$.

$=$ is equivalent to $(m \geq n) \wedge (m \leq n)$.

$m \neq n$ is logically equivalent to $\neg (m = n)$.

\item A $\lambda$-term ``MapPair'', i.e a function that applies a function
to both elements of the pair.
\[
\mathop{MapPair} \triangleq \lambda\ f\ p. \mathop{Pair}\ (f\ (\mathop{fst}\ n))
(f\ (\mathop{snd}\ n))
\]

This destructs the pair, applies the function $f$ to both halves and
reconstructs the pair. This has the desired effect of ``MapPair''.

\item ``SquareSum'', i.e a function representing $(m, n) \mapsto m^2 + n^2$
\[
\mathop{SquareSum} \triangleq \lambda p. \mathop{Add}\ (\mathop{Times}\
(\mathop{fst}\ p)\ (\mathop{fst}\ p))\ (\mathop{Times}\ (\mathop{snd}\ p)\
(\mathop{snd}\ p))
\]

This $\lambda$-term destructs the pair, squares each and adds them.

\item Explain why Curry's $\mathbf{Y}$ combinator is needed and how it works.

Curry's $\mathbf{Y}$ combinator has the effect of passing a function to
itself as argument. This allows recursion. Recursion is necessary to
implement partial recursive functions; so the usage of Curry's $\mathbf{Y}$
combinator is an important part in the proof that any
computable function is $\lambda$-computable.

\item Give a $\lambda$-term which is $\beta$-equivalent to the $\mathbf{Y}$
combinator, but only uses its $f$ argument once.
\[
\mathbf{Y}' \triangleq \lambda\ f. (\lambda\ y.\ y\ y)\ (\lambda\ x.\ f\ (x\ x))
\]

\end{enumerate}

\begin{examquestion}{2009}{6}{6}

\begin{enumerate}

\item Define what it means for a $\lambda$-calculus term to be in normal
form. Is it possible for a $\lambda$-term to have two normal forms that are
not $\alpha$-equivalent? Provide justification for your answer?

A $\lambda$-calculus term is in normal form if there do not exist any
$\beta$-reductions.

A $\beta$-reduction is defined as follows:

\begin{gather*}
\dfrac{}{(\lambda\ x. M)N \to M[N/x]}
\quad
\dfrac{M \to M'}{(\lambda x.M)N\to (\lambda\ x.M')N}
\quad
\dfrac{N \to N'}{(\lambda\ x. M)N'}\\
\dfrac{M \to M'}{M\ N \to M'\ N}
\quad
\dfrac{N \to N'}{M\ N \to M\ N'}
\end{gather*}

It's not possible for a $\lambda$-term to have two normal forms which are
not $\alpha$-equivalent.

\begin{gather}
\dfrac{M =_\alpha M'}{M \twoheadrightarrow M'}
\quad
\dfrac{M \to M'}{M \twoheadrightarrow M'}
\quad
\dfrac{M \twoheadrightarrow M'\quad M' \to M''}{M \twoheadrightarrow M''}
\end{gather}

Many-step $\beta$-reduction is confluent. Formally,
\[
M \twoheadrightarrow M' \wedge M \twoheadrightarrow M'' \Longrightarrow
\exists M'''. M' \twoheadrightarrow M''' \wedge M'' \twoheadrightarrow M'''
\]
This also applies to many-step $\beta$-reduction. So if there was a
$\lambda$-term which had multiple $\beta$ normal forms which were not
$\alpha$ then confluence for many-step $\beta$-reduction \textit{could not}
be true. Which is a contradiction. Therefore, the assumption that there
exists a $\lambda$-term which has multiple $\beta$ normal forms.

\item For each of the following, give an example of a $\lambda$-term that

\begin{enumerate}

\item is in normal form
\[
\lambda\ x.\ x
\]

\item is not in normal form but has a normal form
\[
(\lambda\ x.\ x)y
\]
The normal form for this expression is:
\[
y
\]

\item does not have a normal form
\[
\Omega = (\lambda\ x.\ x\ x)(\lambda\ x.\ x\ x)
\]

\end{enumerate}

\item We define a $\lambda$-term $N$ to be \textit{non-trivial} iff there
exist $A$ and $B$ such that $N\ A \to^* \mathbf{true}$ and $N\ B \to^*
\mathbf{false}$, where $\mathbf{true}$ and $\mathbf{false}$ encode the
Booleans.

Give an example of a $\lambda$-term that is non-trivial and show that it is
non-trivial.

\[
N = \lambda\ x.\ x
\]

$N$ is \textit{non-trivial} because $N\ \mathbf{true} = \mathbf{true}$ and
$N\ \mathbf{false} = \mathbf{false}$.

\item We define a $\lambda$-term $N$ as \textit{total} iff for each
$\lambda$-term $M$, either $N\ M \to^* \mathbf{true}$ or $N\ M \to^*
\mathbf{false}$.

Give an example of a $\lambda$-term that is total, and show that it is total.

\[
\lambda\ x. \mathbf{true}
\]

Clearly, $\forall M. (\lambda\ x. \mathbf{true}) M \to \mathbf{true}$. So
the criteria is fulfilled.

\item Prove that there is no non-trivial and total $\lambda$-term.

Consider a \textit{non-trivial} $\lambda$-term $N$ and without loss of
generality let $A$, $B$ be $\lambda$-terms such that $N\ A \to^*
\mathbf{true}$ and $N\ B to^* \mathbf{false}$.

Firstly, use the identity:
\[
N(\mathbf{Y}L) =_{\beta} N(\mathbf{if}\ (N\ (\mathbf{Y}\ L)) B\ A)
\]

I show that $N(\mathbf{Y}\ L)$ can evaluate to neither $\mathbf{true}$ nor
$\mathbf{false}$ -- providing an arbitrary counterexample to the existence
of a \textit{non-trivial} and \textit{total} $\lambda$-term.

\begin{itemize}

\item Case $N(\mathbf{Y}\ L) \to^* \mathbf{false}$
\begin{align*}
N(\mathbf{Y}\ L) &\to^* \mathbf{false} \Longrightarrow \\
N(\mathbf{if}\ (N\ (\mathbf{Y}\ L)) B\ A) &\to^* \mathbf{false}
\Longrightarrow \\
N(\mathbf{if}\ \mathbf{false}\ B\ A) &\to^* \mathbf{false}
\Longrightarrow \\
N\ A &\to^* \mathbf{false} \Longrightarrow \\
\mathbf{true} &\to^* \mathbf{false}
\end{align*}
Since a contradiction has been reached, we can conclude that
$N(\mathbf{Y}\ L) \not\to^* \mathbf{false}$

\item Case $N(\mathbf{Y}\ L) \to^* \mathbf{true}$
\begin{align*}
N(\mathbf{Y}\ L) &\to^* \mathbf{true} \Longrightarrow \\
N(\mathbf{if}\ (N\ (\mathbf{Y}\ L)) B\ A) &\to^* \mathbf{true}
\Longrightarrow \\
N(\mathbf{if}\ \mathbf{true}\ B\ A) &\to^* \mathbf{true}
\Longrightarrow \\
N\ B &\to^* \mathbf{true} \Longrightarrow \\
\mathbf{false} &\to^* \mathbf{true}
\end{align*}
Since a contradiction has been reached, we can conclude that
$N(\mathbf{Y}\ L) \not\to^* \mathbf{true}$

\end{itemize}
Therefore, $N(\mathbf{Y}\ L)$ reduces to neither $\mathbf{true}$ nor
$\mathbf{false}$. So $N(\mathbf{Y}\ L)$ is not total. So there can exist no
$\lambda$-term which is both \textit{non-trivial} and total.

\item What consequences does this have for defining a general equality
$\lambda$-term such that
\begin{align*}
\mathop{\text{equal}}\ A\ B & \to^* \mathbf{true} & & \mathop{\text{if}}\ A
 = B\\
\mathop{\text{equal}}\ A\ B & \to^* \mathbf{true} && \mathop{\text{otherwise}}
\end{align*}

This shows that there can exist no general equality $\lambda$-term. If
there was a general equality $\lambda$-term, it would be both
\textit{non-trivial} and \textit{total}. By the proof above, there exists
no such term and therefore there exists no general equality $\lambda$-term.

\end{enumerate}

\end{examquestion}

\begin{examquestion}{2019}{6}{6}

\begin{enumerate}

\item \begin{enumerate}

\item Give an inductive definition of the relation $M =_\beta N$ of
$\beta$-\textit{conversion} between $\lambda$-terms $M$ and $N$.
\[
\dfrac{}{x =_\alpha x}
\quad
\dfrac{Z\#(M\ N) M[z/x] =_\alpha N[z/y]}{
\lambda x. M =_\alpha \lambda y. N
}
\quad
\dfrac{M =_\alpha N \quad M' =_\alpha N'}{M N =_\alpha M' N'}
\]

\begin{gather*}
\dfrac{}{(\lambda x. M)N \to M[N/x]}
\quad
\dfrac{M \to M'}{\lambda x. M \to \lambda x. M'}
\quad
\dfrac{M \to M'}{M N \to M' N}
\\
\dfrac{N \to N'}{M N \to M N'}
\quad
\dfrac{
M =_\alpha N \quad M \to M' \quad M' =_\alpha N
}{
N \to N'
}
\end{gather*}

\iffalse
\[
\dfrac{M =_\alpha N}{M \twoheadrightarrow N}
\quad
\dfrac{M \to M'}{M \twoheadrightarrow M'}
\quad
\dfrac{M \twoheadrightarrow M'\quad M' \to M''}{
M \twoheadrightarrow M''
}
\]
\fi

\begin{gather*}
\dfrac{M =_\alpha N}{M =_\beta N}
\quad
\dfrac{
M \to N
}{
M =_\beta N
}
\quad
\dfrac{M =_\beta M'}{M' =_\beta M}
\quad
\dfrac{M =_\beta M' \quad M' =_\beta M''}{M =_\beta M''}
\\
\dfrac{M =_\beta M'}{\lambda x. M =_\beta \lambda x. M'}
\quad
\dfrac{M =_\beta M' \quad N =_\beta N'}{
M\ N =_\beta M'\ N'
}
\end{gather*}

\item What is meant by a term in $\beta$-\textit{normal form}?

A $\lambda$-term $M$ is in $\beta$-\textit{normal form} if and only if
there does not exist any $\lambda$-term such that $M \to N$ and
$\neg(M =_\alpha N)$.

\item If $M$ and $N$ are in $\beta$-normal form, explain why $M =_\beta N$
implies that $M$ and $N$ are $\alpha$-equivalent $\lambda$-terms.

Assume $M =_\beta N$ for $M, N$ in $\beta$-normal form.
\begin{align*}
\Longrightarrow &\exists M'. M \twoheadrightarrow M' \twoheadleftarrow N &
& \text{by definition} \\
\intertext{By definition of $\beta$-normal form, the only reduction which
either can be made is an $\alpha$-renaming}
\Longrightarrow& \exists M'. M =_\alpha M' =_\alpha N &
& \\
\Longrightarrow& M =_{\alpha} N & & \text{since $=_\alpha$ is an equivalence}
\end{align*}

\end{enumerate}

\item Show that there are $\lambda$-terms $\mathbf{True}$, $\mathbf{False}$
and $\mathbf{If}$ satisfying $\mathbf{If}\ \mathbf{True}\ M\ N =_\beta M$
and $\mathbf{If}\ \mathbf{False}\ M\ N =_\beta N$ and with $\mathbf{True}
\neq \mathbf{False}$.

Define the $\lambda$-terms as follows:
\begin{align*}
\mathbf{If} &\triangleq \lambda fxy. fxy \\
\mathbf{True} &\triangleq \lambda xy. x \\
\mathbf{False} &\triangleq \lambda xy. y
\end{align*}

\[
\begin{prooftree}
\hypo{\mathbf{If}\ \mathbf{True}\ M\ N \to \mathbf{True}\ M\ N}
\infer1{
\mathbf{If}\ \mathbf{True}\ M\ N =_\beta \mathbf{True}\ M\ N
}
\hypo{\mathbf{True}\ M\ N \to M}
\infer1{
\mathbf{True}\ M\ N =_\beta M
}
\infer2{
\mathbf{If}\ \mathbf{True}\ M\ N =_\beta M
}
\end{prooftree}
\]

\[
\begin{prooftree}
\hypo{\mathbf{If}\ \mathbf{False}\ M\ N \to \mathbf{False}\ M\ N}
\infer1{
\mathbf{If}\ \mathbf{False}\ M\ N =_\beta \mathbf{False}\ M\ N
}
\hypo{\mathbf{False}\ M\ N \to N}
\infer1{
\mathbf{False}\ M\ N =_\beta N
}
\infer2{
\mathbf{If}\ \mathbf{False}\ M\ N =_\beta N
}
\end{prooftree}
\]

\item Define \textit{Curry's fixed point combinator} $\mathbf{Y}$ and prove
its fixed-point property.

\[
\mathbf{Y} \triangleq \lambda f. (\lambda x. f (x\ x))(\lambda x. f (x\ x))
\]

The fixed-point property is that for any $\lambda$-term $M$: $\mathbf{Y}M
=_\beta M(\mathbf{Y}M)$. A derivation that the fixed-point property holds for
any arbitrary $\lambda$-term $M$ is given below:
\[
\mathlarger{\nabla} =
\begin{prooftree}
\hypo{\mathbf{Y}M \to (\lambda x. M(x\ x))(\lambda x. M(x\ x))}
\infer1{
\mathbf{Y}M =_\beta (\lambda x. M(x\ x))(\lambda x. M(x\ x))
}
\hypo{(\lambda x. M(x\ x))(\lambda x. M(x\ x)) \to M((\lambda x. M(x\ x))(\lambda x. M(x\ x)))}
\infer1{
(\lambda x. M(x\ x))(\lambda x. M(x\ x)) =_\beta M((\lambda x. M(x\ x))(\lambda x. M(x\ x)))
}
\infer2{
\mathbf{Y}M =_\beta M((\lambda x. M(x\ x))(\lambda x. M(x\ x)))
}
\end{prooftree}
\]
\[
\begin{prooftree}
\hypo{\mathlarger{\nabla}}
\hypo{M(\mathbf{Y}M) \to M((\lambda x. M(x\ x))(\lambda x. M(x\ x)))}
\infer1{
M(\mathbf{Y}M) =_\beta M((\lambda x. M(x\ x))(\lambda x. M(x\ x)))
}
\infer2{
\mathbf{Y}M =_\beta M(\mathbf{Y}M)
}
\end{prooftree}
\]

\item Consider the following two properties of $\lambda$-term $M$:
\begin{enumerate}[label=(\Roman*)]

\item there exist $\lambda$-terms $A$ and $B$ with $M\ A =_\beta
\mathbf{True}$ and $M\ B =_\alpha \mathbf{False}$.

\item for all $\lambda$-terms $N$, either $M\ N =_\beta \mathbf{True}$ or
$M\ N =_\beta \mathbf{False}$.

\end{enumerate}

Prove that $M$ cannot have both properties (I) and (II).

I will assume property (I) holds and perform case analysis to prove that
property (II) cannot hold by contradiction, using the
identity below:
\[
\mathbf{Y}(\lambda. x \mathbf{If}\ (M\ x)B\ A) =_{\beta} \mathbf{If}\ (M(\mathbf{Y}(\lambda. x \mathbf{If}\ (M\ x)B\ A)))B\ A
\]
\begin{itemize}


\item Case $M (\mathbf{Y}(\lambda. x \mathbf{If}\ (M\ x)B\ A)) =_\beta
\mathbf{True}$
\begin{align*}
\mathbf{True} &=_\beta
M (\mathbf{Y}(\lambda. x \mathbf{If}\ (M\ x)B\ A)) \Longrightarrow \\
\mathbf{True} &=_\beta
M (\mathbf{If}\ \mathbf{True}\ B\ A) \Longrightarrow \\
&=_{\beta} M B \Longrightarrow \\
&=_{\beta} \mathbf{False} \text{\ by definition of $M$}
\end{align*}
A contradiction has been reached. Therefore, this case cannot hold!

\item Case $M (\mathbf{Y}(\lambda. x \mathbf{If}\ (M\ x)B\ A)) =_\beta
\mathbf{False}$
\begin{align*}
\mathbf{False} &=_\beta
M (\mathbf{Y}(\lambda. x \mathbf{If}\ (M\ x)B\ A)) \Longrightarrow \\
\mathbf{True} &=_\beta
M (\mathbf{If}\ \mathbf{False}\ B\ A) \Longrightarrow \\
&=_{\beta} M A \Longrightarrow \\
&=_{\beta} \mathbf{True} \text{\ by definition of $M$}
\end{align*}
A contradiction has been reached. Therefore, this case cannot hold!

\end{itemize}

Therefore $M (\mathbf{Y}(\lambda. x \mathbf{If}\ (M\ x)B\ A))$ is not
$\beta$-equivalent to either $\mathbf{True}$ \textit{or} $\mathbf{False}$.
So if $M$ has property (I) then it cannot have property (II). So there
exists no $\lambda$-term $M$ which has both property (I) and property (II).

\item Deduce that there is no $\lambda$-term $E$ such that for all
$\lambda$-terms $M$ and $N$
\[
E\ M\ N =_{\beta}
\begin{cases}
\mathbf{True} & \quad \text{if}\ M =_{\beta} N \\
\mathbf{False} & \quad \text{otherwise}
\end{cases}
\]

If such an $E$ existed, it would have both property (I) and property (II).
Therefore, no such $E$ can exist using the proof above.

Clearly $(E M)\ M = \mathbf{true}$, and there exist $\lambda$-terms $N$
which are not $\beta$-equivalent to $M$ -- so $(E M)\ N = \mathbf{false}$.
So $EM$ is totally defined and takes the value $\mathbf{true}$ on some
inputs and $\mathbf{false}$ on others. By the proof above, $EM$ cannot
exist. $M$ was an arbitrary $\lambda$-term and so must exist.
We can therefore conclude that $E$ cannot exist.

\end{enumerate}

\end{examquestion}

\section{Question 8}

\begin{enumerate}[label=(\alph*)]

\item Define the $\lambda$-term $\mathop{Fact}$ that computes the factorial
of a Church numeral.

For this question and the next, I implement recursion by explicitly
passing the $\lambda$-term a reference to itself using Curry's $\mathbf{Y}$
combinator.

\[
\mathop{Fact} \triangleq
\mathbf{Y}(\lambda \ f\  n. \mathop{if}\ (\mathop{Eq_0}\ n)\ 1\
(\mathop{Times}\ n\ (\mathbf{Y}\ f\ (\mathop{Pred}\ n))))
\]

\item Define the $\lambda$-term $\mathop{Fib}$ such that
$\mathop{Fib} n =_\beta \mathop{Fib}_n$ where $F_n$ is the $n$-th Fibonacci
number defined recursively as $F_0 = 0, F_1=1, F_n = F_{n-1} + F_{n-2}$.
\[
\mathop{Fib} \triangleq \mathbf{Y} (
\lambda\ f\ n.\ \mathop{if}\ (\mathop{Eq_0}\ n)\ 0\
(
	\mathop{if}\ (\mathop{Eq_0}\ (\mathop{Pred}\ n))\ 1\
	(\mathop{Add}\
		(\mathbf{Y}\ f\ (\mathop{Pred}\ n))
		(\mathbf{Y}\ f\ (\mathop{Pred}\ (\mathop{Pred}\ n)))
	)
)
)
\]

\end{enumerate}

\end{document}
