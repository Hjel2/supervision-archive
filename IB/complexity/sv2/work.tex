\newcommand{\svcourse}{CST Part IA: Introduction to Probability}
\newcommand{\svnumber}{1}
\newcommand{\svvenue}{Churchill, Room TBD}
\newcommand{\svdate}{2022-05-14}
\newcommand{\svtime}{11:00}
\newcommand{\svuploadkey}{PO5ogKIM8KQA22FZS8IAf8gxA8XKi19jxIBVHIfFZ+3GCBXuNUXS9lVN6bNYjxM/}

\newcommand{\svrname}{Mr Matthew Ireland}
\newcommand{\jkfside}{twoside}
\newcommand{\jkfhanded}{right}

\newcommand{\studentname}{Harry Langford}
\newcommand{\studentemail}{hjel2@cam.ac.uk}


\documentclass[10pt,\jkfside,a4paper]{article}
\usepackage{stmaryrd}

% DO NOT add \usepackage commands here.  Place any custom commands
% into your SV work files.  Anything in the template directory is
% likely to be overwritten!

\usepackage{fancyhdr}

\usepackage{lastpage}       % ``n of m'' page numbering
\usepackage{lscape}         % Makes landscape easier

\usepackage{verbatim}       % Verbatim blocks
\usepackage{epsfig}         % Embed encapsulated postscript
\usepackage{array}          % Array environment
\usepackage[nolinks]{qrcode}         % QR codes
\usepackage{enumitem}       % Required by Tom Johnson's exam question header

\usepackage{hhline}         % Horizontal lines in tables
\usepackage{siunitx}        % Correct spacing of units
\usepackage{amsmath}        % American Mathematical Society
\usepackage{amssymb}        % Maths symbols
\usepackage{amsthm}         % Theorems

\usepackage{ifthen}         % Conditional processing in tex

\usepackage[top=3cm,
            bottom=3cm,
            inner=2cm,
            outer=5cm]{geometry}

% PDF metadata + URL formatting
\usepackage[
            pdfauthor={\studentname},
            pdftitle={\svcourse, SV \svnumber},
            pdfsubject={},
            pdfkeywords={9d2547b00aba40b58fa0378774f72ee6},
            pdfproducer={},
            pdfcreator={},
            hidelinks]{hyperref}

\renewcommand{\headrulewidth}{0.4pt}
\renewcommand{\footrulewidth}{0.4pt}
\fancyheadoffset[LO,LE,RO,RE]{0pt}
\fancyfootoffset[LO,LE,RO,RE]{0pt}
\pagestyle{fancy}
\fancyhead{}
\fancyhead[LO,RE]{{\bfseries \studentname}\\\studentemail}
\fancyhead[RO,LE]{{\bfseries \svcourse, SV~\svnumber}\\\svdate\ \svtime, \svvenue}
\fancyfoot{}
\fancyfoot[LO,RE]{For: \svrname}
\fancyfoot[RO,LE]{\today\hspace{1cm}\thepage\ / \pageref{LastPage}}
\fancyfoot[C]{\qrcode[height=0.8cm]{\svuploadkey}}
\setlength{\headheight}{22.55pt}

\ifthenelse{\equal{\jkfside}{oneside}}{

 \ifthenelse{\equal{\jkfhanded}{left}}{
  % 1. Left-handed marker, one-sided printing or e-marking, use oneside and...
  \evensidemargin=\oddsidemargin
  \oddsidemargin=73pt
  \setlength{\marginparwidth}{111pt}
  \setlength{\marginparsep}{-\marginparsep}
  \addtolength{\marginparsep}{-\textwidth}
  \addtolength{\marginparsep}{-\marginparwidth}
 }{
  % 2. Right-handed marker, one-sided printing or e-marking, use oneside.
  \setlength{\marginparwidth}{111pt}
 }

}{
 % 3. Alternating margins, two-sided printing, use twoside.
}

\setlength{\parindent}{0em}
\addtolength{\parskip}{1ex}

% Exam question headings, labels and sensible layout (courtesy of Tom Johnson)
\setlist{parsep=\parskip, listparindent=\parindent}
\newcommand{\examhead}[3]{\section{#1 Paper #2 Question #3}}
\newenvironment{examquestion}[3]{
    \examhead{#1}{#2}{#3}\setlist[enumerate, 1]{label=(\alph*)}\setlist[enumerate, 2]{label=(\roman*)}
    \marginpar{\qrcode{https://www.cl.cam.ac.uk/teaching/exams/pastpapers/y#1p#2q#3.pdf}}
    \marginpar{\footnotesize \url{https://www.cl.cam.ac.uk/teaching/exams/pastpapers/y#1p#2q#3.pdf}}
}{}



\begin{document}

\begin{examquestion}{2019}{6}{4}

A Boolean formula $\phi$ is in \textit{conjunctive normal form} (CNF) if it
is the conjunction of clauses, each of which is the disjunction of literals.
It is said to be in $k$-CNF (for $k \in \mathbb N$) if each clause has
exactly $k$ literals in it.

An assignment $\sigma : V \to \{\text{true}, \text{false}\}$ of truth values
to the variables is a \textit{satisfying assignment} for a CNF formula $\phi$
if it makes at least one literal in each clause of $\phi$ true. It is said
to be a \textit{not-all-equals} assignment for $\phi$ if it makes at least
one literal in each clause of $\phi$ true \textit{and} at least one literal
in each clauses of $\phi$ false.

Let CNF-SAT denote the problem of determining, given a formula in CNF,
whether it has a satisfying assignment.

Let $k$-SAT denote the problem of determining, given a formula in $k$-CNF,
whether it has a satisfying assignment.

Let $k$-NAE denote the problem of determining, given a formula in $k$-CNF,
whether it has a not-all-equals assignment.

\begin{enumerate}[label=(\alph*)]

\item Explain why CNF-SAT is NP-complete. Your explanation should include a
full definition of NP-completeness and a brief sketch on the proof of the
Cook-Levin theorem.

A language $L$ is NP-complete if and only if it is both in NP and is NP-hard.
CNF-SAT is both in NP and is NP-hard -- therefore it is NP-complete.

A language $L$ is in NP if and only if it is recognisable in polynomial time
by a Nondeterministic Turing Machine.

A language $L$ is NP-hard if and only if, for all languages $A \in NP. A \le
_P L$ -- all languages in NP are polynomially reducible to $L$.

A language $L_1 \subseteq \Sigma_1^*$ is polynomially reducible to $L_2 \subseteq \Sigma^*$
if and only if there exists a function $f: \Sigma_1^* \to \Sigma_2^*$ such
that $x \in L_1 \iff f(x) \in L_2$.

CNF-SAT is solvable by a Nondeterministic Turing Machine in polynomial time.
By nondeterministically writing an interpretation, replacing all variables
with their values under the interpretation; and running the (polynomial)
algorithm for circuit value problem (specified in the lectures). The
Nondeterministic Turing Machine will then accept if and only if the formula
is satisfiable.

We can prove that CNF-SAT is NP-hard by reducing the computation of an
arbitrary nondeterministic Turing machine into a CNF-SAT problem which is
polynomial in the length of the input. This is the approach taken in the
standard proof of the Cook-Levin theorem.

Let $H_{t, p}$ hold if the head is in position $p$ at time $t$. Let
$T_{t, p, \sigma}$ be true if at time $t$ the tape cell at position $p$
contains symbol $\sigma$. Let $S_{t, q}$ hold if at time $t$ the state of
the machine is $q$.

We can use these definitions to encode formulae which represent the
computation of an arbitrary Turing machine. We can then and the expression
with conditions which restrain the computation and ask if the formula is
satisfiable.

A boolean expression which is satisfiable if and only if an arbitrary
nondeterministic Turing Machine accepts a string is given by:
\begin{gather*}
\bigwedge_{t \in T} \bigwedge_{q \in Q} S_{t, q} \implies \bigwedge_{q' \in T
\setminus \{q\}} \neg S_{t, q'}\\
\bigwedge_{t \in T} \bigwedge_{p \in P} H_{t, p} \implies \bigwedge_{p' \in P
\setminus \{p\}} \neg H_{t, p'}\\
\bigwedge_{t \in T} \bigwedge_{p \in P} \bigwedge_{\sigma \in \Sigma} T_{t,
p, \sigma} \implies \bigwedge_{\sigma' \in \Sigma \setminus \{\sigma\}} \neg
T_{t, p, \sigma'}\\
\bigwedge_{t \in T} \bigwedge_{p \in P} \bigwedge_{p' \in P\setminus\{p\}}
\left( H_{t, p} \wedge T_{t, p', \sigma} \right) \implies T_{t + 1, p',
\sigma}\\
\bigwedge_{t \in T} \bigwedge_{p \in P} \bigwedge_{q \in Q}
S_{t, q} \wedge T_{t, p} \wedge T_{t, p, \sigma} \implies
\bigvee_{\Delta} S_{t+1, q'} \wedge H_{t+1, p'} \wedge T_{t+1, p, \sigma'} \\
S_{1, s} \wedge H_{1, 1}\\
\bigwedge_{p \le p'} T_{1, p, x_p} \wedge \bigwedge_{n < p} T_{1, p, \sqcup}
\end{gather*}
\[
\bigvee_{t} S_{t, \text{acc}}
\]

However, these formulae are not in conjunctive normal form. We can rewrite
them (in polynomial time) such that they are (providing a reduction from SAT
to CNF-SAT):

\begin{gather*}
\bigwedge_{t \in T} \bigwedge_{q \in Q} \bigwedge_{q' \in T
\setminus \{q\}} \left(\neg S_{t, q} \vee \neg S_{t, q'}\right)\\
\bigwedge_{t \in T} \bigwedge_{p \in P} \bigwedge_{p' \in P
\setminus \{p\}} \left( \neg H_{t, p} \vee \neg H_{t, p'} \right) \\
\bigwedge_{t \in T} \bigwedge_{p \in P} \bigwedge_{\sigma \in \Sigma}
\bigwedge_{\sigma' \in \Sigma \setminus \{\sigma\}}
\left(\neg T_{t, p, \sigma} \vee \neg T_{t, p, \sigma'}\right)\\
\bigwedge_{t \in T} \bigwedge_{p \in P} \bigwedge_{p' \in P\setminus\{p\}}
\neg H_{t, p} \vee \neg T_{t, p', \sigma} \vee T_{t + 1, p', \sigma}\\
\bigwedge_{t \in T} \bigwedge_{p \in P} \bigwedge_{q \in Q}
\bigwedge_{\Delta}
\neg S_{t, q} \vee \neg T_{t, p} \vee \neg T_{t, p, \sigma} \vee
AUX_{t+1, q', p', \sigma'} \\
\bigwedge_{t \in T} \bigwedge_{q \in Q} \bigwedge_{p \in P} \bigwedge_{\sigma \in \Sigma}
\neg AUX_{t, q, p, \sigma} \vee H_{t, p}\\
\bigwedge_{t \in T} \bigwedge_{q \in Q} \bigwedge_{p \in P} \bigwedge_{\sigma \in \Sigma}
\neg AUX_{t, q, p, \sigma} \vee S_{t, q}\\
\bigwedge_{t \in T} \bigwedge_{q \in Q} \bigwedge_{p \in P} \bigwedge_{\sigma \in \Sigma}
\neg AUX_{t, q, p, \sigma} \vee T_{t, p, \sigma}\\
S_{1, s} \wedge H_{1, 1}\\
\bigwedge_{p \le p'} T_{1, p, x_p} \wedge \bigwedge_{n < p} T_{1, p, \sqcup}
\end{gather*}
\[
\bigvee_{t} S_{t, \text{acc}}
\]

\item Show that 3-SAT is NP-complete by means of a suitable reduction.

To prove that 3-SAT is NP-hard; I provide a reduction from CNF-SAT to a 3-SAT
formula which is equisatisfiable. This satisfies the criteria for a
polynomial time reduction. Given a CNF-SAT formula $\phi$ let $\phi'$ be the
3-SAT formula which is equisatisfiable.

Firstly, copy all the conjuncts containing 1, 2 or 3 literals into $\phi'$.

Next, take all conjuncts containing $\ge 4$ literals $A_1 \vee A_2 \vee \dots \vee A_m$
and add $(A_1 \vee A_2 \vee n_1) \wedge (\overline{n_1} \vee A_3 \vee n_2) \wedge \dots \wedge (\overline{n_{m - 4}} \vee A_{m-1} \vee A_{m})$
to $\phi'$. This formula is satisfiable if and only if $\phi$ is satisfiable.
Therefore, this is a valid reduction from CNF-SAT to 3-SAT\@.

3-SAT is clearly in NP -- the algorithm which nondeterministically guesses
an allocation and then tests whether this allocation satisfies the formula
is a nondeterministic algorithm for 3-SAT\@.

\item Give a polynomial time reduction from 3-SAT to 4-NAE. What can you
conclude about the complexity of the latter problem?

I provide a reduction $f$ from a formula $\phi$ in 3-CNF to a 4-CNF formula
$\psi$ which is in 4-NAE if and only if $\phi$ is in 3-SAT\@.

Define $f$ as the function which adds a fresh literal $X$ to all CNF clauses.

\begin{itemize}

\item Proof that $\phi \in \mathsf{3-SAT} \implies f(\phi) \in \mathsf{4-NAE}$

Assume $\phi$ is satisfiable. Therefore there is some assignment $A$ such
that under assignment $A$, $\phi$ is satisfied.

By applying $A + \{X \mapsto \mathsf{false}\}$ to $f(\phi)$, we have an
interpretation where every clause contains at least one positive literal
(since $\phi$ is satisfied by $A$) and at least one negative literal (since
$X$ is in every clause). This is the definition of \textit{not-all-equal}.

So $\phi \in \mathsf{3-SAT} \implies f(\phi) \in \mathsf{4-NAE}$ as required.

\item Proof that $\phi \notin \mathsf{3-SAT} \implies f(\phi) \notin \mathsf{4-NAE}$

Assume $\phi \notin \mathsf{3-SAT}$, so there is no interpretation under
which all clauses are satisfied.

For all interpretations $A \setminus \{X\}$, we have that some clause is
all negated. So for all clauses to have at least one positive literal, we
require $X \mapsto \mathsf{true}$. If $f(\phi) \in \mathsf{4-NAE}$ then we
all clauses in $f(\phi)$ must have at least one negative literal. Since
$X \mapsto \mathsf{true}$ (by above), we have that some literal in every
clause in $\phi$ must be negated. Consider inverting the assignment $A$.
This would result in an assignment where every clause in $\phi$ contains a
literal which is positive. So $\phi \in \mathsf{3-SAT}$. However, this
contradicts the assumption that $\phi \notin \mathsf{3-SAT}$.

This proves that $\phi \notin \mathsf{3-SAT} \implies f(\phi) \notin \mathsf{4-NAE}$

\end{itemize}

\item Show that the problem 3-NAE is NP-complete

I shall describe a reduction from 4-NAE to 3-NAE which shows that 3-NAE is
NP-complete. This reduction shall take a similar structure to the reduction
from SAT to 3-SAT\@.

Start with an expression in 4-NAE\@. For every clause $(A \vee B \vee C \vee
D)$ containing exactly 4 literals, split it into two clauses: $(A \vee B
\vee n_i) \wedge (\neg n_i \vee C \vee D)$ where $n_i$ is a fresh variable
which does not occur anywhere else in $\phi$. The expression formed by
this translation is in 3-NAE if and only if the original expression was in
4-NAE\@. Therefore this is a valid reduction from 4-NAE to 3-NAE\@.

\begin{itemize}

\item Assume that $\phi$ is in 4-NAE\@.

So there exists an interpretation $I$ under which every clause in $I$ is
satisfiable and not-all-equals. Let $\psi$ be the translated expression.

Since all clauses with 1--3 literals are unchanged, this interpretation
means that all clauses with 1--3 literals are also in 3-NAE\@.

All clauses with 4-literals $(A \vee B \vee C \vee D)$ are split into two
clauses of the form $(A \vee B \vee n_i) \wedge (\neg n_i \vee C \vee D)$.
Under the interpretation $I$, at least one of $A$, $B$, $C$, $D$ is true and
at least one is false.

If all expressions in the lhs clause are true then at least one expression
in the rhs clause is false. So $n_1 \mapsto 0$ will result in both clauses
being in 3-NAE\@. Similar logic applies if all expressions in the lhs clause
are true, rhs clause are false and rhs clause are true. If none of these
cases hold then both clauses are already in 3-NAE\@. Therefore, if $\phi$ is
in 4-NAE then $\psi$ is in 3-NAE\@.

\item Assume that $\phi$ is not in 4-NAE

So there exists some clause in $\phi$ which is not satisfiable and
not-all-equals.

If it has fewer than 4 literals, this clause is also in $\psi$ so $\psi$ is
also unsatisfiable and not-all-equals.

If this clause has 4 literals -- if it is unsatisfiable, then the
corresponding clauses in $\psi$ are unsatisfiable. If it is all-true, then
the resulting clauses are $(A \vee B \vee n_i) \wedge (\neg n_i \vee C \vee
D)$. Therefore, all $A$, $B$, $C$ and $D$ hold. Whatever value $n_i$ is
assigned to, at least one of the clauses will be all equals.

Therefore, $\psi$ is not in 3-NAE if $\psi$ is not in 4-NAE\@.

\end{itemize}

Since we have informally proved both directions, we can conclude that $\psi$
is in 3-NAE if and only if $\phi$ is in 4-NAE -- therefore the function
described is a (polynomial-time) reduction. Since 4-NAE is a NP-complete
problem, we can conclude that 3-NAE is also an NP-complete problem.

\end{enumerate}

\end{examquestion}

\begin{examquestion}{2015}{6}{1}

\begin{enumerate}[label=(\alph*)]

\setcounter{enumi}{1}

\item An instance of a \textit{linear programming} problem consists of a set
$X = \{x_1, \dots, x_n\}$ of variables and a set of \textit{integer
constraints},e ach of which is of the form
\[
\sum_{1 \le i \le n} a_i x_i \le b,
\]
where each $a_i$ and $b$ is an integer.

The $0-1$ Integer Linear Programming feasibility problem ($\mathsf{ILP}$) is,
to determine, given such a linear programming problem, whether there is an
assignment of values from the set $\{0, 1\}$ to the variables in $X$ such that
substituting these values into the constraints leads to all constraints being
simultaneously satisfied.

\begin{enumerate}[label=(\roman*)]

\item Consider a \textit{clause} $c$, i.e.\ a disjunction of Boolean
literals. Show how such a clause can eb converted to an integer constraint
which has a $\{0, 1\}$-solution if, and only if, $c$ is satisfiable.

From a clause $\{X_1, X_2, \dots, X_n\}$ we can create an equisatisfiable
linear programming problem of the form $\sum_{1 \le i \le n} a_i x_i \le b,$
by taking $a_1 = a_2 = \dots a_n = b = -1$.

If $c$ is satisfiable then there exists some interpretation such that at
least one of the clauses is true. Consider all true variables to have value
1 and all false variables to have value 0. So if $c$ is satisfied by some
interpretation then $\sum_{1 \le i \le n } X_i \ge 1 \simeq \sum_{1 \le i \le n} -1 \cdot X_i \ge -1$.
This second linear constraint is the mapping I defined above. So if $c$ is
satisfiable then the second ILP expression is satisfiable.

If the clause $c$ is not satisfiable then for all interpretations, every
variable in the clause is false. Considering all false variables to be 0, we
have for all interpretations, $\sum_{1 \le i \le n} X_i = 0 \simeq \sum_{1 \le
i \le n} -X_i = 0 \implies \sum_{1 \le i \le n} -X_i > -1$.
Therefore the integer constraint is unsatisfiable.

So the integer constraint formed is satisfiable if and only if the
\textit{clause} $c$ is satisfiable.

\item Use part (b)(i) to show that there is ap olynomial-time reduction from
the problem $\mathsf{CNF-SAT}$ to $\mathsf{ILP}$.

Define $f$ as follows: take as input an expression $\phi$ in CNF and convert
each clause into an integer constraint as described above. This is clearly
polynomial.

If $\phi$ is satisfiable then there exists some interpretation $\mathcal I$
such that every clause is simultaneously satisfied. By (b)(i), we have that
this implies all the integer constriants in $f(\phi)$ can be simultaneously
satisfied. Therefore, the expression formed is in $\mathsf{ILP}$.

If $\phi$ is not satisfiable then for every interpretation $\mathcal I$, we
have that at least one clause is not satisfied. By (b)(i) we have that the
integer constraint formed by this clause is also not satisfied. So for every
interpretation, there is at least one integer constraint in $f(\phi)$
which is not satisfied. So $f(\phi)$ is not in $\mathsf{ILP}$.

Therefore, $f$ satisfies the required criteria to be a polynomial
-time reduction from $\mathsf{CNF-SAT}$ to $\mathsf{ILP}$:
\[
x \in \mathsf{CNF-SAT} \iff f(x) \in \mathsf{ILP}
\]

\item Is there a polynomial-time reduction from $\mathsf{ILP}$ to
$\mathsf{CNF-SAT}$? Justify your answer.

There is a polynomial-time reduction from $\mathsf{ILP}$ to $\mathsf{CNF-SAT}$.

By the Cook-Levin theorem, $\mathsf{CNF-SAT}$ is $\mathsf{NP}$-complete. By
the definition of $\mathsf{NP}$-completeness, this implies that
$\forall A \in \mathsf{NP}. A \le_P \mathsf{CNF-SAT}$.

So to prove that there is a reduction from $\mathsf{ILP}$ to
$\mathsf{CNF-SAT}$, it suffices to prove that $\mathsf{ILP}$ is in
$\mathsf{NP}$.

$\mathsf{ILP}$ can be solved in polynomial time by a Nondeterministic Turing
Machine by nondeterministically writing an assignment to all variables and
(which takes polynomial time) and then testing whether all the integer
constraints are simultaneously satisfied (which also takes polynomial time).
Therefore, $\mathsf{ILP} \in \mathsf{NP}$. So by the Cook-Levin Theorem
there exists a polynomial time reduction from $\mathsf{ILP}$ to
$\mathsf{CNF-SAT}$.

\item What can you conclude about the complexity of $\mathsf{ILP}$?

From (b)(ii), we have that $\mathsf{CNF-SAT} \le_P \mathsf{ILP}$. Since
$\mathsf{CNF-SAT}$ is $\mathsf{NP}$-complete, we have that $\mathsf{ILP}$ is
$\mathsf{NP}$-hard.

From (b)(iii), we have that $\mathsf{ILP} \in \mathsf{NP}$.

Combining these (using the definition of $\mathsf{NP}$-completeness from (a)),
we have that $\mathsf{ILP}$ is $\mathsf{NP}$-complete.

\end{enumerate}

\end{enumerate}

\end{examquestion}

\begin{examquestion}{2014}{6}{2}

\begin{enumerate}[label=(\alph*)]

\setcounter{enumi}{1}

\item Consider the following two decision problems.

\vspace{1em}

\hfill\begin{minipage}{\dimexpr\textwidth-1.66cm}
\textbf{Problem 1:} Given an undirected graph $G = (V, E)$ with $|V|$ even,
does $G$ contain a clique with at least $|V|/2$ vertices?
\end{minipage}

\vspace{1em}

\hfill\begin{minipage}{\dimexpr\textwidth-1.66cm}
\textbf{Problem 2:} Given an undirected graph $G = (V, E)$, does $G$ contain
a clique with at least $|V| - 3$ vertices?
\end{minipage}

\vspace{1em}

\begin{enumerate}[label=(\roman*)]

\item Which of the two problems is in $\mathsf{P}$ and which one is
$\mathsf{NP}$-complete?

Problem 1 is $\mathsf{NP}$-complete.

Problem 2 is in $\mathsf P$.

\item For the problem in $\mathsf{P}$, describe a polynomial-time algorithm.

\begin{itemize}

\item For every triple of vertices $(v_i, v_j, v_k)$, let $E'$ be the
cardinality of the set of edges which contain one of the three vertices. If
$|E| - |E'| = (|V| - 3)!$ then succeed

\item If no triple of vertices met the criteria then fail

\end{itemize}

There are $|V|^3$ vertices, at each of which we do $|E|$ work. Therefore,
the complexity of this algorithm is $\mathcal O(|V|^3|E|)$. This proves that
$\text{Problem 1} \in \mathsf{P}$.

\item For the other problem, prove that it is $\mathsf{NP}$-complete.

I provide a reduction from $\mathsf{CLIQUE}$ to Problem 1. Since
$\mathsf{CLIQUE}$ is $\mathsf{NP}$-complete, this proves that
Problem 1 is $\mathsf{NP}$-hard.

I then show that Problem 1 is in $\mathsf{NP}$.

Combining these results shows that Problem 1 is $\mathsf{NP}$-complete.

The input to $\mathsf{CLIQUE}$ is a pair with a graph and an integer
$(G, k)$, where $G$ itself is a pair $(V, E)$. $\mathsf{CLIQUE}$ accepts if
there is a clique with at least $k$ nodes in the graph $G$. Define the
reduction $f$ as follows:
\[
f(((V, E), k)) \triangleq
\begin{cases}
(V + \{v_1, \dots, v_{2k - |V|}\}, E) & \text{ if } k \ge \lceil|V|/2\rceil \\
(V' = (V + \{v_1, \dots, v_{|V| - 2k}\}), E + \{(v_1, v), \dots, (v_{|V| - 2k}, v)|v \in
V'\}) & \text{ if } k < \lceil|V|
/2\rceil
\end{cases}
\]

Intuitively, if $k$ is greater than $\lceil |V|/2\rceil$, we add $2k - |V|$
nodes with no neighbours such that $k$ is exactly half the number of nodes
in the augmented graph and the size of the largest clique is unaffected.
If $k$ is less than half, then we add $|V| - 2k$ nodes which are connected
to everything to increase the size of the maximum clique by $|V| - 2k$.

This is clearly polynomial-time computable.

I prove that $f$ satisfies the other requirements of a polynomial-time
reduction from $\mathsf{IND}$ to Problem 2:
\[
x \in \mathsf{IND} \iff f(x) \in \text{Problem 2}
\]

\begin{itemize}

\item $(\implies)$

Assume $G \in \mathsf{IND}$

Let $G' = f(G)$.

\begin{itemize}

\item Case $k \ge |V|/2$

If there is a clique in $G$ of size $k$, then (since $G'$ contains $G$ as a
subgraph) there must be a clique of size $k$ in $G'$. By the way $G'$ is
constructed, we have $|V'| = 2k$ so there is a clique containing at least
half the nodes in the graph $G'$.

So we have $f(x) \in \text{Problem 2}$ in this case.

\item Case $k < |V|/2$

In this case, there is a clique of size $k$ in $G$. $f$ adds $|V| - 2k$
fully connected vertices into the graph. So this implies there is a clique
of size $k + |V| - 2k = |V| - k$ in $G'$. $G'$ has $2|V| - 2k$ vertices. So
there is a clique containing at least half of the vertices in the graph.

So $f(G) \in \text{Problem 2}$ in this case.

\end{itemize}

\item $(\Longleftarrow)$

Assume $f(G) \in \text{Problem 2}$.

\begin{itemize}

\item Case $k \ge |V|/2$

If $f(G) \in \text{Problem 2}$, then there is a clique of size at least
$|V'|/2$. By the way $V'$ is constructed, we know $|V'|=2k$. So there is a
clique in $G'$ of size at least $k$. $f$ added $2k - |V|$ new nodes to the
$G$ which are unconnected. So these nodes must not appear in the clique of
size $k$. So the clique of size $k$ must be in the original graph $G$.

So we have $G \in \mathsf{IND}$

\item Case $k < |V|/2$

If $f(x) \in \text{Problem 2}$ then there is a clique of size at least
$|V'|/2$ in $G'$. This clique contains at least $|V| - k$ vertices.
$G'$ is $G$ but with $|V| - 2k$ fully connected vertices. So at least $k$ of
these vertices must be from the original graph $G$. So these $k$ vertices
form a clique with each other. Hence the original graph $G$ contains a
clique with at least $k$ vertices.

So we have $G \in \mathsf{IND}$

\end{itemize}

\end{itemize}

This proves that $\mathsf{IND} \le_P \text{Problem 2}$. By the composition
of polynomial-time reductions, we have that $\forall A \in \mathsf{NP}. A \le_P \text{Problem 2}$.
So Problem 2 is $\mathsf{NP}$-hard.

Problem 2 is clearly in $\mathsf{NP}$. A nondeterministic Turing Machine can
nondeterministically select a set of vertices of size $|V|/2$ and determine
whether they form a clique in polynomial time.

Since Problem 2 is both $\mathsf{NP}$-hard and in $\mathsf{NP}$, we have
that Problem 2 is $\mathsf{NP}$-complete.

\end{enumerate}

\end{enumerate}

\end{examquestion}

\end{document}
