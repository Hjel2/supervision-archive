\newcommand{\svcourse}{CST Part IA: Software Engineering and Security}
\newcommand{\svnumber}{1}
\newcommand{\svvenue}{Microsoft Teams}
\newcommand{\svdate}{2022-05-11}
\newcommand{\svtime}{15:00}
\newcommand{\svuploadkey}{CBd13xmL7PC1zqhNIoLdTiYUBnxZhzRAtJxv/ytRdM1r7qIfwMsxeVwM/pPcIo8l}

\newcommand{\svrname}{Dr Sam Ainsworth}
\newcommand{\jkfside}{oneside}
\newcommand{\jkfhanded}{yes}

\newcommand{\studentname}{Harry Langford}
\newcommand{\studentemail}{hjel2@cam.ac.uk}


\documentclass[10pt,\jkfside,a4paper]{article}

% DO NOT add \usepackage commands here.  Place any custom commands
% into your SV work files.  Anything in the template directory is
% likely to be overwritten!

\usepackage{fancyhdr}

\usepackage{lastpage}       % ``n of m'' page numbering
\usepackage{lscape}         % Makes landscape easier

\usepackage{verbatim}       % Verbatim blocks
\usepackage{listings}       % Source code listings
\usepackage{graphicx}
\usepackage{float}
\usepackage{epsfig}         % Embed encapsulated postscript
\usepackage{array}          % Array environment
\usepackage{qrcode}         % QR codes
\usepackage{enumitem}       % Required by Tom Johnson's exam question header

\usepackage{hhline}         % Horizontal lines in tables
\usepackage{siunitx}        % Correct spacing of units
\usepackage{amsmath}        % American Mathematical Society
\usepackage{amssymb}        % Maths symbols
\usepackage{amsthm}         % Theorems

\usepackage{ifthen}         % Conditional processing in tex

\usepackage[top=3cm,
            bottom=3cm,
            inner=2cm,
            outer=5cm]{geometry}

% PDF metadata + URL formatting
\usepackage[
            pdfauthor={\studentname},
            pdftitle={\svcourse, SV \svnumber},
            pdfsubject={},
            pdfkeywords={9d2547b00aba40b58fa0378774f72ee6},
            pdfproducer={},
            pdfcreator={},
            hidelinks]{hyperref}

\renewcommand{\headrulewidth}{0.4pt}
\renewcommand{\footrulewidth}{0.4pt}
\fancyheadoffset[LO,LE,RO,RE]{0pt}
\fancyfootoffset[LO,LE,RO,RE]{0pt}
\pagestyle{fancy}
\fancyhead{}
\fancyhead[LO,RE]{{\bfseries \studentname}\\\studentemail}
\fancyhead[RO,LE]{{\bfseries \svcourse, SV~\svnumber}\\\svdate\ \svtime, \svvenue}
\fancyfoot{}
\fancyfoot[LO,RE]{For: \svrname}
\fancyfoot[RO,LE]{\today\hspace{1cm}\thepage\ / \pageref{LastPage}}
\fancyfoot[C]{\qrcode[height=0.8cm]{\svuploadkey}}
\setlength{\headheight}{22.55pt}


\ifthenelse{\equal{\jkfside}{oneside}}{

 \ifthenelse{\equal{\jkfhanded}{left}}{
  % 1. Left-handed marker, one-sided printing or e-marking, use oneside and...
  \evensidemargin=\oddsidemargin
  \oddsidemargin=73pt
  \setlength{\marginparwidth}{111pt}
  \setlength{\marginparsep}{-\marginparsep}
  \addtolength{\marginparsep}{-\textwidth}
  \addtolength{\marginparsep}{-\marginparwidth}
 }{
  % 2. Right-handed marker, one-sided printing or e-marking, use oneside.
  \setlength{\marginparwidth}{111pt}
 }

}{
 % 3. Alternating margins, two-sided printing, use twoside.
}


\setlength{\parindent}{0em}
\addtolength{\parskip}{1ex}

% Exam question headings, labels and sensible layout (courtesy of Tom Johnson)
\setlist{parsep=\parskip, listparindent=\parindent}
\newcommand{\examhead}[3]{\section{#1 Paper #2 Question #3}}
\newenvironment{examquestion}[3]{
\examhead{#1}{#2}{#3}\setlist[enumerate, 1]{label=(\alph*)}\setlist[enumerate, 2]{label=(\roman*)}
\marginpar{\href{https://www.cl.cam.ac.uk/teaching/exams/pastpapers/y#1p#2q#3.pdf}{\qrcode{https://www.cl.cam.ac.uk/teaching/exams/pastpapers/y#1p#2q#3.pdf}}}
\marginpar{\footnotesize \href{https://www.cl.cam.ac.uk/teaching/exams/pastpapers/y#1p#2q#3.pdf}{https://www.cl.cam.ac.uk/\\teaching/exams/pastpapers/\\y#1p#2q#3.pdf}}
}{}

\usepackage{textcomp}
\usepackage{xcolor}
\usepackage{filecontents}

\makeatletter

% initialisation of user macros
\newcommand\PrologPredicateStyle{}
\newcommand\PrologVarStyle{}
\newcommand\PrologAnonymVarStyle{}
\newcommand\PrologAtomStyle{}
\newcommand\PrologOtherStyle{}
\newcommand\PrologCommentStyle{}

% useful switches (to keep track of context)
\newif\ifpredicate@prolog@
\newif\ifwithinparens@prolog@

% save definition of underscore for test
\lst@SaveOutputDef{`_}\underscore@prolog

% local variables
\newcount\currentchar@prolog

\newcommand\@testChar@prolog%
{%
  % if we're in processing mode...
  \ifnum\lst@mode=\lst@Pmode%
    \detectTypeAndHighlight@prolog%
  \else
    % ... or within parentheses
    \ifwithinparens@prolog@%
      \detectTypeAndHighlight@prolog%
    \fi
  \fi
  % Some housekeeping...
  \global\predicate@prolog@false%
}

% helper macros
\newcommand\detectTypeAndHighlight@prolog
{%
  % First, assume that we have an atom.
  \def\lst@thestyle{\PrologAtomStyle}%
  % Test whether we have a predicate and modify the style accordingly.
  \ifpredicate@prolog@%
    \def\lst@thestyle{\PrologPredicateStyle}%
  \else
    % Test whether we have a predicate and modify the style accordingly.
    \expandafter\splitfirstchar@prolog\expandafter{\the\lst@token}%
    % Check whether the identifier starts by an underscore.
    \expandafter\ifx\@testChar@prolog\underscore@prolog%
      % Check whether the identifier is '_' (anonymous variable)
      \ifnum\lst@length=1%
        \let\lst@thestyle\PrologAnonymVarStyle%
      \else
        \let\lst@thestyle\PrologVarStyle%
      \fi
    \else
      % Check whether the identifier starts by a capital letter.
      \currentchar@prolog=65
      \loop
        \expandafter\ifnum\expandafter`\@testChar@prolog=\currentchar@prolog%
          \let\lst@thestyle\PrologVarStyle%
          \let\iterate\relax
        \fi
        \advance \currentchar@prolog by 1
        \unless\ifnum\currentchar@prolog>90
      \repeat
    \fi
  \fi
}
\newcommand\splitfirstchar@prolog{}
\def\splitfirstchar@prolog#1{\@splitfirstchar@prolog#1\relax}
\newcommand\@splitfirstchar@prolog{}
\def\@splitfirstchar@prolog#1#2\relax{\def\@testChar@prolog{#1}}

% helper macro for () delimiters
\def\beginlstdelim#1#2%
{%
  \def\endlstdelim{\PrologOtherStyle #2\egroup}%
  {\PrologOtherStyle #1}%
  \global\predicate@prolog@false%
  \withinparens@prolog@true%
  \bgroup\aftergroup\endlstdelim%
}

% language name
\newcommand\lang@prolog{Prolog-pretty}
% ``normalised'' language name
\expandafter\lst@NormedDef\expandafter\normlang@prolog%
  \expandafter{\lang@prolog}

% language definition
\expandafter\expandafter\expandafter\lstdefinelanguage\expandafter%
{\lang@prolog}
{%
  language            = Prolog,
  keywords            = {},      % reset all preset keywords
  showstringspaces    = false,
  alsoletter          = (,
  alsoother           = @$,
  moredelim           = **[is][\beginlstdelim{(}{)}]{(}{)},
  MoreSelectCharTable =
    \lst@DefSaveDef{`(}\opparen@prolog{\global\predicate@prolog@true\opparen@prolog},
}

% Hooking into listings to test each ``identifier''
\newcommand\@ddedToOutput@prolog\relax
\lst@AddToHook{Output}{\@ddedToOutput@prolog}

\lst@AddToHook{PreInit}
{%
  \ifx\lst@language\normlang@prolog%
    \let\@ddedToOutput@prolog\@testChar@prolog%
  \fi
}

\lst@AddToHook{DeInit}{\renewcommand\@ddedToOutput@prolog{}}

\makeatother
%
% --- end of ugly internals ---


% --- definition of a custom style similar to that of Pygments ---
% custom colors
\definecolor{PrologPredicate}{RGB}{000,031,255}
\definecolor{PrologVar}      {RGB}{024,021,125}
\definecolor{PrologAnonymVar}{RGB}{000,127,000}
\definecolor{PrologAtom}     {RGB}{186,032,032}
\definecolor{PrologComment}  {RGB}{063,128,127}
\definecolor{PrologOther}    {RGB}{000,000,000}

% redefinition of user macros for Prolog style
\renewcommand\PrologPredicateStyle{\color{PrologPredicate}}
\renewcommand\PrologVarStyle{\color{PrologVar}}
\renewcommand\PrologAnonymVarStyle{\color{PrologAnonymVar}}
\renewcommand\PrologAtomStyle{\color{PrologAtom}}
\renewcommand\PrologCommentStyle{\itshape\color{PrologComment}}
\renewcommand\PrologOtherStyle{\color{PrologOther}}

% custom style definition
\lstdefinestyle{pstyle}
{
  language     = Prolog-pretty,
  upquote      = true,
  stringstyle  = \PrologAtomStyle,
  commentstyle = \PrologCommentStyle,
  literate     =
    {:-}{{\PrologOtherStyle :- }}2
    {,}{{\PrologOtherStyle ,}}1
    {.}{{\PrologOtherStyle .}}1
}

% global settings
\lstset
{
  captionpos = below,
  columns    = fullflexible,
  basicstyle = \ttfamily,
  style= pstyle
}


\begin{document}

\part{Sean's exercise sheet part 3}

\section{Knowledge representation and reasoning}

\begin{enumerate}

\item There were in fact \textit{two} queries suggested in the notes for
obtaining a sequence of actions. The details for
\[
\exists a \exists s. \text{Sequence}(a, s_0, s) \wedge \text{Goal}(s)
\]
were provided, but earlier in the notes the format
\[
\exists \text{actionList}.\text{Goal}(\dots \text{actionList}\dots)
\]
was suggested. Explain how this alternative form of query might be made to
work.

Informally, we could hardcode the initial state $s_0$ into the Goal
predicate. We then use it as a wrapper to the initial query. Using prolog
notation, we would encode this as:

\begin{lstlisting}[style=pstyle]
goal(ActionList) :- sequence(ActionList, s0, S), isGoal(S).
\end{lstlisting}

This has the disadvantage of requiring the query to hardcode the start state
-- and not exposing the final state to the caller -- perhaps for the
purposes of further queries or actions.

\item Making correct use of the situation calculus, write the sentences in
FOL required to implement the \texttt{Shoot} action in Wumpus World. Write
further sentences in FOL to allow movement and change of orientations.

\begin{gather*}
Poss(a, s) \implies (Poss(\text{Shoot}, result(a, s)) \iff (
\\
(a = \text{grab} \wedge At(\ell, s) \wedge Available(\text{Arrow}, \ell, s))\\
\vee \\
(Poss(\text{Shoot}, s) \wedge \neg (a = \text{Shoot} \vee a = release(Arrow)))
\\
))
\end{gather*}

\begin{gather*}
Poss(a, s) \implies (
At(\ell, result(a, s)) \iff (
\\
At(\ell', s) \wedge a = go(\ell', \ell)
\\
\vee \\
At(\ell', s) \wedge a \neq go \wedge \ell = \ell'
\\
))
\end{gather*}

\begin{gather*}
Poss(a, s) \implies (
Direction(north, result(a, s)) \iff ( \\
Direction(north, s) \wedge \neg (a = turnLeft \vee a = turnRight)
\\
\vee \\
Direction(east, s) \wedge a = turnLeft
\\
\vee \\
Direction(west, s) \wedge a = turnRight
\\
))
\end{gather*}

\begin{gather*}
Poss(a, s) \implies (
Direction(east, result(a, s)) \iff ( \\
Direction(east, s) \wedge \neg (a = turnLeft \vee a = turnRight)
\\
\vee \\
Direction(south, s) \wedge a = turnLeft
\\
\vee \\
Direction(north, s) \wedge a = turnRight
\\
))
\end{gather*}

\begin{gather*}
Poss(a, s) \implies (
Direction(south, result(a, s)) \iff ( \\
Direction(south, s) \wedge \neg (a = turnLeft \vee a = turnRight)
\\
\vee \\
Direction(west, s) \wedge a = turnLeft
\\
\vee \\
Direction(east, s) \wedge a = turnRight
\\
))
\end{gather*}

\begin{gather*}
Poss(a, s) \implies (
Direction(west, result(a, s)) \iff ( \\
Direction(west, s) \wedge \neg (a = turnLeft \vee a = turnRight)
\\
\vee \\
Direction(north, s) \wedge a = turnLeft
\\
\vee \\
Direction(south, s) \wedge a = turnRight
\\
))
\end{gather*}

\end{enumerate}

\begin{examquestion}{2003}{9}{8}

\begin{enumerate}[label=(\alph*)]

\item
Explain what the terms \textit{ontological commitment} and \textit{
epistemological commitment} mean in the context of a language for knowledge
representation and reasoning. what are the ontological and epistemological
commitments made by propositional and by first order logic?

Ontological Commitment is something which is true in the world. In a
language for knowledge representation and reasoning, Ontological commitments
represent the set of things which are actually true.

Epistemological Commitment is something which an agent believes to be true.
In a language for knowledge representation and reasoning, this is the set of
things which the agent can infer to be true.

\item You wish to construct a robotic pet cat for the purposes of
entertainment. One purpose of the cat is to scratch valuable objects when
the owner is not present. Give a brief general description of
\textit{situation calculus} and describe how it might be used for knowledge
representation by the robot. Include in your answer one example each of a
\textit{frame axiom}, an \textit{effect axiom} and a \textit{successor-state
axiom}, along with example definitions of suitable predicates and functions.

The situation calculus is a restricted form of first order logic which operates
on ``situations''. In the calculus, we use a knowledge base $\texttt{KB}$ to
reason about predicates that are true. This knowledge base defines the
initial situation $s_0$ and a set of rules for subsequent situations. The
``user'' can then input the sequence of actions which they have taken and
use the situation calculus to determine whether this is legal and if so what
the resulting situation will be.

A frame axiom is used to model the ways in which the world stays the same.
For example, if a predicate is true and we do not make it false, then it is
still true!
\[
Have(\text{Arrow}, s) \wedge a \neq \text{shoot} \wedge
a \neq \text{drop}(\text{Arrow}) \implies Have(\text{Arrow}, result(a, s))
\]

An effect axiom determines the predicates which have been made true as a
result of performing an action:
\[
Poss(\text{grab}, o, s) \implies Have(o, result(\text{grab}, s))
\]

A successor-state axiom combines a frame axiom and an effect axiom into a
single axiom which determines under what conditions a predicate will hold.
For example:
\begin{gather*}
Poss(a, s) \implies (Have(\text{gold}, result(a, s)) \iff ((\\
a = \text{grab}
\wedge Available(\text{gold}, s)) \\
\vee \\
(Have(\text{gold}, s) \wedge \neg (a = drop)) \\
))
\end{gather*}

\item Give a brief description of the \textit{representational frame problem},
the \textit{inferential frame problem}, the \textit{qualification problem}
and the \textit{ramification problem}.

In most environments, there are many fluents and each action only changes a
very small proportion of them. This is the frame problem.

\begin{itemize}

\item Representational Frame Problem

The Representational Frame Problem is a subset of the frame problem which
focuses on using a minimal number of frame axioms -- proportional to the
amount of variables which an action actually changes.

\item Inferential Frame Problem

The Inferential Frame Problem is a subset of the frame problem which focuses
on efficiently carrying all the unchanged fluents through many timesteps.

\item Qualification problem

Often, the number of prerequisites which an action requires is immense and
cannot be efficiently modelled. The qualification problem refers to the
problem of efficiently listing the requirements for every action.

\item Ramification problem

Actions tend to have a wide range of implicit consequences at a very high
level of detail. The Ramification problem is the problem of carrying
through all the implicit effects of actions.

For example, if I pick up a pen then I have not directly encoded the
property that I am also picking up the ink inside the pen -- this could lead
to strange plans involving ``pick up pen'' and ``pick up ink'' as separate
steps.

\end{itemize}

\end{enumerate}

\end{examquestion}

\end{document}
