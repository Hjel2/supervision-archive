\newcommand{\svcourse}{CST Part IA: Software Engineering and Security}
\newcommand{\svnumber}{1}
\newcommand{\svvenue}{Microsoft Teams}
\newcommand{\svdate}{2022-05-11}
\newcommand{\svtime}{15:00}
\newcommand{\svuploadkey}{CBd13xmL7PC1zqhNIoLdTiYUBnxZhzRAtJxv/ytRdM1r7qIfwMsxeVwM/pPcIo8l}

\newcommand{\svrname}{Dr Sam Ainsworth}
\newcommand{\jkfside}{oneside}
\newcommand{\jkfhanded}{yes}

\newcommand{\studentname}{Harry Langford}
\newcommand{\studentemail}{hjel2@cam.ac.uk}


\documentclass[10pt,\jkfside,a4paper]{article}

% DO NOT add \usepackage commands here.  Place any custom commands
% into your SV work files.  Anything in the template directory is
% likely to be overwritten!

\usepackage{fancyhdr}

\usepackage{lastpage}       % ``n of m'' page numbering
\usepackage{lscape}         % Makes landscape easier

\usepackage{verbatim}       % Verbatim blocks
\usepackage{listings}       % Source code listings
\usepackage{graphicx}
\usepackage{float}
\usepackage{epsfig}         % Embed encapsulated postscript
\usepackage{array}          % Array environment
\usepackage{qrcode}         % QR codes
\usepackage{enumitem}       % Required by Tom Johnson's exam question header

\usepackage{hhline}         % Horizontal lines in tables
\usepackage{siunitx}        % Correct spacing of units
\usepackage{amsmath}        % American Mathematical Society
\usepackage{amssymb}        % Maths symbols
\usepackage{amsthm}         % Theorems

\usepackage{ifthen}         % Conditional processing in tex

\usepackage[top=3cm,
            bottom=3cm,
            inner=2cm,
            outer=5cm]{geometry}

% PDF metadata + URL formatting
\usepackage[
            pdfauthor={\studentname},
            pdftitle={\svcourse, SV \svnumber},
            pdfsubject={},
            pdfkeywords={9d2547b00aba40b58fa0378774f72ee6},
            pdfproducer={},
            pdfcreator={},
            hidelinks]{hyperref}

\renewcommand{\headrulewidth}{0.4pt}
\renewcommand{\footrulewidth}{0.4pt}
\fancyheadoffset[LO,LE,RO,RE]{0pt}
\fancyfootoffset[LO,LE,RO,RE]{0pt}
\pagestyle{fancy}
\fancyhead{}
\fancyhead[LO,RE]{{\bfseries \studentname}\\\studentemail}
\fancyhead[RO,LE]{{\bfseries \svcourse, SV~\svnumber}\\\svdate\ \svtime, \svvenue}
\fancyfoot{}
\fancyfoot[LO,RE]{For: \svrname}
\fancyfoot[RO,LE]{\today\hspace{1cm}\thepage\ / \pageref{LastPage}}
\fancyfoot[C]{\qrcode[height=0.8cm]{\svuploadkey}}
\setlength{\headheight}{22.55pt}


\ifthenelse{\equal{\jkfside}{oneside}}{

 \ifthenelse{\equal{\jkfhanded}{left}}{
  % 1. Left-handed marker, one-sided printing or e-marking, use oneside and...
  \evensidemargin=\oddsidemargin
  \oddsidemargin=73pt
  \setlength{\marginparwidth}{111pt}
  \setlength{\marginparsep}{-\marginparsep}
  \addtolength{\marginparsep}{-\textwidth}
  \addtolength{\marginparsep}{-\marginparwidth}
 }{
  % 2. Right-handed marker, one-sided printing or e-marking, use oneside.
  \setlength{\marginparwidth}{111pt}
 }

}{
 % 3. Alternating margins, two-sided printing, use twoside.
}


\setlength{\parindent}{0em}
\addtolength{\parskip}{1ex}

% Exam question headings, labels and sensible layout (courtesy of Tom Johnson)
\setlist{parsep=\parskip, listparindent=\parindent}
\newcommand{\examhead}[3]{\section{#1 Paper #2 Question #3}}
\newenvironment{examquestion}[3]{
\examhead{#1}{#2}{#3}\setlist[enumerate, 1]{label=(\alph*)}\setlist[enumerate, 2]{label=(\roman*)}
\marginpar{\href{https://www.cl.cam.ac.uk/teaching/exams/pastpapers/y#1p#2q#3.pdf}{\qrcode{https://www.cl.cam.ac.uk/teaching/exams/pastpapers/y#1p#2q#3.pdf}}}
\marginpar{\footnotesize \href{https://www.cl.cam.ac.uk/teaching/exams/pastpapers/y#1p#2q#3.pdf}{https://www.cl.cam.ac.uk/\\teaching/exams/pastpapers/\\y#1p#2q#3.pdf}}
}{}


\begin{document}

\part{Seed Labs}

\section{TCP Attacks Lab}

\subsection{SYN Flooding Attack}

I launched the attack through Python, while I noticed an increase in
response time from the client, I was still able to telnet into it. When I
retried the attack with C, the increased response time made the system
unusable. However, I was still able to telnet into it until I decreased the
queue size to 80. After enabling SYN cookies, the client worked perfectly
and I did not observe any impacts of the attack.

\subsection{TCP TST Attacks on \texttt{telnet} Connections}

I setup a telnet connection between two docker images and was able to use 
wireshark to find the source port, destination port, source IP, destination 
IP and sequence number. I was able to successfully close the connection. 

\subsection{TCP Session Hijacking}

I setup a telnet connection between two docker images and was able to use
wireshark to find the source port, destination port, source IP, destination
IP and sequence number. I was able to send packets which got acknowledged
and accepted but did not find commands which actually \textit{did} anything.

\textbf{I got the attack vector working but couldn't find a payload}

\part{Exam Questions}

\begin{examquestion}{2006}{3}{9}

\begin{enumerate}[label=(\alph*)]

\item Name \textit{three} types of software vulnerability; give an example
for each and a brief description of how each could be exploited:

\begin{itemize}

\item Metacharacter Vulnerability

SQL Injection, CSRF and XSS are examples of vulnerabilities which can arise
as a result of Metacharacter Vulnerability. Metacharacter Vulnerability is
defined as when a user is allowed to input characters which have special
meanings and are not interpreted as data. In SQL Injection, the user can
close the quote, add a comment and write their own query.

\item Buffer Overflow

Return to LibC attacks in a setuid program can yield a root shell. In Return
to LibC, the attacker overwrites the buffer, sets the return address of the
buffer to be a library function and passes it an argument which causes it to
act maliciously. For example, jumping to \texttt{system()} with the argument
being a pointer to an environment variable storing \texttt{/bin/sh} will
open a shell.

\item Race Condition

A race condition is where two operations happen concurrently; and the one
which finishes first will succeed. An example of this is a TCP RST attack --
the attacker eavesdrops the sequence number and infers the next
sequence number. They then spoof a TCP RST message. If this RST message
arrives before the senders next message, then the connection will be
forcibly closed.

\end{itemize}

\item Alice wants to attack Bob's computer via the Internet, by sending IP
packets to it, directly from her own computer. she does not want Bob to find
out the IP address of her computer.

\begin{enumerate}

\item Is this easier for Alice with TCP or UDP based application protocols?
Explain why.

It's easier for Alice with UDP\@. UDP is connectionless and only uses
destination port and IP address to multiplex messages -- UDP also does not
have a handshaking protocol. If Alice sends UDP messages (with random
source IP addresses) then they are treated exactly the same as messages
from any other device. If they are sent at a sufficiently fast rate then
Bob will not have enough time to process the real UDP messages.

TCP has a handshaking protocol which establishes a connection. Alice
would not receive the response from Bob (since she has to use a fake IP
address). Alice would then be unable to setup the connection to send
messages. Alice is also unable to send messages to Bob at a sufficiently
fast rate since only a small number of messages would be accepted (due to
the window size).

\item For the more difficult protocol, explain \textit{one} technique that
Alice could try to overcome this obstacle and \textit{one} countermeasure
that Bob could implement in his computer.

Alice could use SYN flooding instead. Rather than attempting to overwhelm
Bob with normal messages, Alice could flood him with SYN requests, filling
up his TCB and making him unable to process any legitimate requests.

The first stage of handshaking is for the client to send a SYN to the server
``I'd like to send data to you''. The server then allocates some memory
storing this (known as a TCB) and sends its own SYN + ACK ``I hear you and
would like to send data to you''. The server will then resend its SYN + ACK
if they go missing and eventually give up. However, during this process, the
TCB is still allocated -- using up memory.

The attack is for Alice to send many SYN requests from random IP addresses.
Bob will then store TCBs and eventually run out of memory and be unable to
accept any new communications.

This can be combated by using a SYN cookie as the sequence number. A SYN
cookie is a MAC (Message Authentication Code); a hash of the servers private
key, the time (at low precision i.e 64 seconds), the clients IP address,
port number and the sequence number they sent. On receiving any ACK, Bob
will use the data in the ACK to reverse engineer the SYN cookie he would
have made for the SYN -- if the sequence number of the ACK is exactly one
larger than this then Bob accepts the connection, else he rejects it. Bob
will not allocate any state for a connection until it is established.
Connections can only be established if the client \textit{actually receives
the SYN + ACK response}. An additional protection is to store a list of
``known'' clients and allocate a certain proportion (i.e a quarter) of TCBs
for their exclusive use. Usage of a SYN cookie has a very low probability
of failure (due to storing a timestamp). It also adds some overhead in the
form of hashing. It's therefore common practice to dynamically enable these
protections when Bob detects he is under a SYN flooding attack.

\item Name \textit{three} functions that Alice's Internet Service provider
could implement to make it more difficult or Alice to achieve her goal.

\begin{itemize}

\item Alice's ISP could drop any outgoing packets with a source
address which is not found on the subnet.

\item Alice's ISP could limit the number of open flows which can be made
through a single home router.

\item Alice's ISP could limit the rate at which Alice can make SYN requests.

\end{itemize}

\end{enumerate}

\item In what way are TCP/UDP port numbers below 1024 special?

Ports below 1024 are ``privileged'' and can only be used by root.

\end{enumerate}

\end{examquestion}

\begin{examquestion}{2018}{4}{7}

\begin{enumerate}

\item An application process receives information via a UDP packet over a
wired Ethernet LAN connection. If the packet carries a source port number
below 1024, under which conditions can information be trusted.

I was unable to find a convincing answer to this. I will therefore state
conditions under which this is guaranteed to be trustworthy:

\begin{itemize}

\item The source address is from a node on the LAN and the LAN is not connected
to the wider internet or the router / switch filters packets such that no
adversary can spoof packets with IP addresses on the subnet; AND we trust
no adversary on the LAN has root access.

\item The UDP packet contains information which authorises it via a secure
protocol. For example, a private key signature of: a challenge sent to the
node, the content of the data and the metadata.

\end{itemize}

\item What is a \textit{UDP-based amplification attack} and why are similar
attacks far less practical via TCP?

A \textit{UDP-based amplification attack} is an attack where the attacker
sends a small packet to a server with the return address of the victim. The
server then responds to this request by sending a large amount of data to
the victim. The actual attacker can cause the victim to receive hundreds of
times more data than they actually sent. UDP is connectionless and does not
verify source addresses -- so this attack is trivial.

TCP is not connectionless, for the attacker to get the server to send data
to the victim, they would have to establish a connection with the server as
the victim. This would be challenging. Furthermore, on receiving the first
packet, the victim would send RST to the server closing the connection. So
the attacker would have to re-establish the connection repeatedly.

\item Name and briefly explain \textit{four} techniques that the designers
of C compilers and operating system kernels can implement to reduce the risk
of stack-based buffer-overflow attacks.

\begin{itemize}

\item Address Space Layout Randomisation (ASLR)

ALSR randomises the virtual addresses at which different parts of the stack
are placed. This means the absolute addresses cannot be determined with
certainty and reduces the chance of a successful attack. This can be
implemented by Operating System designers.

\item Non-executable Stack

Marking the stack as non-executable prevents the attacker from inputting the
bytecode which they wish to execute. However, the machine is still
vulnerable to a return-to-libc attack. This can be mitigated by using shells
which immediately revert to the real user ID -- so the worst that a user can
do is execute a shell with the permissions they already have. This approach
is taken by default in Ubuntu.

\item Shadow Stack

The program can keep a second copy of the stack (known as a shadow stack)
which only contains the return address and other basic information (no local
variables). When a function is called, the shadow stack is updated and when
a function returns, the value on the real stack is compared against the
value on the shadow stack. If they are inconsistent then the program
terminates. This must be implemented by C compiler designers.

\item Stack Canary

A Stack Canary is a word which comes immediately below the return address of
each function. It is a random sequence of bytes (of which a copy is stored
elsewhere, off the stack). On returning from a function, the program will
check that the stack canary matches the copy. If it does not, the program
will terminate, having detected ``stack smashing''.

\end{itemize}

\item How can an implementation of the C function \texttt{strcmp()} cause a
vulnerability to a side-channel attack, and how can this be mitigated.

If the C function \texttt{strcmp()} takes time proportional to the length of
the input which are the same (i.e  returns once failure is known) then we
can determine the value of the string we are comparing against by using a
timing attack.

A timing attack on a non-constant length function is as follows: try a set of
random strings. The one which took the longest time had the longest
matching prefix. Work forwards, adjusting characters until changing a
character makes the comparison no faster. This is the first character which
is incorrect. Alter it until the comparison is slower. Repeat for
subsequent characters.

This can be mitigated by making \texttt{strcmp()} take time which is only
dependent on the \textit{length} of the input. For example (assuming s1 is
the input string):

\end{enumerate}

\end{examquestion}

\part{Additional Questions}

\begin{enumerate}

\item Think of another real-world security protocol that predates
cryptography, similar to the example of ordering wine in a restaurant
described in lectures. Discuss any interesting properties of integrity,
authentication, non-repudiability, confidentiality, anonymity or deniability.

% TODO

\item Debunk each of the following software myths with reference to one
concrete case study or example which demonstrates the opposite can
sometimes be true:

\begin{itemize}

\item Computers are cheaper than analogue devices

% TODO

\item Software is easy to change

% TODO

\item Reuse of software increases safety

% TODO

\item Formal verification removes all errors

% TODO

\item More testing makes software more reliable

% TODO

\item Automation reduces risk

% TODO

\end{itemize}



\end{enumerate}

\end{document}
