\newcommand{\studentname}{Harry Langford}
\newcommand{\studentemail}{hjel2@cam.ac.uk}

\documentclass[10pt, a4paper]{article}

\input{./template/includes.tex}
% DO NOT add \usepackage commands here.  Place any custom commands
% into your SV work files.  Anything in the template directory is
% likely to be overwritten!

\usepackage{fancyhdr}

\usepackage{lastpage}       % ``n of m'' page numbering
\usepackage{lscape}         % Makes landscape easier

\usepackage{verbatim}       % Verbatim blocks
\usepackage{epsfig}         % Embed encapsulated postscript
\usepackage{array}          % Array environment
\usepackage[nolinks]{qrcode}         % QR codes
\usepackage{enumitem}       % Required by Tom Johnson's exam question header

\usepackage{hhline}         % Horizontal lines in tables
\usepackage{siunitx}        % Correct spacing of units
\usepackage{amsmath}        % American Mathematical Society
\usepackage{amssymb}        % Maths symbols
\usepackage{amsthm}         % Theorems

\usepackage{ifthen}         % Conditional processing in tex

\usepackage[top=3cm,
            bottom=3cm,
            inner=2cm,
            outer=5cm]{geometry}

% PDF metadata + URL formatting
\usepackage[
            pdfauthor={\studentname},
            pdftitle={\svcourse, SV \svnumber},
            pdfsubject={},
            pdfkeywords={9d2547b00aba40b58fa0378774f72ee6},
            pdfproducer={},
            pdfcreator={},
            hidelinks]{hyperref}

\renewcommand{\headrulewidth}{0.4pt}
\renewcommand{\footrulewidth}{0.4pt}
\fancyheadoffset[LO,LE,RO,RE]{0pt}
\fancyfootoffset[LO,LE,RO,RE]{0pt}
\pagestyle{fancy}
\fancyhead{}
\fancyhead[LO,RE]{{\bfseries \studentname}\\\studentemail}
\fancyhead[RO,LE]{{\bfseries \svcourse, SV~\svnumber}\\\svdate\ \svtime, \svvenue}
\fancyfoot{}
\fancyfoot[LO,RE]{For: \svrname}
\fancyfoot[RO,LE]{\today\hspace{1cm}\thepage\ / \pageref{LastPage}}
\fancyfoot[C]{\qrcode[height=0.8cm]{\svuploadkey}}
\setlength{\headheight}{22.55pt}

\ifthenelse{\equal{\jkfside}{oneside}}{

 \ifthenelse{\equal{\jkfhanded}{left}}{
  % 1. Left-handed marker, one-sided printing or e-marking, use oneside and...
  \evensidemargin=\oddsidemargin
  \oddsidemargin=73pt
  \setlength{\marginparwidth}{111pt}
  \setlength{\marginparsep}{-\marginparsep}
  \addtolength{\marginparsep}{-\textwidth}
  \addtolength{\marginparsep}{-\marginparwidth}
 }{
  % 2. Right-handed marker, one-sided printing or e-marking, use oneside.
  \setlength{\marginparwidth}{111pt}
 }

}{
 % 3. Alternating margins, two-sided printing, use twoside.
}

\setlength{\parindent}{0em}
\addtolength{\parskip}{1ex}

% Exam question headings, labels and sensible layout (courtesy of Tom Johnson)
\setlist{parsep=\parskip, listparindent=\parindent}
\newcommand{\examhead}[3]{\section{#1 Paper #2 Question #3}}
\newenvironment{examquestion}[3]{
    \examhead{#1}{#2}{#3}\setlist[enumerate, 1]{label=(\alph*)}\setlist[enumerate, 2]{label=(\roman*)}
    \marginpar{\qrcode{https://www.cl.cam.ac.uk/teaching/exams/pastpapers/y#1p#2q#3.pdf}}
    \marginpar{\footnotesize \url{https://www.cl.cam.ac.uk/teaching/exams/pastpapers/y#1p#2q#3.pdf}}
}{}



\usepackage{listings}
\usepackage{pdfpages}
\let\oldpdfpagesdonotuseitonlyincludesonepageunlessyousupplyanoptionalargument\includepdf
\renewcommand{\includepdf}[9]{}
\newcommand{\image}[1]{\oldpdfpagesdonotuseitonlyincludesonepageunlessyousupplyanoptionalargument[pages=-]{#1}}

\begin{document}

\begin{enumerate}[label=(\alph*)]

\item 

\begin{enumerate}

\item We're using a Na\"ive Bayes classifier. So the 
approach we're using is a smoothed probabilistic generative multiclass classifier.

The features I would use are whether each pixel is black or white. Given that the majority of all 
pixels will be black, I will model the presence of black pixels. The features are the colour of each of 
the individual pixels: so there are $150 \times 200 = 30000$ features. With this many features and any reasonably-sized 
training dataset, we will have a very high uncertainty on the probabilities of each feature given a 
class and each feature will be very uninformative.

The probability of an image $i$ being in a class $c$

\[
\log P(c|i) = \log P(c) + \sum_{w_i \in i} \lg P(w_i|c)
\]

Note that we are using log-probabilities for numerical stability (else all probabilities would tend to zero with an unusably 
high uncertainty rapidly).

Unfortunately, the probabilities are not independent -- so the Na\"ive Bayes assumption is proken since the 
probability of one pixel being black massively increases the probability of one of it's neighbours being black.

\item The training data is the set of pixels from each of the hand-drawn images. It was collected by human annotators who 
annotated each picture according to what it represented. 

The models parameters were estiamted according to the following formula:

\[
P(c) = \frac{\mathit{Count}(c)}{\sum_{c_i \in C}\mathit{Count}(c_i)}
\]

\[
P(w_i|c) = \frac{\mathit{Count}(w_i, c) + 1}{\mathit{Count}_{w_i \in V}(w_i) + |V|}
\]

Where $i$ is the image, $C$ is the set of classes (house, tent, boat and igloo) and 
$V$ is the set of all pixels.

\end{enumerate}

\item We should granulate the images -- reducing the resolution by merging several adjacent pixels into one 
pixel (this pixel being black if 3 or more constituent pixels are black). This will greatly reduce the uncertainty on the data. Currently the probability of any individual pixel 
being black is so low that it's got a very high uncertainty and fairly informative. If we were to merge 
4 adjacent pixels and turn the images into a $75 \times 100$ pixel images (now only having 7500 features rather 
than 300000 we would be able to get much better predictions). Further improvements could involve a higher degree 
of merging however we would need to experimentally test to find out the optimal degree of merging and what the feature should be 
(ie I suggest the resultant pixel should be black if 3 or more of the subpixels are black -- but maybe 4 or more or 2 or more would be more appropriate).

This should improve results since it massively decreases the uncertainty on probabilities.

\item 

\begin{enumerate}

\item Door or no Door, Sea or no Sea, lots of Curved lines or few Curved lines (what constitutes lots of curved 
lines would require testing), Roof or no Roof. 

Igloos and tent's don't have doors, only boats and houses can have the sea (although houses are far less 
likely to have the sea, while tents can not have the sea at all), houses and tents are less likely than 
igloos or boats to have curved lines while boats and igloos are 
likely to have more curved lines and only houses and boats will have rooves.

I think these features would be discriminative and would be possible to predict automatically.

These features would allow us to discriminate between the classes fairly well.

\begin{itemize}

\item Door or no Door

Houses and boats have doors so this feature would be discriminative. This would also be relatively easy 
to predict automatically since doors have a fairly uniform shape (square with a circle inside it).

\item Sea or no Sea

Almost all drawings of boats would have the sea while very few of any other category would have the sea. This 
feature would be very good at discriminating boats and very prevelant while still being moderatelly easy to 
predict automatically.

\item Lots of curved lines or few curved lines.

Igloos have a very high proportion of curved lines while the other classes have significantly fewer curved lines. 
This feature would be very discriminative for igloos -- although it would require testing to find out an appropriate 
boundary for this. This would also be a prevelant feature.

\item Roof or no roof

Houses and boats have rooves while the other classes don't usually. This would be discriminative and easy to make automatically  
due to the amount of shading on rooves.

\end{itemize}

\item We could incorporate the features described above as features in our pixel classifier. Our classifier 
would now have 7504 features -- four of which are high level.

The high level features would have a very low uncertainty compared to the pixels -- for example tents cannot be 
on the sea so anything on the sea would not be a tent. Our Na\"ive Bayes classifier is (should be) smoothed and so 
would move probability mass from higher probability features to this impossible feature. 
We could solve this by decreasing the smoothing for the high level features to account for this (so they'd 
still be smoothed but significantly less so -- for example add-one smoothing for the pixel values and add-0.001 smoothing 
for the high-level features).

\end{enumerate}

\end{enumerate}

\end{document}