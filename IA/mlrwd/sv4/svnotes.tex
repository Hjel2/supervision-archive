\documentclass[10pt, a4paper]{article}

% DO NOT add \usepackage commands here.  Place any custom commands
% into your SV work files.  Anything in the template directory is
% likely to be overwritten!

\usepackage{fancyhdr}

\usepackage{lastpage}       % ``n of m'' page numbering
\usepackage{lscape}         % Makes landscape easier

\usepackage{verbatim}       % Verbatim blocks
\usepackage{listings}       % Source code listings
\usepackage{graphicx}
\usepackage{float}
\usepackage{epsfig}         % Embed encapsulated postscript
\usepackage{array}          % Array environment
\usepackage{qrcode}         % QR codes
\usepackage{enumitem}       % Required by Tom Johnson's exam question header

\usepackage{hhline}         % Horizontal lines in tables
\usepackage{siunitx}        % Correct spacing of units
\usepackage{amsmath}        % American Mathematical Society
\usepackage{amssymb}        % Maths symbols
\usepackage{amsthm}         % Theorems

\usepackage{ifthen}         % Conditional processing in tex

\usepackage[top=3cm,
            bottom=3cm,
            inner=2cm,
            outer=5cm]{geometry}

% PDF metadata + URL formatting
\usepackage[
            pdfauthor={\studentname},
            pdftitle={\svcourse, SV \svnumber},
            pdfsubject={},
            pdfkeywords={9d2547b00aba40b58fa0378774f72ee6},
            pdfproducer={},
            pdfcreator={},
            hidelinks]{hyperref}

\renewcommand{\headrulewidth}{0.4pt}
\renewcommand{\footrulewidth}{0.4pt}
\fancyheadoffset[LO,LE,RO,RE]{0pt}
\fancyfootoffset[LO,LE,RO,RE]{0pt}
\pagestyle{fancy}
\fancyhead{}
\fancyhead[LO,RE]{{\bfseries \studentname}\\\studentemail}
\fancyhead[RO,LE]{{\bfseries \svcourse, SV~\svnumber}\\\svdate\ \svtime, \svvenue}
\fancyfoot{}
\fancyfoot[LO,RE]{For: \svrname}
\fancyfoot[RO,LE]{\today\hspace{1cm}\thepage\ / \pageref{LastPage}}
\fancyfoot[C]{\qrcode[height=0.8cm]{\svuploadkey}}
\setlength{\headheight}{22.55pt}


\ifthenelse{\equal{\jkfside}{oneside}}{

 \ifthenelse{\equal{\jkfhanded}{left}}{
  % 1. Left-handed marker, one-sided printing or e-marking, use oneside and...
  \evensidemargin=\oddsidemargin
  \oddsidemargin=73pt
  \setlength{\marginparwidth}{111pt}
  \setlength{\marginparsep}{-\marginparsep}
  \addtolength{\marginparsep}{-\textwidth}
  \addtolength{\marginparsep}{-\marginparwidth}
 }{
  % 2. Right-handed marker, one-sided printing or e-marking, use oneside.
  \setlength{\marginparwidth}{111pt}
 }

}{
 % 3. Alternating margins, two-sided printing, use twoside.
}


\setlength{\parindent}{0em}
\addtolength{\parskip}{1ex}

% Exam question headings, labels and sensible layout (courtesy of Tom Johnson)
\setlist{parsep=\parskip, listparindent=\parindent}
\newcommand{\examhead}[3]{\section{#1 Paper #2 Question #3}}
\newenvironment{examquestion}[3]{
\examhead{#1}{#2}{#3}\setlist[enumerate, 1]{label=(\alph*)}\setlist[enumerate, 2]{label=(\roman*)}
\marginpar{\href{https://www.cl.cam.ac.uk/teaching/exams/pastpapers/y#1p#2q#3.pdf}{\qrcode{https://www.cl.cam.ac.uk/teaching/exams/pastpapers/y#1p#2q#3.pdf}}}
\marginpar{\footnotesize \href{https://www.cl.cam.ac.uk/teaching/exams/pastpapers/y#1p#2q#3.pdf}{https://www.cl.cam.ac.uk/\\teaching/exams/pastpapers/\\y#1p#2q#3.pdf}}
}{}


\begin{document}

\section*{Machine Learning Supervision 4}

Third question is likely two questions on the first topic. Very unlikely to have two questions on 
HMM's. Could be a high level question ``if you want to do machine learning how od you split your 
datasets'' ``what's the point of the development set'' etc. Guesses. Fairly safe there will be 
something in Na\"ive Bayes, HMM's and perhaps more general machine learning methodology.



\section*{Sample}

In machine learning a lot of things are ``this would require empirical testing''

Note that you should answer what the question asks -- they ask you about 
probabilities then think of probabilities.

Default is including states and then saying they're all zero.

``Showing both probabilities'': include the probabilities. Can be shown by boxes 
with boxes and dotted lines showing the probability of meitting these states.

(c)

(i) The probabliity $a_{FL_2} = 0$ if no smoothing is used

(ii) In the training data $L_2E$ will never occur and $L_1E$ will never occur.

$a_{L_1E} a_{L_2E} = 0$

Note that all we can say about $a_{FE}$ is that it's nonzero.

(iii) In the training data, we will not see $L_2L_2L_2$. with the first order model, we cannot 
model this since the next state only occurs on the current state. However, we could change the model to 
a second order HMM.

We can say $a_{L_2L_2} \leq \frac{1}{2}$.

(iv) Think about edge cases.

\section*{Na\"ive Bayes}

Review binary features!!


(a)(i) Write the formula for na\"ive bayes and define the classes etc.

\[
\^c = \text{argmax} P(x) \times \prod_i P(f_i|c)
\]

\[
c \in \{\text{igloo, house, tent, boat}\}
\]

\[
\begin{split}
f_i = 1 \text{ if pixel $i$ is black}\\
      0 \text{ otherwise}\\
\end{split}
\]

(ii) Read the question! How was the data labelled? How did you \textit{get} the training data. Did you get people 
to annotate drawins or did you get people to draw houses and igloos etc.

(b) Explain better

(c) (i) Note they didn't ask you to \textit{find} the features -- just to say what the best would be.

(ii) Use the abstract feature classifier. If your confidence level si not great then backtrack to the 
raw pixel classifier.

If you just append features then they'll get swallowed by the 7500 other features.

\section*{General}

Features having values:

If a feature is a continuous variable then how do you model it in na\"ive bayes.

\end{document}
