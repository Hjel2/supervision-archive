\newcommand{\svcourse}{CST Part IA: Introduction to Probability}
\newcommand{\svnumber}{1}
\newcommand{\svvenue}{Churchill, Room TBD}
\newcommand{\svdate}{2022-05-14}
\newcommand{\svtime}{11:00}
\newcommand{\svuploadkey}{PO5ogKIM8KQA22FZS8IAf8gxA8XKi19jxIBVHIfFZ+3GCBXuNUXS9lVN6bNYjxM/}

\newcommand{\svrname}{Mr Matthew Ireland}
\newcommand{\jkfside}{twoside}
\newcommand{\jkfhanded}{right}

\newcommand{\studentname}{Harry Langford}
\newcommand{\studentemail}{hjel2@cam.ac.uk}


\documentclass[10pt,\jkfside,a4paper]{article}

\input{../../template/includes.tex}
% DO NOT add \usepackage commands here.  Place any custom commands
% into your SV work files.  Anything in the template directory is
% likely to be overwritten!

\usepackage{fancyhdr}

\usepackage{lastpage}       % ``n of m'' page numbering
\usepackage{lscape}         % Makes landscape easier

\usepackage{verbatim}       % Verbatim blocks
\usepackage{epsfig}         % Embed encapsulated postscript
\usepackage{array}          % Array environment
\usepackage[nolinks]{qrcode}         % QR codes
\usepackage{enumitem}       % Required by Tom Johnson's exam question header

\usepackage{hhline}         % Horizontal lines in tables
\usepackage{siunitx}        % Correct spacing of units
\usepackage{amsmath}        % American Mathematical Society
\usepackage{amssymb}        % Maths symbols
\usepackage{amsthm}         % Theorems

\usepackage{ifthen}         % Conditional processing in tex

\usepackage[top=3cm,
            bottom=3cm,
            inner=2cm,
            outer=5cm]{geometry}

% PDF metadata + URL formatting
\usepackage[
            pdfauthor={\studentname},
            pdftitle={\svcourse, SV \svnumber},
            pdfsubject={},
            pdfkeywords={9d2547b00aba40b58fa0378774f72ee6},
            pdfproducer={},
            pdfcreator={},
            hidelinks]{hyperref}

\renewcommand{\headrulewidth}{0.4pt}
\renewcommand{\footrulewidth}{0.4pt}
\fancyheadoffset[LO,LE,RO,RE]{0pt}
\fancyfootoffset[LO,LE,RO,RE]{0pt}
\pagestyle{fancy}
\fancyhead{}
\fancyhead[LO,RE]{{\bfseries \studentname}\\\studentemail}
\fancyhead[RO,LE]{{\bfseries \svcourse, SV~\svnumber}\\\svdate\ \svtime, \svvenue}
\fancyfoot{}
\fancyfoot[LO,RE]{For: \svrname}
\fancyfoot[RO,LE]{\today\hspace{1cm}\thepage\ / \pageref{LastPage}}
\fancyfoot[C]{\qrcode[height=0.8cm]{\svuploadkey}}
\setlength{\headheight}{22.55pt}

\ifthenelse{\equal{\jkfside}{oneside}}{

 \ifthenelse{\equal{\jkfhanded}{left}}{
  % 1. Left-handed marker, one-sided printing or e-marking, use oneside and...
  \evensidemargin=\oddsidemargin
  \oddsidemargin=73pt
  \setlength{\marginparwidth}{111pt}
  \setlength{\marginparsep}{-\marginparsep}
  \addtolength{\marginparsep}{-\textwidth}
  \addtolength{\marginparsep}{-\marginparwidth}
 }{
  % 2. Right-handed marker, one-sided printing or e-marking, use oneside.
  \setlength{\marginparwidth}{111pt}
 }

}{
 % 3. Alternating margins, two-sided printing, use twoside.
}

\setlength{\parindent}{0em}
\addtolength{\parskip}{1ex}

% Exam question headings, labels and sensible layout (courtesy of Tom Johnson)
\setlist{parsep=\parskip, listparindent=\parindent}
\newcommand{\examhead}[3]{\section{#1 Paper #2 Question #3}}
\newenvironment{examquestion}[3]{
    \examhead{#1}{#2}{#3}\setlist[enumerate, 1]{label=(\alph*)}\setlist[enumerate, 2]{label=(\roman*)}
    \marginpar{\qrcode{https://www.cl.cam.ac.uk/teaching/exams/pastpapers/y#1p#2q#3.pdf}}
    \marginpar{\footnotesize \url{https://www.cl.cam.ac.uk/teaching/exams/pastpapers/y#1p#2q#3.pdf}}
}{}


\usepackage[super]{nth}
\usepackage{listings}
\usepackage{multirow}
\DeclareMathOperator*{\argmax}{argmax}

\begin{document}

\section*{Markov assumption:}

State the two Markov assumptions, and explain why they are important in the
definition of Hidden Markov Models.

The two (first-order) Markov assumptions are:

\begin{itemize}

\item The next state is only dependent on the previous state.

\item The observation is dependent only on the current hidden state

\end{itemize}

These two observations mean that the hidden state sequence displays the 
optimal substructure 
property -- so we can employ dynamic programming to efficiently calculate the 
most likely sequence using the Viterbi algorithm -- or the most likely state 
when using the Forward algorithm.

\section*{HMM Artificial data:}

The data you were given with task 7 (parallel sequence of observations and states
created by the “dice” HMM) was artificially created using an HMM (remember
that we called HMMs and Naive Bayes \textbf{generative models}). In this exercise,
you will explore how this was done.

\begin{enumerate}

\item What is the information you need in order to be able to design an algorithm
for generating artifical data using an HMM?

The transition probabilities and emission probabilities.

\item Describe an algorithm for creating artificial data.

Begin in the start state. 

While you are not in the end state:

Move to the next state and emit an emission with probabilities determined by the 
transition and emission probabilities.

Either return the data in an appropriate format or save it to disk depending on the 
size of data we are creating.

\begin{lstlisting}[language=python]
def generate_data(n, start_state, end_state, start_emission, 
end_emission, transition_probs, emission_probs):
    import random
    stateset = list({[state for transition in transition_probs 
	for state in transition]})
    emissionset = list({map(lambda x: x[1], emission_probs)})
    transitions = {start: ([end for end in stateset], [
	transition_probs.get((start, end), 0) for end in stateset]) 
	for start in stateset}
    emissions = {start: ([end for end in emissionset], [
	emission_probs.get((start, end), 0) for end in emissionset]) 
	for start in stateset}
    sample_data = []
    for _ in range(n):
        hidden = [start_state]
        observed = [start_emission]
        state = random.choices(transitions[start_state][0], 
		transitions[start_state][1])
        while state != end_state:
            state = random.choices(transitions[state][0], 
			transitions[state][1])[0]
            observation = random.choices(emissions[state][0], 
			emissions[state][1])[0]
            hidden.append(state)
            observed.append(observation)
        observed.append(end_emission)
        sample_data.append((hidden, observed))
    return sample_data
\end{lstlisting}

\item Transition probabilities into the final state are expressed as an extra
parameter for an HMM. In some models these final transition probabilities
are irrelevant. Under what circumstances would the prediction result be
affected by transitions into the final state? Can you think of some examples
of real world situations where this might happen?

The transition into the final state is relevant when the probabilities of 
ending on different states have different probabilities. For example if we 
were speech tagging, the probability of ending on a conjunction is significantly 
less than ending on a noun.

To illustrate this, I downloaded a text version of the ``The Hobbit'' and ``The Lord of the Rings'' 
by J.R.R.Tolkien and tokenised it using nltk. These are the results:

\begin{center}
\begin{tabular}{c|c|c|c}
Type & Times Seen & Final Token & Final Probability \\
\hline
coordinating conjunction & 32254 & 31 & 0.00096 \\
cardinal digit & 3372 & 161 & 0.048 \\
determiner & 58625 & 460 & 0.0078 \\
existential there & 2460 & 13 & 0.0053 \\
foreign word & 34 & 4 & 0.12 \\
preposition/subordinating conjunction & 71251 & 903 & 0.013 \\
adjective (large) & 31525 & 2206 & 0.070 \\
adjective, comparative (larger) & 1817 & 138 & 0.076 \\
adjective, superlative (largest) & 978 & 67 & 0.069 \\
list market & 8 & 3 & 0.38 \\
modal (could, will) & 10185 & 232 & 0.023 \\
noun, singular (cat, tree) & 70923 & 12312 & 0.17 \\
proper noun, singular (sarah) & 30689 & 5933 & 0.19 \\
proper noun, plural (indians or americans) & 245 & 43 & 0.18 \\
noun plural (desks) & 24991 & 3560 & 0.14 \\
predeterminer (all, both, half) & 957 & 61 & 0.064 \\
possessive ending (parent's) & 2791 & 3450 & 0.0022 \\
personal pronoun (hers, herself, him, himself) & 48886 & 3450 & 0.071 \\
possessive pronoun (her, his, mine, my, our ) & 12341 & 16 & 0.0013 \\
adverb (occasionally, swiftly) & 40118 & 4211 & 0.10 \\
adverb, comparative (greater) & 1066 & 121 & 0.11 \\
adverb, superlative (biggest) & 146 & 5 & 0.034 \\
particle (about) & 3724 & 380 & 0.10 \\
infinite marker (to) & 12023 & 66 & 0.0055 \\
interjection (goodbye) & 459 & 78 & 0.17 \\
verb (ask) & 22263 & 1618 & 0.073 \\
verb past tense (pleaded) & 42149 & 1940 & 0.045 \\
verb gerund (judging) & 10149 & 684 & 0.067 \\
verb past participle (reunified) & 14527 & 1539 & 0.11 \\
verb, present tense not \nth{3} person (wrap) & 11408 & 536 & 0.046 \\
verb, present tense with \nth{3} person singular & 7858 & 243 & 0.031 \\
determiner (that, what) & 2281 & 5 & 0.0022 \\
pronoun (who) & 2347 & 25 & 0.010 \\
adverb (how) & 2679 & 17 & 0.018 \\
Total & 679628 & 41191 & \\
\end{tabular}
\end{center}

In this circumstance we can see that final state transitions are very relevant. 
There is a factor of 200 difference between the probability of ending on a conjunction 
and ending on a proper noun!

\begin{lstlisting}[language=python]
import nltk
from tqdm import tqdm


endings = {}
totals = {}
with open('lotr.txt') as lotr:
	for line in tqdm(lotr.readlines(10000)):
		tokens = nltk.word_tokenize(line)
		tagged = nltk.pos_tag(tokens)
		for i in range(len(tagged) - 1):
			totals[tagged[i][1]] = totals.get(
			tagged[i][1], 0) + 1
			if tagged[i + 1][0] in ('.', '!', '?'):
				endings[tagged[i][1]] = endings.get(
				tagged[i][1], 0) + 1
		if tagged:
			totals[tagged[-1][1]] = totals.get(
			tagged[-1][1], 0) + 1
print(totals)
print(endings)
\end{lstlisting}

% [("''", 0.004082), (')', 0.217857), (',', 8.2e-05), ('.', 0.000291), (':', 0.000459), ('CC', 0.000961), ('CD', 0.047746), ('DT', 0.007846), ('EX', 0.005285), ('FW', 0.117647), 
% ('IN', 0.012674), ('JJ', 0.069976), ('JJR', 0.075949), ('JJS', 0.068507), ('LS', 0.375), ('MD', 0.022779), ('NN', 0.173597), ('NNP', 0.193327), ('NNPS', 0.17551), ('NNS', 0.142451), 
% ('POS', 0.021856), ('PRP', 0.070572), ('PRP$', 0.001296), ('RB', 0.104965), ('RBR', 0.113508), ('RBS', 0.034247), ('RP', 0.102041), ('TO', 0.005489), ('UH', 0.169935), ('VB', 0.072677), 
% ('VBD', 0.04496), ('VBG', 0.067396), ('VBN', 0.105941), ('VBP', 0.046985), ('VBZ', 0.030924), ('WDT', 0.002192), ('WP', 0.010476), ('WP$', 0.017857), ('WRB', 0.006346)]

% Total Occurrences:
% [("''", 12494), ('(', 294), (')', 280), (',', 36458), ('.', 41193), (':', 6535), ('CC', 32254), ('CD', 3372), ('DT', 58625), ('EX', 2460), ('FW', 34), ('IN', 71251), ('JJ', 31525), 
% ('JJR', 1817), ('JJS', 978), ('LS', 8), ('MD', 10185), ('NN', 70923), ('NNP', 30689), ('NNPS', 245), ('NNS', 24991), ('PDT', 957), ('POS', 2791), ('PRP', 48886), ('PRP$', 12341), 
% ('RB', 40118), ('RBR', 1066), ('RBS', 146), ('RP', 3724), ('SYM', 11), ('TO', 12023), ('UH', 459), ('VB', 22263), ('VBD', 43149), ('VBG', 10149), ('VBN', 14527), ('VBP', 11408), 
% ('VBZ', 7858), ('WDT', 2281), ('WP', 2291), ('WP$', 56), ('WRB', 2679), ('``', 3834)]

% Times a sentence ended on one:
% [("''", 51), (')', 61), (',', 3), ('.', 12), (':', 3), ('CC', 31), ('CD', 161), ('DT', 460), ('EX', 13), ('FW', 4), ('IN', 903), ('JJ', 2206), ('JJR', 138), ('JJS', 67), ('LS', 3), 
% ('MD', 232), ('NN', 12312), ('NNP', 5933), ('NNPS', 43), ('NNS', 3560), ('POS', 61), ('PRP', 3450), ('PRP$', 16), ('RB', 4211), ('RBR', 121), ('RBS', 5), ('RP', 380), ('TO', 66), 
% ('UH', 78), ('VB', 1618), ('VBD', 1940), ('VBG', 684), ('VBN', 1539), ('VBP', 536), ('VBZ', 243), ('WDT', 5), ('WP', 24), ('WP$', 1), ('WRB', 17)]

% Raw dictionaries:
% {'DT': 58625, 'NNP': 30689, 'IN': 71251, 'NN': 70923, 'RB': 40118, 'VBD': 43149, '.': 41193, 'JJ': 31525, ',': 36458, 'VBN': 14527, 'NNS': 24991, 'CC': 32254, 'PRP': 48886, 'TO': 12023, 'VB': 22263, 'RP': 3724, ':': 6535, 'VBZ': 7858, 'VBG': 10149, 'PDT': 957, 'CD': 3372, '(': 294, ')': 280, 'WP': 2291, 'JJS': 978, 'PRP$': 12341, 'MD': 10185, 'WRB': 2679, 'POS': 2791, 'VBP': 11408, 'JJR': 1817, 'EX': 2460, 'WDT': 2281, 'RBS': 146, 'RBR': 1066, '``': 3834, "''": 12494, 'FW': 34, 'UH': 459, 'NNPS': 245, 'WP$': 56, 'LS': 8, 'SYM': 11}
% {'NN': 12312, 'NNS': 3560, 'DT': 460, 'NNP': 5933, 'PRP': 3450, 'JJ': 2206, 'RP': 380, ')': 61, 'IN': 903, 'JJR': 138, 'RB': 4211, 'VB': 1618, 'VBD': 1940, 'VBN': 1539, 'VBP': 536, 'VBG': 684, 'CD': 161, 'POS': 61, 'CC': 31, 'VBZ': 243, 'RBR': 121, 'TO': 66, 'WRB': 17, 'MD': 232, 'WP': 24, 'RBS': 5, 'JJS': 67, 'UH': 78, 'WDT': 5, 'FW': 4, 'NNPS': 43, '.': 12, 'PRP$': 16, 'EX': 13, ':': 3, 'LS': 3, ',': 3, "''": 51, 'WP$': 1}

\end{enumerate}

\section*{Smoothing in HMMs:}

We did not smooth the Dice HMM in task 7 nor did you smooth the protein
HMM in task 9.

\begin{enumerate}

\item In which situations can smoothing be counterproductive, and why?

Smoothing makes models better when we have classes with zero or near-zero probabilities 
due to lack of data rather than a true zero probability. However with HMM's, there are 
some cases where we do not want this.

\begin{itemize}

\item Sometimes for HMM's, it's easy to get large amounts of training data -- for example if we 
have historical data, can generate data ourselves or can get data by web scraping or from a large 
online repository.

If we have a 
large amount of training data and a relatively small amount of hidden states and 
classes of observations (such as in the dice databset), then we have a very low uncertainty in 
the transition and emission probabilities. Since we are so certain our probabilities are 
accurate, smoothing only skews the model marginally with no benefit. In this case we would not 
want to smooth.

\item Note that in HMM's we have ``dummy'' start and end states which should have emission and transition 
probabilities of 0 or 1. Smoothing would mean there was a nonzero probability of introducing a 
start or end state in the middle of the sequence -- which is not possible.

\item Sometimes with things we model with HMM's, certain transitions or emissions are physically impossible. 
We want them to ahve a zero probability.

For example if the loaded dice in task 7 did not have a 1, then we 
would never want our HMM to predict that the loaded dice rolled a 1.
If we smoothed the probabilities then this impossibility observation becomes possible in our model.

\end{itemize}

\item In the case of the protein model, which of the two types of probability are
better candidates for smoothing and why?

We would want smoothing in a HMM if we have a large amount of states and emissions and so a 
high uncertainty on the transition and emission probabilities. Particularly if there are 
real sequences of observations where every hidden-state sequence has zero probability.

If we have $n$ hidden states and $m$ emissions, then the number of transition probabilities we have is 
$n^2$ and the number of emission probabilities we have is $nm$. Since $n > m$ in the protein model, I would 
suggest smoothing the transition probabilities since they have the highest uncertainty.

\end{enumerate}

\section*{Veterbi and Forward algorithm:}

Study the Forward algorithm in the Jurafsky and Martin textbook. This is the
algorithm for estimating the likelihood of an observation. It is another instance
of the dynamic programming paradigm.

\begin{enumerate}

\item Give and explain the recursive formula for this dynamic programming
algorithm in terms of $a_{ij}$ and $b_i(o_t)$.

Let $P(i_t)$ be the probability of being in state $i$ at time $t$.

\[
P(j_{t+1}) = b_j(o_{t+1}) \sum^{n-1}_{i=0} P(i_t) \cdot a_{ij}
\]

The probability of being in state $j$ at time $t + 1$ is equal to the sum of the 
probabilities of being in state $i$ at time $t$ and going from state $i$ to $j$ multiplied 
by the probability of state $j$ emitting the observation seen at time $t + 1$.

Note that the base case is that the probability of being in the start state at $t=0$ is 1.

\item Explain why there is a summation over the paths.

We are trying to work out the state which is most likely at a given time. This can 
be from any path. So we have to sum over the endpoints of all possible paths.

\end{enumerate}

\section*{Parts of Speech tagging with HMM:}

Hidden Markov Models (HMM) can be used for \textbf{Part of Speech Tagging}.
This is the task of assigning parts of speech, such as \textbf{verb}, 
\textbf{noun}, \textbf{pronoun}, determiner to words in a text sample.

A particular HMM is defined as follows: $S_e$ = \{$s_1$ = verb; $s_2$ = noun, $s_3$ =
personal pronoun, $s_4$ = auxiliary verb\}; $s_0$, $s_F$ designated start state and end
state.

\[A = 
\begin{bmatrix}
a_{01} = 0.01 & a_{02} = 0.10 & a_{03} = 0.60 & a_{04} = 0.29 & & \\
a_{11} = 0.02 & a_{12} = 0.63 & a_{13} = 0.07 & a_{14} = 0.13 & a_{1f} = 0.15 \\
a_{21} = 0.49 & a_{22} = 0.20 & a_{23} = 0.10 & a_{24} = 0.01 & a_{2f} = 0.20 \\
a_{31} = 0.40 & a_{32} = 0.05 & a_{33} = 0.05 & a_{34} = 0.40 & a_{3f} = 0.10 \\
a_{41} = 0.73 & a_{42} = 0.01 & a_{43} = 0.15 & a_{44} = 0.01 & a_{4f} = 0.10 \\
\end{bmatrix}
\]

\[B = 
\begin{bmatrix}
b_1(fish) = 0.89 & b_2(fish) = 0.75 & b_3(fish) = 0 & b_4(fish) = 0 \\
b_1(can) = 0.10 & b_2(can) = 0.24 & b_3(can) = 0 & b_4(can) = 1 \\
b_1(we) = 0.01 & b_2(we) = 0.01 & b_3(we) = 1 & b_4(we) = 0 \\
\end{bmatrix}
\]

The observation sequence $O$ is the following: $O$ = We can fish

\begin{enumerate}

\item Consider the two state sequences $X_a$ = $s_0$, $s_3$, $s_4$, $s_1$, $s_f$ and $X_b$ =
$s_0$, $s_3$, $s_1$, $s_2$, $s_f$ . Which interpretations of the above 
observation sequence do they represent?

The first sequence interprets the sequence as meaning that the people 
speaking are able to go fishing.

The second sequence interprets the sequence as meaning that the people 
speaking are ``canning'' fish -- ie they are putting the fish into cans.

\item Give the probabilities $P(X_a)$, $P(X_b)$, $P(X_a, O)$ and $P(X_b, O)$, and state
which of these probabilities are used in the HMM.

\begin{equation}
\begin{split}
P(X_a) &= a_{03} \times a_{34} \times a_{41} \times a_{1f} \\
P(X_a) &= 0.60 \times 0.40 \times 0.73 \times 0.15 \\
p(X_a) &= 0.02628 \\
\end{split}
\end{equation}

\begin{equation}
\begin{split}
P(X_a) &= a_{03} \times a_{31} \times a_{12} \times a_{2f} \\
P(X_a) &= 0.60 \times 0.40 \times 0.63 \times 0.20 \\
p(X_a) &= 0.03024 \\
\end{split}
\end{equation}

\begin{equation}
\begin{split}
P(X_a) &= a_{03} \times b_3(we) \times a_{34} \times b_4(can) \times a_{41} \times b_1(fish) \times a_{1f} \\
P(X_a) &= 0.60 \times 1 \times 0.40 \times 1 \times 0.73 \times 0.89 \times 0.15 \\
p(X_a) &= 0.0233892 \\
\end{split}
\end{equation}

\begin{equation}
\begin{split}
P(X_a) &= a_{03} \times b_3(we) \times a_{31} \times b_1(can) \times a_{12} \times b_2(fish) \times a_{2f} \\
P(X_a) &= 0.60 \times 1 \times 0.40 \times 0.10 \times 0.63 \times 0.75 \times 0.20 \\
p(X_a) &= 0.002268 \\
\end{split}
\end{equation}

The HMM will use $P(X_a, O)$ and $P(X_b, O)$.

\item Demonstrate the use of the Viterbi algorithm for deriving the most probable
sequence of parts of speech given $O$ above. Explain your notation and
intermediate results.

\begin{center}
\begin{tabular}{|c|c c c c c c|}
 & \multicolumn{6}{c}{State} \\
\hline 
Word & $s_0$ & $s_1$ & $s_2$ & $s_3$ & $s_4$ & $s_f$ \\
 & 1 & 0 & 0 & 0 & 0 & 0 \\
``we'' & 0 & 0.0001, $s_0$ & 0.001, $s_0$ & 0.6, $s_0$ & 0.29, $s_0$ & 0 \\
``can'' & 0 & 0.024, $s_3$ &  0.0072, $s_3$ & 0 & 0.24, $s_3$ & 0 \\
``fish'' & 0 & 0.155928, $s_4$ & 0.01134, $s_1$ & 0 & 0 & 0 \\
 & 0 & 0 & 0 & 0 & 0 & 0.0233892, $s_1$ \\
\end{tabular}
\end{center}

We can backtrack from the most likely end state. This gives that the most likely 
sequence is:
\[s_0 \longrightarrow s_3 \longrightarrow s_4 \longrightarrow s_1 \longrightarrow s_f\]

\item Does the model arrive at the correct disambiguation? If so, how does it
achieve this? If not, what could you change so that it does?

This model did arrive at the disambiguation which I believe is most likely. Note that 
without further context or a labelling from the author, it is impossible to know what 
the ``correct'' disambiguation is -- for all we know the author could work in a cannery.

It achieved this by using the transition and emission 
probabilities to work out the probability of arriving at every state 
at every word by it's most likely sequence. We keep references to which state precedes 
this state in its most likely sequence. 

Since HMM's exhibit optimal substructure, we 
know that the most likely sequencce leading to the previous state is a subsequence of 
the most likely sequence leading to the end state.

We then used this fact at the final iteration: we take the state at the end of the path 
with the highest probability and work backwards. This will give us the path with the highest 
probability.

\item If a labelled sample of text is available, then the emission probability
matrix $B$ can be estimated from a labelled sample of text. Describe one
way how this can be done.

We can estimate the probablility of a hidden state $S$ emitting an observation 
$O$ by dividing the total amount of times which the state $S$ appeared.

\item The statistical laws of language imply that there is a potential problem
when training emission probabilities for words. This problem manifests
itself in the probability of the word \textit{can} in the state sequence $X_b$ from (a)
above. What is the problem, and how could it be fixed?

I had two ideas about issues with viterbi which relate to the statistical laws of language -- 
unseen words (no dataset is large enough to contain every word you could possibly see by Heaps' Law); 
and the probabilities of most words occuring being very small by Zipf's law. I could not see 
any strong links between either of these and $X_b$ so I tried to invent a tenuous one.

Zipf's Law states that the frequency of the $\text{i}^{\text{th}}$ most 
common word is inversely proportional to its rank. This means that many less frequent words 
will have \textit{very} low probabilities of being emitted by particular states. 
This can be problematic.

Since we multiply 
probabilities (and all probabilities are positive and $\leq 1$), we will end up with 
$\lim_{P(x_t) \longrightarrow 0}$ -- meaning that either the uncertainty on values 
due to float precision will make the algorithm behave seemingly randomly or we will 
have to use such a high float precision that computation is intractable.

Note that in $X_b$, the probability reached 0.002 after only three (moderately common) words! 
If we were to analyse a nontrivial example -- say one page of text discussing a niche topic 
using this method then we would have 
paths with probabilities in the order of $1.36 \times 10^{-350}$ (extrapolating our toy 
example gives this value -- far smaller values would be expected in practice). 
Even for this example, I had to use a special library to get high enough 
precision. We would either have to use special libraries or all probabilities would 
round to 0, leaving us with a completely useless model.

The way to solve this is to use log-probabilities. So rather than multiplying the original 
probabilities, we should sum the logarithmic probabilities. This ensures that the numbers we 
deal with are large enough to be represented and can be manipulated easily. Note also 
that the cost of multiplication is far higher than that of addition.

Using this method, we can comfortably represent probabilities down to $2^{-10^{300}}$ without 
using any special libraries. This covers the whole sample space of possible inputs.

\end{enumerate}

\section*{Viterbi with higher order HMMs:}

Viterbi is a clever algorithm that allows you to process the input in time that is
linear to the observation sequence. With a first order HMM, we keep $N$ (number
of states) maximum probabilities per observation at each step.

\begin{enumerate}

\item How many states do we need to keep for an $N$ order HMM?

If there are $M$ hidden states then to keep an $N$ order HMM we need 
$M^{N}$ states.

IE for a first order HMM we need $M$ states, for a second order HMM 
we need $M^2$ states etc.

\item What are the implications for the asymptotic complexity of Viterbi?

The complexity of Viterbi is $\Theta(LM^{N + 1})$ where $L$ is the length of the 
input.

\end{enumerate}

\begin{examquestion}{2021}{3}{8}

You are a 22nd century historian researching the “FEE” (First Epidemic Era) of
2019–2025, for which records are patchy. You research which government policy was
in place in any given week during this historic phase. Policies, in order of severity,
are: No restrictions, Tier 1, Tier 2, Tier 3, and Lockdown.

\begin{enumerate}

\item From other historic sources, you know the following about sequences of policy
levels: if you are in a given policy level, there is a 40\% chance you will stay
there, a 20\% chance that you will be upgraded to the next-highest (more
severe) level next week, and a 10\% chance that you will be downgraded to
the next-lowest (less severe) policy level. The background lockdown probability
(which applies if nothing more informative is known about lockdown) is 10\%. For
each observation sequence, there is also a 5\% chance of the sequence ending at
any point. Transitions to any other policy level beyond those already described
are equally likely. Observation sequences begin with each policy level at equal
likelihood.

Using the information given above, construct the full transition probability table.

\begin{center}
\begin{tabular}{c|c|c c c c c c c|}
& & \multicolumn{7}{c}{Next State} \\
\hline
& & Start & No Restriction & Tier 1 & Tier 2 & Tier 3 & Lockdown & End \\
\hline
\multirow{7}{*}{\rotatebox[origin=c]{90}{~Current~State}} & Start & 0 & 0.2 & 0.2 & 0.2 & 0.2 & 0.2 & 0 \\
& No Restriction & 0 & 0.4 & 0.2 & 0.125 & 0.125 & 0.1 & 0.05 \\
& Tier 1 & 0 & 0.1 & 0.4 & 0.2 & 0.15 & 0.1 & 0.05 \\
& Tier 2 & 0 & 0.15 & 0.1 & 0.4 & 0.2 & 0.1 & 0.05 \\
& Tier 3 & 0 & 0.125 & 0.125 & 0.1 & 0.4 & 0.2 & 0.05 \\
& Lockdown & 0 & 0.15 & 0.15 & 0.15 & 0.1 & 0.4 & 0.05 \\ 
& End & 0 & 0 & 0 & 0 & 0 & 0 & 0 \\
\hline
\end{tabular}
\end{center}

\item You want to estimate which policy was in place for the first six weeks of 2025,
but unfortunately, the only information you have about this is a sequence of
Covid case numbers for these six weeks:

\begin{equation}
[0 - 99], [0 - 99], [> 200], [> 200], [> 200], [100 - 199]
\end{equation}

You know that case loads are associated with policy levels as in the Table below.
Describe how you can calculate the sequence of most likely policy levels for these
6 weeks, giving numbers for at least three steps of the calculation. Assume that
all policies are equally likely in the week preceding the first week.

\begin{tabular}{|l|l|l|l|l|l|}
\hline
& No Restriction & Tier 1 & Tier 2 & Tier 3 & Lockdown \\
\hline
0 - 99 cases & 5\% & 20\% & 20\% & 50\% & 90\% \\
100 - 199 cases & 15\% & 40\% & 30\% & 40\% & 9\% \\
> 200 cases & 80\% & 50\% & 20\% & 20\% & 1\% \\
\hline
\end{tabular}

We can calculate the sequence of most likely policy levels by building a hidden markov model with the transition 
probabilities found in part (a) and emission probabilities given above. Then perform the viterbi algorithm 
on it to find the most likely sequence of restriction levels.

To make the maths more understandable, I will use normal probabilities rather than log probabilities. In a 
real hidden markov model, we would either use log probabilities or normalise probabilities so the values 
do not tend to zero.

Let $R$ denote the set of possible policies. Let $R_n$ denote the probabilities of policies on the 
$n^\text{th}$ week.

The probability of a given policy $S$ on the $(k + 1)^\text{th}$ week is equal to $P(o|S)\times \argmax_{r \in R_i}(P(r) \times P(r\rightarrow S))$ 
where $P(o|S)$ is the probability of emitting the observation seen on the $(k + 1)^{\text{th}}$ week given the 
hidden state is $S$ and $P(r \longrightarrow S)$ is the probability of transitioning from state $r$ to state $S$. 
We start in the start state and at the end multiply the probability by the probability of a transition from that 
state to the end state.

When we have run out of all observations we take $\argmax_{r \in R_n} P(r)$ and work backwards using the 
pointers we have to the previous state in that sequence.

Applying this gives the following table:

In the table the following abbreviations are used:

\begin{tabular}{c c}
Start $\rightarrow$ ST & No Restrictions $\rightarrow$ NR \\
Tier 1 $\rightarrow$ T1 & Tier 2 $\rightarrow$ T2 \\
Tier 3 $\rightarrow$ T3 & Lockdown $\rightarrow$ LD \\
End $\rightarrow$ ED & $\times 10^{n} \rightarrow e^{n}$ \\
\end{tabular}

Note that all probabilities are rounded to 3 S.F. for presentation -- actual calculation was 
done at full precision.

\begin{tabular}{|c| c c c c c c c|}
\hline
Week & ST & NR & T1 & T2 & T3 & LD & ED \\
\hline
0 & 1, \null & 0, \null & 0, \null & 0, \null & 0, \null & 0, \null & 0, \null \\
1 & 0, \null & $1.00e^{-2}$, ST & $4.00e^{-2}$, ST & $4.00e^{-2}$, ST & $1.00e^{-1}$, ST & $1.80e^{-1}$, ST & 0, \null \\
2 & 0, \null & $1.35e^{-3}$, LD & $5.40e^{-3}$, LD & $5.40e^{-2}$, LD & $2.00e^{-2}$, T3 & $6.48e^{-2}$, LD & 0, \null \\
3 & 0, \null & $7.78e^{-3}$, LD & $4.86e^{-3}$, LD & $1.94e^{-2}$, LD & $1.60e^{-3}$, T3 & $2.59e^{-3}$, LD & 0, \null \\
4 & 0, \null & $2.49e^{-3}$, NR & $9.72e^{-4}$, Tier 1 & $1.94e^{-3}$, NR & $1.94e^{-4}$, NR & $7.78e^{-6}$, NR & 0, \null \\
5 & 0, \null & $7.96e^{-4}$, NR & $2.49e^{-4}$, NR & $6.22e^{-5}$, NR & $6.22e^{-5}$, NR & $2.49e^{-6}$, NR & 0, \null \\
6 & 0, \null & $4.78 e^{-5}$, NR & $6.37e^{-5}$, NR & $2.99e^{-5}$, NR & $3.98e^{-5}$, NR & $7.17e^{-6}$, NR & 0, \null \\
7 & 0, \null & 0, \null & 0, \null & 0, \null & 0, \null & 0, \null & $3.19e^{-6}$, T1 \\
\hline
\end{tabular}

By this table we can conclude that the last state in the most likely sequence was ``Tier 1''.

We can repeatedly backtrack to find the states in the most likely sequence.

\begin{tabular}{c|c c c c c c}
Week & 1 & 2 & 3 & 4 & 5 & 6 \\
\hline
Policy & Lockdown & Lockdown & No Restrictions & No Restrictions & No Restrictions & Tier 1
\end{tabular}

\item In which respects is the modelling described above not fully adequate to describe
an actual epidemic?

I don't really know where to begin. Put simply this is an \textit{awful} model and 
basically no part of it is suitable when scrutinized.

I saw six reasons why the modelling is not adequate to describe an actual 
epidemic:

\begin{itemize}

\item The first-order markov assumption is not suitable since it takes more than one week for restrictions 
to have any affect on case numbers.

Coronavirus and other viruses have incubation periods -- this means that there is a lag between 
infection and recorded cases. As a result it takes more than one week for any restrictions to be 
reflected in cases.

\item The next state is not only dependent on the previous state but also on the previous \textit{observations}.

Restrictions are implemented in \textit{response} to case numbers, which change as a result of restrictions. 
This is not a dependency relationship between case numbers and restrictions and so is unsuitable 
as a hidden markov model.

\item The emission probabilities are unrealistic.

I couldn't tell if this was intentional or if the question is just dated but it was so
obvious to me that I had to mention it -- historically lockdowns are implemented in response 
to high case numbers -- so high case numbers would indicate that there was a lockdown. Not the 
other way round as this suggests.

\item Transition and emission probabilities are not constant for different times and different countries.

As new variantes emerge, testing programs get better and vaccination rates get better, the meaning 
behind case rates changes. IE in the first lockdown we had far fewer recorded cases than we do now 
and yet the impact was far higher.

Additionally, different countries had different reasons for case rates. IE as an extreme example 
some island nations had low or zero case rates throughout despite having no restrictions because 
of a closed-border policy. This model would predict they were in lockdown throughout.

\item The hidden states are not well-defined.

This model is meant to be generalised across different countries. Different countries had different approaches. 
IE New Zealand had very few domestic restrictions but a very tight travel policy throughout the pandemic -- 
which category does this belong to?

Some countries also had regional approaches. IE when the tiered approach was implemented and regions had 
restrictions ranging from tier 1 to tier 3 -- what would the restrictions of the UK be?

\item The observations are neither broad nor fine-grained enough and are not well-defined.

The category 200+ is the overwhelming majority weeks in every country throughout the pandemic. 
As an example, Britain, France, Germany and Spain have all had 200+ cases in every single one of the 
108 weeks since March 2020 -- despite all four countries having varying levels of restrictions throughout.

Additionally, there is a huge and important difference between 0 weekly cases and 1 - 99. This is not reflected 
in the model.

The case rates are not specified to be ``total cases'' or ``cases per 100k''. There is a very important 
distinction which is never explicitly stated. I have assumed these are total cases (as is implied) 
although the classes would match cases per 100k better.

\end{itemize}

\end{enumerate}

\end{examquestion}

\end{document}