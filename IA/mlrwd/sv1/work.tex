\newcommand{\svcourse}{CST Part IA: Introduction to Probability}
\newcommand{\svnumber}{1}
\newcommand{\svvenue}{Churchill, Room TBD}
\newcommand{\svdate}{2022-05-14}
\newcommand{\svtime}{11:00}
\newcommand{\svuploadkey}{PO5ogKIM8KQA22FZS8IAf8gxA8XKi19jxIBVHIfFZ+3GCBXuNUXS9lVN6bNYjxM/}

\newcommand{\svrname}{Mr Matthew Ireland}
\newcommand{\jkfside}{twoside}
\newcommand{\jkfhanded}{right}

\newcommand{\studentname}{Harry Langford}
\newcommand{\studentemail}{hjel2@cam.ac.uk}


\documentclass[10pt,\jkfside,a4paper]{article}

\input{../../template/includes.tex}
% DO NOT add \usepackage commands here.  Place any custom commands
% into your SV work files.  Anything in the template directory is
% likely to be overwritten!

\usepackage{fancyhdr}

\usepackage{lastpage}       % ``n of m'' page numbering
\usepackage{lscape}         % Makes landscape easier

\usepackage{verbatim}       % Verbatim blocks
\usepackage{epsfig}         % Embed encapsulated postscript
\usepackage{array}          % Array environment
\usepackage[nolinks]{qrcode}         % QR codes
\usepackage{enumitem}       % Required by Tom Johnson's exam question header

\usepackage{hhline}         % Horizontal lines in tables
\usepackage{siunitx}        % Correct spacing of units
\usepackage{amsmath}        % American Mathematical Society
\usepackage{amssymb}        % Maths symbols
\usepackage{amsthm}         % Theorems

\usepackage{ifthen}         % Conditional processing in tex

\usepackage[top=3cm,
            bottom=3cm,
            inner=2cm,
            outer=5cm]{geometry}

% PDF metadata + URL formatting
\usepackage[
            pdfauthor={\studentname},
            pdftitle={\svcourse, SV \svnumber},
            pdfsubject={},
            pdfkeywords={9d2547b00aba40b58fa0378774f72ee6},
            pdfproducer={},
            pdfcreator={},
            hidelinks]{hyperref}

\renewcommand{\headrulewidth}{0.4pt}
\renewcommand{\footrulewidth}{0.4pt}
\fancyheadoffset[LO,LE,RO,RE]{0pt}
\fancyfootoffset[LO,LE,RO,RE]{0pt}
\pagestyle{fancy}
\fancyhead{}
\fancyhead[LO,RE]{{\bfseries \studentname}\\\studentemail}
\fancyhead[RO,LE]{{\bfseries \svcourse, SV~\svnumber}\\\svdate\ \svtime, \svvenue}
\fancyfoot{}
\fancyfoot[LO,RE]{For: \svrname}
\fancyfoot[RO,LE]{\today\hspace{1cm}\thepage\ / \pageref{LastPage}}
\fancyfoot[C]{\qrcode[height=0.8cm]{\svuploadkey}}
\setlength{\headheight}{22.55pt}

\ifthenelse{\equal{\jkfside}{oneside}}{

 \ifthenelse{\equal{\jkfhanded}{left}}{
  % 1. Left-handed marker, one-sided printing or e-marking, use oneside and...
  \evensidemargin=\oddsidemargin
  \oddsidemargin=73pt
  \setlength{\marginparwidth}{111pt}
  \setlength{\marginparsep}{-\marginparsep}
  \addtolength{\marginparsep}{-\textwidth}
  \addtolength{\marginparsep}{-\marginparwidth}
 }{
  % 2. Right-handed marker, one-sided printing or e-marking, use oneside.
  \setlength{\marginparwidth}{111pt}
 }

}{
 % 3. Alternating margins, two-sided printing, use twoside.
}

\setlength{\parindent}{0em}
\addtolength{\parskip}{1ex}

% Exam question headings, labels and sensible layout (courtesy of Tom Johnson)
\setlist{parsep=\parskip, listparindent=\parindent}
\newcommand{\examhead}[3]{\section{#1 Paper #2 Question #3}}
\newenvironment{examquestion}[3]{
    \examhead{#1}{#2}{#3}\setlist[enumerate, 1]{label=(\alph*)}\setlist[enumerate, 2]{label=(\roman*)}
    \marginpar{\qrcode{https://www.cl.cam.ac.uk/teaching/exams/pastpapers/y#1p#2q#3.pdf}}
    \marginpar{\footnotesize \url{https://www.cl.cam.ac.uk/teaching/exams/pastpapers/y#1p#2q#3.pdf}}
}{}



\usepackage{dsfont}

\begin{document}

\section{Sentiment lexicon:}

\begin{enumerate}

\item (To be done with other members of your supervision group) Each find
a short piece of text (100 words or less) expressing an opinion about
something, without showing it to the others. This could be a review, but
don’t use a movie review or similar. Tokenize and then sort the words so
they are in alphabical order, removing duplicates. Now swap lists. Mark
each word in the list you’ve been given as positive or negative sentiment
and say whether you think the piece the words have come from is overall
positive or negative. Compare your answer with the original text. Did
you get the overall sentiment right? Were the words used in the way you
thought they might be?

Does your Task 1 system get this right? What about your Task 2 system?

The words Ben gave me were:

\begin{tabular}{c c|c c|c c}
word & polarity & word & polarity & word & polarity\\
\hline
! &  & 's &  & ( & \\
) &  & , &  & . & \\
a &  & advanced & positive & algorithms & positive\\
also &  & am &  & an & \\
and &  & are &  & as & \\
at &  & beginner-friendly & positive & being & \\
bought & negative & companion & positive & concepts & positive\\
consider & positive & course & positive & covered & positive\\
definitely & positive & despite & negative & ece358 & \\
especially & positive & everything & positive & for & \\
go &  & great & positive & hardcover & positive\\
heavy & negative & helpful & positive & i & \\
if &  & includes & positive & introductory & positive\\
is &  & it &  & large & negative\\
like & positive & loving & positive & maybe & negative\\
myself & negative & not & negative & of & \\
overall & negative & pseudocode & positive & quite & negative\\
recommend & positive & said &  & snippets & positive\\
some & negative & still & negative & textbook & \\
that &  & the &  & this & \\
uoft &  & very &  & visualization & positive\\
would &  & you &  &  & \\
\end{tabular}

I will not be associating polarities to tokens such as ``!'', ``'s'' or ``('' which are not words but 
punctuation -- nor will I be assigning polarities to words such as ``a'', ``and'' or ``an'' which are 
grammatical necessities and so can convey little-to-no opinion whatsoever.

There are 22 positive words but only 11 negative words. I will therefore conclude that this text is most 
likely a positive review. \\
A more human reason for this would be that most of the positive words are stronger than the negative words -- 
most of the negative words are words which I interpret to be negative but could also be used in positive 
contexts -- ie ``heavy'': is the reviewer complaining the book is physically heavy, is a heavy read or 
saying there is a heavy emphasis on algorithmic proof?

The full text was:

``I bought this as a course companion for ECE358 at UofT (an algorithms course) and am loving it. 
The pseudocode snippets for everything is very helpful and it also includes visualization of algorithms. 
Despite being an introductory textbook, some of the concepts covered are quite advanced and maybe not 
as beginner-friendly. That being said, it's still a great algorithms textbook overall and I would definitely 
recommend it! Also consider that it is a very large and heavy textbook, especially if you go for hardcover 
like myself.''

This was a 4-star review to \textit{Introduction to Algorithms}. A positive review. This has the overall 
sentiment of what I expected. However -- not everything was used as I anticipated.\\
For example I saw ``beginner-friendly'' and introductory and assumed they would be used in conjunction 
rather than disjunction: ie ``this textbook is beginner-friendly and introductory''.

Both of my tick1 predictors (the sentiment analyser with a cutoff at zero and the one with a nonzero cutoff) 
predicted that the sentiment was positive. The lexicon agreed with me that there were significantly more 
positive words than negative. This makes sense since the lexicon was not trained on film reviews and so is 
generalisable to other sources.

However, both of my tick2 predictors (unsmoothed and smoothed) predicted that the sentiment was negative. 
This was surprising but on more thought also makes sense -- they were trained on film reviews which 
have a totally different set of interpretations to an advanced algorithms textbook: for example in film 
review some words (such as ``introductory'') have different polarities and many of the positive words 
simply will never turn up -- ``advanced'', ``hardcover'', ``beginner-friendly''.

This shows two things: 
\begin{itemize}
\item Predictors are only usable on the type of data that they were trained on
\item Predictors that are better in one situation are not always better than other predictors in 
every situation. Shows by the tick1 lexicon predictor (which achieved no higher than a 60\% accuracy on the 
film reviews) getting the correct result while the Na\"ive Bayes predictor which achieved 82\% got the 
incorrect result.
\end{itemize}

\item What sort of words change the polarity of the sentiment words? not is an
obvious example: can you think of 10 others? Are there any examples in
the text you looked at in 1? Which words in a sentence can have their
sentiment flipped if there’s a not in the sentence?

``But'' and ``however'' negate the sentiment of the previous sentence.

In other cases words can change the polarity -- for example ``would'', ``wanted'', ``could'', ``maybe'' 
sometimes change the polarity of subsequent words ie ``I wanted a textbook filled with interesting algorithms 
and examples'', ``maybe this is more useful for beginners''. 

Most positive words have their sentiment flipped when negated (with \textit{not}) but many negative 
words do not have their sentiment flipped. For example ``It's a bad book'' is obviously negative, but 
``It's not a bad book'' is still a negative statement since the implication is that it is not a good 
book either.

\item Try looking at some social media posts and work out whether you could
find words which indicated different types of sentiment: e.g. could you use
a lexicon to classify posts according to how emotionally involved someone
was feeling?

You could -- but it wouldn't be as effective since ``emotional involvement'' is not binary and so the data would be 
more varied -- and especially given the broad range of topics which people discuss on social media the lexicon 
required would be vast. Training data about emotional envolvement in politics or sport could not be 
applied to emotional envolvement in breakups for example. So in summary: yes you could but it would take a lot of effort and at the end of it 
still wouldn't be particularly good unless you greatly narrowed down the criteria and only 
analysed one very particular type of social media post.

\item In a test set with 412 examples, 328 are correctly classified. What is the
accuracy

Accuracy is the total correct predictions divided by the total number of predictions.

\begin{equation}
\begin{split}
A &= \frac{328}{412}\\
A &= 0.796 \text{ to 3.S.F}\\
\end{split}
\end{equation}

\item Why is accuracy not necessarily a good measure of success if the classes
have very different probabilities?

If one class has a significantly higher probability of occuring than another then 
a predictor may have a negligible (or zero) probability of choosing the smaller 
class -- and so be near-useless in most applications ($\dagger$) -- however this 
predictor may still have a very high accuracy.

Often better metrics to use are recall and precision (ideally a combination of both)-- 
recall is the proportion of false negatives and precision is the proportion of false positives.

A simple but powerful metric to use is the F-measure and is a weighted geometric average of 
recall and precision.

\begin{equation}
F_\beta = \frac{(\beta^2 + 1)PR}{\beta^2P + R}
\end{equation}

A higher $\beta$ weights recall higher and a lower beta weights precision higher. $\beta$ should 
be chosen dependent on the relative cost of false-positives and false-negatives.

Take a cancer screening test. Assume that there is a very low prevelance of this cancer 
and only 1 in 1000 people have it.
If the test doesn't work and always gives negative results then it has a 99.9\% accuracy.
Using accuracy as a metric we would conclude that this test is very good -- despite 
the fact that it never says anyone has cancer and is worse than useless.

In this specific case recall would be more important -- since the 
cost of a false negative is far higher then the cost of a false positive.
So an appropriate metric would be $F_\beta$ with a very high $\beta$.

($\dagger$) The main exceptions to ``low probabilities of one class'' being bad are where 
false-positives must never happen or where there is a secondary predictor 
that is far more expensive to run so you wish to run a cheaper predictor to deal with the 
``obvious'' cases with and make the whole system faster.

\end{enumerate}

\section{Na\"ive Bayes}

\begin{enumerate}

\item

\begin{enumerate}

\item 
Suppose that you are using Naive Bayes on a task where you have 100
documents in a training set, which is equally divided between class A and
class B. There are three features F1, F2 and F3: each may occur at most
once in a document. (Note that the set up here is a little different from
the way we used NB in Task 2.) The distribution for the three features
among documents is as follows:

\begin{center}
\begin{tabular}{c c c}
& A & B\\
F1 & 5 & 5\\
F2 & 0 & 10\\
F3 & 3 & 27\\
\end{tabular}
\end{center}

\item
\begin{equation}
\begin{split}
\hat{\mathds{P}}(A|F1) &= \frac{A\wedge F1}{\sum_{c \in \{A, B\}} c \wedge F1}\\
					 &= \frac{5}{10}\\
					 &= 0.5\\
\hat{\mathds{P}}(A|F2) &= \frac{A\wedge F2}{\sum_{c \in \{A, B\}} c \wedge F2}\\
					 &= \frac{0}{10}\\
					 &= 0\\
\hat{\mathds{P}}(A|F3) &= \frac{A\wedge F2}{\sum_{c \in \{A, B\}} c \wedge F3}\\
					 &= \frac{3}{30}\\
					 &= 0.1\\
\end{split}
\end{equation}
\begin{equation}
\begin{split}
\hat{\mathds{P}}(B|F1) &= \frac{B\wedge F1}{\sum_{c \in \{A, B\}} c \wedge F1}\\
					 &= \frac{5}{10}\\
					 &= 0.5\\
\hat{\mathds{P}}(B|F2) &= \frac{B\wedge F2}{\sum_{c \in \{A, B\}} c \wedge F2}\\
					 &= \frac{10}{10}\\
					 &= 1.0\\
\hat{\mathds{P}}(B|F3) &= \frac{B\wedge F2}{\sum_{c \in \{A, B\}} c \wedge F3}\\
					 &= \frac{27}{30}\\
					 &= 0.9\\
\end{split}
\end{equation}

\item Assume that you are trying to classify a document which contains only
the features F1 and F3: how would you estimate the relative probability
of A and B (without add-one smoothing)?

\begin{equation}
\begin{split}
\hat{\mathds{P}}(A|(F1\wedge F3)) &\approx \frac{\hat{\mathds{P}}(A) \times \hat{\mathds{P}}(F1|A) \times \hat{\mathds{P}}(F3|A)}{\hat{\mathds{P}}(F1\wedge F3)}\\
							&= \frac{\frac{1}{2} \times \frac{5}{50} \times \frac{3}{50}}{\frac{1}{10}\times \frac{3}{10}}\\
							&= \frac{1}{10}\\
							&= 0.1\\
\hat{\mathds{P}}(B|(F1\wedge F3)) &\approx \frac{\hat{\mathds{P}}(B) \times \hat{\mathds{P}}{F1|B} \times \hat{\mathds{P}}(F3|B)}{\hat{\mathds{P}}(F1\wedge F3)}\\
							&= \frac{\frac{1}{2} \times \frac{5}{50} \times \frac{27}{50}}{\frac{1}{10}\times \frac{3}{10}}\\
							&= \frac{9}{10}\\
							&= 0.9\\
\end{split}
\end{equation}

\item What difference would it make if there were 25 documents in class A in
the training set and 75 in class B?

This would change $\hat{\mathds{P}}(A)$ from 0.5 to 0.25 and $\hat{\mathds{P}}(B)$ from 0.5 to 0.75.\\
However it would also double the probabilities $\hat{\mathds{P}}(F1|A)$, 
$\hat{\mathds{P}}(F2|A)$ and $\hat{\mathds{P}}(F3|A)$ (since the observed number is the same while the 
number of observations halved). $\hat{\mathds{P}}(F1|B)$, $\hat{\mathds{P}}(F2|B)$ and $\hat{\mathds{P}}(F3|B)$ 
would decrease by $\frac{1}{3}$ since there were now more observations of $B$.

With these new probabilities:

\begin{equation}
\begin{split}
\hat{\mathds{P}}(A|(F1\wedge F3)) &\approx \frac{\hat{\mathds{P}}(A) \times \hat{\mathds{P}}(F1|A) \times \hat{\mathds{P}}(F3|A)}{\hat{\mathds{P}}(F1\wedge F3)}\\
							&= \frac{\frac{1}{4} \times \frac{5}{25} \times \frac{3}{25}}{\frac{1}{10}\times \frac{3}{10}}\\
							&= \frac{1}{5}\\
							&= 0.2\\
\hat{\mathds{P}}(B|(F1\wedge F3)) &\approx \frac{\hat{\mathds{P}}(B) \times \hat{\mathds{P}}{F1|B} \times \hat{\mathds{P}}(F3|B)}{\hat{\mathds{P}}(F1\wedge F3)}\\
							&= \frac{\frac{3}{4} \times \frac{5}{75} \times \frac{27}{75}}{\frac{1}{10}\times \frac{3}{10}}\\
							&= \frac{9}{10}\\
							&= 0.9\\
\end{split}
\end{equation}

So the new $\hat{\mathds{P}}(A|(F1 \wedge F2))$ is 0.2 and the new $\hat{\mathds{P}}(B|(F1 \wedge F2))$ is 0.8.

\item Which of the features F1, F2 and F3 would be more useful for classification
in general? Explain your answer.

F2 and F3 are the features which would be most useful for classification.

F1 is evenly split between both A and B so (assuming the original model in which the classes A and B have the same size), 
the probabilities $\hat{\mathds{P}}(A|F1)$ and $\hat{\mathds{P}}(B|F1)$ are the same. So we cannot tell anything about 
which class the document belongs to based on F1.

In the observed data, F2 never occurred with A and is uncommon with B. Na\"ively, one may think that F2 would be the best 
indicator of whether a class belonged to A or B. However, due to the small size of the training set we cannot use this 
feature like this. The probability of observing F1 given A may not be zero. The sample is so small that we can rule 
very little out and eliminating A completely irrelevant of what the other evidence says is na\"ive. This is why smoothing 
is useful -- we deal with the case where we do not observe any feature while keeping the probability low.

A truly na\"ive algorithm (like the unsmoothed one we were forced to use for the first part of tick 2) would keep F2 as a feature on B but not as a feature on A 
and so multiply B by $\hat{\mathds{P}}(B|F2)$ and not multiply A therefore decreasing the relative probability of B despite observing a 
feature uniquely seen in B.

In general F3 is the best indicator -- we have a high number of positives and so a good confidence that $\hat{\mathds{P}}(B|F3) >> \hat{\mathds{P}}(A|F3)$ 
and F3 is also common in B so can be used in many instances.

\item Given reasonable amounts of training and test data and a feature set with
10 features, how could you establish which features were most useful?

To remove ambiguity with the question I will clarify that I am assessing the usefulness of 
given features for a given (Na\"ive Bayes) model rather than their overall predictive power.

Unlike many other models; Na\"ive Bayes has no intrinsic score for the usefulness of a feature.
$\hat{\mathds{P}}(C|F_i)$ is not a good measure since if $F_i$ is a very rare feature than it can 
have a high $\hat{\mathds{P}}(C|F_i)$ but very low overall importance. A similar argument applies to 
using $\hat{\mathds{P}}(F_i)$ as a measure of importance -- $F_i$ could be a very frequent feature with 
no correllation at all to any class -- ie $F_i$ is whether the $mm$ digit of a persons height is 
odd and the classes are whether the film review they wrote was positive.

For Na\"ive Bayes the best way of working out feature importance is experimentally. A good way to 
work this out is called Permutation Importance. You first train and test the data with all features 
as normal. Record the accuracy (or F-measure or recall or precision -- whichever metric you are using) 
of the full model. Then for each feature: shuffle that one feature randomly and redo the training and 
testing. Record the resulting drop in metric score. A high decrease means that the feature is more 
important while a low decrease means that the feature is less important (and an increase means that 
we have bad data or a bad predictor [like the unsmoothed predictor in tick2]).

\end{enumerate}

\item (Difficult) The approach we asked you to take for NB incorporated 
probabilities for all the positions in the document (i.e., all the tokens). An
alternative approach, often used for document classification, is to count
words only once no matter how many times they appear in a document.
This model is clearly less informative than the approach we used, but it
usually works better. Why?

One of the Na\"ive Bayes assumptions is that the probabilities of different words occuring 
is independent. This assumption is broken between the same word. For example if one author 
uses the word ``favourite'' then they are more likely to reuse it. This is a clear breach 
of the Na\"ive Bayes assumption. 

As with many things in machine learning if we do not handle edge cases then we will end up 
with worse results. A simple way to handle this edge case that breaks our assumption 
is to simply ignore multiple occurrences of the same word and consider the text as a 
set-of-words rather than a bag-of-words (only consider a word once irrelevant of how many 
times it occurs in the text). So although we consider less data and information; 
the model of the data that we use fits our assumptions and the Na\"ive Bayes model of text 
better.

Since our predictive algorithm is now better suited to the data, we can often end up with 
better results than if we considered the data as it really is even if we have less actual 
data to process.

\end{enumerate}

\section{Statistical properties of language:}

\begin{enumerate}

\item Given that we will always see new words given a sufficiently large corpus,
how is it that most people would confidently say that the following are not
English words: \textit{pferd}, \textit{abtruce}, \textit{Kx'a}. Are they right?

Pferd is Horse in German.
Abtruce is not a word used for anything.
Kx'a is the name of a language group in Africa.

While none of these words are commonly used -- some of them do have meaning -- and it is 
undoubtable that they have all been used at some point. Regulation of \textit{what} words 
constitute a language is very informal and aims to model accurately what the population 
actually uses. For example there are words which are simply never used anymore that are 
sometimes removed from the lexicon even though they are written down and were used for hundreds 
of years -- and there are other words which are always being added. For example ``food-baby'' and 
``NFT'' were recently added to the dictionary. So one argument is that words are only part of 
English if they are actively being used in some English-speaking community. This is the approach 
that most people would take. Under this definition only \text{Kx'a} would be an English word -- 
it is the recognised name of a language group and used in linguistics.

However one could take a more liberal approach. For example in the BBC comedy Blackadder a 
dictionarian was attempting to write a dictionary of every word. To which Blackadder offered 
his contrafibularities upon it's completion. A liberal view would say that since this word 
had been used to convey meaning it was now an English word. Under this interpretation 
\textit{pferd} and \textit{Kx'a} would be English words since both have undoubtably been 
used to convey meaning (the former likely among german speakers speaking english) and the latter 
in linguistics. This is in practice a reasonable approach -- any attempt to map the english language 
will not be completely successful since new words are emerging all the time. Under this interpretation 
it is almost undoubtable that there are people who have used all three words \textit{pferd}, 
\textit{abtruce} and \textit{Kx'a} to convey meaning and so all three should be considered as 
English words.

In my opinion the latter is a more accurate representation of the reality of English -- the language 
is always evolving and so does not keep a static or even consistent (ie literally -- it is its 
own antonym) lexicon. However like with everything we must make practical modelling assumptions 
to have a feasible chance of using anything consistently or well and so the first interpretation 
is more practical.

My interpretation is that \textit{pferd} is primarily a German word -- and not English, \textit{abtruce} 
is not a word in any mainstream context so should in general not be considerd a word, \textit{Kx'a} is a word.

\end{enumerate}

\end{document}