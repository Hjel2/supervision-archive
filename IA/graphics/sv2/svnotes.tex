\documentclass[10pt, a4paper]{article}

% DO NOT add \usepackage commands here.  Place any custom commands
% into your SV work files.  Anything in the template directory is
% likely to be overwritten!

\usepackage{fancyhdr}

\usepackage{lastpage}       % ``n of m'' page numbering
\usepackage{lscape}         % Makes landscape easier

\usepackage{verbatim}       % Verbatim blocks
\usepackage{listings}       % Source code listings
\usepackage{graphicx}
\usepackage{float}
\usepackage{epsfig}         % Embed encapsulated postscript
\usepackage{array}          % Array environment
\usepackage{qrcode}         % QR codes
\usepackage{enumitem}       % Required by Tom Johnson's exam question header

\usepackage{hhline}         % Horizontal lines in tables
\usepackage{siunitx}        % Correct spacing of units
\usepackage{amsmath}        % American Mathematical Society
\usepackage{amssymb}        % Maths symbols
\usepackage{amsthm}         % Theorems

\usepackage{ifthen}         % Conditional processing in tex

\usepackage[top=3cm,
            bottom=3cm,
            inner=2cm,
            outer=5cm]{geometry}

% PDF metadata + URL formatting
\usepackage[
            pdfauthor={\studentname},
            pdftitle={\svcourse, SV \svnumber},
            pdfsubject={},
            pdfkeywords={9d2547b00aba40b58fa0378774f72ee6},
            pdfproducer={},
            pdfcreator={},
            hidelinks]{hyperref}

\renewcommand{\headrulewidth}{0.4pt}
\renewcommand{\footrulewidth}{0.4pt}
\fancyheadoffset[LO,LE,RO,RE]{0pt}
\fancyfootoffset[LO,LE,RO,RE]{0pt}
\pagestyle{fancy}
\fancyhead{}
\fancyhead[LO,RE]{{\bfseries \studentname}\\\studentemail}
\fancyhead[RO,LE]{{\bfseries \svcourse, SV~\svnumber}\\\svdate\ \svtime, \svvenue}
\fancyfoot{}
\fancyfoot[LO,RE]{For: \svrname}
\fancyfoot[RO,LE]{\today\hspace{1cm}\thepage\ / \pageref{LastPage}}
\fancyfoot[C]{\qrcode[height=0.8cm]{\svuploadkey}}
\setlength{\headheight}{22.55pt}


\ifthenelse{\equal{\jkfside}{oneside}}{

 \ifthenelse{\equal{\jkfhanded}{left}}{
  % 1. Left-handed marker, one-sided printing or e-marking, use oneside and...
  \evensidemargin=\oddsidemargin
  \oddsidemargin=73pt
  \setlength{\marginparwidth}{111pt}
  \setlength{\marginparsep}{-\marginparsep}
  \addtolength{\marginparsep}{-\textwidth}
  \addtolength{\marginparsep}{-\marginparwidth}
 }{
  % 2. Right-handed marker, one-sided printing or e-marking, use oneside.
  \setlength{\marginparwidth}{111pt}
 }

}{
 % 3. Alternating margins, two-sided printing, use twoside.
}


\setlength{\parindent}{0em}
\addtolength{\parskip}{1ex}

% Exam question headings, labels and sensible layout (courtesy of Tom Johnson)
\setlist{parsep=\parskip, listparindent=\parindent}
\newcommand{\examhead}[3]{\section{#1 Paper #2 Question #3}}
\newenvironment{examquestion}[3]{
\examhead{#1}{#2}{#3}\setlist[enumerate, 1]{label=(\alph*)}\setlist[enumerate, 2]{label=(\roman*)}
\marginpar{\href{https://www.cl.cam.ac.uk/teaching/exams/pastpapers/y#1p#2q#3.pdf}{\qrcode{https://www.cl.cam.ac.uk/teaching/exams/pastpapers/y#1p#2q#3.pdf}}}
\marginpar{\footnotesize \href{https://www.cl.cam.ac.uk/teaching/exams/pastpapers/y#1p#2q#3.pdf}{https://www.cl.cam.ac.uk/\\teaching/exams/pastpapers/\\y#1p#2q#3.pdf}}
}{}


\begin{document}

\section*{Graphics Supervision 2}

\subsection*{Model Transform}
Changing the object in object coordinates to the object in world
coordinates -- scaling, rotating and translating etc
The time you need to use object coordinates are texture mapping.
finding the intersection of a ray with an object.
In world coordinates the camera can be anywhere.
In world coordinates you are just constructing the scene and
converting everything into the same coordinate system so that you can say where everything is.
There is one model transform per object.

\subsection*{View Transform}
This transforms everything from the world coordinates so that the
camera is at the origin in these coordinates and everything moves respective to the camera.
This is just so that you make the maths easier.
There is one transform per camera.

\subsection*{Projection Transformation}
Projecting the 3D coordinates onto a 2D plane.
Perspective projection is the most common, however there are other
projections such as ones which just go ``first in the z-axis''
one transform per camera

You can precompute the view and projection transformations for the whole scene -- chaining
these together means you have to do significantly less work.

Perspective transformations scale down the vector to the point so that
they are just all on the same plane but that the unit vector in their direction is unchanged.

With perspective transformations you don't map the $z$ onto the plane --
you map it to $\frac{1}{z}$ so that you can know which order things should be in. This is
called the ``$z$-buffer''.

\subsection*{Matrices}

To try and figure out what a complicated matrix does is to throw in some
arbitrary points, see what happens. After that's done, throw in the
corners of a unit square etc and get an intuition for roughly what the matrix is doing.

There is a very limited set of transformations that a matrix could be doing
(rotation, scaling, transformation, projection)
and it's usually obvious which it is. For example if there are no $\sqrt{}$'s
in the matrix then there is no rotation. If one of the right hand rows is non-zero then
there is a transformation, if the bottom row (except $w$) is nonzero then there is a
projection.

In graphics, dividing by a basis vector almost always means projection.

\subsection*{MIPmap's}

Bilinear interpolation interpolates in the x-y axis while Trilinear interpolation
is bilinear interpolation between pixels in a MIPmap plus interpoltion between
two adjacent layers of a MIPmap.

All a MIPMap is is a cache of precomputed downsampled values which can be looked up in
constant time rather than computed for each pixel in quadratic time.

If a question asks how do you do a thing -- you do need to say manually how to do it.

\end{document}
