\documentclass[10pt, a4paper]{article}

% DO NOT add \usepackage commands here.  Place any custom commands
% into your SV work files.  Anything in the template directory is
% likely to be overwritten!

\usepackage{fancyhdr}

\usepackage{lastpage}       % ``n of m'' page numbering
\usepackage{lscape}         % Makes landscape easier

\usepackage{verbatim}       % Verbatim blocks
\usepackage{listings}       % Source code listings
\usepackage{graphicx}
\usepackage{float}
\usepackage{epsfig}         % Embed encapsulated postscript
\usepackage{array}          % Array environment
\usepackage{qrcode}         % QR codes
\usepackage{enumitem}       % Required by Tom Johnson's exam question header

\usepackage{hhline}         % Horizontal lines in tables
\usepackage{siunitx}        % Correct spacing of units
\usepackage{amsmath}        % American Mathematical Society
\usepackage{amssymb}        % Maths symbols
\usepackage{amsthm}         % Theorems

\usepackage{ifthen}         % Conditional processing in tex

\usepackage[top=3cm,
            bottom=3cm,
            inner=2cm,
            outer=5cm]{geometry}

% PDF metadata + URL formatting
\usepackage[
            pdfauthor={\studentname},
            pdftitle={\svcourse, SV \svnumber},
            pdfsubject={},
            pdfkeywords={9d2547b00aba40b58fa0378774f72ee6},
            pdfproducer={},
            pdfcreator={},
            hidelinks]{hyperref}

\renewcommand{\headrulewidth}{0.4pt}
\renewcommand{\footrulewidth}{0.4pt}
\fancyheadoffset[LO,LE,RO,RE]{0pt}
\fancyfootoffset[LO,LE,RO,RE]{0pt}
\pagestyle{fancy}
\fancyhead{}
\fancyhead[LO,RE]{{\bfseries \studentname}\\\studentemail}
\fancyhead[RO,LE]{{\bfseries \svcourse, SV~\svnumber}\\\svdate\ \svtime, \svvenue}
\fancyfoot{}
\fancyfoot[LO,RE]{For: \svrname}
\fancyfoot[RO,LE]{\today\hspace{1cm}\thepage\ / \pageref{LastPage}}
\fancyfoot[C]{\qrcode[height=0.8cm]{\svuploadkey}}
\setlength{\headheight}{22.55pt}


\ifthenelse{\equal{\jkfside}{oneside}}{

 \ifthenelse{\equal{\jkfhanded}{left}}{
  % 1. Left-handed marker, one-sided printing or e-marking, use oneside and...
  \evensidemargin=\oddsidemargin
  \oddsidemargin=73pt
  \setlength{\marginparwidth}{111pt}
  \setlength{\marginparsep}{-\marginparsep}
  \addtolength{\marginparsep}{-\textwidth}
  \addtolength{\marginparsep}{-\marginparwidth}
 }{
  % 2. Right-handed marker, one-sided printing or e-marking, use oneside.
  \setlength{\marginparwidth}{111pt}
 }

}{
 % 3. Alternating margins, two-sided printing, use twoside.
}


\setlength{\parindent}{0em}
\addtolength{\parskip}{1ex}

% Exam question headings, labels and sensible layout (courtesy of Tom Johnson)
\setlist{parsep=\parskip, listparindent=\parindent}
\newcommand{\examhead}[3]{\section{#1 Paper #2 Question #3}}
\newenvironment{examquestion}[3]{
\examhead{#1}{#2}{#3}\setlist[enumerate, 1]{label=(\alph*)}\setlist[enumerate, 2]{label=(\roman*)}
\marginpar{\href{https://www.cl.cam.ac.uk/teaching/exams/pastpapers/y#1p#2q#3.pdf}{\qrcode{https://www.cl.cam.ac.uk/teaching/exams/pastpapers/y#1p#2q#3.pdf}}}
\marginpar{\footnotesize \href{https://www.cl.cam.ac.uk/teaching/exams/pastpapers/y#1p#2q#3.pdf}{https://www.cl.cam.ac.uk/\\teaching/exams/pastpapers/\\y#1p#2q#3.pdf}}
}{}


\begin{document}

\section*{Graphics Supervision 1}

\subsection*{Exam Advice}

look at the syllabus page
use the website to see what the structure of the exams is
the important things on this page are the exam questions
(this new course is based off the old IB course. Some of the old past paper questions are relevant but ONLY some).
rendering and ray tracing comes up all the time
Do past paper questions.

Graphics hardware and OpenGL is the new stuff that's not in the old exams --
it doesn't come up so much (especially about GPU frameworks and APIs)
Fundamentals of Computer graphics book for the stuff on colour.
Read the colour and photometry sections.
Just read chapters on the things you've got difficulty with.
The mock exams in January are really hard. They are made purposefully difficult to
``shock'' you. Do SOME revision over Christmas.

If the question is ambitious then either answer both cases or answer one case very well
and explain the case that you're answering. If you don't know something then make an
assumption. If it's something you aren't expected to know then you get the marks.
If you are expected to know it then you will lose a couple of marks but will still get
the marks for the rest of the question. Anything more than 6 or so marks and you really
want to break your solution down into chunks.

\subsection*{Metamers}

We abuse metamers by rather than creating continuous waves of light by having only
3 discrete lights which appear to have the same colour as the continuous distribution.
Metamers make things easier by only having to use RGB rather than make a proper
continuous wave function since the eye cannot tell the difference.
Review and remember the diagram for the density of the rod and cone cells in the eye
Also the graph of the sensitivity of the different cone cells to the eye.
Learn how to read and write answers. You can structure it better with bullet
points, diagrams etc. That's the biggest challenge.
VERY low proportion of the cone cells are short cone cells (only 2\% - 5\%).
Blue cones have a lower intensity response. Blue light is refracted differently in your eye.

\subsection*{Light}

Ambient light is very poor for ambient-only scenes and high-contrast scenes. However you
could model this with two ambient constants. Ambient illumination is accurate for scenes
which are relatively uniformly lit. Diffuse illumination is good on FLAT surfaces.
But if you don't do it on flat surfaces then it is not good (ie crystalline structures or grass).
Original: ``Specular illumination can be accurate but isn't right by nature since
there are so many degrees of freedom.'' Specular illumination can be made to correctly
approximate light since there are so many degrees of freedom (parameters to change).
But it isn't correct by it's very nature.

\subsection*{Eyes}

Generally regular things should be avoided -- your eyes are used to noise/blurs/static but
are not used to jagged/regular things so it really stands out in a bad way.
The more noise you can add the better - such as for soft shadows.
With motion blur you can more heavily weight things towards the end of the motion.
Multiple samples are good for: DoF, Anti-Aliasing, Motion-blur and area light source.

\subsection*{Diagrams}

Always include an accompanying explanation about why your diagram is correct.
It makes it easier for the examiner and makes it less likely for you to forget things.
You are expected to know the illumination equation.
Exam questions are meant to demonstrate knowledge. Don't make things more difficult if it
doesn't demonstrate any new knowledge.
Include all the detail though.

With reflection (and other algorithms) you should include the termination conditions (either
maximum depth or ray hits nothing more) and then also include how the light is added back in
(reflexivity) (how the stack is unwound).

\end{document}
