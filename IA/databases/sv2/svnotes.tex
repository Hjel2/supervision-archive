\newcommand{\svrname}{Dr John Fawcett}
\newcommand{\jkfside}{oneside}
\newcommand{\jkfhanded}{right}

\newcommand{\studentname}{Harry Langford}
\newcommand{\studentemail}{hjel2@cam.ac.uk}

\documentclass[10pt,\jkfside,a4paper]{article}

\newcommand{\svcourse}{CST Part IA: Introduction to Probability}
\newcommand{\svnumber}{1}
\newcommand{\svvenue}{Churchill, Room TBD}
\newcommand{\svdate}{2022-05-14}
\newcommand{\svtime}{11:00}
\newcommand{\svuploadkey}{PO5ogKIM8KQA22FZS8IAf8gxA8XKi19jxIBVHIfFZ+3GCBXuNUXS9lVN6bNYjxM/}

\newcommand{\svrname}{Mr Matthew Ireland}
\newcommand{\jkfside}{twoside}
\newcommand{\jkfhanded}{right}

\newcommand{\studentname}{Harry Langford}
\newcommand{\studentemail}{hjel2@cam.ac.uk}


% DO NOT add \usepackage commands here.  Place any custom commands
% into your SV work files.  Anything in the template directory is
% likely to be overwritten!

\usepackage{fancyhdr}

\usepackage{lastpage}       % ``n of m'' page numbering
\usepackage{lscape}         % Makes landscape easier

\usepackage{verbatim}       % Verbatim blocks
\usepackage{epsfig}         % Embed encapsulated postscript
\usepackage{array}          % Array environment
\usepackage[nolinks]{qrcode}         % QR codes
\usepackage{enumitem}       % Required by Tom Johnson's exam question header

\usepackage{hhline}         % Horizontal lines in tables
\usepackage{siunitx}        % Correct spacing of units
\usepackage{amsmath}        % American Mathematical Society
\usepackage{amssymb}        % Maths symbols
\usepackage{amsthm}         % Theorems

\usepackage{ifthen}         % Conditional processing in tex

\usepackage[top=3cm,
            bottom=3cm,
            inner=2cm,
            outer=5cm]{geometry}

% PDF metadata + URL formatting
\usepackage[
            pdfauthor={\studentname},
            pdftitle={\svcourse, SV \svnumber},
            pdfsubject={},
            pdfkeywords={9d2547b00aba40b58fa0378774f72ee6},
            pdfproducer={},
            pdfcreator={},
            hidelinks]{hyperref}

\renewcommand{\headrulewidth}{0.4pt}
\renewcommand{\footrulewidth}{0.4pt}
\fancyheadoffset[LO,LE,RO,RE]{0pt}
\fancyfootoffset[LO,LE,RO,RE]{0pt}
\pagestyle{fancy}
\fancyhead{}
\fancyhead[LO,RE]{{\bfseries \studentname}\\\studentemail}
\fancyhead[RO,LE]{{\bfseries \svcourse, SV~\svnumber}\\\svdate\ \svtime, \svvenue}
\fancyfoot{}
\fancyfoot[LO,RE]{For: \svrname}
\fancyfoot[RO,LE]{\today\hspace{1cm}\thepage\ / \pageref{LastPage}}
\fancyfoot[C]{\qrcode[height=0.8cm]{\svuploadkey}}
\setlength{\headheight}{22.55pt}

\ifthenelse{\equal{\jkfside}{oneside}}{

 \ifthenelse{\equal{\jkfhanded}{left}}{
  % 1. Left-handed marker, one-sided printing or e-marking, use oneside and...
  \evensidemargin=\oddsidemargin
  \oddsidemargin=73pt
  \setlength{\marginparwidth}{111pt}
  \setlength{\marginparsep}{-\marginparsep}
  \addtolength{\marginparsep}{-\textwidth}
  \addtolength{\marginparsep}{-\marginparwidth}
 }{
  % 2. Right-handed marker, one-sided printing or e-marking, use oneside.
  \setlength{\marginparwidth}{111pt}
 }

}{
 % 3. Alternating margins, two-sided printing, use twoside.
}

\setlength{\parindent}{0em}
\addtolength{\parskip}{1ex}

% Exam question headings, labels and sensible layout (courtesy of Tom Johnson)
\setlist{parsep=\parskip, listparindent=\parindent}
\newcommand{\examhead}[3]{\section{#1 Paper #2 Question #3}}
\newenvironment{examquestion}[3]{
    \examhead{#1}{#2}{#3}\setlist[enumerate, 1]{label=(\alph*)}\setlist[enumerate, 2]{label=(\roman*)}
    \marginpar{\qrcode{https://www.cl.cam.ac.uk/teaching/exams/pastpapers/y#1p#2q#3.pdf}}
    \marginpar{\footnotesize \url{https://www.cl.cam.ac.uk/teaching/exams/pastpapers/y#1p#2q#3.pdf}}
}{}



\begin{document}

The strategy for supervisions and exam questions is just to dump as much raw knowledge 
in the question as you possibly can. Throw anything that is every slightly related to 
the question that is being asked into it. This is how you demonstrate not just that you 
know the topic well but that you \textbf{know everything} about a topic.

Document oriented databases:
\begin{itemize}
\item Are keyed.
\item The keys are used to look for the address of the document in a hash table.
\item The keys must be stored in a separate data structure.
\item There can be multiple keys each of which point to the lookup in the hash table. 
IE in a ``Elastic Search'' you could search with any of 4 or 5 keys.
\item The document itself contains attributes. Each document may not contain every 
attribute and attributes can be multivalued. 
\item This is in some ways like a Java Object -- if you have the name of it then you 
can access it in O(1) time. It has attributes, it has links to other classes -- but 
any one instance of an object does not have to have every possible attribute initialised.
\item Are \textbf{awful} when appending documents. \\ Because you have to search 
through every other document to add the data in.
\item Use two-way keys (ie if a movies document references an actor, then both the 
actor and the movie will have the key to the other).
\item Document oriented databases are \textbf{very} fast to access for most common 
queries. This is because all you do is look up an address in a hash table.
\item Document oriented databases are usually okay (not as bad as relational databases) 
for transitive queries. This is because nodes point to other nodes so you have the same 
sort of graph-relations that graph oriented databases have. However, if you don't have 
the key then you have to look through the entire document oriented database - which is 
just unfeasible (since NoSQL databases and especially document oriented databases just 
store huge amounts of data).
\end{itemize}

OLAP:
\begin{itemize}
\item \textbf{THIS IS ABOUT A USE NOT A SPECIFIC IMPLEMENTATION.}
\item These are what you think of as normal relational databases.
They, however are not ``normalised''. They are databases which are optimised for large 
amounts of quick queries, concurrency and ACID transactions.
\item They do \textbf{almost no} reads which require multiple records.
\item etc. This is AN APPLICATION NOT AN IMPLEMENTATION
\end{itemize}

OLTP:
\begin{itemize}
\item These databases are just huge -- so large that you cannot do any foreign keys etc. 
When you perform a query on them all you can do is just parse through the database and 
touch each record once and only once.
\item You essentially merge a ton of records together, remove all foreign keys and merge 
the tables in. This mega-record is known as a fact and OLTP databases are known as ``fact 
tables''.
\item This is basically used for business analytics.
\end{itemize}
\end{document}