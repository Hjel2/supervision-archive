\newcommand{\svrname}{Dr John Fawcett}
\newcommand{\jkfside}{oneside}
\newcommand{\jkfhanded}{right}

\newcommand{\studentname}{Harry Langford}
\newcommand{\studentemail}{hjel2@cam.ac.uk}

\documentclass[10pt,\jkfside,a4paper]{article}

\newcommand{\svcourse}{CST Part IA: Introduction to Probability}
\newcommand{\svnumber}{1}
\newcommand{\svvenue}{Churchill, Room TBD}
\newcommand{\svdate}{2022-05-14}
\newcommand{\svtime}{11:00}
\newcommand{\svuploadkey}{PO5ogKIM8KQA22FZS8IAf8gxA8XKi19jxIBVHIfFZ+3GCBXuNUXS9lVN6bNYjxM/}

\newcommand{\svrname}{Mr Matthew Ireland}
\newcommand{\jkfside}{twoside}
\newcommand{\jkfhanded}{right}

\newcommand{\studentname}{Harry Langford}
\newcommand{\studentemail}{hjel2@cam.ac.uk}

\input{../../template2/includes.tex}
% DO NOT add \usepackage commands here.  Place any custom commands
% into your SV work files.  Anything in the template directory is
% likely to be overwritten!

\usepackage{fancyhdr}

\usepackage{lastpage}       % ``n of m'' page numbering
\usepackage{lscape}         % Makes landscape easier

\usepackage{verbatim}       % Verbatim blocks
\usepackage{epsfig}         % Embed encapsulated postscript
\usepackage{array}          % Array environment
\usepackage[nolinks]{qrcode}         % QR codes
\usepackage{enumitem}       % Required by Tom Johnson's exam question header

\usepackage{hhline}         % Horizontal lines in tables
\usepackage{siunitx}        % Correct spacing of units
\usepackage{amsmath}        % American Mathematical Society
\usepackage{amssymb}        % Maths symbols
\usepackage{amsthm}         % Theorems

\usepackage{ifthen}         % Conditional processing in tex

\usepackage[top=3cm,
            bottom=3cm,
            inner=2cm,
            outer=5cm]{geometry}

% PDF metadata + URL formatting
\usepackage[
            pdfauthor={\studentname},
            pdftitle={\svcourse, SV \svnumber},
            pdfsubject={},
            pdfkeywords={9d2547b00aba40b58fa0378774f72ee6},
            pdfproducer={},
            pdfcreator={},
            hidelinks]{hyperref}

\renewcommand{\headrulewidth}{0.4pt}
\renewcommand{\footrulewidth}{0.4pt}
\fancyheadoffset[LO,LE,RO,RE]{0pt}
\fancyfootoffset[LO,LE,RO,RE]{0pt}
\pagestyle{fancy}
\fancyhead{}
\fancyhead[LO,RE]{{\bfseries \studentname}\\\studentemail}
\fancyhead[RO,LE]{{\bfseries \svcourse, SV~\svnumber}\\\svdate\ \svtime, \svvenue}
\fancyfoot{}
\fancyfoot[LO,RE]{For: \svrname}
\fancyfoot[RO,LE]{\today\hspace{1cm}\thepage\ / \pageref{LastPage}}
\fancyfoot[C]{\qrcode[height=0.8cm]{\svuploadkey}}
\setlength{\headheight}{22.55pt}

\ifthenelse{\equal{\jkfside}{oneside}}{

 \ifthenelse{\equal{\jkfhanded}{left}}{
  % 1. Left-handed marker, one-sided printing or e-marking, use oneside and...
  \evensidemargin=\oddsidemargin
  \oddsidemargin=73pt
  \setlength{\marginparwidth}{111pt}
  \setlength{\marginparsep}{-\marginparsep}
  \addtolength{\marginparsep}{-\textwidth}
  \addtolength{\marginparsep}{-\marginparwidth}
 }{
  % 2. Right-handed marker, one-sided printing or e-marking, use oneside.
  \setlength{\marginparwidth}{111pt}
 }

}{
 % 3. Alternating margins, two-sided printing, use twoside.
}

\setlength{\parindent}{0em}
\addtolength{\parskip}{1ex}

% Exam question headings, labels and sensible layout (courtesy of Tom Johnson)
\setlist{parsep=\parskip, listparindent=\parindent}
\newcommand{\examhead}[3]{\section{#1 Paper #2 Question #3}}
\newenvironment{examquestion}[3]{
    \examhead{#1}{#2}{#3}\setlist[enumerate, 1]{label=(\alph*)}\setlist[enumerate, 2]{label=(\roman*)}
    \marginpar{\qrcode{https://www.cl.cam.ac.uk/teaching/exams/pastpapers/y#1p#2q#3.pdf}}
    \marginpar{\footnotesize \url{https://www.cl.cam.ac.uk/teaching/exams/pastpapers/y#1p#2q#3.pdf}}
}{}



\begin{document}

\begin{enumerate}
\item{Discuss the practical exercises for graph databases. Do you 
understand the queries and what they are doing?}

For the first query; the task is to select all actors who have co-acted with 
Jennifer Lawrence more than once and output their names and how many times 
they have co-acted with her.\\
The way to do this is to select all the actors who have co-acted with 
Jennifer Lawrence, group them by name (using a with clause) and return only 
the actors who have more than one occurrence.\\
Because all the actors who have been selected have 1 or more occurrences, we 
can use ``<>'' rather than ``>''.

\begin{verbatim}
match (p1:Person)-[:ACTS_IN]->(:Movie)<-[:ACTS_IN]-(p2:Person)
where p1.name = 'Lawrence, Jennifer (III)'
  and p2.name <> 'Lawrence, Jennifer (III)'
with p2.name as name, count(*) as total
where total <> 1
return name, total
order by name, total;
\end{verbatim}

The second query asks us to find the distance between every genre and every 
other genre. The query to do this first matches all genres. Then works out 
the shortest path from that genre to every other genre. This, however will 
contain paths from g1 to g2 \textbf{and} from g2 to g1. To remove this we 
use an ordering and only take the relations which satisfy that (alphabetical) 
ordering (where genre1 < g2.genre). Then, since every path has two HAS\_GENRE 
relationships between them for every film in the path; the length is 
length(path)/2.

\begin{verbatim}
match (g:Genre)
with g.genre as genre1
match path = allshortestpaths( (g:genre)-[:HAS_GENRE*]-(g2:genre) )
where genre1 < g2.genre
return distinct genre1, g2.genre as genre2, length(path)/2 as length
order by length desc, genre1, genre2;
\end{verbatim}

The third query asks us to calculate a ``similarity'' score between two movies 
based on the number of genres and actors that the movies have in common.

To do this we first have to establish all the movies which have actors 
and genres in common. We can do this by two consecutive match clauses. 
The first will select all movies which have a common actor, the second will 
select all movies from the first set which have a common genre. 
The third match clause then filters out all those movies in that group 
which do not have a common keyword.\\
The with clause renames the variables so that they can be returned in 
the next line. The only thing of particular note is count(). This returns 
the number of distinct paths between m1 and m2 in the path it is passed. 
In path1, count(path1) represents the number of common actors while count(path2) 
is the number of genres that the two movies have in common. There is no need to count 
path3 -- matching it is enough to filter out any movies which do not have any keywords 
in common.

\begin{verbatim}
match (m1:Movie)
where m1.title = 'Skyfall (2012)'
match path1 = (m1:Movie)-[:ACTS_IN*2]-(m2:Movie)
match path2 = (m1:Movie)-[:HAS_GENRE*2]-(m2:Movie)
match path3 = (m1:Movie)-[:HAS_KEYWORD*2]-(m2:Movie)
with m2.title as title, (10 * count(path1) + count(path2)) as score
return title, score
order by score desc;
\end{verbatim}

I'm aware that this query will not return movies which do not have BOTH a common 
actor and a common genre. But: I spent a substantial amount of time on this 
and am happy that this does something close enough for me to talk about it.

\item{In SQL, the creator of a database needs to define the schema up front: 
that is, what columns each table is going to have, and the datatype of each 
column. This means that all rows in the table have the same set of columns 
(although some of them may be set to NULL if the value is unknown).\\
On the other hand, Cypher does not have an explicit schema: a node can have 
any set of properties, and you can always add a property with a new name to 
an existing node.\\
Discuss the pros and cons of these two approaches.}

In SQL; defining the schema on creation means that the database is very rigid 
and inflexible. This is good if the purpose of the database is known before 
creation. However, it does limit the table in certain situations --
 if there are lots of optional 
fields then the database will either have to create many tables or accept a high 
proportion of NULL values. This \textit{can be} okay however is either wasteful 
of space OR means queries will require lots of joins.\\
Take the example where a movie has alternative names in different countries.
In a SQL database the options are either to create a new table containing the 
fields \begin{verbatim} ALTNAMES (\overline{movie_id}, \overline{country_id}, alt_name)\end{verbatim} 
or to add fields onto the movies table for each country for which the movie could 
have an alternative name for.\\
The first approach would be preferred by OLTP databases (since it has lower redundancy 
and is easier to write to) while the second would be preferred by OLAP databases 
(since it doesn't require joins).
However, both have downfalls and are not ideal.

Compared to a graph database which can add any properties to a node:
if the movie node wants to have an alternate name for that country then it can 
simply add a new property with the alternate name. This approach does not fill 
the whole node with NULLs or force expensive joins for basic queries.

SQL's schema enforcement does have positives too: a well-designed schema can 
result in very low data redundancy. It's impossible to implement the same thing 
using Cypher since users can simply add properties to any node -- if this data is 
stored elsewhere in the database then there is redundancy.

SQL also is more resilient to user-error. Take an numeric field (ie for the movies table: 
``star\_rating''). 
In a SQL database this would be integer. So any other data type would not be allowed. 
This means if a user was to input totally invalid data (say ``four''), the database 
would not allow it to be entered. This keeps the database in a consistent state. 

However, with a graph database since properties are defined individually by the user 
they could create the property ``star\_rating'' and set it to ``four''. Despite this 
data being invalid, since the graph database did not have a defined schemaa it would 
be accepted.

\item{In the graph databases tutorial we showed a Cypher query for find the 
number of co-actors for Jennifer Lawrence. How would you write the same 
query in SQL?}

\begin{verbatim}
select distinct p1.name
from people as p1
join plays_role as pr1 on pr1.person_id = p1.person_id
join movies as m on m.movie_id = pr1.movie_id
join plays_role as pr2 on pr2.movie_id = m.movie_id
join people as p2 on p2.person_id = pr2.person_id
where p2.name = 'Jennifer Lawrence'
order by p1.name;
\end{verbatim}

\item{We discussed several examples of Cypher queries that have close 
equivalents in SQL, and some examples that are hard to express in SQL. 
Discuss: would it make sense to move existing SQL-based applications to 
Cypher? In what circumstances would you choose one query language over the 
other?}

While Cypher has positives when compared to SQl: I believe that 
SQL is better suited for most applications. 

In general: most of the largest queries executed on SQL-based databases are done 
for business analytics on OLAP databases. These sort of queries aim to only touch 
each fact once and then proceed. Graph-oriented databases do not offer any speedup 
on analytics such as these. They would likely slow-down the queries -- 
on a sufficiently large database (such as one which analytics queries 
would be executed on), you cannot load the whole database into RAM. In a SQL-based 
application you load the facts which you are currently processing into RAM.
However, for a graph-database you would not be able to know in advance which 
nodes connect to the node you are analysing. So would be unable to load them into RAM. 
This means that each node would require multiple fetches from main memory -- greatly slowing 
down the query.

For some queries Cypher is far faster than SQL-based applications. 
It is especially good at finding routes from one node to another -- something which 
SQL requires many $\Theta(n^2)$ joins to do. However, this sort of query is uncommon 
and there is very little demand for it in industy -- so little that recursive queries were 
not included in the SQL standard until 1999.

Another argument for not changing existing SQL-based applications to Cypher is 
that SQL is well-known. Cypher is nicher and there would be large integration issues. 
In addition to this: there are many different SQL providers all of which would 
require a different implementation to change their product to a graph-oriented database.

SQL-based OLTP databases are faster to write to and update than graph-based applications.
For most day-to-day usage all you care about is the speed of writing and updating and 
in this regard SQL is superior to Cypher.

Since SQL defines the schema when creating the table; a good OLTP implementation 
enforces low data redundancy. This reduces the amount of data that you have 
to store when compared to a graph-based database. It also removes the possibility 
of inconsistent states -- again something which Cypher struggles to do.

Also: since SQL is very established many companies who offer SQL databases have very 
optimised implementations -- which collect data about the database and can 
greatly decrease the amount of time taken on many queries. While Cypher can also 
be optimised: in general it is not optimised to the same extent -- and collecting 
analytical data on the database would be impractical since properties are 
defined when creating the nodes.

\item{In the example graph database, there are 22 nodes of label Genre, but 
in the relational database, the genres table has 273 rows. How do you 
explain this discrepancy? What are the consequences of this difference in 
data model?}

In the graph database, different ``HAS\_GENRE'' relations point to the same genre 
node. This means that the number of genre nodes is equal to the number of genre 
relations.

While in the \textit{old, bad and totally un-normalised} relational 
database schema: each genre has a foreign key pointing to one movie associated.
This means that in the relational database: the number of records in the genres 
table is equal to the number of genre relations NOT the number of genres.
In short: this discrepancy is caused by a bad data model.

The consequences of this difference is that the relational database is 
very un-normalised and stores many duplicate copies of the name of the genre. 
So there could be a situation in the relational database where the name of a 
genre changes and the whole table would be in an inconsistent state, or where 
the name of one genre was input incorrectly. This problem would not happen with 
the graph database. 

In more general differences between the graph database schema and the relational 
database: the graph database also makes it easier to query 
for movies with a specific genre. In the relational database, you have to search 
through the genres table for a record with a specific genre, then search the 
movies table for the movie with that movie\_id. This will take time $\Theta{mn}$ 
where m is the number of movies and n is the number of movies with that genre. 

While in the graph database you only have to navigate to the genres node and 
then select all movies who have a pointer to that node. This will take 
$\Theta(n)$ time with respect to either (depending on the implementation) 
the number of movies in the database OR the number of movies with that genre. 
Either is better than the time complexity of the same query in the relational 
database.

\item{What do you think the database software is doing internally when you 
ask it to find the shortest path between two nodes? Describe it in words 
(no code required).}

To find the path; the software must search the graph. It can either do this by 
a breadth-first-search or by iterative deepening. The space complexity of a breadth
-first-search may be prohibitive for larger graphs though. So I think it likely 
that the software would use a heuristic to decide whether to use an iterative 
deepening strategy or a breadth-first-search.

The software first uses some heuristic to find out whether the branching factor 
and the depth of the tree will make the space complexity of a bfs prohibitive. 
If not: the software will perform a bfs until it finds a path between the nodes.
If the space complexity of a bfs is prohibitive, then the software would 
run a series of dfs's using an iterative deepening strategy to search for the 
shortest path between two nodes.

Due to the nature of a breadth-first-search; any path found using one is guaranteed 
to be a shortest path (since all possible shorter paths have already been searched).\\
However, the iterative deepening strategy may lead to cases where the first path found 
is not the shortest path. This would only occur if the depth was increased by more than 
one at each iteration.

\end{enumerate}

\end{document}