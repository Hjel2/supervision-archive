\newcommand{\svrname}{Dr John Fawcett}
\newcommand{\jkfside}{oneside}
\newcommand{\jkfhanded}{right}

\newcommand{\studentname}{Harry Langford}
\newcommand{\studentemail}{hjel2@cam.ac.uk}

\documentclass[10pt,\jkfside,a4paper]{article}

% DO NOT add \usepackage commands here.  Place any custom commands
% into your SV work files.  Anything in the template directory is
% likely to be overwritten!

\usepackage{fancyhdr}

\usepackage{lastpage}       % ``n of m'' page numbering
\usepackage{lscape}         % Makes landscape easier

\usepackage{verbatim}       % Verbatim blocks
\usepackage{listings}       % Source code listings
\usepackage{epsfig}         % Embed encapsulated postscript
\usepackage{array}          % Array environment
\usepackage{qrcode}         % QR codes
\usepackage{enumitem}       % Required by Tom Johnson's exam question header

\usepackage{hhline}         % Horizontal lines in tables
\usepackage{siunitx}        % Correct spacing of units
\usepackage{amsmath}        % American Mathematical Society
\usepackage{amssymb}        % Maths symbols
\usepackage{amsthm}         % Theorems

\usepackage{ifthen}         % Conditional processing in tex

\usepackage[top=3cm,
            bottom=3cm,
            inner=2cm,
            outer=5cm]{geometry}

% PDF metadata + URL formatting
\usepackage[
            pdfauthor={\studentname},
            pdftitle={\svcourse, SV \svnumber},
            pdfsubject={},
            pdfkeywords={9d2547b00aba40b58fa0378774f72ee6},
            pdfproducer={},
            pdfcreator={},
            hidelinks]{hyperref}


% DO NOT add \usepackage commands here.  Place any custom commands
% into your SV work files.  Anything in the template directory is
% likely to be overwritten!

\usepackage{fancyhdr}

\usepackage{lastpage}       % ``n of m'' page numbering
\usepackage{lscape}         % Makes landscape easier

\usepackage{verbatim}       % Verbatim blocks
\usepackage{listings}       % Source code listings
\usepackage{graphicx}
\usepackage{float}
\usepackage{epsfig}         % Embed encapsulated postscript
\usepackage{array}          % Array environment
\usepackage{qrcode}         % QR codes
\usepackage{enumitem}       % Required by Tom Johnson's exam question header

\usepackage{hhline}         % Horizontal lines in tables
\usepackage{siunitx}        % Correct spacing of units
\usepackage{amsmath}        % American Mathematical Society
\usepackage{amssymb}        % Maths symbols
\usepackage{amsthm}         % Theorems

\usepackage{ifthen}         % Conditional processing in tex

\usepackage[top=3cm,
            bottom=3cm,
            inner=2cm,
            outer=5cm]{geometry}

% PDF metadata + URL formatting
\usepackage[
            pdfauthor={\studentname},
            pdftitle={\svcourse, SV \svnumber},
            pdfsubject={},
            pdfkeywords={9d2547b00aba40b58fa0378774f72ee6},
            pdfproducer={},
            pdfcreator={},
            hidelinks]{hyperref}

\renewcommand{\headrulewidth}{0.4pt}
\renewcommand{\footrulewidth}{0.4pt}
\fancyheadoffset[LO,LE,RO,RE]{0pt}
\fancyfootoffset[LO,LE,RO,RE]{0pt}
\pagestyle{fancy}
\fancyhead{}
\fancyhead[LO,RE]{{\bfseries \studentname}\\\studentemail}
\fancyhead[RO,LE]{{\bfseries \svcourse, SV~\svnumber}\\\svdate\ \svtime, \svvenue}
\fancyfoot{}
\fancyfoot[LO,RE]{For: \svrname}
\fancyfoot[RO,LE]{\today\hspace{1cm}\thepage\ / \pageref{LastPage}}
\fancyfoot[C]{\qrcode[height=0.8cm]{\svuploadkey}}
\setlength{\headheight}{22.55pt}


\ifthenelse{\equal{\jkfside}{oneside}}{

 \ifthenelse{\equal{\jkfhanded}{left}}{
  % 1. Left-handed marker, one-sided printing or e-marking, use oneside and...
  \evensidemargin=\oddsidemargin
  \oddsidemargin=73pt
  \setlength{\marginparwidth}{111pt}
  \setlength{\marginparsep}{-\marginparsep}
  \addtolength{\marginparsep}{-\textwidth}
  \addtolength{\marginparsep}{-\marginparwidth}
 }{
  % 2. Right-handed marker, one-sided printing or e-marking, use oneside.
  \setlength{\marginparwidth}{111pt}
 }

}{
 % 3. Alternating margins, two-sided printing, use twoside.
}


\setlength{\parindent}{0em}
\addtolength{\parskip}{1ex}

% Exam question headings, labels and sensible layout (courtesy of Tom Johnson)
\setlist{parsep=\parskip, listparindent=\parindent}
\newcommand{\examhead}[3]{\section{#1 Paper #2 Question #3}}
\newenvironment{examquestion}[3]{
\examhead{#1}{#2}{#3}\setlist[enumerate, 1]{label=(\alph*)}\setlist[enumerate, 2]{label=(\roman*)}
\marginpar{\href{https://www.cl.cam.ac.uk/teaching/exams/pastpapers/y#1p#2q#3.pdf}{\qrcode{https://www.cl.cam.ac.uk/teaching/exams/pastpapers/y#1p#2q#3.pdf}}}
\marginpar{\footnotesize \href{https://www.cl.cam.ac.uk/teaching/exams/pastpapers/y#1p#2q#3.pdf}{https://www.cl.cam.ac.uk/\\teaching/exams/pastpapers/\\y#1p#2q#3.pdf}}
}{}


\usepackage{listings}
\usepackage{enumitem}

\begin{document}

\subsection*{Paper 1 Question 1}

\begin{enumerate}[label=(\alph*)]

\item The findSix program counts the number of sixes which are reachable from the root 
without passing any other sixes. It returns unit if it can find none and raises an exception 
gotIt(n) if it finds n $\neq$ 0 6's. It has a type of tree $\rightarrow$ unit. 

\begin{lstlisting}[language=Caml]

let t1 = Br( 4, Lf 6 , Br(6, Lf 3 , Lf 6) );;
let t2 = Br( 4, Lf 7 , Br(8, Lf 3 , Lf 1) );;

\end{lstlisting}

When run on the first tree, it checks 4. Realises that 4 $\neq$ 6. So it recursively calls itself on 
the left subtree -- Lf 6. This raises GotIt(1). So v1 = 1. findSix is then called on the right subtree. 
v = 6 so this raises gotIt(1). So v2 = 1. v = v1 + v2 = 2. So the first call raises GotIt(2).

When run on the second tree, it checks 4. 4 $\neq$ 6. So findSix is called on (Lf 7). 7 $\neq$ 6. 
So () is returned. v1 = 0. findSix is then called on the right subtree -- Br(8, Lf 3 , Lf 1).
8 $\neq$ 6. So findSix is called on both subtrees. $3 \neq$ 6. 1 $\neq$ 6. So both return (). This means 
findSix Br(8, Lf 3 , Lf 1) returns (). Hence the whole function returns ().

\item 

\begin{lstlisting}[language=Caml]

let rec compute n = function
	| Lf(x) -> if x = 6 then 1 else 0
        | Br(v, l, r) -> if v = 6 then 1
                         else
                         let v1 = compute n l in
                         let v2 = compute n r in
                         v1 + v2
;;
\end{lstlisting}

\item 

\begin{lstlisting}[language=Caml]
let containsroot t =
    let rec predicate p = function
        | Lf(x) -> p x
        | Br(v, l, r) -> p v || predicate p l || predicate p r
    in
    match t with
        | Lf(_) -> false
        | Br(root, l, r) -> predicate ((=) root) l || predicate ((=) root) r
;;
\end{lstlisting}

\item

\begin{lstlisting}[language=Caml]
let doublepath t =
    let rec contains x = function
        | [] -> false
        | hd::tl -> (x = hd) || contains x tl
    in
    let rec twoinpath acc = function
        | Lf(v) -> contains v acc
        | Br(v, l, r) -> (contains v acc) || (twoinpath (v::acc) l) || 
					(twoinpath (v::acc) r)
    in
    twoinpath [] t
;;
\end{lstlisting}

\end{enumerate}

\end{document}