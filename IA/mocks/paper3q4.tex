\newcommand{\svrname}{Dr John Fawcett}
\newcommand{\jkfside}{oneside}
\newcommand{\jkfhanded}{right}

\newcommand{\studentname}{Harry Langford}
\newcommand{\studentemail}{hjel2@cam.ac.uk}

\documentclass[10pt,\jkfside,a4paper]{article}

% DO NOT add \usepackage commands here.  Place any custom commands
% into your SV work files.  Anything in the template directory is
% likely to be overwritten!

\usepackage{fancyhdr}

\usepackage{lastpage}       % ``n of m'' page numbering
\usepackage{lscape}         % Makes landscape easier

\usepackage{verbatim}       % Verbatim blocks
\usepackage{listings}       % Source code listings
\usepackage{epsfig}         % Embed encapsulated postscript
\usepackage{array}          % Array environment
\usepackage{qrcode}         % QR codes
\usepackage{enumitem}       % Required by Tom Johnson's exam question header

\usepackage{hhline}         % Horizontal lines in tables
\usepackage{siunitx}        % Correct spacing of units
\usepackage{amsmath}        % American Mathematical Society
\usepackage{amssymb}        % Maths symbols
\usepackage{amsthm}         % Theorems

\usepackage{ifthen}         % Conditional processing in tex

\usepackage[top=3cm,
            bottom=3cm,
            inner=2cm,
            outer=5cm]{geometry}

% PDF metadata + URL formatting
\usepackage[
            pdfauthor={\studentname},
            pdftitle={\svcourse, SV \svnumber},
            pdfsubject={},
            pdfkeywords={9d2547b00aba40b58fa0378774f72ee6},
            pdfproducer={},
            pdfcreator={},
            hidelinks]{hyperref}


% DO NOT add \usepackage commands here.  Place any custom commands
% into your SV work files.  Anything in the template directory is
% likely to be overwritten!

\usepackage{fancyhdr}

\usepackage{lastpage}       % ``n of m'' page numbering
\usepackage{lscape}         % Makes landscape easier

\usepackage{verbatim}       % Verbatim blocks
\usepackage{listings}       % Source code listings
\usepackage{graphicx}
\usepackage{float}
\usepackage{epsfig}         % Embed encapsulated postscript
\usepackage{array}          % Array environment
\usepackage{qrcode}         % QR codes
\usepackage{enumitem}       % Required by Tom Johnson's exam question header

\usepackage{hhline}         % Horizontal lines in tables
\usepackage{siunitx}        % Correct spacing of units
\usepackage{amsmath}        % American Mathematical Society
\usepackage{amssymb}        % Maths symbols
\usepackage{amsthm}         % Theorems

\usepackage{ifthen}         % Conditional processing in tex

\usepackage[top=3cm,
            bottom=3cm,
            inner=2cm,
            outer=5cm]{geometry}

% PDF metadata + URL formatting
\usepackage[
            pdfauthor={\studentname},
            pdftitle={\svcourse, SV \svnumber},
            pdfsubject={},
            pdfkeywords={9d2547b00aba40b58fa0378774f72ee6},
            pdfproducer={},
            pdfcreator={},
            hidelinks]{hyperref}

\renewcommand{\headrulewidth}{0.4pt}
\renewcommand{\footrulewidth}{0.4pt}
\fancyheadoffset[LO,LE,RO,RE]{0pt}
\fancyfootoffset[LO,LE,RO,RE]{0pt}
\pagestyle{fancy}
\fancyhead{}
\fancyhead[LO,RE]{{\bfseries \studentname}\\\studentemail}
\fancyhead[RO,LE]{{\bfseries \svcourse, SV~\svnumber}\\\svdate\ \svtime, \svvenue}
\fancyfoot{}
\fancyfoot[LO,RE]{For: \svrname}
\fancyfoot[RO,LE]{\today\hspace{1cm}\thepage\ / \pageref{LastPage}}
\fancyfoot[C]{\qrcode[height=0.8cm]{\svuploadkey}}
\setlength{\headheight}{22.55pt}


\ifthenelse{\equal{\jkfside}{oneside}}{

 \ifthenelse{\equal{\jkfhanded}{left}}{
  % 1. Left-handed marker, one-sided printing or e-marking, use oneside and...
  \evensidemargin=\oddsidemargin
  \oddsidemargin=73pt
  \setlength{\marginparwidth}{111pt}
  \setlength{\marginparsep}{-\marginparsep}
  \addtolength{\marginparsep}{-\textwidth}
  \addtolength{\marginparsep}{-\marginparwidth}
 }{
  % 2. Right-handed marker, one-sided printing or e-marking, use oneside.
  \setlength{\marginparwidth}{111pt}
 }

}{
 % 3. Alternating margins, two-sided printing, use twoside.
}


\setlength{\parindent}{0em}
\addtolength{\parskip}{1ex}

% Exam question headings, labels and sensible layout (courtesy of Tom Johnson)
\setlist{parsep=\parskip, listparindent=\parindent}
\newcommand{\examhead}[3]{\section{#1 Paper #2 Question #3}}
\newenvironment{examquestion}[3]{
\examhead{#1}{#2}{#3}\setlist[enumerate, 1]{label=(\alph*)}\setlist[enumerate, 2]{label=(\roman*)}
\marginpar{\href{https://www.cl.cam.ac.uk/teaching/exams/pastpapers/y#1p#2q#3.pdf}{\qrcode{https://www.cl.cam.ac.uk/teaching/exams/pastpapers/y#1p#2q#3.pdf}}}
\marginpar{\footnotesize \href{https://www.cl.cam.ac.uk/teaching/exams/pastpapers/y#1p#2q#3.pdf}{https://www.cl.cam.ac.uk/\\teaching/exams/pastpapers/\\y#1p#2q#3.pdf}}
}{}


\usepackage{enumitem}

\usepackage{listings}

\begin{document}

\subsection*{Paper 3 Question 4}

\begin{enumerate}[label = (\alph*)]

\item Modelling transformations transform an object from object coordinates (where the object is 
at the origin and has an arbitrary size) into world coordinates with the rest of the scene.

Viewing transformations change an object from world coordinates into viewing coordinates such that the camera 
is at the origin and is pointing down the positive $z$-axis.

Projection transformations project all objects in a scene onto a given plane. This is usually $z=d$ for some 
constant d. They usually project $z$ to $\frac{1}{z}$ rather than $d$ so that this can be used to calculate 
the distance to the object.

\item
\begin{enumerate}[label=(\roman*)]

\item If an object was moving then the modelling transformation would change.

\item If an object in the scene was changing shape then the modelling transformation would change.

\item Light bending due to the extreme gravity of a black hole would affect all shapes in the same way -- 
assuming that the shapes were near to each other. In this case we would change the viewing transformation.

\end{enumerate}

\item 
We would compute the following in the GPU:

\begin{lstlisting}
in vec2 xy;
GLposition = vec3(xy, f(xy.x, xy.y))
\end{lstlisting}

The return from this could be taken as a vec3[].

\begin{lstlisting}
in vec3[] xyz;
out xy;
int min = vec3(0, 0, float("inf"));
for each in xyz:
	if each.z < min.z:
		min = each;
xy = min.xy;
\end{lstlisting}

The z-buffer would essentially be used to compute $f(x, y)$ in parallel on the GPU and then we could 
find the minimum in series (finding the minimum could be parallelised using a recursive algorithm to $\Theta(\lg n)$ 
however that is not relevant to the z-buffer).

\item Bump mapping finds the normal at a point and offsets it slightly according to a ``bump map''. Which is 
a 2d array indicating whether the location is raised or inset. This allows the user to map a texture onto an object 
making it appear to have an uneven surface.

Some people may prefer displacement mapping 
because displacement mapping changes the \textbf{shape} of the object rather than just appearing to add insets 
and raised portions. For example bump mapping could not be used to model a sphere with a hole cut out of it -- while 
displacement mapping could.

\item we triangularise 3D objects because it removes all ambiguity about where lines on any non-coplanar points 
may go.

Triangles allow easy baryocentric interpolation of colours in the vertex shader.

GPU's are optimised to deal with trianges (as opposed to arbitrary-sided n-gons).

Triangles are simpler shapes and it requires far simpler maths to do operations on trianlges than on n-gons.

\end{enumerate}

\end{document}