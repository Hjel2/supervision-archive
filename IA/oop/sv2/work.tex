\newcommand{\svrname}{Dr John Fawcett}
\newcommand{\jkfside}{oneside}
\newcommand{\jkfhanded}{right}

\newcommand{\studentname}{Harry Langford}
\newcommand{\studentemail}{hjel2@cam.ac.uk}

\documentclass[10pt,\jkfside,a4paper]{article}

\newcommand{\svcourse}{CST Part IA: Introduction to Probability}
\newcommand{\svnumber}{1}
\newcommand{\svvenue}{Churchill, Room TBD}
\newcommand{\svdate}{2022-05-14}
\newcommand{\svtime}{11:00}
\newcommand{\svuploadkey}{PO5ogKIM8KQA22FZS8IAf8gxA8XKi19jxIBVHIfFZ+3GCBXuNUXS9lVN6bNYjxM/}

\newcommand{\svrname}{Mr Matthew Ireland}
\newcommand{\jkfside}{twoside}
\newcommand{\jkfhanded}{right}

\newcommand{\studentname}{Harry Langford}
\newcommand{\studentemail}{hjel2@cam.ac.uk}

\input{../../template2/includes.tex}
% DO NOT add \usepackage commands here.  Place any custom commands
% into your SV work files.  Anything in the template directory is
% likely to be overwritten!

\usepackage{fancyhdr}

\usepackage{lastpage}       % ``n of m'' page numbering
\usepackage{lscape}         % Makes landscape easier

\usepackage{verbatim}       % Verbatim blocks
\usepackage{epsfig}         % Embed encapsulated postscript
\usepackage{array}          % Array environment
\usepackage[nolinks]{qrcode}         % QR codes
\usepackage{enumitem}       % Required by Tom Johnson's exam question header

\usepackage{hhline}         % Horizontal lines in tables
\usepackage{siunitx}        % Correct spacing of units
\usepackage{amsmath}        % American Mathematical Society
\usepackage{amssymb}        % Maths symbols
\usepackage{amsthm}         % Theorems

\usepackage{ifthen}         % Conditional processing in tex

\usepackage[top=3cm,
            bottom=3cm,
            inner=2cm,
            outer=5cm]{geometry}

% PDF metadata + URL formatting
\usepackage[
            pdfauthor={\studentname},
            pdftitle={\svcourse, SV \svnumber},
            pdfsubject={},
            pdfkeywords={9d2547b00aba40b58fa0378774f72ee6},
            pdfproducer={},
            pdfcreator={},
            hidelinks]{hyperref}

\renewcommand{\headrulewidth}{0.4pt}
\renewcommand{\footrulewidth}{0.4pt}
\fancyheadoffset[LO,LE,RO,RE]{0pt}
\fancyfootoffset[LO,LE,RO,RE]{0pt}
\pagestyle{fancy}
\fancyhead{}
\fancyhead[LO,RE]{{\bfseries \studentname}\\\studentemail}
\fancyhead[RO,LE]{{\bfseries \svcourse, SV~\svnumber}\\\svdate\ \svtime, \svvenue}
\fancyfoot{}
\fancyfoot[LO,RE]{For: \svrname}
\fancyfoot[RO,LE]{\today\hspace{1cm}\thepage\ / \pageref{LastPage}}
\fancyfoot[C]{\qrcode[height=0.8cm]{\svuploadkey}}
\setlength{\headheight}{22.55pt}

\ifthenelse{\equal{\jkfside}{oneside}}{

 \ifthenelse{\equal{\jkfhanded}{left}}{
  % 1. Left-handed marker, one-sided printing or e-marking, use oneside and...
  \evensidemargin=\oddsidemargin
  \oddsidemargin=73pt
  \setlength{\marginparwidth}{111pt}
  \setlength{\marginparsep}{-\marginparsep}
  \addtolength{\marginparsep}{-\textwidth}
  \addtolength{\marginparsep}{-\marginparwidth}
 }{
  % 2. Right-handed marker, one-sided printing or e-marking, use oneside.
  \setlength{\marginparwidth}{111pt}
 }

}{
 % 3. Alternating margins, two-sided printing, use twoside.
}

\setlength{\parindent}{0em}
\addtolength{\parskip}{1ex}

% Exam question headings, labels and sensible layout (courtesy of Tom Johnson)
\setlist{parsep=\parskip, listparindent=\parindent}
\newcommand{\examhead}[3]{\section{#1 Paper #2 Question #3}}
\newenvironment{examquestion}[3]{
    \examhead{#1}{#2}{#3}\setlist[enumerate, 1]{label=(\alph*)}\setlist[enumerate, 2]{label=(\roman*)}
    \marginpar{\qrcode{https://www.cl.cam.ac.uk/teaching/exams/pastpapers/y#1p#2q#3.pdf}}
    \marginpar{\footnotesize \url{https://www.cl.cam.ac.uk/teaching/exams/pastpapers/y#1p#2q#3.pdf}}
}{}



\begin{document}

\subsection*{6.1} 
Explain the difference between a class, an abstract class and an interface in Java:

A class can be instantiated and has methods with definitions and attributes. 

An abstract class is a template for a class -- which contains methods with definitions and 
also abstract methods -- which are declarations that the class which inherits the abstract class 
must declare the methods before it can be instantiated. Abstract classes cannot be instantiated.
Abstract classes can have attributes.

An interface is a template for a class. It contains no method definitions. Every mathod in 
an interface is abstract (the abstract keyword is hidden). Interfaces can have attributes.

\subsection*{6.8}
\begin{enumerate}

\item Create a Java interface for a standard queue (i.e. FIFO).

\begin{verbatim}
interface QueueInterface{
    void enqueue(int x);
    int dequeue();
    boolean isEmpty();
}
\end{verbatim}

\item Implement OOPListQueue, which should use two OOPLinkedList objects as per 
the queues you constructed in your FoCS course. You may need to implement a method 
to reverse lists.

\begin{verbatim}
package uk.ac.cam.hjel2.oop.sv2;

import java.util.NoSuchElementException;

public class OOPListQueue implements QueueInterface{

    private OOPLinkedList headlist;
    private OOPLinkedList taillist;

    OOPListQueue(){
        headlist = new OOPLinkedList();
        taillist = new OOPLinkedList();
    }

    @Override
    public void enqueue(int x) {
        taillist.add(x);
    }

    @Override
    public int dequeue() {
        if (headlist.length != 0){
            return headlist.remove();
        }
        else if(taillist.length != 0) {
            taillist.reverse();
            headlist = taillist;
            taillist = new OOPLinkedList();
            return dequeue();
        }
        else {
            throw new NoSuchElementException();
        }
    }

    @Override
    public boolean isEmpty() {
        return (headlist.length==0 && taillist.length==0);
    }
}
\end{verbatim}

\item Implement OOPArrayQueue. Use integer indices to keep track of the head and tail positions.

\begin{verbatim}
package uk.ac.cam.hjel2.oop.sv2;

import java.util.Arrays;
import java.util.NoSuchElementException;

public class OOPArrayQueue implements QueueInterface{
    private int[] queue;
    private int hd;
    private int tl;
    private int length;

    OOPArrayQueue(){
        queue = new int[10];
        this.length = 10;
        // hd is the first occupied element
        // tl is the next free element
        // if the queue fills up on insertion then we immediately create
        // a new array (with double the length) to be the queue
        // so the only case that hd=tl is when the queue is empty
    }

    @Override
    public void enqueue(int x) {
        if ((tl + 1) % length == hd) {
            System.out.println("called");
            System.out.printf("hd: %s tl: %s len: %s%n", hd, tl, length);
            int[] temp = new int[2 * length];
            System.arraycopy(queue, hd, temp, 0, length - hd);
            System.arraycopy(queue, tl, temp, length - hd + 1, (hd - tl - 1) % length);
            queue = temp;
            hd = 0;
            tl = length;
            length *= 2;
        }
        queue[tl] = x;
        tl++;
        tl %= length;
    }

    @Override
    public int dequeue() {
        if (!isEmpty()) {
            int first = queue[hd];
            hd++;
            hd %= length;
            return first;
        }
        else {
            throw new NoSuchElementException();
        }
    }

    @Override
    public boolean isEmpty() {
        return hd==tl;
    }

    @Override
    public String toString(){
        return Arrays.toString(queue);
    }
}
\end{verbatim}

\item For OOPListQueue; dequeue has a $\Theta(1)$ ammortized cost (however an indivual dequeue could have $O(n)$ time). 
Enqueing and isEmpty are all $\Theta(1)$.

OOPArrayQueue has a $\Theta(1)$ time for dequeue and isEmpty -- and a $\Theta(1)$ ammortized cost for enqueue.

\end{enumerate}

\subsection*{9.3} Write a Java program that reads in a text file that contains two integers on each line, 
separated by a comma (i.e. two columns in a comma-separated file). Your program should print out the 
same set of numbers, but sorted by the first column and subsorted by the second.

\begin{verbatim}
package uk.ac.cam.hjel2.oop.sv2;

import java.io.File;
import java.io.FileNotFoundException;
import java.util.ArrayList;
import java.util.Comparator;
import java.util.List;
import java.util.Scanner;

public class OOPFileReader {

    private static class OOPComparator implements Comparator<int[]>{

        @Override
        public int compare(int[] o1, int[] o2) {
            switch (Integer.compare(o1[0], o2[0])){
                case -1 -> {return -1;}
                case 0 -> {return Integer.compare(o1[1], o2[1]);}
                default -> {return 1;}
            }
        }
    }

    public void read(String path) throws FileNotFoundException {
        List<int[]> tuples = new ArrayList<>();
        Scanner scanner = new Scanner(new File(path));
        String[] line;
        while (scanner.hasNext()){
            line = scanner.next().split(",");
            tuples.add(new int[]{Integer.parseInt(line[0]), Integer.parseInt(line[1])});
        }
        tuples.sort(new OOPComparator());
        for (int[] t: tuples){
            System.out.printf("%s,%s%n", t[0], t[1]);
        }
    }
}
\end{verbatim}

\end{document}