\newcommand{\svcourse}{CST Part IA: Introduction to Probability}
\newcommand{\svnumber}{1}
\newcommand{\svvenue}{Churchill, Room TBD}
\newcommand{\svdate}{2022-05-14}
\newcommand{\svtime}{11:00}
\newcommand{\svuploadkey}{PO5ogKIM8KQA22FZS8IAf8gxA8XKi19jxIBVHIfFZ+3GCBXuNUXS9lVN6bNYjxM/}

\newcommand{\svrname}{Mr Matthew Ireland}
\newcommand{\jkfside}{twoside}
\newcommand{\jkfhanded}{right}

\newcommand{\studentname}{Harry Langford}
\newcommand{\studentemail}{hjel2@cam.ac.uk}


\documentclass[10pt,\jkfside,a4paper]{article}

% DO NOT add \usepackage commands here.  Place any custom commands
% into your SV work files.  Anything in the template directory is
% likely to be overwritten!

\usepackage{fancyhdr}

\usepackage{lastpage}       % ``n of m'' page numbering
\usepackage{lscape}         % Makes landscape easier

\usepackage{verbatim}       % Verbatim blocks
\usepackage{epsfig}         % Embed encapsulated postscript
\usepackage{array}          % Array environment
\usepackage[nolinks]{qrcode}         % QR codes
\usepackage{enumitem}       % Required by Tom Johnson's exam question header

\usepackage{hhline}         % Horizontal lines in tables
\usepackage{siunitx}        % Correct spacing of units
\usepackage{amsmath}        % American Mathematical Society
\usepackage{amssymb}        % Maths symbols
\usepackage{amsthm}         % Theorems

\usepackage{ifthen}         % Conditional processing in tex

\usepackage[top=3cm,
            bottom=3cm,
            inner=2cm,
            outer=5cm]{geometry}

% PDF metadata + URL formatting
\usepackage[
            pdfauthor={\studentname},
            pdftitle={\svcourse, SV \svnumber},
            pdfsubject={},
            pdfkeywords={9d2547b00aba40b58fa0378774f72ee6},
            pdfproducer={},
            pdfcreator={},
            hidelinks]{hyperref}

\renewcommand{\headrulewidth}{0.4pt}
\renewcommand{\footrulewidth}{0.4pt}
\fancyheadoffset[LO,LE,RO,RE]{0pt}
\fancyfootoffset[LO,LE,RO,RE]{0pt}
\pagestyle{fancy}
\fancyhead{}
\fancyhead[LO,RE]{{\bfseries \studentname}\\\studentemail}
\fancyhead[RO,LE]{{\bfseries \svcourse, SV~\svnumber}\\\svdate\ \svtime, \svvenue}
\fancyfoot{}
\fancyfoot[LO,RE]{For: \svrname}
\fancyfoot[RO,LE]{\today\hspace{1cm}\thepage\ / \pageref{LastPage}}
\fancyfoot[C]{\qrcode[height=0.8cm]{\svuploadkey}}
\setlength{\headheight}{22.55pt}

\ifthenelse{\equal{\jkfside}{oneside}}{

 \ifthenelse{\equal{\jkfhanded}{left}}{
  % 1. Left-handed marker, one-sided printing or e-marking, use oneside and...
  \evensidemargin=\oddsidemargin
  \oddsidemargin=73pt
  \setlength{\marginparwidth}{111pt}
  \setlength{\marginparsep}{-\marginparsep}
  \addtolength{\marginparsep}{-\textwidth}
  \addtolength{\marginparsep}{-\marginparwidth}
 }{
  % 2. Right-handed marker, one-sided printing or e-marking, use oneside.
  \setlength{\marginparwidth}{111pt}
 }

}{
 % 3. Alternating margins, two-sided printing, use twoside.
}

\setlength{\parindent}{0em}
\addtolength{\parskip}{1ex}

% Exam question headings, labels and sensible layout (courtesy of Tom Johnson)
\setlist{parsep=\parskip, listparindent=\parindent}
\newcommand{\examhead}[3]{\section{#1 Paper #2 Question #3}}
\newenvironment{examquestion}[3]{
    \examhead{#1}{#2}{#3}\setlist[enumerate, 1]{label=(\alph*)}\setlist[enumerate, 2]{label=(\roman*)}
    \marginpar{\qrcode{https://www.cl.cam.ac.uk/teaching/exams/pastpapers/y#1p#2q#3.pdf}}
    \marginpar{\footnotesize \url{https://www.cl.cam.ac.uk/teaching/exams/pastpapers/y#1p#2q#3.pdf}}
}{}



\usepackage{float}

\begin{document}

\begin{examquestion}{2010}{5}{4}

In an application, processes may be identified as ``readers'' or ``writers''
of a certain data object. Multiple-reader, single-writer access to this
object must be implemented, with priority for writers over readers. Readers
execute procedures \textit{startread} and \textit{endread} before and after
reading. Writers execute procedures \textit{startwrite} and
\textit{endwrite} before and after writing one-at-a-time.

The following variables are used in an implementation of the algorithm:
\begin{table}[H]
\centering
\begin{tabular}{l l}
\textit{ar} & is the count of active readers \\
\textit{rr} & is the count of reading readers \\
\textit{aw} & is the count of active writers \\
\textit{ww} & is the count of writing writers (who write one-at-a-time) \\
\end{tabular}
\end{table}

\begin{enumerate}[label=(\alph*)]

\item For mutual exclusion:\\
\textit{SemCountGuard} is a Semaphore under which the above contents are
read and written.\\
\textit{SemWrite} is for writers to wait on, in order to write one-at-a-time.

For condition synchronisation:\\
\textit{SemOKtoRead} is for readers to wait until all writers have finished.\\
\textit{SemOKtoWrite} is for writers to wait until currently reading readers
have finished.

Discuss the following pseudocode for an attempted implementation of
\textit{startread}:

\begin{lstlisting}

procedure startread()
wait(SemCountGuard);
ar := ar + 1;
if aw > 0 then wait(SemOKtoRead);
rr := rr + 1;
signal(SemCountGuard);
return;

\end{lstlisting}

This code will deadlock or fail to implement priority correctly. startread
waits on SemOKtoRead while still holding onto SemCountGuard. This will
either prevent any writing writer from decrementing ww or force it to signal
SemOKtoRead first -- which would allow new readers to start before the next
writer.

Additionally, startread does not update SemOKtoWrite. startread changes the
number of readers -- if we go from 0 to 1 readers, we should acquire
SemOKtoWrite so that writers cannot start. This will prevent writers writing
while readers are reading. If startread does not updates SemOKtoWrite, then
it is redundant and writers will have to spinlock, repeatedly polling until
there are no readers left.

\iffalse % TODO

If we have a writer who finishes, the reader will have SemCountGuard and all
readers who started after that writer will be in the queue for SemCountGuard.
If the writer tries to update the aw count before it updates SemOKtoRead, it
will fail. If the writer tries to update aw count after signaling
SemOKtoRead, every reader which arrived after the writer will serialise.
Any new reader will wait on SemOKtoRead. Presumably the last Semaphore will
raise this. If not the whole situation deadlocks. So either this deadlocks
or the process serialises. This is awful.

This does not update SemOKtoWrite. This means the writer must now manage
whether it is allowed to write. This will require the SemCountGuard. So
either it acquires the SemCountGuard and deadlocks the whole system as it
waits for readers to decrement -- which they can't do because it has the
SemCountGuard -- or loops and keeps asking for it until eventually there are
no readers. This implements no priority.

\fi

\item Using the above example, comment on the ease of monitor programming
and implementation, compared with Semaphore programming. Assume a monitor
\textit{ReadersWriters} defines condition variables \textit{SemOKtoRead} and
\textit{SemOKtoWrite}.

Monitors are blocks of code with the requirement that at most one thread be
executing any of them. This is a very intuitive way of thinking about
concurrency and makes programming far easier.

Monitors have condition variables. These are queues of threads which are
waiting for particular predicate to be true. These allow us to wake threads
waiting on conditions when that condition becomes true without having those
threads repeatedly retrying the condition. In this example, the condition is
simple, however it can be arbitrarily complex and computationally expensive.

When a thread starts waiting on a condition variable, it will release the
monitor. This allows other threads to enter the monitor. Therefore, threads
which wait on condition variables must ensure all shared objects are in
consistent states. This can lead to bugs.

Monitors are a very easy way of getting good performance when code is
written well. However, it is very easy to forget to signal a condition
variable (leading to threads waiting unnecessarily) or to signal it when the
predicate hasn't actually changed (leading to wasted work).

Furthermore, monitors can overly serialise code. Consider the example below.
It's safe for Readers to become active (not reading) and writers to become
active at the same time. However, under this monitor implementation that
cannot occur.

Semaphores require a much deeper understanding of the system to
implement properly. It's very easy to deadlock or have race conditions when
using Semaphores -- and avoiding either requires thorough analysis of code.
Semaphores give much lower-level control of the code and can therefore be
more efficient.

\begin{lstlisting}

monitor ReadersWriters:
	ar = rr = aw = ww = 0
	condition SemOKtoRead, SemOKtoWrite
	def startread():
		ar += 1
		while aw > 0:
			wait(SemOKtoRead, ReadersWriters)
		rr += 1

	def endread():
		ar -= 1
		rr -= 1
		if rr == 0:
			signal(SemOKtoWrite)

	def startwrite():
		aw += 1
		while rr > 0:
			wait(SemOKtoWrite, ReadersWriters)
		ww += 1

	def endwrite():
		ww -= 1
		aw -= 1
		if aw == 1:
			signal(SemOKtoRead)
\end{lstlisting}

\iffalse

Monitor programming is much easier and more intuitive than Semaphores.
However, it does serialise code more than Semaphores and other primitives do.

Monitors are a set of blocks of code; which have the requirement that at most
one thread can be executing any of them at once. Notice that this
requirement is only extended to ``executing'' that thread. Any number of
threads can be waiting on a condition variable inside a monitor.

Condition Variables are queues of threads which are waiting for the same
predicate to become true. They should always be held inside a while loop to
ensure the thread waiting on one only continues if the predicate is actually
true.

Condition Variables are not themselves related to any predicate. They are
heads of queues. If you wish to use a condition variable, it must be placed
in a while loop:

\begin{lstlisting}[language=C]
while (!p()){
	wait(&condition_variable, &monitor);
}
\end{lstlisting}

If a thread starts waiting on a condition variable inside a monitor, other
threads can enter the monitor. It is therefore essential that threads only
wait on condition variables after leaving objects in consistent states.

It's quite easy to wait on a condition variable and leave a thread in an
inconsistent state. Or to have a condition variable outside a while loop --
leading to edge cases where a thread may condition with a prerequisite false.

It's also easy to forget to signal a condition variable after changing
internal state such that it could become true.

\fi

\item Describe and comment on the Java approach to supporting mutual
exclusion and condition synchronisation.

The Java primitive ``synchronized'' implements monitors. It can be passed an
object and Java will by default create a mutex around it which will prevent
any other procedure to run a ``synchronized'' block. For instance readers
could synchronize on a shared counter before changing the ar or aw counts.

It's very common to synchronize on the item itself so there is additional
syntactic sugar to simplify this. The following two functions are identical
-- both take out a mutex on the object itself.

\begin{lstlisting}[language=Java]
public T1 f(T2 val){
	synchronized(this){
		...
	}
}

public synchronized T1 f(T2 val){
	...
}
\end{lstlisting}

Java mutual exclusion is intuitive and removes a lot of low-level
implementations, leaving that to the compiler. However, mutual exclusion and
concurrency control in general is still difficult.

\item Explain how active objects and guarded commands avoid some of the
issues arising in the above programs.

% TODO

\iffalse % TODO

An active object is like a monitor but only one thread can be executing at
once. This makes everything simpler. It's like having a thread which is
called externally in parallel. This simplifies concurrency greatly (but can
also serialise unnecessarily).

\fi

\end{enumerate}

\end{examquestion}

\section{Implementation of other procedures}

Write out the other three methods (endread, startwrite and endwrite) and
state exactly what you are using the four variables ar, rr, aw and ww for.

Rather than implementing 3 and trying reason about the behaviour of the
$4^{\text{th}}$, I decided to implement all 4 methods.

Writers take priority over readers in all situations. New readers cannot start
until there are no active writers. On creation, a new writer only has to
wait for reading readers or other writers. There is no situation where a
reader starts reading while there is an active writer.

I used only four Semaphores. However, I did not use them for the same
things as in the question. I have therefore renamed them:

\begin{itemize}

\item ReaderGuard

This locks updates to ar and rr.

\item WriterGuard

This locks updates to ww

\item ReaderLock

This is used to implement priority, preventing new readers from starting
until there are no active writers.

\item WriterLock

This is used to prevent multiple writers writing at once and to prevent
writers starting writing while there are still any reading readers.

\end{itemize}

Uses of the four variables:

\begin{itemize}

\item ar

ar is a count of the number of active readers. This is not necessary for the
implementation -- it is only kept up to date for consistency. ar is locked
by ReaderGuard.

\item rr

rr is a count of the number of readers which are reading from the file. This
is used to keep a count of how many readers the active writers are waiting for.
When the number of reading readers changes from 0 to 1, WriterLock is
acquired so that new writers cannot start writing while readers are still
reading. When the number of reading readers decreases to zero, WriterLock is
signalled allowing writers to start writing. rr is locked by ReaderGuard.

\item aw

aw is the count of the number of writers which want to write to the file.
When writing writers finish writing, they check if aw is zero before releasing
ReaderLock and allow new readers. New writers check aw before attempting to
acquire ReaderLock. aw is locked by WriterGuard.

\item ww

ww is the number of writers who are currently writing to the file. It is not
necessary for the implementation, it is just kept updated for consistency.
ww is locked by WriterLock.

\end{itemize}

\begin{lstlisting}
ReaderLock = Semaphore(1)
WriterLock = Semaphore(1)
ReaderGuard = Semaphore(1)
WriterGuard = Semaphore(1)

def startread():
	wait(ReaderGuard)
	ar += 1
	signal(ReaderGuard)
	wait(ReaderLock)
	wait(ReaderGuard)
	rr += 1
	if rr == 1:
		wait(WriterLock)
	signal(ReaderGuard)
	signal(ReaderLock)

\end{lstlisting}

\begin{lstlisting}

def endread():
	wait(ReaderGuard)
	rr -= 1
	if rr == 0:
		signal(WriterLosck)
	ar -= 1
	signal(ReaderGuard)

\end{lstlisting}

\begin{lstlisting}

def startwrite():
	wait(WriterGuard)
	aw += 1
	if aw == 1:
		wait(ReaderLock)
	signal(WriterGuard)
	wait(WriterLock)
	ww += 1

\end{lstlisting}

\begin{lstlisting}

def endwrite():
	ww -= 1
	signal(WriterLock)
	wait(WriterGuard)
	aw -= 1
	if aw == 0:
		signal(ReaderLock)
	signal(WriterGuard)

\end{lstlisting}

\begin{examquestion}{2000}{3}{1}

\begin{enumerate}[label=(\alph*)]

\item A software module controls a car park of known capacity. Calls to the
module's procedures \textit{enter}() and \textit{exit}() are triggered when
cars enter and leave via the barriers.

Give pseudocode for the \textit{enter} and \textit{exit} procedures

\begin{enumerate}[label=(\roman*)]

\item if the module is a monitor

This felt very light for 8 marks -- however I could not see what else the
question could be asking for.

\begin{lstlisting}

public class CarPark{
	private int cars = 0;

	public synchronized void enter(){
		cars++;
	}

	public synchronized void exit(){
		cars--;
	}
}

\end{lstlisting}

\item if the programming language in which the module is written provides
only Semaphores

\begin{lstlisting}

WriteLock = Semaphore(1)
cars = 0;

def enter():
	wait(WriteLock)
	cars += 1
	signal(WriteLock)

def exit():
	wait(WriteLock)
	cars -= 1
	signal(WriteLock)

\end{lstlisting}

\end{enumerate}

\item Outline the implementation of

\begin{enumerate}[label=(\roman*)]

\item Semaphores

% TODO talk about concurrent queues?

Semaphores have an integer variable, a queue of threads and two methods
(wait and signal). Operations on both the variable and the queue need to be
atomic.

A Semaphores is initialised with a value. The variable is set to this value.
Calling ``wait'' when the variable is non-zero decrements the
variable atomically (using compare-and-swap or load-linked,
store-conditional). Calling ``wait'' when the variable is zero will
atomically place the thread onto the tail of the queue waiting on the
Semaphore.

Threads can also call ``signal'', which wakes the thread on the head of the
queue (if the queue is non-empty) or increments the variable.

Here is an implementation of atomic increment using compare-and-swap; and
load-linked, store-conditional.
\begin{lstlisting}
void inc_cas(int *p){
	do{
		int i = *p;
		i++;
	}
	while (cmpxchg p, i)
}
\end{lstlisting}

\item monitors

Monitors can be implemented by attaching a binary Semaphore to an object.
When entering any method of this object, wait on the Semaphore; and signal
the Semaphore when leaving from it. This includes entering and leaving while
waiting on Condition Variables. Condition Variables are threads queues
which are implemented in the same way as the threads queues in Semaphores.

\iffalse % TODO

A monitor is an object with an associated Semaphore and several associated
atomic queues. On entry to a code block, wait on the Semaphore. On exiting
the code block, signal. On calling wait(conditionvariable, monitor) signal.
On re-entry wait on the Semaphore. This permits only one thread running at
once. The condition variables are just the thread component of a sempahore.
This allows you to wake threads as a one-off.

Broadcast just iterates through all the threads on the condition variable 
queue and wakes all of them.

\fi

\end{enumerate}

\end{enumerate}

\end{examquestion}

\end{document}
