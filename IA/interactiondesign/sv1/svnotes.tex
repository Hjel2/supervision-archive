\documentclass[10pt, a4paper]{article}

% DO NOT add \usepackage commands here.  Place any custom commands
% into your SV work files.  Anything in the template directory is
% likely to be overwritten!

\usepackage{fancyhdr}

\usepackage{lastpage}       % ``n of m'' page numbering
\usepackage{lscape}         % Makes landscape easier

\usepackage{verbatim}       % Verbatim blocks
\usepackage{epsfig}         % Embed encapsulated postscript
\usepackage{array}          % Array environment
\usepackage[nolinks]{qrcode}         % QR codes
\usepackage{enumitem}       % Required by Tom Johnson's exam question header

\usepackage{hhline}         % Horizontal lines in tables
\usepackage{siunitx}        % Correct spacing of units
\usepackage{amsmath}        % American Mathematical Society
\usepackage{amssymb}        % Maths symbols
\usepackage{amsthm}         % Theorems

\usepackage{ifthen}         % Conditional processing in tex

\usepackage[top=3cm,
            bottom=3cm,
            inner=2cm,
            outer=5cm]{geometry}

% PDF metadata + URL formatting
\usepackage[
            pdfauthor={\studentname},
            pdftitle={\svcourse, SV \svnumber},
            pdfsubject={},
            pdfkeywords={9d2547b00aba40b58fa0378774f72ee6},
            pdfproducer={},
            pdfcreator={},
            hidelinks]{hyperref}

\renewcommand{\headrulewidth}{0.4pt}
\renewcommand{\footrulewidth}{0.4pt}
\fancyheadoffset[LO,LE,RO,RE]{0pt}
\fancyfootoffset[LO,LE,RO,RE]{0pt}
\pagestyle{fancy}
\fancyhead{}
\fancyhead[LO,RE]{{\bfseries \studentname}\\\studentemail}
\fancyhead[RO,LE]{{\bfseries \svcourse, SV~\svnumber}\\\svdate\ \svtime, \svvenue}
\fancyfoot{}
\fancyfoot[LO,RE]{For: \svrname}
\fancyfoot[RO,LE]{\today\hspace{1cm}\thepage\ / \pageref{LastPage}}
\fancyfoot[C]{\qrcode[height=0.8cm]{\svuploadkey}}
\setlength{\headheight}{22.55pt}

\ifthenelse{\equal{\jkfside}{oneside}}{

 \ifthenelse{\equal{\jkfhanded}{left}}{
  % 1. Left-handed marker, one-sided printing or e-marking, use oneside and...
  \evensidemargin=\oddsidemargin
  \oddsidemargin=73pt
  \setlength{\marginparwidth}{111pt}
  \setlength{\marginparsep}{-\marginparsep}
  \addtolength{\marginparsep}{-\textwidth}
  \addtolength{\marginparsep}{-\marginparwidth}
 }{
  % 2. Right-handed marker, one-sided printing or e-marking, use oneside.
  \setlength{\marginparwidth}{111pt}
 }

}{
 % 3. Alternating margins, two-sided printing, use twoside.
}

\setlength{\parindent}{0em}
\addtolength{\parskip}{1ex}

% Exam question headings, labels and sensible layout (courtesy of Tom Johnson)
\setlist{parsep=\parskip, listparindent=\parindent}
\newcommand{\examhead}[3]{\section{#1 Paper #2 Question #3}}
\newenvironment{examquestion}[3]{
    \examhead{#1}{#2}{#3}\setlist[enumerate, 1]{label=(\alph*)}\setlist[enumerate, 2]{label=(\roman*)}
    \marginpar{\qrcode{https://www.cl.cam.ac.uk/teaching/exams/pastpapers/y#1p#2q#3.pdf}}
    \marginpar{\footnotesize \url{https://www.cl.cam.ac.uk/teaching/exams/pastpapers/y#1p#2q#3.pdf}}
}{}



\begin{document}

\section*{Interaction Design Supervision 1}

\section{Why Interaction Design?}

\begin{itemize}

\item Interaction design is about building projects which are practical, usable and encourage users to stay with your
product.

\item Your software engineers need to have some idea as to who the users are. It's the designers job to decide who
the user is.

\item ui often gets overlooked as people don't come with the right background. People are very ambiguous as usually
they don't understand their users and its' often hard to know whether you've made the right decision.

\item Bad interaction design in a safety critical system can make a system more dangerous. (consider the kegworth crash)

\item Software designers are not representative of the people in this country. If you're trying to build software for
people other than yourself, then you need to learn to understand people who aren't like you. Interaction Design
teaches you how to do this in a UCD fashion.

\end{itemize}

\section{}

\begin{itemize}

\item

Lab Based observation: you do things in a special lab where you're meant to use it.

Lab based environments are controlled.

Usually you'd observe it to have a learning affect.

Lab studies are planned.

Lab studies collect data -- you ask the user questions afterwards and have a formal data collection process.

In Lab Based observation, we are impartial and don't usually try to help the user. We just let them get on with what
they're trying to do.

You shouldn't have much pressure in a lab based observation.

We're often worried about the power balance between the researcher and the people who are being researched. You try
to get this balance by having 1--2 people.

Usually you have one person in the room with the participant and a large number of other people behind a one-way
mirror. Usually lab based studies will have senior management looking at how the users use the product.

\item

With questionnaires you need to be very careful not to use any leading questions.

When you do questionnaires you will have a lot of sampling bias. For example people who have more time will be more
likely to answer your survey.

Your target audience may be overly small. It can be quite difficult to find people to answer them.

\item

Focus groups:

Get a bunch of people to talk about something.

You can have a lot of people and often there will be a very small group of people who are actually saying stuff.

There is often peer pressure to agree and peer pressure not to appear stupid. This leads to bias.

You can have difficultly staying on topic.

Depending on the problem, this can make people either more of less sensitive to talking about issues. For example
you're.

\item

Competitor research: (market analysis)

Just google what you want to do.

\item

Interviews:

\item

Task focused documentation:

It's often very important to consider things in context.

However, this is not particularly useful if the range of things is overly wide.

For example microsoft word where you may want to use it for a million things -- it's stupid to have a 1000 page
manual -- nobody will ever understand all of it.

\item Ethnography

This is where you follow people around for ages while they do their tasks.

\end{itemize}

\section{Stakeholders}

\begin{itemize}

\item Think about all stakeholders:

For example Astronomy Weather App:

\begin{itemize}

\item Astronomy/physics teachers

\item Amateur astronomers

\item Telescope vendors

\item weather API provider

\item smartphone vendors on which they install the app

\item Advertisers in our app

\item people who write articles for astronomy

\item people who do street lighting (city council)

\end{itemize}

Often we're unsure who the primary stakeholder is -- we just have to pick one group and decide these are our primary
stakeholders.

\subsection{Have we missed anyone}

\begin{itemize}

\item Ask people in interviews who else you should be talking to

\item Walk through uses of your app

\item Many websites have a ``what are you'' when you log into a website -- include an ``other'' section

\item Observation in the field

\item how did you come across us

\item If you're overly serious, then do things like taking ads out and ask people what they should bother with? ``We're
 thinking about doing this -- do you care''.

\item In general this is a pretty difficult problem -- it's hard to figure out who you missed

\end{itemize}

\subsection{Persona}

\begin{itemize}

\item A representative user which gives a face and a name to a user.

\item This gives us a lot of empathy with users and makes it easier to make decision since you have specific users in
mind.

\item You should usually have multiple personas and can think about whether it will help specific users.

\item It often reveals assumptions which turn out not to be true

For example with 3D printers, you might think it would be professionals in industrial. In production labs you have
one person who knows the 3D printer. They have a phd in maths and take a model and think about how this works. You
need to think about all personas. You may have someone doing the industrial and the fashion design. One is a mechEng
and the other is a graphical designer.

\item It helps us to distinguish different types of users

\end{itemize}

\subsection{What goes wrong with personas}

\begin{itemize}

\item you make assumptions about personas which turn out to be wrong

\item you make personas more ideal than they are actually

\item You can be specific about the wrong things

\item You could overspecialise the app to help with one individual

\item You can have too many personas that you totally forget what you're doing anymore.

\item You can also make embarrassing mistakes -- like giving personas bad names etc. This looks very bad if it gets
leaked since it can give users a perception that you think they're idiots.

\item You can end up stereotyping niche users.

\item Users may overlap with multiple personas

\item it is slightly hard to do this exercise and get taken seriously by software engineers as they think ``what are
you doing with this BS?'' This is a recurring problem with interaction design.

\end{itemize}

\subsection{Finding personas}

\begin{itemize}

\item Interviews

\item Large scale data analysis

There is often clustering in the set of features into a few small groups.

You then build a persona from this by finding a set of these people and talking to them.

Alternatively interview people and figure out which cluster they fall into.

\end{itemize}

\end{itemize}

\section{Requirements}

\begin{itemize}

\item Ask users in a questionnaire

Often the things people give you is very, very obvious. For example you get told you want the thing to be ``easy to
use'' in every single regard.

People don't really care about questionnaires and so they don't often answer it properly. They don't think about the
answers and often just give the easiest answer.

You don't get good answers since the amount of effort required to think is too much to bother to give the survey a
good answer.

If users instantly know what you want then you may get good answers.

Giving understandable answer and not leading questions are mutually exclusive. So questionnaires are generally pretty
bad.

Don't use questionnaires in general. Questionnaires, especially longer questionnaires tend not to be particularly
good tools. You usually get very short questionnaires -- give me a rating and tell me why.

Questionnaires are usually used when the amount of attention required is really short.

\item Ask users in interviews

\end{itemize}

\section{Visual Design}

Interfaces are usually based on analogy to the physical world.

Design is often done by people assuming who the world is like them.

Often people think that the way to go is to make the best thing. And completely ignore that every single user is used
to doing things in a different and contradictory way. Often the best design to do is to make one that's a hybrid
between the best thing and the current (bad) convention and then gradually change towards the good thing.

\subsection{Gestalt Descriptions}

\begin{itemize}

\item Law of Common fate

Scrolling up

\item Similarity

Messages sent by other people are a different colour to the message you sent

All messages have the same shape.

\item Enclosure

Messages are in a bubble

\item Proximity

\item Figure/Ground

Messages are the foreground and the wallpaper is the background

\item Anomaly

You just don't use this because it isn't great. People do notice it but it isn't great.

\end{itemize}

\section{Attention}

\begin{itemize}

\item Visual stimulus attract peoples attention

\item Auditory stimulus attract peoples attention

\item Peoples attention is limited. To make things safer you can have priorities (hierarchical prioritisation).
However, it's often hard to figure out which order to do things in.

\item In power stations, you have white background noise and when it turns off it's very obvious. This is a way of
alerting people to issues. This would be one of many alarms.

\item People are sensitive both to stimulus and to the absence of stimulus.

\end{itemize}

\section{Java}

You have to group people. Almost everyone is somehow linked.
You then have to start finding common traits etc. You then abstract the differences and then start
designing.

With problems like this there are a lot of resources. Programming languages often employ almost no
interaction designers -- which is a serious problem.

Highly technical problems almost always have very bad interaction design. This is often because users are highly
qualified and so care more about \textit{what} they can do rather than how difficult it is to do it.

Designers are therefore focused on professionals. 

\end{document}
