\newcommand{\svcourse}{CST Part IA: Software Engineering and Security}
\newcommand{\svnumber}{1}
\newcommand{\svvenue}{Microsoft Teams}
\newcommand{\svdate}{2022-05-11}
\newcommand{\svtime}{15:00}
\newcommand{\svuploadkey}{CBd13xmL7PC1zqhNIoLdTiYUBnxZhzRAtJxv/ytRdM1r7qIfwMsxeVwM/pPcIo8l}

\newcommand{\svrname}{Dr Sam Ainsworth}
\newcommand{\jkfside}{oneside}
\newcommand{\jkfhanded}{yes}

\newcommand{\studentname}{Harry Langford}
\newcommand{\studentemail}{hjel2@cam.ac.uk}


\documentclass[10pt,\jkfside,a4paper]{article}

% DO NOT add \usepackage commands here.  Place any custom commands
% into your SV work files.  Anything in the template directory is
% likely to be overwritten!

\usepackage{fancyhdr}

\usepackage{lastpage}       % ``n of m'' page numbering
\usepackage{lscape}         % Makes landscape easier

\usepackage{verbatim}       % Verbatim blocks
\usepackage{listings}       % Source code listings
\usepackage{graphicx}
\usepackage{float}
\usepackage{epsfig}         % Embed encapsulated postscript
\usepackage{array}          % Array environment
\usepackage{qrcode}         % QR codes
\usepackage{enumitem}       % Required by Tom Johnson's exam question header

\usepackage{hhline}         % Horizontal lines in tables
\usepackage{siunitx}        % Correct spacing of units
\usepackage{amsmath}        % American Mathematical Society
\usepackage{amssymb}        % Maths symbols
\usepackage{amsthm}         % Theorems

\usepackage{ifthen}         % Conditional processing in tex

\usepackage[top=3cm,
            bottom=3cm,
            inner=2cm,
            outer=5cm]{geometry}

% PDF metadata + URL formatting
\usepackage[
            pdfauthor={\studentname},
            pdftitle={\svcourse, SV \svnumber},
            pdfsubject={},
            pdfkeywords={9d2547b00aba40b58fa0378774f72ee6},
            pdfproducer={},
            pdfcreator={},
            hidelinks]{hyperref}

\renewcommand{\headrulewidth}{0.4pt}
\renewcommand{\footrulewidth}{0.4pt}
\fancyheadoffset[LO,LE,RO,RE]{0pt}
\fancyfootoffset[LO,LE,RO,RE]{0pt}
\pagestyle{fancy}
\fancyhead{}
\fancyhead[LO,RE]{{\bfseries \studentname}\\\studentemail}
\fancyhead[RO,LE]{{\bfseries \svcourse, SV~\svnumber}\\\svdate\ \svtime, \svvenue}
\fancyfoot{}
\fancyfoot[LO,RE]{For: \svrname}
\fancyfoot[RO,LE]{\today\hspace{1cm}\thepage\ / \pageref{LastPage}}
\fancyfoot[C]{\qrcode[height=0.8cm]{\svuploadkey}}
\setlength{\headheight}{22.55pt}


\ifthenelse{\equal{\jkfside}{oneside}}{

 \ifthenelse{\equal{\jkfhanded}{left}}{
  % 1. Left-handed marker, one-sided printing or e-marking, use oneside and...
  \evensidemargin=\oddsidemargin
  \oddsidemargin=73pt
  \setlength{\marginparwidth}{111pt}
  \setlength{\marginparsep}{-\marginparsep}
  \addtolength{\marginparsep}{-\textwidth}
  \addtolength{\marginparsep}{-\marginparwidth}
 }{
  % 2. Right-handed marker, one-sided printing or e-marking, use oneside.
  \setlength{\marginparwidth}{111pt}
 }

}{
 % 3. Alternating margins, two-sided printing, use twoside.
}


\setlength{\parindent}{0em}
\addtolength{\parskip}{1ex}

% Exam question headings, labels and sensible layout (courtesy of Tom Johnson)
\setlist{parsep=\parskip, listparindent=\parindent}
\newcommand{\examhead}[3]{\section{#1 Paper #2 Question #3}}
\newenvironment{examquestion}[3]{
\examhead{#1}{#2}{#3}\setlist[enumerate, 1]{label=(\alph*)}\setlist[enumerate, 2]{label=(\roman*)}
\marginpar{\href{https://www.cl.cam.ac.uk/teaching/exams/pastpapers/y#1p#2q#3.pdf}{\qrcode{https://www.cl.cam.ac.uk/teaching/exams/pastpapers/y#1p#2q#3.pdf}}}
\marginpar{\footnotesize \href{https://www.cl.cam.ac.uk/teaching/exams/pastpapers/y#1p#2q#3.pdf}{https://www.cl.cam.ac.uk/\\teaching/exams/pastpapers/\\y#1p#2q#3.pdf}}
}{}


\begin{document}

\section{Analytical Evaluation}

Consider the website www.ikea.com/gb/en/.

As the new User Researcher on the team, you have been asked by the product manager to evaluate the usability of the
website.

\begin{enumerate}

\item Using the 10 usability heuristics, perform a Heuristic Evaluation. Identify what are the problems with the
website and which heuristics are violated for each problem. Rate the severity of each problem and suggest UI redesigns
through which the problems could be addressed.

The ten heuristics are:

\begin{itemize}

\item Visibility of system status

\begin{itemize}

\item

\end{itemize}

\item Match between the system and the real world

\begin{itemize}

\item Truck to symbolise postcode -- pretty good match

\item camera icon to indicate image search

\item magnifying glass to indicate search (standard)

\item outline of a person for account information (standard trend)

\item heart icon for favourites

\item shopping basket for basket

\item Overall it matches really well

\item sheet of paper with a line on it and a ``+'' to mean ``create a wish list''.

\item Shopping basket with a ``+'' to mean add to basket. Minor inconsistency with the
symbol for creating a list since it implies we are ``creating a new basket'' rather than adding
to the one basket we have. Might be more consistent to have a picture of a basket with
an arrow pointing into it.

\item Image of a shop means select store

\end{itemize}

Viewing the page in Welsh:

Some Welsh speakers can't speak english. Ikea should be able to support this by having an option
to change language to minority languages. I couldn't find an option for this.
It seems to imply you can by having a change country icon with the normal icon for
changing language but it links you to worldwide ikea. If I didn't speak english I'd be
unable to find a language I could use. The website did have an auto-translated version through
google -- however this website didn't work and didn't allow you to find products or navigate properly,
saying after any search that no products with those filters were found (sometimes with the error message in english!).

\item User control and freedom

\begin{itemize}

\item Pretty good overall -- there are breadcrumbs if you go more than one
click deep into a link.

However, there are some routes where the breadcrumbs link to websites you didn't visit.
The way in this case to go back is to click the $\leftarrow$ button in the browser.

This inconsistency may not be intuitive to some users who may click on the breadcrumbs and
end up somewhere totally different (although I appreciate that the breadcrumbs are for each
item and the website is an arbitrary graph so having the full link would be unsuitable).

\end{itemize}

\item Consistency and standards

``+'' in the bottom corner of an icon can mean either ``create'' or ``add to''

The icon for ``Shopping Bag'' is a picture of a basket

Many links are in bubbles but some are not and don't have icons. For example if you're about to buy a
product then you don't see the symbol for remove or add to wishlist -- only the text.

If you look in the bag in split screen then there is a menu which opens up to give only one more
option despite the fact the menu icon takes up exactly as much space as the icon for add to wishlist and
the screen is not close to being full.

\item Error prevention

In some screens (after adding to a shopping bag), icons which add to shopping bag and
wishlists pop up when you hover over them.

in some screens the ``add to basket'' icon is very close to other buttons (for example to see variants) of the product.

\item Recognition rather than recall

I could use the site without needing any documentation simply from the icons and
recognising what they did. This was quite efficient. I decided to test this by
navigating the page after translating it to Chinese -- a language I do not speak.
After doing this, I was unable to navigate the website at all -- it was not as
usable as I expected.

\item Flexibility and efficiency of use

Really liked this, found a few shortcuts which would make use faster for experienced
users (although use was not slow elsewhere).

\item Aesthetic and minimalist design

Very good.

\item Help users recognize and recover from errors

Poor. Some ``page not founds'' link you to a text page containing normal text
saying ``not found'' in the top left and nothing else (no links, nothing). Others link to pages which give you links
to the main ikea page -- not the ikea GB page. A third brings you to a page with a
message

Oops! Something went wrong :(\\
The page you are looking for can't be found

And a link to the ikea GB homepage.

\item Help and documentation

Couldn't find any -- although the page was pretty intuitive. There was no information about what a
shopping bag or wishlist was etc -- although it's not unintuitive.

\end{itemize}

\item Do a Cognitive Walkthrough to buy something (e.g a desk lamp). Define the
inputs, step through the action sequences, record the important information at
each step and make a list of suggestions for revising the UI to solve the issues
discovered.

\begin{itemize}

\item Start on the homepage. Big cookie window opens up. I had to click on cookie
settings. A pane then opened on the side. I had to look around for a bit to figure
out how to disable them (the button at the bottom was not even on my screen when the
panel opened). This was very inconvenient.

\item I clicked on the search bar on the top and typed in desk lamp. The window
then opened and a black popup came up (Hej! Add your postcode to see delivery information for
our products) which covered the text saying what the results were. I had to click an
``x'' button.

\item The screen for this is pretty good. It says all the information I need to
know.

\item I then see the compare button in the top right. I click on it and buttons in
the top left appear which allow me to compare lamps.

\item I find a lamp I like and click on compare. The screen then jolts downwards
1cm as a bar along the top appears. I find a second lamp I like and click compare.

\item I reach the bottom of the page and click show more.

\item At this point I start seeing less irrelevant results so decide I'll compare the
two products I've already found.

\item It brings me to a new page.

\item I start scrolling down the page and the page starts jolting and flashing
rapidly as the page reorganises itself hundreds of times per second.
This is very uncomfortable and looking at it for a while gives me a mild
headache.

\item After scrolling back up I decide not to scroll down again.

\item next I click ``show only differences''. Nothing changes.

\item I realise the two lamps I've chosen are in fact different colour variants
of the same lamp! They've got a different description but seem otherwise identical.

\item I decide on the white lamp and click on it to go to the main information page
about it.

\item After clicking on it, I'm presented with a (very) large image (so large
it can't all be displayed in my browser at once) and a slide saying
1 of 9. I'm confused as to how to see the other images.

\item After slight thought I realise I should swipe right. I do so. This displays
the next picture.

\item I then swipe left to go back. Unfortunately this conflicts with the
builtin ``back'' command for my browser and brings me back to the compare
page. (this didn't happen consistently but it did if I was at the first picture
or swiped left while the pictures were still swapping).

\item I swipe through and see the images. I'm excited. It's a good lamp.

\item Now I scroll down to see the product details -- checking for any major flaws
or cool features.

\item A narrow (awkward to use) panel pops up on the right hand side of the screen.
I was expecting the menu to drop-down. The product information is more technical --
nothing like how bright it is or any adjustment options or buttons.

\item I click off and have a look at the technical information. It's technical and not
useful (except for the lifetime -- 25000 hours -- sounds good).

\item I then look at the measurements, nothing useful.

\item I've still not seen a specification saying what buttons it has or how it works.
At this point I'm a bit annoyed and start to look through the reviews. As expected,
pretty positive.

\item Now I look back at the pictures in the hope of seeing something that'll tell me
more about it. I think I see a switch on the cable. I scroll up to see the picture in full.

\item Now I see a switch on a cable. I'm not sure if the angle is adjustable. I find
one image in which the angle is adjusted and conclude that it must be adjustable. I'm ready
to buy it.

\item I now click add to shopping bag and a large blue popup covers the screen advertising
lamps and similar products. I've just bought one lamp -- I don't need another!

\item I see the small white underlined text saying ``continue to shopping bag''.
I scroll down and see ``continue to checkout''. It brings me to a new page. I scroll down
and see a question. Delivery or Collection and a box inviting me to input my postcode.
I do so.

\item It suggests that I can collect from ikea for \pounds 0! I'm sold. I
click ``select a store'' and it gives suggestions over 100 miles away! I then
click map view and see they don't have any stores closer than that. So I decide
I'll have it delivered. I click off and click delivery. It suggests tuesday.
Unfortunately I'm busy then. I ask to change the date and it allows me to select
a day but gives no finer grained control than 7am-7pm. Damn! I work full time and
can't do these times. I relent and accept it'll be delivered when I'm out.

\item I input my details and it brings me to a payment screen.

\end{itemize}

Suggestions:

\begin{itemize}

\item Stop the popups! Just. Don't do popups they make websites much worse.

\item information on the side are award and difficult to use! Do drop-down
information instead.

\item Be careful about sudden jolts! I had two times where the screen jolted
downwards awkwardly.

\item Do careful debugging. I found a bug which could cause epileptic fits
or discourage or scare users!

\item Don't list different colours of products as different products.

\item Use smaller images.

\item Make it obvious how to do things. I was confused about how to see the next
images.

\item Consider interactions between the website and browser defaults -- I went
to the previous page when swiping an image.

\item Give more product specifications -- what was listed as product information was
actually irrelevant! I just wanted to know if there were controls or adjustments
and couldn't find that out.

\item Give a guide as to how far away stores are. Displaying pick-up options 100 miles
away is not helpful and I want to know how far away they are.

\end{itemize}

\end{enumerate}

Compare your findings between the two evaluation methods. What differences do you observe?

Heuristic evaluation is good for finding \textit{all} of the large problems across a whole
system. However, it often misses out lots of the annoying little things.

Cognitive Walkthrough is the opposite -- it's very good at finding all of the annoying
things with certain parts of the website -- however you're unlikely to actually
visit the whole website and as a result will end up missing lots of things out!

Create a table that summarises the findings, benefits, costs and limitations of HE and CW .

% TODO Do this.

I simply don't know what you want.

\section{Evaluation with Users}

Consider the same website.

Your team is working on a new feature for the product page: showing photos of the product or similar products in real
homes. The team has now designed and implemented this feature:

You have been asked to conduct usability testing to evaluate this new feature.

Create a research plan that includes the objectives of the study, the stakeholders, the user research method(s)
to be used in the study, participant recruitment, type of data collected and how it will be analysed.

\subsection*{Objectives}

\begin{itemize}

\item Establish whether including photos of the product or similar products in real
homes has a positive or negative effect on the website and on sales in general.

\item Establish where the best place to show this information is.

\end{itemize}

\subsection*{Stakeholders}

\begin{itemize}

\item Ikea customers

\item People posting photos of ikea furniture in their homes

\item Ikea staff

\item Competitor companies

\item The website provider on which the ikea website is based

\end{itemize}

\subsection*{User Research Methods}

\begin{itemize}

\item Lab based observation

We can ask some users to use the ikea website and see how the pictures affect their
perception of the website and how much it made them want to buy those products compared
to having the products in normal peoples homes.

\item Card sorting

We could use this to decide where to put the pictures of it in real homes --
should it go in the normal pictures or in the reviews section or around the page?
Do people associate it with a positive review or as product information or as part
of the product itself.

\end{itemize}

\subsection*{Participant Recruitment}

\begin{itemize}

\item We should choose subsets of ikea customers and prospective ikea customers
at random -- we must always be careful about how we select users.

\end{itemize}

\subsection*{Type of Data Collected}

\begin{itemize}

\item Mass data about the users activity and purchases on the website.
Compared between the version which does have the ``see this in real homes''
and the version which does not allow you to see the product in real homes.

\end{itemize}

\subsection*{How the data will be analysed}

\begin{itemize}

\item Unsupervised learning.

\end{itemize}

\begin{examquestion}{2019}{3}{6}

\begin{enumerate}[label=(\alph*)]

\item During your practical session you were asked to create a working app for a chosen
primary stakeholder which works on both a desktop and a laptop.

Describe the primary stakeholder the app was developed for, and describe three
data gathering techniques your group used for the app to identify the user
requirements. Explain the reasons behind this choice.

Our app was developed for cambridge rowers. These people have very specific
requirements, things they do and don't care about the weather -- and there is also
a system called the CUCBC flag which currently can only be checked on a 21 year old website.
We felt that incorporating that into our app would be very useful.

\begin{itemize}

\item Questionnaires

We felt that we should get a broad idea of what rowers want in general. We
gave them a lot of basic questions to understand the things that they care about
and then had some more open-ended questions which allowed them to tell us more
about the problem. This gave us a broad overview -- however questionnaires are
flawed and many people don't put enough effort into them. We tried to mitigate this
by making our questionnaire as short as possible. From the questionnaire we
established that around 80\% of rowers use a weather app! This was far higher than
any of us were expecting.

\item Interviews

Speaking directly to the userbase is the best way to find out more about them.
So we spoke to six rowers and asked them to tell us about their last experiences,
which types of weather mattered to them and had extended conversations. These
gave us great insights. From the interviews, we established that wind-speed and
\textit{perceived temperature} (not the actual temperature) are the most important
information for rowers.

\item Competitor Research

there are lots of existing weather apps which have already done research into the
field and are very established. These apps have been operating for years and know
what does and does not work for the general population. Many of these features are
transferable to rowers. We decided to research them and decide which of these features
we wanted to incorporate into our own app (to avoid re-inventing the wheel).

\end{itemize}

\item Consider a website for purchasing clothing similar to that in the figure below.
Would it be more appropriate to use Cognitive Walkthrough or Heuristic
Evaluation to evaluate this website? Give three criteria on which to base your
decision.

For this website I would recommend Cognitive Walkthrough:

\begin{itemize}

\item Heuristic evaluation is very good at picking up all the \textit{big}
or obvious things that are wrong with a website. However, if you can't immediately
see problems then there's probably not too much to be gained from Heuristic Evaluation.
Cognitive Walkthrough, however is very good at finding all the little, annoying things
which will collectively annoy users to the extend of disliking the website.

\item The main downside with cognitive walkthrough is that it only tests the
paths which you actually test -- in a large domain it can be infeasible to test
every path -- leaving lots of sites unchecked. However, with shopping sites
almost every interface is shared by a large number of pages. So a cognitive
walkthrough to buy a rucksack will find faults with the process of buying
boots or tents! So the main disadvantage of cognitive walkthrough is non-existent
in this specific case.

\item Heuristic Evaluation yielded great improvements in the early days of
the web, before most people knew how to design websites well. Since then,
design has become much more standardised and most sites are designed by
experienced professionals. As a result, the value of Heuristic Evaluation
has gradually decreased since it's conception. The same has not happened to
Cognitive Walkthrough.

\end{itemize}

\item What does Gestalt theory describe and what is its implication for
interaction design? Describe which principles(s) are being applied for each
item in the figure below, and how, and what it tells us about the interface and
the interaction.

There are several Gestalt principles which relate items into groups. The gestalt principles are:

\begin{itemize}

\item similarity

We tend to group items which are similar together. For example in google maps the
dots indicate the current path -- although they are not connected, we group them
together because they all look the same.

\item Common fate

If the same thing happens to a group of objects then we group them together. For
example in the interface o create a blog post, if we click off the profile, all the
icons will disappear at the same time.

\item Enclosure

We associate things which are grouped together as being of the same type.
For example the interface to create the blog post has all the icons in a
bubble -- this bubble encloses the group and helps us associate them as
related.

\item Closure

Our brain will automatically fill in the gaps between lines. Consider the example
on google maps -- although none of the dots are connected, our brain fills in the
gaps between them and so we view them as one contiguous line (rather than a
set of dots in a certain pattern).

\item Good continuation

\item anomaly

Our brains are impossibly good at spotting things that don't fit in. Consider the
notification icon in the twitter interface! The notification stands out and gets
our attention -- as it is intended to!

\item proximity

We associate things which are nearby as being related. Consider the twitter profile
photo -- the tweets, following and followers information are close together and close
to the persons name -- so we associate them as being related to that person -- similarly
we associate the profile picture as being a picture of that person.

\item background and foreground split

Our brains usually split things into the background and the foreground.
Consider the twitter profile information. We split the profile into the background
(the picture of the village) and the foreground (the name, photo and information).

\end{itemize}

\end{enumerate}

\end{examquestion}

\end{document}
