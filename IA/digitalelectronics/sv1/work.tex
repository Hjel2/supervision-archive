\newcommand{\svrname}{Mr Matthew Ireland}
\newcommand{\jkfside}{oneside}
\newcommand{\jkfhanded}{right}

\newcommand{\studentname}{Harry Langford}
\newcommand{\studentemail}{hjel2@cam.ac.uk}

\documentclass[10pt,\jkfside,a4paper]{article}

\newcommand{\svcourse}{CST Part IA: Software Engineering and Security}
\newcommand{\svnumber}{1}
\newcommand{\svvenue}{Microsoft Teams}
\newcommand{\svdate}{2022-05-11}
\newcommand{\svtime}{15:00}
\newcommand{\svuploadkey}{CBd13xmL7PC1zqhNIoLdTiYUBnxZhzRAtJxv/ytRdM1r7qIfwMsxeVwM/pPcIo8l}

\newcommand{\svrname}{Dr Sam Ainsworth}
\newcommand{\jkfside}{oneside}
\newcommand{\jkfhanded}{yes}

\newcommand{\studentname}{Harry Langford}
\newcommand{\studentemail}{hjel2@cam.ac.uk}

% DO NOT add \usepackage commands here.  Place any custom commands
% into your SV work files.  Anything in the template directory is
% likely to be overwritten!

\usepackage{fancyhdr}

\usepackage{lastpage}       % ``n of m'' page numbering
\usepackage{lscape}         % Makes landscape easier

\usepackage{verbatim}       % Verbatim blocks
\usepackage{listings}       % Source code listings
\usepackage{epsfig}         % Embed encapsulated postscript
\usepackage{array}          % Array environment
\usepackage{qrcode}         % QR codes
\usepackage{enumitem}       % Required by Tom Johnson's exam question header

\usepackage{hhline}         % Horizontal lines in tables
\usepackage{siunitx}        % Correct spacing of units
\usepackage{amsmath}        % American Mathematical Society
\usepackage{amssymb}        % Maths symbols
\usepackage{amsthm}         % Theorems

\usepackage{ifthen}         % Conditional processing in tex

\usepackage[top=3cm,
            bottom=3cm,
            inner=2cm,
            outer=5cm]{geometry}

% PDF metadata + URL formatting
\usepackage[
            pdfauthor={\studentname},
            pdftitle={\svcourse, SV \svnumber},
            pdfsubject={},
            pdfkeywords={9d2547b00aba40b58fa0378774f72ee6},
            pdfproducer={},
            pdfcreator={},
            hidelinks]{hyperref}


% DO NOT add \usepackage commands here.  Place any custom commands
% into your SV work files.  Anything in the template directory is
% likely to be overwritten!

\usepackage{fancyhdr}

\usepackage{lastpage}       % ``n of m'' page numbering
\usepackage{lscape}         % Makes landscape easier

\usepackage{verbatim}       % Verbatim blocks
\usepackage{listings}       % Source code listings
\usepackage{graphicx}
\usepackage{float}
\usepackage{epsfig}         % Embed encapsulated postscript
\usepackage{array}          % Array environment
\usepackage{qrcode}         % QR codes
\usepackage{enumitem}       % Required by Tom Johnson's exam question header

\usepackage{hhline}         % Horizontal lines in tables
\usepackage{siunitx}        % Correct spacing of units
\usepackage{amsmath}        % American Mathematical Society
\usepackage{amssymb}        % Maths symbols
\usepackage{amsthm}         % Theorems

\usepackage{ifthen}         % Conditional processing in tex

\usepackage[top=3cm,
            bottom=3cm,
            inner=2cm,
            outer=5cm]{geometry}

% PDF metadata + URL formatting
\usepackage[
            pdfauthor={\studentname},
            pdftitle={\svcourse, SV \svnumber},
            pdfsubject={},
            pdfkeywords={9d2547b00aba40b58fa0378774f72ee6},
            pdfproducer={},
            pdfcreator={},
            hidelinks]{hyperref}

\renewcommand{\headrulewidth}{0.4pt}
\renewcommand{\footrulewidth}{0.4pt}
\fancyheadoffset[LO,LE,RO,RE]{0pt}
\fancyfootoffset[LO,LE,RO,RE]{0pt}
\pagestyle{fancy}
\fancyhead{}
\fancyhead[LO,RE]{{\bfseries \studentname}\\\studentemail}
\fancyhead[RO,LE]{{\bfseries \svcourse, SV~\svnumber}\\\svdate\ \svtime, \svvenue}
\fancyfoot{}
\fancyfoot[LO,RE]{For: \svrname}
\fancyfoot[RO,LE]{\today\hspace{1cm}\thepage\ / \pageref{LastPage}}
\fancyfoot[C]{\qrcode[height=0.8cm]{\svuploadkey}}
\setlength{\headheight}{22.55pt}


\ifthenelse{\equal{\jkfside}{oneside}}{

 \ifthenelse{\equal{\jkfhanded}{left}}{
  % 1. Left-handed marker, one-sided printing or e-marking, use oneside and...
  \evensidemargin=\oddsidemargin
  \oddsidemargin=73pt
  \setlength{\marginparwidth}{111pt}
  \setlength{\marginparsep}{-\marginparsep}
  \addtolength{\marginparsep}{-\textwidth}
  \addtolength{\marginparsep}{-\marginparwidth}
 }{
  % 2. Right-handed marker, one-sided printing or e-marking, use oneside.
  \setlength{\marginparwidth}{111pt}
 }

}{
 % 3. Alternating margins, two-sided printing, use twoside.
}


\setlength{\parindent}{0em}
\addtolength{\parskip}{1ex}

% Exam question headings, labels and sensible layout (courtesy of Tom Johnson)
\setlist{parsep=\parskip, listparindent=\parindent}
\newcommand{\examhead}[3]{\section{#1 Paper #2 Question #3}}
\newenvironment{examquestion}[3]{
\examhead{#1}{#2}{#3}\setlist[enumerate, 1]{label=(\alph*)}\setlist[enumerate, 2]{label=(\roman*)}
\marginpar{\href{https://www.cl.cam.ac.uk/teaching/exams/pastpapers/y#1p#2q#3.pdf}{\qrcode{https://www.cl.cam.ac.uk/teaching/exams/pastpapers/y#1p#2q#3.pdf}}}
\marginpar{\footnotesize \href{https://www.cl.cam.ac.uk/teaching/exams/pastpapers/y#1p#2q#3.pdf}{https://www.cl.cam.ac.uk/\\teaching/exams/pastpapers/\\y#1p#2q#3.pdf}}
}{}


\usepackage{multirow}
\usepackage{multicol}
\usepackage{pifont}

% If you have any additional \usepackage commands, or other
% macros or directives, put them here.  Remember not to edit
% files in the template directory because any changes will
% be overwritten when template updates are issued.

\begin{document}

\begin{enumerate}
\item
\begin{enumerate}
\item{Use Boolean algebra to show that $(X + Y).(X + Z) = X + Y.Z$}
\begin{equation*}
\begin{split}
(X + Y).(X + Z) &= X.X + X.Y + X.Z + Y.Z \\
&= X.(X + Y + Z) + Y.Z \\
&= X + Y.Z \\
\end{split}
\end{equation*}
\item{Use Boolean algebra to show that $(X + Y).(\overline X + Z) = X.Z + \overline X.Y$}
\begin{equation*}
\begin{split}
(X + Y).(\overline X + Z) &= X.\overline X + X.Z + \overline X.Y+ Y.Z \\
&= 0 + X.Z + \overline X.Y + X.Y.Z + \overline X.Y.Z \\
&= X.Z.(1 + Y) + \overline X.Y.(1 + Z) \\
&= X.Z.1 + \overline X.Y.1 \\
&= X.Z + \overline X.Y
\end{split}
\end{equation*}
\item{Which common functional unit (i.e. standard digital circuit component) does the Boolean
expression in part (b) describe?}

A switch - where X controls whether the output from Y or the output from Z is transmitted.
\end{enumerate}

\item 
The months of the year are coded in binary with January represented by $A_3A_2A_1A_0$ =
(0001) and December by (1100). Find a simplified sum-of-products expression in terms
of $A_3$, $A_2$, $A_1$, $A_0$ for the months without an $r$ in their name.
Show that a simpler expression is obtained by changing the coding so January is represented by (0000) and December by (1011).

The months which don't have $r$ in their name are:
May, June, July, August --
Which have binary representations of:
(0101), (0110), (0111), (1000) respectively \\
There are also several numbers which do not represent any months and so can be considered as ``don't cares'': (0000), (1101), (1110) and (1111).
A Karnaugh Map of this gives: \\\\
\begin{center}
\begin{tabular}{|c|c|c|c|c|c|} 
\hline
& & \multicolumn{4}{|c|}{$A_3A_2$} \\
\hline
& & 00 & 01 & 11 & 10 \\ 
\hline
\multirow{4}{2em}{$A_1A_0$} 
& 00 & X & 0 & 0 & 1 \\
& 01 & 0 & 1 & X & 0 \\
& 11 & 0 & 1 & X & 0 \\
& 10 & 0 & 1 & X & 0 \\
\hline
\end{tabular}
\end{center}
Using the Karnaugh Map we can see that the SOP form is $A_2A_1 + A_2A_0 + \overline{A_2A_1A_0}$
\\\\
If January is represented by (0000) and December by (1011) then the minterms are: (0100), (0101). (0110) and (0111). The ``don't cares'' will be (1100), (1101), (1110) and (1111).\\
On a Karnaugh Map this gives:
\begin{center}
\begin{tabular}{|c|c|c|c|c|c|} 
\hline
& & \multicolumn{4}{|c|}{$A_3A_2$} \\
\hline
& & 00 & 01 & 11 & 10 \\ 
\hline
\multirow{4}{2em}{$A_1A_0$} 
& 00 & 0 & 1 & X & 0 \\
& 01 & 0 & 1 & X & 0 \\
& 11 & 0 & 1 & X & 0 \\
& 10 & 0 & 1 & X & 0 \\
\hline
\end{tabular}
\end{center}
We can see that there is now only one prime implicant: $A_2$.\\
So the simplified SOP form is $A_2$.

\item Question 3
\begin{enumerate}
\item{Explain the Quine-McCluskey method for a function of four variables, by analogy to the
“equivalent” steps on a 4-variable Karnaugh Map. Show side-by-side the execution of the
Quine-McCluskey method and what the equivalent steps on the Karnaugh map for the
function Q(A, B, C, D) = $\sum$(0, 1, 3, 4, 7, 12, 13, 15).}
The Quine-McCluskey method starts by writing down all the minterms. This is the equivalent of marking out the minterms on a Karnaugh Map.\\
It then checks all the minterms to see whether any differ by only one bit. If so it ``ticks'' those two terms and puts the combined version in the next column. If there are any at one stage which are not used then they are prime implicants and are starred. This is the equivalent of checking Karnaugh Maps to try and find groups which cannot be made any larger (the prime implicants).\\
Finally, we would use a prime implicant chart to determine which prime implicants were in the covering set - and which were not. In the Karnaugh Map we can do this by inspection.

\begin{multicols}{2}
Karnaugh Map Method:
\begin{center}
\begin{tabular}{|c|c|c|c|c|c|} 
\hline
& & \multicolumn{4}{|c|}{$AB$} \\
\hline
& & 00 & 01 & 11 & 10 \\ 
\hline
\multirow{4}{2em}{$CD$} 
& 00 & 1 & 1 & 1 & 0 \\
& 01 & 1 & 0 & 1 & 0 \\
& 11 & 1 & 1 & 1 & 0 \\
& 10 & 0 & 0 & 0 & 0 \\
\hline
\end{tabular}
\end{center}
We would then look at the Karnaugh Map and see the largest groups we could make. In this case they would be: 
$\overline{ACD}$, $B\overline{CD}$, $\overline{ABC}$, $\overline{AB}D$, $AB\overline C$, $ABD$, $\overline ACD$, $BCD$\\
Visually we can look at this and see that there are lots of overlaps and so we only need half of them to make the covering set.\\
So a covering set (in this case there are two) is: $\overline{ABC},  \overline ACD,  ABC,  B\overline{CD}$.
\\
\begin{center}
\begin{tabular}{c c}
\multicolumn{2}{c}{The Quine-McCluskey Method:}\\
0000 \ding{51} & 000-\\
0001 \ding{51} & 0-00\\
0100 \ding{51} & 00-1\\
0011 \ding{51} & -100\\
1100 \ding{51} & 0-11\\
0111 \ding{51} & 110-\\
1101 \ding{51} & -111\\
1111 \ding{51} & 11-1\\
\end{tabular}
\end{center}
From this we can see all the prime implicants.\\
Quine-McCluskey would then use a prime-implicant chart to decide which prime implicants are redundant.
\begin{center}
\begin{tabular}{c|c c c c c c c c}
000- & X & X & & & & & & \\
0-00 & X & & & X & & & & \\
00-1 & & X & X & & & & & \\
-100 & & & & X & & X & & \\
0-11 & & & X & & X & & & \\
110- & & & & & & X & X & \\
-111 & & & & & X & & & X \\
11-1 & & & & & & & X & X \\
\hline
 & 0 & 1 & 3 & 4 & 7 & 12 & 13 & 15
\end{tabular}
\end{center}
From this Quine-McCluskey would see that there are no minterms which are represented by only one prime implicant.
So Quine-McCluskey would select a prime implicant, say $000-$ and make the covering set from that. 
This would lead to the same covering set as the Karnaugh Map: $\overline{ABC}$, $\overline ACD$, $ABC$, $B\overline{CD}$
\end{multicols}
\item{(Optional.) Write a program in your favourite programming language that implements the
Quine-McCluskey algorithm to simplify a combinatorial logic function. I’m not going to
have time to look through your source code in detail, so show me some examples of input
and output to prove it works!}

Implemented in Python.

\end{enumerate}

\item
\begin{enumerate}
\item{What is a static hazard in a combinatorial logic circuit?}

A static hazard in combinatorical logic is where the delay in logic gates causes a change in input which should not change the output to cause a short change in output.

\item{Consider the four-variable function Z(A, B, C, D) = $\sum$(1, 3, 5, 7, 8, 9, 12, 13). Identify
potential static 0 or static 1 hazards when the function is implemented in}

\begin{enumerate}
\item{Sum-of-products form}

A SOP expression is $\overline CD$ + $\overline ACD$ + $A\overline{CD}$.\\
Summary: 1001 -> 1000 causes a static 0 hazard.\\
If $A=1$, $B=0$, $C=0$, $D=1$ then the logic circuit is 1. 
If then $D$ changes to 0, then the logic circuit is in a positive state (since $A\overline{CD}$ is 1).
However, since $C\overline D$ takes one fewer logic gate to turn to 0 than $A\overline{CD}$ does to change to 1, the logic circuit will think that both are 0 for a short time and pulse 0.

This can be prevented by using the SOP form: $\overline AD + B\overline C$ instead.

\item{Product-of-sums form}

A POS expression is $(A + D).(\overline A + \overline C)$ \\
Summary: 0010 -> 1010 causes a static 1 hazard.\\
Consider if initially $A=0$, $B=0$ $C=1$, $D=0$ and then $A$ changes to 1. $A$ will change $(A + D)$ to 1 and then compare that to $(\overline A + \overline C)$ before $(\overline A + \overline C)$ has changed to 0.
Meaning the circuit will pulse as 1 despite having transitioned from a negative state to another negative state.

This can be prevented by adding another term - so that the expression is now $(\overline A + \overline C).(A + D).(C + D)$.

\end{enumerate}
\end{enumerate}

\item{How many Boolean functions of two arguments are there? Why?}

For a Boolean function with two arguments, there are 4 possible inputs. For each input there are two possible outputs. So the number of possible boolean functions with two arguments is given by 

$2^4$, which is $16$.

So there are 16 possible Boolean functions with two arguments.

\item{Using a four-variable Karnaugh map, fill it with 1s and 0s to find a function for which the minimised POS form is simpler than the minimised SOP form.}

The Karnaugh Map below is simpler when solved using a POS form than a SOP form:
\begin{center}
\begin{tabular}{|c|c|c|c|c|c|} 
\hline
& & \multicolumn{4}{|c|}{$AB$} \\
\hline
& & 00 & 01 & 11 & 10 \\ 
\hline
\multirow{4}{2em}{$CD$} 
& 00 & 0 & 0 & 0 & 0 \\
& 01 & 0 & 1 & 0 & 1 \\
& 11 & 0 & 1 & 0 & 1 \\
& 10 & 0 & 1 & 0 & 1 \\
\hline
\end{tabular}
\end{center}
Let the expression be represented by $f$.\\
In SOP form: \\
$f = \overline ABC + \overline ABD + A\overline BC + A\overline BD$\\
Which has 12 literals and 15 logic gates.\\
In POS form: \\
$\overline f = \overline{AB} + AB + \overline{CD}$\\
$f = (A + B).(\overline A + \overline B).(C + D)$\\
Which has 6 literals and 7 logic gates.\\
So the POS form for this function is simpler than the SOP form.

\item
\begin{enumerate}
\item{Using the results in Q1 (a) and (b) simplify\\
$P = (A + B + \overline C).(A + B + D).(A + B + \overline E).(A + \overline D + E).(\overline A + C)$}

\begin{equation*}
\begin{split}
P &= (A + B + \overline C).(A + B + D).(A + B + \overline E).(A + \overline D + E).(\overline A + C)\\
  &= ((A + B) + \overline C).((A + B) + D)((A + B) + \overline E).(A + \overline D + E).(\overline A + C)\\
  &= ((A + B) + \overline CD).((A + B) + \overline E).(A + \overline D + E).(\overline A + C)\\
  &= ((A + B) + \overline CD\overline E).(A + \overline D + E).(\overline A + C)\\
  &= (A + (B + \overline CD\overline E)).(\overline A + C).(A + \overline D + E)\\
  &= (AC + \overline AB+ \overline {AC}D\overline E).(A + \overline D + E)\\
  &= A.(AC + \overline AB + \overline {AC}D\overline E) + (\overline D + E).(AC + \overline AB + \overline {AC}D\overline E)\\
  &= AAC + A\overline AB + A\overline{AC}D\overline E + AC\overline D + ACE + \overline AB\overline D + \overline ABE + \overline{AC}D\overline{DE} + \overline{AC}D\overline EE\\
  &= AC.(1 + \overline D + E) + 0B + 0\overline CD\overline E + \overline AB\overline D + \overline ABE + \overline{AC}0\overline E + \overline{AC}D0\\
  &= AC + \overline AB\overline D + \overline ABE\\
  &= AC + \overline AB.(\overline D + E)\\
\end{split}
\end{equation*}
\item{Confirm your answer using a 5-variable Karnaugh map. (HINT: you may have to introduce a new kind of cell adjacency in order to be able to represent 5 variables on the map.
Alternatively, you may find your answer from (a) useful in organising the Karnaugh map.)}

In order to use a Karnaugh Map with 5 variables we need to have 2 separate 4 variable tables and create a relation between position i in table 0 and position i in table 1.

For the variables $ABCDE$ and the curcuit $P = (A + B + \overline C).(A + B + D).(A + B + \overline E).(A + \overline D + E).(\overline A + C)$ we can have the first table with $E=0$ and the second table with $E=1$.
\begin{center}
$E = 0$
\begin{tabular}{|c|c|c|c|c|c|} 
\hline
& & \multicolumn{4}{|c|}{$AB$} \\
\hline
& & 00 & 01 & 11 & 10 \\ 
\hline
\multirow{4}{2em}{$CD$} 
& 00 & 0 & 1 & 0 & 0 \\
& 01 & 0 & 1 & 0 & 0 \\
& 11 & 0 & 0 & 1 & 1 \\
& 10 & 0 & 0 & 1 & 1 \\
\hline
\end{tabular}
$E = 1$
\begin{tabular}{|c|c|c|c|c|c|} 
\hline
& & \multicolumn{4}{|c|}{$AB$} \\
\hline
& & 00 & 01 & 11 & 10 \\ 
\hline
\multirow{4}{2em}{$CD$} 
& 00 & 0 & 1 & 0 & 0 \\
& 01 & 0 & 1 & 0 & 0 \\
& 11 & 0 & 1 & 1 & 1 \\
& 10 & 0 & 1 & 1 & 1 \\
\hline
\end{tabular}
\end{center}
In the first table (where $E = 0$), $P = 1$ where $AC = 1$ or $\overline AB\overline D$.

And in the second table (where $E = 1$), $P = 1$ where $AC = 1$ or $\overline AB\overline D$ or $\overline AB$.

So:
\begin{equation*}
\begin{split}
P &= AC\overline E + ACE + \overline AB\overline{DE} + \overline AB\overline{D}E + \overline ABE\\
&= AC(E + \overline E) + \overline AB\overline D(E + \overline E) + \overline ABE\\
&= AC1 + \overline AB\overline D1 + \overline ABE\\
&= AC + \overline AB(\overline D + E)\\
\end{split}
\end{equation*}
Which is the same as the algebraic manipulation approach.
\end{enumerate}
\end{enumerate}
\end{document}