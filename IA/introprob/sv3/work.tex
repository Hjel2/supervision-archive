\newcommand{\svcourse}{CST Part IA: Introduction to Probability}
\newcommand{\svnumber}{1}
\newcommand{\svvenue}{Churchill, Room TBD}
\newcommand{\svdate}{2022-05-14}
\newcommand{\svtime}{11:00}
\newcommand{\svuploadkey}{PO5ogKIM8KQA22FZS8IAf8gxA8XKi19jxIBVHIfFZ+3GCBXuNUXS9lVN6bNYjxM/}

\newcommand{\svrname}{Mr Matthew Ireland}
\newcommand{\jkfside}{twoside}
\newcommand{\jkfhanded}{right}

\newcommand{\studentname}{Harry Langford}
\newcommand{\studentemail}{hjel2@cam.ac.uk}


\documentclass[10pt,\jkfside,a4paper]{article}
\usepackage{amsfonts}
\usepackage{mathtools}

% DO NOT add \usepackage commands here.  Place any custom commands
% into your SV work files.  Anything in the template directory is
% likely to be overwritten!

\usepackage{fancyhdr}

\usepackage{lastpage}       % ``n of m'' page numbering
\usepackage{lscape}         % Makes landscape easier

\usepackage{verbatim}       % Verbatim blocks
\usepackage{epsfig}         % Embed encapsulated postscript
\usepackage{array}          % Array environment
\usepackage[nolinks]{qrcode}         % QR codes
\usepackage{enumitem}       % Required by Tom Johnson's exam question header

\usepackage{hhline}         % Horizontal lines in tables
\usepackage{siunitx}        % Correct spacing of units
\usepackage{amsmath}        % American Mathematical Society
\usepackage{amssymb}        % Maths symbols
\usepackage{amsthm}         % Theorems

\usepackage{ifthen}         % Conditional processing in tex

\usepackage[top=3cm,
            bottom=3cm,
            inner=2cm,
            outer=5cm]{geometry}

% PDF metadata + URL formatting
\usepackage[
            pdfauthor={\studentname},
            pdftitle={\svcourse, SV \svnumber},
            pdfsubject={},
            pdfkeywords={9d2547b00aba40b58fa0378774f72ee6},
            pdfproducer={},
            pdfcreator={},
            hidelinks]{hyperref}

\renewcommand{\headrulewidth}{0.4pt}
\renewcommand{\footrulewidth}{0.4pt}
\fancyheadoffset[LO,LE,RO,RE]{0pt}
\fancyfootoffset[LO,LE,RO,RE]{0pt}
\pagestyle{fancy}
\fancyhead{}
\fancyhead[LO,RE]{{\bfseries \studentname}\\\studentemail}
\fancyhead[RO,LE]{{\bfseries \svcourse, SV~\svnumber}\\\svdate\ \svtime, \svvenue}
\fancyfoot{}
\fancyfoot[LO,RE]{For: \svrname}
\fancyfoot[RO,LE]{\today\hspace{1cm}\thepage\ / \pageref{LastPage}}
\fancyfoot[C]{\qrcode[height=0.8cm]{\svuploadkey}}
\setlength{\headheight}{22.55pt}

\ifthenelse{\equal{\jkfside}{oneside}}{

 \ifthenelse{\equal{\jkfhanded}{left}}{
  % 1. Left-handed marker, one-sided printing or e-marking, use oneside and...
  \evensidemargin=\oddsidemargin
  \oddsidemargin=73pt
  \setlength{\marginparwidth}{111pt}
  \setlength{\marginparsep}{-\marginparsep}
  \addtolength{\marginparsep}{-\textwidth}
  \addtolength{\marginparsep}{-\marginparwidth}
 }{
  % 2. Right-handed marker, one-sided printing or e-marking, use oneside.
  \setlength{\marginparwidth}{111pt}
 }

}{
 % 3. Alternating margins, two-sided printing, use twoside.
}

\setlength{\parindent}{0em}
\addtolength{\parskip}{1ex}

% Exam question headings, labels and sensible layout (courtesy of Tom Johnson)
\setlist{parsep=\parskip, listparindent=\parindent}
\newcommand{\examhead}[3]{\section{#1 Paper #2 Question #3}}
\newenvironment{examquestion}[3]{
    \examhead{#1}{#2}{#3}\setlist[enumerate, 1]{label=(\alph*)}\setlist[enumerate, 2]{label=(\roman*)}
    \marginpar{\qrcode{https://www.cl.cam.ac.uk/teaching/exams/pastpapers/y#1p#2q#3.pdf}}
    \marginpar{\footnotesize \url{https://www.cl.cam.ac.uk/teaching/exams/pastpapers/y#1p#2q#3.pdf}}
}{}



\begin{document}

\section*{Lecture 8--9}

\begin{enumerate}

\setcounter{enumi}{2}

\item Let $X$ be a single sample from a Binomial distribution $B(n, p)$.
In each of the following four cases, decide whether there exists an unbiased estimator and justify your answer.

\begin{enumerate}

\item Assume $n$ is known but $p$ is unknown and we would like to estimate $p$.

If $n \neq 0$ then $\frac{X}{n}$ is an unbiased estimator for $p$.

\[
\begin{split}
\mathbb{E}(X) &= np \Longrightarrow \\
\frac{\mathbb{E}(X)}{n} &= p \Longrightarrow \\
\mathbb{E}\left( \frac{X}{n} \right) &= p \\
\end{split}
\]

\item Assume $p$ is known but $n$ is unknown and we would like to estimate $n$.

If $p \neq 0$ then $\frac{X}{p}$ is an unbiased estimator for $n$.

\[
\begin{split}
\mathbb{E}(X) &= np \Longrightarrow \\
\frac{\mathbb{E}(X)}{p} &= n \Longrightarrow \\
\mathbb{E}\left( \frac{X}{p} \right) &= n \\
\end{split}
\]

\item Assume $n$ and $p \in (0, 1)$ are both unknown and we would like to estimate $n + p$.

There is no unbiased estimator for $n + p$. The only information we have is an unbiased estimator for
$n \cdot p$ -- this cannot be manipulated into an estimator for $n + p$.

\item Assume $n$ and $p$ are both unknown and we would like to estimate $n \cdot p$.

$X$ is an unbiased estimator for $n \cdot p$.

\[
\begin{split}
X &\sim B(n, p) \Longrightarrow \\
\mathbb{E}(X) &= np \\
\end{split}
\]

\end{enumerate}

\setcounter{enumi}{4}

\item Let $X_1, X_2 \ldots$ be a sequence of independent and identically distributed samples from the discrete
uniform distribution over $\{1, 2,\dots, N\}$.
Let $Z \coloneqq \min\{i \geq 1: X_i = X_{i + 1}\}$.
Compute $\mathbb{E}[Z]$ and $\mathbb{E}\left[(Z - N)^2\right]$.
How can you obtain an unbiased estimator for $N$.


In a given sequence of random samples over a discrete uniform distribution $\{1, 2, \dots, N\}$, the
probability that the next sample is equal to the current sample is $\frac{1}{N}$ and is constant. Hence
the first $i$ such that $X_i = X_{i + 1}$ forms a geometric distribution with parameter $\frac{1}{N}$.
So $Z \sim Geo\left(\frac{1}{N}\right)$.

\[
\begin{split}
\mathbb{E}(Z) &= \mathbb{E}\left( Geo\left( \frac{1}{N} \right) \right) \\
&= N \\
\end{split}
\]

\[
\begin{split}
\mathbb{E}\left( (Z - N)^2 \right) &= \mathbb{E}\left( Z^2 \right) - 2N \mathbb{E}(Z) + N^2 \\
&= \mathbb{E}\left( Geo\left( \frac{1}{N} \right)^2 \right) - 2N^2 + N^2 \\
&= \frac{2 - \frac{1}{N}}{\frac{1}{N^2}} - N^2 \\
&= N^2 - N \\
\end{split}
\]

$Z$ is an unbiased estimator for $N$:
\[
\mathbb{E}(Z) = N \\
\]

\setcounter{enumi}{7}

\item Let $X_1, X_2, \ldots, X_n$ be i.i.d.\ samples from a normal distribution $N(\mu, \sigma^2)$, where
$\mu$ is unknown but $\sigma$ is known. 

\begin{enumerate}

\item Prove that $Z_1 = X_1$, $Z_2 = \overline{X_n}$ and $Z_3 = \frac{(Z_1 + Z_2)}{2}$ are all unbiased estimators.

\[
\begin{split}
Z_1 &= X_1 \Longrightarrow \\
\mathbb{E}(Z_1) &= \mathbb{E}(X_1) \Longrightarrow \\
\mathbb{E}(Z_1) &= \mathbb{E}(X) \\
\end{split}
\]
So $Z_1 = X_1$ is an unbiased estimator of $X$.

\[
\begin{split}
Z_2 &= \overline{X_n} \\
\mathbb{E}(Z_2) &= \mathbb{E}\left( \overline{X_n} \right) \\
&= \frac{\mathbb{E}\left(\sum^{n}_{i=0} X_i\right)}{n} \\
&= \frac{\sum^{n}_{i=1} \mathbb{E}(X_i)}{n} \\
&= \frac{n \mathbb{E}(X)}{n} \\
&= \mathbb{E}(X) \\
\end{split}
\]
So $Z_2$ is an unbiased estimator for $X$.

\[
\begin{split}
Z_3 &= \frac{Z_1 + Z_2}{2} \\
\mathbb{E}(Z_3) &= \mathbb{E}\left(\frac{Z_1 + Z_2}{2}\right) \\
&= \frac{\mathbb{E}(Z_1) + \mathbb{E}(Z_2)}{2} \\
&= \frac{\mathbb{E}(X) + \mathbb{E}(X)}{2} \\
&= \mathbb{E}(X) \\
\end{split}
\]
So $Z_3$ is an unbiased estimator for $X$.

\item Which of the three estimators would you choose?

I would choose $Z_2$ since it has the lowest mean squared error (since the 
estimator is unbiased the MSE is equal to the variance of $Z_2$ -- which is $\frac{\sigma^2}{n}$).
$Z_2$ is therefore likely to have the least error.

Although if work to compute was a significant concern or if we didn't need a
good estimate or knew that $Var(X)$ was very low, then $Z_1$ would
also be suitable.

\end{enumerate}

\setcounter{enumi}{8}

\item Consider the following modification of the problem of estimating the population size. 
Instead of sampling without replacement, we sample with replacement. 
What is the expected number of items we need to sample until we have seen $k$ different ID's (so that we can apply the 
estimators from the lectures)?

Let $K$ be the number of samples we have to choose to see $k$ different people if we sample from a
population $p$ without replacement.

Consider this to be a sum of $k$ geometric distributions where the $i^\text{th}$ geometric distribution has
parameter $\frac{p - i + 1}{p}$. If $p \geq k$ then the sum of the expectations of these geometric distributions is:
\[
\begin{split}
\mathbb{E}(K) &= \sum^{k}_{i=1} \frac{p}{p - i + 1} \\
&= p(H_p - H_{p - k}) \\
\end{split}
\]
Where $H_i$ is the $i^\text{th}$ harmonic.

This leads to the expression:
\[
\mathbb{E}(K) =
\begin{cases}
0 & \text{ if } k = 0 \vee k = 1\\
p(H_p - H_{p - k}) & \text{ if } 1 < k \leq p \\
\infty & \text{ otherwise } \\
\end{cases}
\]

\setcounter{enumi}{10}

\item Let $X$ be a random variable such that $\mu = \mathbb{E}[X] = \frac{1}{2}$ and $\mathbf{V}\left[ X \right] = 1$. 
What can you deduce about $\mathbb{E}[\ln(2X)]$?

We can deduce two things about $\mathbb{E}[\ln(2X)]$:

Firstly:

\[
\begin{split}
\mathbb{E}(\ln(2X)) &= \mathbb{E}(\ln(1 + (2X - 1))) \\
&= \mathbb{E}(\ln(1 + 2(X - \mathbb{E}(X)))) \\
&\approx \mathbb{E}\left(2(X - \mathbb{E}(X)) - \frac{4(X - \mathbb{E}(X))^2}{2}\right) \\
&\approx 2\mathbb{E}(X - \mathbb{E}(X)) - 2\mathbb{E}((X - \mathbb{E}(X))^2) \\
&\approx -2\mathbf{V}(X) \\
&\approx -1 \\
\end{split}
\]

Secondly:

Consider $Z = \ln 2X$. So $\mathbb{E}(X) = \frac{1}{2}\mathbb{E}(e^{Z})$. Since $e^Z$ is a strictly convex
function, we can apply Jensen's inequality:
\[
\begin{split}
\frac{1}{2}^{\mathbb{E}(Z)} &< \frac{1}{2}\mathbb{E}(e^Z) \Longrightarrow \\
\frac{1}{2}e^{\mathbb{E}(\ln 2X)} &< \mathbb{E}(X) \\
\mathbb{E}(\ln 2X) &< \ln 2\mathbb{E}(X) \\
\mathbb{E}(\ln 2X) &< \ln 1 \\
\mathbb{E}(\ln 2X) &< 0 \\
\end{split}
\]

\end{enumerate}

\section*{Lecture 12--13}

\begin{enumerate}

\item Let $X$ count the number of collisions among $k$ independent samples from a discrete uniform distribution over
$\{1, 2, \ldots, n\}$.

\begin{enumerate}

\item What is $\mathbb{E}[X]$?

I will define a collision to be when a sample is on a sample where at least one other sample is.
So the set $\{1, 1, 2, 3, 4\}$ has only 1 collision and the set $\{1, 1, 1\}$ has two collisions.

The expected number of distinct values in $i$ samples ($\mu_i$) can be formed by the recurrence relation:
\[
\begin{split}
\mu_{i + 1} &= \mu_i + P(\text{new value}) \\
&= \mu_i + \frac{n - \mu_i}{n} \\
&= \left( \frac{n - 1}{n} \right)\mu_i + 1 \\
\end{split}
\]
With the boundary conditions $\mu_0 = 0$, this has the solution:
\[
\mu_k = n\left( 1 - \left(\frac{n - 1}{n}\right)^k \right)
\]

So the expected number of collisions $\mathbb{E}(C)$ after $k$ samples is the number of samples
minus the number of distinct samples. This is given by:
\[
\mathbb{E}(C) = k - n\left( 1 - \left(\frac{n - 1}{n}\right)^k \right)
\]

\item Prove that $\mathbb{P}[X > 0] \approx 1 - \exp\left( - \begin{pmatrix} k \\ 2 \\ \end{pmatrix} \cdot \frac{1}{n} \right)$.

The exact probability of having 0 collisions in $k$ samples from a uniform distribution of
size $n$ is:
\[
\frac{n\mathbf{P}k}{n^k}
\]

So the exact probability of having more than 0 collisions in $k$ samples is given by:
\[
\begin{split}
& 1 - \frac{n\mathbf{P}k}{n^k} \\
=& 1 - \exp{\left(\ln n! - \ln (n - k)! - \ln n^k\right)} \\
\approx& 1 - \exp{\left(n\ln n - n - (n - k)\ln (n - k) + n - k - k\ln n\right)} \text{ since } \ln n! \approx n \ln
n - n\\
=& 1 - \exp{\left((n - k)\ln n - (n - k)\ln (n - k) - k\right)} \\
=& 1 - \exp{\left((n - k)\ln \frac{n}{n - k} - k\right)} \\
=& 1 - \exp{\left((n - k)\ln \left(1 + \frac{k}{n - k}\right) - k\right)} \\
\approx& 1 - \exp{\left((n - k)\left(\frac{k}{n - k} - \frac{k^2}{2(n - k)^2}\right) - k\right)} \text{ using the Taylor
 expansion } \ln (1 + x)\\
=& 1 - \exp{\left(k - \frac{k^2}{2(n - k)} - k\right)} \\
=& 1 - \exp{\left(-\frac{k^2}{2(n - k)}\right)} \\
\approx& 1 - \exp{\left(- \frac{k(k - 1)}{2n}\right)} \text{ for } 1 \ll k \ll n \\
=& 1 - \exp{\left(-\begin{pmatrix} k \\ 2 \\ \end{pmatrix} \cdot \frac{1}{n}\right)} \\
\end{split}
\]

\item Describe how this could be used to obtain an estimator for the population size?
Your estimator does not need to be unbiased.

We can perform the experiment many times with $k$ to estimate the probability $p$ that
the number of collisions is greater than zero.

Then rearrange the expression above to give:
\[
n \approx \begin{pmatrix} k \\ 2 \\ \end{pmatrix} \frac{1}{-\ln(1 - p)}
\]
Substitute our sample $p$ into this expression to give an expression for $n$.

\end{enumerate}

\setcounter{enumi}{2}

\item Prove that the set of random variables $\sigma_{i, j}$ are not pairwise independent.

\[
\begin{split}
\mathbb{E}(\sigma_{i, j})
&= \mathbb{E}\left(\sum^{n}_{0} p_n^2\right) \\
&= \mathbb{E}\left( \frac{1}{n} \right) \\
\end{split}
\]

\[
\begin{split}
\mathbb{E}(\sigma_{i, k} | \sigma_{i, j})
&= \mathbb{E}(P(X = X_i)) \\
&= \frac{2}{n} \\
\end{split}
\]
So the $\sigma_{i, j}$ are not independent. Informally, knowing that $\sigma_{i, j} = 1$ increases
our estimate of the probability of $X = X_i$ and so $\mathbb{E}(\sigma_{i, k}) \geq \frac{1}{n}$.

\item Prove formally that any testing algorithm must have a two-sided error.

A testing algorithm which always says any distribution it tests is uniformly
distributed would only have a one-sided error. So the statement is incorrect
unless we add some further criteria into our definition of a test:
a test must have a nonzero probability of predicting either result and must
terminate for all inputs.

Consider three discrete random variables $X, Y \sim Po(\lambda)$ and $Z = X - Y$.
So $Z$ can take any positive or negative value with nonzero probability.
By our definition of a test,
there exists some finite sequence of results for which the test will predict
that a variable is a uniform variable. Since the probability of $Z$ taking any
discrete value is nonzero and $Z$ is not affected by previous values, any
finite-length discrete sequence of values has a nonzero probability of occurring.
So $Z$ has a nonzero probability of being incorrectly identified as a uniform
variable in this test. An analogous argument holds for a random uniform variable
$X' \sim U(-\infty, \infty)$.

Since our test was arbitrary, every test which always terminates and has a
nonzero probability of predicting either result must have two-sided error.

\item What is the expected number of local maxima in the secretary problem for $n$ candidates?
Based on this result, suggest an algorithm that outperforms the primitive approach.

The expected number of local maxima $\mathbb{E}(M)$ is the sum of the probabilities that the current secretary is
better than all previous secretaries. This is given by:
\[
\begin{split}
\mathbb{E}(M) &= \sum^{n}_{i=1} \frac{1}{i} \\
&= H_{n} \\
&\approx \ln n \\
\end{split}
\]

Wait until you've seen $\ln n - 1$ local maxima and then select the next person who beats
all previous candidates -- else select the last person.

\setcounter{enumi}{6}

\item Assume $n = 4$ in the secretary problem, and for any $1 \leq k \leq 4$ consider the strategy that accepts the
first candidate that is better than the previous $k - 1$ candidates. For each possible value of $k$, compute the
probability of hiring the best candidate.

For $k = 1$, the probability that we choose the best candidate is the probability that the
first candidate is the best candidate. This is $\frac{1}{4}$.

For $k = 2$, the probability that we choose the best candidate is the probability that the
best candidate is the second candidate or the best candidate is the third and the second best
is either first or last or the best candidate is last and the second best was first.
This is given by $\frac{1}{4} + \frac{1}{4}\times\frac{1}{2} + \frac{1}{4} \times \frac{1}{4} = \frac{7}{16}$.

For $k = 3$, the probability that we choose the best candidate is the probability that the best candidate is third
or the best candidate is fourth and the second best is not third. This is given by $\frac{1}{4} +
\frac{1}{4}\times\frac{3}{4} = \frac{7}{16} $.

For $k = 4$, the probability that we choose the best candidate is the probability that the best candidate is the
fourth candidate. Which is $\frac{1}{4}$.

\item Consider the secretary problem and let $I_1, I_2, \dots, I_n$ be the $n$ random variables where $I_j = 1$ if
and only if the $j^{\text{th}}$ candidate is the best among the first $j$ candidates. Prove that these $n$ random
variables are independent.

Prove by induction the lemma that:
\[
\int^{y_{k+1}}_0 \dots \int^{y_3}_0 \int^{y_2}_0 y_1^{x_1 - 1} y_2^{x_2 - x_1 - 1} \dots y_k^{x_k - x_{k - 1} - 1} dy_1
dy_2 \dots dy_k = \frac{1}{\prod^k_{i=1}x_i}\left[ y_{k}^{x_k} \right]^{y_{k+1}}_0
\]

For $k = 1$:
\[
\begin{split}
 & \int^{y_2}_0 y_1^{x_1 - 1} dx \\
=& \frac{1}{x_1}\left[ y_1^{x_1} \right]^{y_1}_0 \\
=& \prod^1_{i=1}\frac{1}{x_i}\left[ y_{1}^{x_1} \right]^{y_{2}}_0
\end{split}
\]

So the lemma holds for $k = 1$.

Assume now that it holds for $k = \ell$.

So:
\[
\begin{split}
 & \int^{y_{\ell + 2}}_0\int^{y_{\ell+1}}_0 \dots \int^{y_3}_0 \int^{y_2}_0 y_1^{x_1 - 1} y_2^{x_2 - x_1 - 1} \dots
y_\ell^{x_\ell - x_{\ell - 1} - 1} y_{\ell + 1}^{x_{\ell + 1} - x_{\ell} - 1} dy_1 dy_2 \dots dy_\ell dy_{\ell + 1} \\
=& \int^{y_{\ell + 2}}_{0}x^{x_{\ell + 1} - x_{\ell} - 1} \prod^\ell_{i=1}\frac{1}{x_i}\left[ y_{\ell}^{x_\ell}
\right]^{y_{\ell+1}}_0 dy_{\ell + 1} \text{ by induction hypothesis } \\
=& \frac{1}{\prod^\ell_{i=1}x_i} \int^{y_{\ell + 2}}_0 y_{\ell + 1}^{x_{\ell + 1} - x_\ell - 1} \left( y_{\ell +
1}^{x_\ell} - 0\right) dy_{\ell + 1} \\
=& \prod^\ell_{i=1}\frac{1}{x_i} \int^{y_{\ell + 2}}_0 y_{\ell + 1}^{x_{\ell + 1} - 1} dy_{\ell + 1} \\
=& \prod^{\ell + 1}_{i=1}\frac{1}{x_i} \left[ y_{\ell + 1}^{x_{\ell + 1}} \right]^{y_{\ell + 2}}_0 \text{ as required
 } \\
\end{split}
\]

Since the lemma holds for $k = 1$ and if it holds for $k = \ell$ then it also holds for $k = \ell + 1$, by
induction it must hold for all $k \in \mathbb{N}_{\geq 1}$.

I will prove that $\forall x_i \in \mathbb{N}. P(I_{x_k} \dots I_{x_1}) = P(I_{x_k}) \times \dots \times P
(I_{x_1})$. Without loss of generality, consider $x_k > x_{k - 1} \dots x_1$.

$P(I_{x_k} \dots I_{x_1})$ is equal to the probability that $\forall i. \forall x_{i - 1} < j < x_i. X_j < x_i$. There
are $x_i - x_{i - 1} - 1$ such $X_j$.
This forms the infinite depth integral:
\[
\begin{split}
P(I_{x_k} \dots I_{x_1}) &= \int^1_0\int^{y_{k}}_0\dots \int^{y_2}_0 y_k^{x_k - x_{k - 1}} y_{k-1}^{x_{k - 1} - x_{k
- 2} - 1} \dots y_1^{x_1 - 1} dy_1 \dots dy_{k - 1} dy_{k} \\
&= \prod^{k}_{i=1}\frac{1}{x_i} \left[ y_{k}^{x_{k}} \right]^1_0 \text{ by the lemma proved above } \\
&= \prod^{k}_{i=1}\frac{1}{x_i} \\
&= \prod^{k}_{i=1} P(I_{x_i}) \\
\end{split}
\]
Since the $x_i$ were arbitrary, this holds for any set of $x_i$. So all $n$ random
variables are independent.

\end{enumerate}

\end{document}
