\newcommand{\svcourse}{CST Part IA: Software Engineering and Security}
\newcommand{\svnumber}{1}
\newcommand{\svvenue}{Microsoft Teams}
\newcommand{\svdate}{2022-05-11}
\newcommand{\svtime}{15:00}
\newcommand{\svuploadkey}{CBd13xmL7PC1zqhNIoLdTiYUBnxZhzRAtJxv/ytRdM1r7qIfwMsxeVwM/pPcIo8l}

\newcommand{\svrname}{Dr Sam Ainsworth}
\newcommand{\jkfside}{oneside}
\newcommand{\jkfhanded}{yes}

\newcommand{\studentname}{Harry Langford}
\newcommand{\studentemail}{hjel2@cam.ac.uk}


\documentclass[10pt,\jkfside,a4paper]{article}

% DO NOT add \usepackage commands here.  Place any custom commands
% into your SV work files.  Anything in the template directory is
% likely to be overwritten!

\usepackage{fancyhdr}

\usepackage{lastpage}       % ``n of m'' page numbering
\usepackage{lscape}         % Makes landscape easier

\usepackage{verbatim}       % Verbatim blocks
\usepackage{listings}       % Source code listings
\usepackage{graphicx}
\usepackage{float}
\usepackage{epsfig}         % Embed encapsulated postscript
\usepackage{array}          % Array environment
\usepackage{qrcode}         % QR codes
\usepackage{enumitem}       % Required by Tom Johnson's exam question header

\usepackage{hhline}         % Horizontal lines in tables
\usepackage{siunitx}        % Correct spacing of units
\usepackage{amsmath}        % American Mathematical Society
\usepackage{amssymb}        % Maths symbols
\usepackage{amsthm}         % Theorems

\usepackage{ifthen}         % Conditional processing in tex

\usepackage[top=3cm,
            bottom=3cm,
            inner=2cm,
            outer=5cm]{geometry}

% PDF metadata + URL formatting
\usepackage[
            pdfauthor={\studentname},
            pdftitle={\svcourse, SV \svnumber},
            pdfsubject={},
            pdfkeywords={9d2547b00aba40b58fa0378774f72ee6},
            pdfproducer={},
            pdfcreator={},
            hidelinks]{hyperref}

\renewcommand{\headrulewidth}{0.4pt}
\renewcommand{\footrulewidth}{0.4pt}
\fancyheadoffset[LO,LE,RO,RE]{0pt}
\fancyfootoffset[LO,LE,RO,RE]{0pt}
\pagestyle{fancy}
\fancyhead{}
\fancyhead[LO,RE]{{\bfseries \studentname}\\\studentemail}
\fancyhead[RO,LE]{{\bfseries \svcourse, SV~\svnumber}\\\svdate\ \svtime, \svvenue}
\fancyfoot{}
\fancyfoot[LO,RE]{For: \svrname}
\fancyfoot[RO,LE]{\today\hspace{1cm}\thepage\ / \pageref{LastPage}}
\fancyfoot[C]{\qrcode[height=0.8cm]{\svuploadkey}}
\setlength{\headheight}{22.55pt}


\ifthenelse{\equal{\jkfside}{oneside}}{

 \ifthenelse{\equal{\jkfhanded}{left}}{
  % 1. Left-handed marker, one-sided printing or e-marking, use oneside and...
  \evensidemargin=\oddsidemargin
  \oddsidemargin=73pt
  \setlength{\marginparwidth}{111pt}
  \setlength{\marginparsep}{-\marginparsep}
  \addtolength{\marginparsep}{-\textwidth}
  \addtolength{\marginparsep}{-\marginparwidth}
 }{
  % 2. Right-handed marker, one-sided printing or e-marking, use oneside.
  \setlength{\marginparwidth}{111pt}
 }

}{
 % 3. Alternating margins, two-sided printing, use twoside.
}


\setlength{\parindent}{0em}
\addtolength{\parskip}{1ex}

% Exam question headings, labels and sensible layout (courtesy of Tom Johnson)
\setlist{parsep=\parskip, listparindent=\parindent}
\newcommand{\examhead}[3]{\section{#1 Paper #2 Question #3}}
\newenvironment{examquestion}[3]{
\examhead{#1}{#2}{#3}\setlist[enumerate, 1]{label=(\alph*)}\setlist[enumerate, 2]{label=(\roman*)}
\marginpar{\href{https://www.cl.cam.ac.uk/teaching/exams/pastpapers/y#1p#2q#3.pdf}{\qrcode{https://www.cl.cam.ac.uk/teaching/exams/pastpapers/y#1p#2q#3.pdf}}}
\marginpar{\footnotesize \href{https://www.cl.cam.ac.uk/teaching/exams/pastpapers/y#1p#2q#3.pdf}{https://www.cl.cam.ac.uk/\\teaching/exams/pastpapers/\\y#1p#2q#3.pdf}}
}{}


\usepackage{physics}
\usepackage{amsfonts}

\begin{document}

\begin{enumerate}

\item Suppose that $Y_1,Y_2,\cdots,Y_n$ are independent samples from a fixed Poisson distribution with
an unknown mean $\lambda > 0$.
Suppose that our prior distribution for $\lambda$ is $\Gamma(\alpha, \beta)$
where $\alpha > 0$ and $\beta > 0$.
Compute the probability density function of the posterior distribution.

In the following equations;
let $S_n = \sum^n_{i=0} Y_i$

\[
\begin{split}
\mathbb{P}(\lambda|Y_1,\dots,Y_n) &= \frac{\mathbb{P}(Y_1,\dots,Y_n|\lambda)\mathbb{P}(\lambda)}{\mathbb{P}(Y_1,
\dots,Y_n)} \\
&= \frac{\frac{\lambda^{S_n}e^{-n\lambda}}{\prod^n_{i=1}Y_i!}\times \frac{\beta^\alpha \lambda^{\alpha -
1} e^{-\beta \lambda}}{\Gamma(\alpha)}}{\int^\infty_0
\frac{x^{S_n}e^{-nx}}{\prod^n_{i=1}Y_i!}\times \frac{\beta^\alpha x^{\alpha -
1} e^{-\beta x}}{\Gamma(\alpha)}\dd{x}} \\
&= \frac{\frac{\beta^\alpha}{\Gamma(\alpha)\prod^n_{i=1}Y_i!}\lambda^{S_n}e^{-n\lambda}
\lambda^{\alpha - 1} e^{-\beta \lambda}}{\frac{\beta^\alpha}{\Gamma(\alpha)\prod^n_{i=1}Y_i!}\int^\infty_0
x^{S_n}e^{-nx}x^{\alpha -
1} e^{-\beta x}\dd{x}} \\
&= \frac{\lambda^{\alpha + S_n - 1} e^{-(\beta + n) \lambda}}{\int^\infty_0
x^{\alpha + S_n - 1} e^{-(\beta + n) x}\dd{x}} \\
&= \frac{\lambda^{\alpha + S_n - 1} e^{-(\beta + n) \lambda}}{\frac{1}{\beta + n}\int^\infty_0
\left(\frac{y}{\beta + n}\right)^{\alpha + S_n - 1} e^{-y}\dd{y}} \ \dag \\
&= \frac{\lambda^{\alpha + S_n - 1} e^{-(\beta + n) \lambda}}{\frac{1}{(\beta + n)^{\alpha + S_n}}\int^\infty_0 y^{\alpha + S_n - 1} e^{-y}\dd{y}} \\
&= \frac{(\beta + n)^{\alpha + S_n}\lambda^{\alpha + S_n - 1} e^{-(\beta + n) \lambda}}{\int^\infty_0 x^{\alpha +
S_n - 1} e^{-x}\dd{x}} \\
&= \frac{(\beta + n)^{\alpha + S_n}\lambda^{\alpha + S_n - 1} e^{-(\beta + n) \lambda}}{\Gamma(\alpha + S_n)} \\
&= \Gamma(\alpha + S_n, \beta + n)
\end{split}
\]

\dag:
By substituting $y = (\beta + n)x$. $\frac{1}{\beta + n}\dd{y} = \dd{x}$

So the posterior probability density function is $\Gamma(\alpha + S_n, \beta + n)$.

\item
Supose Wilfred has $k > 1$ light bulbs and the probability of any individual
bulb not working is $p$.
Two strategies for testing the $k$ bulbs are:

\begin{enumerate}[label=(\Alph*)]

\item Test each bulb separately.
This takes $k$ tests.

\item Wire up all $k$ bulbs as a series circuit.
If all the bulbs come on, the testing is complete in just one test,
otherwise revert to strategy $A$ taking a total of $k + 1$ tests.

\end{enumerate}

Let $X$ be a random variable whose value $r$ is the number of tests required
using strategy $B$.
The probability $\mathbb{P}(X = r)$ may be expressed as:

\[
\mathbb{P}(X = r) =
\begin{cases}
(1 - p)^k & \text{ if } r = 1 \\
1 - (1 - p)^k & \text{ if } r = k + 1 \\
0 & \text{ otherwise } \\
\end{cases}
\]

\begin{enumerate}

\item Explain this function and justify why the constraint $k > 1$ is required
mathematically.

$X = 1$ if and only if all $k$ bulbs work.
So $\mathbb{P}(X = 1) = (1 - p)^k$ as required.

If any bulbs don't work, then under method $B$ we do the first test and then do a further $k$ tests
leading to $k + 1$ total tests.
The probability that any bulbs don't work is $1 - \mathbb{P}(\text{all bulbs work}) = 1 - (1 - p)^k$.

No other outcomes are possible. 
So all other outcomes have probability zero. 
This leads to the function as described.

Since we are considering real objects, the constraint $k \in \mathbb{N}$ is obvious. 
To restrict $k > 1$ we have to consider two cases: $k = 0$ and $k = 1$.

\begin{itemize}

\item [$k = 0$]
Since $k + 1 = 1$, the function claims that $\mathbb{P}(X = 1) = 0 \wedge 1$
which is a contradiction.

\item [$k = 1$]
Contraversially, I don't think the constraint $k > 1$ is required
\textit{mathematically}.
It is required for method $B$ to make any sense.
If $k = 1$ then we test 1 bulb and if it's broken test it again --
which wastes a test.

\end{itemize}

\item
Determine the expectation $\mathbb{E}(X)$.

\[
\begin{split}
\mathbb{E}(X)
&= (1 - p)^k + (k + 1)(1 - (1 - p)^k) \\
&= k + 1 + (1 - p)^k - (k + 1)(1 - p) \\
&= k + 1 - k(1 - p)^k \\
\end{split}
\]

\item
Strategy $B$ beats strategy $A$ if $\mathbb{E}(X) < k$ and this condition
is satisfied if $p < f(k)$ where $f(k)$ is some function of $k$.
Derive the function $f(k)$.

\[
\begin{split}
k + 1 - k(1 - p)^k &< k \\
1 - k(1 - p)^k &< 0 \\
1 &< k(1 - p)^k \\
\frac{1}{k} &< (1 - p)^k \\
\sqrt[k]{\frac{1}{k}} &< 1 - p \\
p &< 1 - \sqrt[k]{\frac{1}{k}} \\
\end{split}
\]

\item
Suppose you have $n$ light bulbs, where $n \gg k$ and $k|n. \exists m. n = m \cdot k$
and you partition the $n$ bulbs into $m$ groups of $k$.
Assuming that the groups are independent and again assuming that $k < 1$, show
that the expected number of tests is:
\[
n\left(1 + \frac{1}{k} - (1 - p)^k\right)
\]

\[
\begin{split}
\mathbb{E}(X) &= m(k + 1 - k(1 - p)^k) \\
              &= mk\left(1 + \frac{1}{k} - (1 - p)^k\right) \\
              &= n\left(1 + \frac{1}{k} - (1 - p)^k\right) \\
\end{split}
\]

In the question sheet, neither of the brackets were square.
I will assume the question was asking
about $(1 + \frac{1}{k} - (1 - p)^k)$ rather than $(1 - p)$.

If $p$ is small then $(1 + \frac{1}{k} - (1 - p)^k)$ is close to $\frac{1}{k}$.

If $p$ is large then $(1 + \frac{1}{k} - (1 - p)^k)$ is close to $1 + \frac{1}{k}$.

We can exploit this by using larger $k$ when $p$ is small.

The optimal value of $k$ to choose for a given $p$ is the solution to the equation
$2 \ln k + k\ln(1 - p) + \ln\left(\ln \frac{1}{1 - p}\right) = 0$ which can be solved
numerically.

\end{enumerate}

\item 

You and a friend are presented with a box containing nine
indistinguishable chocolates, three of which are contaminated with a deadly poison.
Each of you must eat a single chocolate.

\begin{enumerate}

\item
If you choose before your friend, what is the probability that you will survive?

\[
\frac{2}{3}
\]

\item 
If you choose first and survive, what is the probability that your friend survives?

\[
\frac{5}{8}
\]

\item 
If you choose first and die, what is the probability that your friend survives?

\[
\frac{3}{4}
\]

\item Is it in your best interests to persuade your ``friend'' to choose first?

No.

\begin{align*}
\mathbb{P}(\text{survive}|\text{first}) &= \frac{2}{3} & \mathbb{P}(\text{survive}|\text{second}) &=
\frac{1}{3}\times \frac{3}{4} + \frac{2}{3} \times \frac{5}{8} \\
& & &= \frac{3}{12} + \frac{5}{12} \\
& & &= \frac{8}{12} \\
& & &= \frac{2}{3}
\end{align*}

\item
If you choose first, what is the probability that you survive given that your
friend survives?

\[
\begin{split}
\mathbb{P}(\text{survive}|\text{friend survives}) &= \frac{\frac{2}{3} \times \frac{5}{8}}{\frac{2}{3} \times
\frac{5}{8} +
\frac{1}{3} \times \frac{6}{8}} \\
                                  &= \frac{2 \times 5}{2 \times 5 + 1 \times 6} \\
                                  &= \frac{10}{16} \\
                                  &= \frac{5}{8} \\
\end{split}
\]

\end{enumerate}

\item
There are $n$ socks in a drawer, three of which are red and the rest black.
Wilfred chooses his socks by selecting two at random from the drawer, and puts
them on.
He is three times more likely to wear socks of different colours than to wear
matching red socks.

\begin{enumerate}

\item Find $n$.

For legibility, I will use $^n C_r$ to mean
$\begin{pmatrix} n \\ r
\\
\end{pmatrix}$.

\[
\begin{split}
\frac{\mathbb{P}(2\text{ colours})}{\mathbb{P}(2\text{ red})} &= \frac{\left(\frac{^3 C_1 \ ^{n - 3} C_1}{^n
C_2}\right)}{\left(\frac{^3
C_2 \ ^{n - 3} C_0}{^n C_2}\right)} \\
3 &= \frac{3 \times (n - 3)}{3} \\
3 &= n - 3 \\
n &= 6 \\
\end{split}
\]

\item For this value of $n$, what is the probability that Wilfred wears matching
black socks?

\[
\begin{split}
\mathbb{P}(2\text{ black}) &= \frac{\begin{pmatrix} 3 \\ 0 \\ \end{pmatrix} \begin{pmatrix} 3 \\ 2 \\ \end{pmatrix}}{\begin{pmatrix} 6 \\ 2 \\ \end{pmatrix}} \\
                  &= \frac{3}{15} \\
                  &= \frac{1}{5} \\
\end{split}
\]

\end{enumerate}

\item
Let $X_1, X_2, \dots$ be a sequence of observations.
Define the sample mean $\overline{X}_n$, of the first $n$ observations by
$\overline{X}_n = \sum^n_{i=1} \frac{X_i}{n}$.
Show that for $n > 1$
\[
\overline{X}_n = \overline{X}_{n-1} + \frac{X_n - \overline{X}_{n-1}}{n} \\
\]

\[
\begin{split}
\overline{X}_n &= \frac{\sum^n_{i=1} X_i}{n} \\
               &= \frac{(n - 1)\overline{X}_{n-1} + X_n}{n} \\
               &= \frac{n\overline{X}_{n-1} + X_n - \overline{X}_{n-1}}{n} \\
               &= \overline{X}_{n-1} + \frac{X_n - \overline{X}_{n-1}}{n} \\
\end{split}
\]

\item A canon fixed at the origin in the $x-y$ plane fires discrete particles.
The $i^\text{th}]$ particle moves within the plane along a straight line which makes a random
angle $\Theta = \theta_i$ with the $x-axis$ and collides with a screen at $(x_0, y_i)$ where
the parameter $x_0 > 0$ defines the position of the screen on the $x$-axis.
The values of $\theta_i$ are uniformly distributed in the range $-\frac{\pi}{2} \leq \theta_i \leq \frac{\pi}{2}$.

\begin{enumerate}

\item
Find $y_i$ in terms of $\theta_i$ and $x_0$.

\[
    y_i = x_0\tan \theta_i
\]

\item
Find the probability density function, $g(\theta)$ and cumulative density function
$G(\theta) = \mathbb{P}(\Theta \leq \theta)$.

\begin{gather*}
    g(\theta) = \frac{1}{\pi} \\\\
    G(\theta) = \frac{\theta + \frac{\pi}{2}}{\pi}\\
\end{gather*}

\item
Letting $Y$ be a continuous random variable denoting the $y$-coordinate of the
collision points.
Find the cumulative distribution function $F(y) = \mathbb{P}(Y \leq y)$.

\[
F(y) = \frac{\arctan{\left(\frac{y}{x_0}\right)} + \frac{\pi}{2}}{\pi} \\
\]

\item
Hence show that the probability distribution function $f(y)$ is given by:
\[
f(y) = \frac{x_0}{\pi(x_0^2 + y^2)}
\]

\[
\begin{split}
f(y) &= \dv{F}{y} \\
     &= \dv{x}\left(\frac{1}{\pi}\arctan\left(\frac{y}{x_0}\right) + \frac{1}{2}\right) \\
     &= \frac{x_0}{\pi(x_0^2 + y^2)} \\
\end{split}
\]

\item
Show that the standard deviation of the $y$-coordinate of the collision position
has no mathematically defined value.

\[
\begin{split}
Var(Y) &= \mathbb{E}(Y^2) - \mathbb{E}(Y)^2 \\
       &= \int^{\infty}_{-\infty} \frac{x_0 y^2}{\pi(x_0^2 + y^2)} \dd{x} - \left(\int^{\infty}_{-\infty}
\frac{x_0 y}{\pi(x_0^2 + y^2)}\dd{y}\right) \\
      &= \left[x_0 y - x_0^2 \arctan\left(\frac{y}{x_0}\right) \right]^{\infty}_{-\infty} -
\left[\frac{1}{2}x_0\ln\left(y^2 + x_0^2\right)\right]^{\infty}_{-\infty} \\
&= (\infty - x_0^2\pi + \infty) - (\infty - \infty) \\
&= \infty - \infty \\
& \text{Which is undefined}
\end{split}
\]

\item
$\theta$ is now uniformly distributed between $-\frac{\pi}{6}$ and $+\frac{\pi}{3}$.
Find $\mathbb{E}(Y)$.

\[
G(y) = \frac{2}{\pi}\arctan\left(\frac{y}{x_0}\right) + \frac{1}{3}
\]

\[
\begin{split}
g(y) &= \dv{y}\left(\frac{2}{\pi}\arctan\left(\frac{y}{x_0}\right) + \frac{1}{3}\right) \\
     &= \frac{2x_0}{\pi(x_0^2 + y^2)} \\
\end{split}
\]

\[
\begin{split}
\mathbb{E}(y) &= \int^{\sqrt{3}x_0}_{-\frac{x_0}{\sqrt{3}}} \frac{2x_0 y}{\pi(x_0^2 + y^2)} \dd{y} \\
              &= \left[\frac{x_0}{\pi} \ln\left(x_0^2 + y^2\right)\right]^{\sqrt{3}x_0}_{-\frac{x_0}{\sqrt{3}}} \\
              &= \frac{x_0}{\pi} \left(\ln\left(4x_0^2\right) - x_0 \ln\left(\frac{4}{3}x_0^2\right)\right) \\
              &= \frac{x_0}{\pi} \ln(3) \\
\end{split}
\]

\item
The distance to the screen now halves instantaneously every time a particle is fired.
Find $\mathbb{E}\left(\sum^\infty_{i=1} Y_i\right)$

\[
\begin{split}
\mathbb{E}\left(\sum^\infty_{i=1} Y_i\right) &= \sum^{\infty}_{i=1} \mathbb{E}(Y_i) \\
											 &= \frac{x_0\ln 3}{\pi} \sum^{\infty}_{i=1} \frac{1}{2}^i \\
                                             &= \frac{x_0\ln 3}{\pi} \\
\end{split}
\]

\end{enumerate}

\end{enumerate}

\end{document}