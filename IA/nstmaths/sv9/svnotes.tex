\newcommand{\svrname}{Dr John Fawcett}
\newcommand{\jkfside}{oneside}
\newcommand{\jkfhanded}{right}

\newcommand{\studentname}{Harry Langford}
\newcommand{\studentemail}{hjel2@cam.ac.uk}

\documentclass[10pt,\jkfside,a4paper]{article}

% DO NOT add \usepackage commands here.  Place any custom commands
% into your SV work files.  Anything in the template directory is
% likely to be overwritten!

\usepackage{fancyhdr}

\usepackage{lastpage}       % ``n of m'' page numbering
\usepackage{lscape}         % Makes landscape easier

\usepackage{verbatim}       % Verbatim blocks
\usepackage{listings}       % Source code listings
\usepackage{epsfig}         % Embed encapsulated postscript
\usepackage{array}          % Array environment
\usepackage{qrcode}         % QR codes
\usepackage{enumitem}       % Required by Tom Johnson's exam question header

\usepackage{hhline}         % Horizontal lines in tables
\usepackage{siunitx}        % Correct spacing of units
\usepackage{amsmath}        % American Mathematical Society
\usepackage{amssymb}        % Maths symbols
\usepackage{amsthm}         % Theorems

\usepackage{ifthen}         % Conditional processing in tex

\usepackage[top=3cm,
            bottom=3cm,
            inner=2cm,
            outer=5cm]{geometry}

% PDF metadata + URL formatting
\usepackage[
            pdfauthor={\studentname},
            pdftitle={\svcourse, SV \svnumber},
            pdfsubject={},
            pdfkeywords={9d2547b00aba40b58fa0378774f72ee6},
            pdfproducer={},
            pdfcreator={},
            hidelinks]{hyperref}


% DO NOT add \usepackage commands here.  Place any custom commands
% into your SV work files.  Anything in the template directory is
% likely to be overwritten!

\usepackage{fancyhdr}

\usepackage{lastpage}       % ``n of m'' page numbering
\usepackage{lscape}         % Makes landscape easier

\usepackage{verbatim}       % Verbatim blocks
\usepackage{listings}       % Source code listings
\usepackage{graphicx}
\usepackage{float}
\usepackage{epsfig}         % Embed encapsulated postscript
\usepackage{array}          % Array environment
\usepackage{qrcode}         % QR codes
\usepackage{enumitem}       % Required by Tom Johnson's exam question header

\usepackage{hhline}         % Horizontal lines in tables
\usepackage{siunitx}        % Correct spacing of units
\usepackage{amsmath}        % American Mathematical Society
\usepackage{amssymb}        % Maths symbols
\usepackage{amsthm}         % Theorems

\usepackage{ifthen}         % Conditional processing in tex

\usepackage[top=3cm,
            bottom=3cm,
            inner=2cm,
            outer=5cm]{geometry}

% PDF metadata + URL formatting
\usepackage[
            pdfauthor={\studentname},
            pdftitle={\svcourse, SV \svnumber},
            pdfsubject={},
            pdfkeywords={9d2547b00aba40b58fa0378774f72ee6},
            pdfproducer={},
            pdfcreator={},
            hidelinks]{hyperref}

\renewcommand{\headrulewidth}{0.4pt}
\renewcommand{\footrulewidth}{0.4pt}
\fancyheadoffset[LO,LE,RO,RE]{0pt}
\fancyfootoffset[LO,LE,RO,RE]{0pt}
\pagestyle{fancy}
\fancyhead{}
\fancyhead[LO,RE]{{\bfseries \studentname}\\\studentemail}
\fancyhead[RO,LE]{{\bfseries \svcourse, SV~\svnumber}\\\svdate\ \svtime, \svvenue}
\fancyfoot{}
\fancyfoot[LO,RE]{For: \svrname}
\fancyfoot[RO,LE]{\today\hspace{1cm}\thepage\ / \pageref{LastPage}}
\fancyfoot[C]{\qrcode[height=0.8cm]{\svuploadkey}}
\setlength{\headheight}{22.55pt}


\ifthenelse{\equal{\jkfside}{oneside}}{

 \ifthenelse{\equal{\jkfhanded}{left}}{
  % 1. Left-handed marker, one-sided printing or e-marking, use oneside and...
  \evensidemargin=\oddsidemargin
  \oddsidemargin=73pt
  \setlength{\marginparwidth}{111pt}
  \setlength{\marginparsep}{-\marginparsep}
  \addtolength{\marginparsep}{-\textwidth}
  \addtolength{\marginparsep}{-\marginparwidth}
 }{
  % 2. Right-handed marker, one-sided printing or e-marking, use oneside.
  \setlength{\marginparwidth}{111pt}
 }

}{
 % 3. Alternating margins, two-sided printing, use twoside.
}


\setlength{\parindent}{0em}
\addtolength{\parskip}{1ex}

% Exam question headings, labels and sensible layout (courtesy of Tom Johnson)
\setlist{parsep=\parskip, listparindent=\parindent}
\newcommand{\examhead}[3]{\section{#1 Paper #2 Question #3}}
\newenvironment{examquestion}[3]{
\examhead{#1}{#2}{#3}\setlist[enumerate, 1]{label=(\alph*)}\setlist[enumerate, 2]{label=(\roman*)}
\marginpar{\href{https://www.cl.cam.ac.uk/teaching/exams/pastpapers/y#1p#2q#3.pdf}{\qrcode{https://www.cl.cam.ac.uk/teaching/exams/pastpapers/y#1p#2q#3.pdf}}}
\marginpar{\footnotesize \href{https://www.cl.cam.ac.uk/teaching/exams/pastpapers/y#1p#2q#3.pdf}{https://www.cl.cam.ac.uk/\\teaching/exams/pastpapers/\\y#1p#2q#3.pdf}}
}{}


\usepackage{physics}

\begin{document}

Using operators to solve second order ODE's:
consider
\begin{equation}
\dv[2]{y}{x} - 5\dv{y}{x} + 6y = e^{2x}\\
\end{equation}

You can convert this into an operator.

This means this is of the form:
\begin{equation}
\left(\dv[2]{x} - 5\dv{x} + 6\right)y = e^{2x}\\
\end{equation}
Factorise this:
\begin{equation}
\left(\dv{x} - 3\right)\left(\dv{x} - 2\right)y = e^{2x}\\
\end{equation}
If you expand this, you do not operate the operator on the constants -- you multiply it.
So when you expand this for example you do not ie you do not write $\dv{x} \times x = 2\dv{x} \neq 0$.

You now set part of this expression to be equal to $f(x)$ -- say $z$. Now you have reduced the order of the 
differential equation. You solve this as a first order differential equation. Then you get an expression for $z$ 
which you can solve for $z$. Then you solve the remaining system as another linear differential equation.

Let $z = \left(\dv{x} - 2\right)y$.
\begin{equation}
\begin{split}
\left(\dv{x} - 3\right)z &= e^{2x}\\
\dv{z}{x} - 3z &= e^{2x} \\
e^{-3x}\dv{z}{x} - 3ze^{-3x} &= e^{-x} \\
ze^{-3x} &= -e^{-x} + c\\
z &= ce^{3x} - e^{2x} \\
\left(\dv{x} - 2\right)y &= ce^{3x} - e^{2x} \\
\dv{y}{x} - 2y &= ce^{3x} - e^{2x} \\
e^{-2x}\dv{y}{x} - 2ye^{-2x} &= ce^{x} - 1 \\
ye^{-2x} &= ce^{x} - x + d \\
y &= ce^{3x} + de^{2x} - xe^{2x} \\
\end{split}
\end{equation}

This method allows you to solve equations without ever having to make a particular integral -- and allows you to 
solve potentially far more difficult operators. For example one of the differentila operators can 
be a function of $x$ and be near-impossible to be solvable via normal methods.

The definition of $\nabla$ is:
\begin{equation}
\underline{\nabla} =: \underline{i}\pdv{x} + \underline{j}\pdv{y} + \underline{k}\pdv{y} \\
\end{equation}
If this acts on a scalar function $f$, then it is the gradient -- this is why it is called grad.
This represents the normal to any given surface. $\nabla f$ is the normal to any part of that surface.

%Because $\nabla$ is a vector differential operator, it can also operate on other vectors too -- say $\underline{v} = (v_1, v_2, v_3)$.
%This is denoted $\underline{\nabla} \dot \underline{v}$ this is called solenoidial if it is equal to zero.

If $\underline{\nabla}\times \underline{v}$ is equal to zero then it is conserative.
\begin{equation}
\underline{\nabla}\cdot \underline{v} = \pdv{v_1}{x} + \pdv{v_2}{y} + \pdv{v_3}{z} \\
\end{equation}
This is a scalar.

the cross product is equal to the cross product of the nabla operator as a vector and the vector v. 
This gives a vector quantity. 
\begin{equation}
\underline{\nabla}\times\underline{v} = \left(\pdv{v_3}{y} - \pdv{v_2}{z}\right)\underline{i} + \left(\pdv{v_1}{z} - \pdv{v_3}{x}\right)\underline{j} + \left(\pdv{v_1}{y} - \pdv{v_2}{x}\right)\underline{k} \\
\end{equation}

\begin{equation}
\nabla^2 f = \nabla(\nabla f) = \pdv[2]{f}{x} + \pdv[2]{f}{y} + \pdv[2]{f}{z} \\
\end{equation}
This is known as the laplacian and is used in laplaces equation.

\end{document}