\newcommand{\svrname}{Dr John Fawcett}
\newcommand{\jkfside}{oneside}
\newcommand{\jkfhanded}{right}

\newcommand{\studentname}{Harry Langford}
\newcommand{\studentemail}{hjel2@cam.ac.uk}

\documentclass[10pt,\jkfside,a4paper]{article}

% DO NOT add \usepackage commands here.  Place any custom commands
% into your SV work files.  Anything in the template directory is
% likely to be overwritten!

\usepackage{fancyhdr}

\usepackage{lastpage}       % ``n of m'' page numbering
\usepackage{lscape}         % Makes landscape easier

\usepackage{verbatim}       % Verbatim blocks
\usepackage{listings}       % Source code listings
\usepackage{epsfig}         % Embed encapsulated postscript
\usepackage{array}          % Array environment
\usepackage{qrcode}         % QR codes
\usepackage{enumitem}       % Required by Tom Johnson's exam question header

\usepackage{hhline}         % Horizontal lines in tables
\usepackage{siunitx}        % Correct spacing of units
\usepackage{amsmath}        % American Mathematical Society
\usepackage{amssymb}        % Maths symbols
\usepackage{amsthm}         % Theorems

\usepackage{ifthen}         % Conditional processing in tex

\usepackage[top=3cm,
            bottom=3cm,
            inner=2cm,
            outer=5cm]{geometry}

% PDF metadata + URL formatting
\usepackage[
            pdfauthor={\studentname},
            pdftitle={\svcourse, SV \svnumber},
            pdfsubject={},
            pdfkeywords={9d2547b00aba40b58fa0378774f72ee6},
            pdfproducer={},
            pdfcreator={},
            hidelinks]{hyperref}


% DO NOT add \usepackage commands here.  Place any custom commands
% into your SV work files.  Anything in the template directory is
% likely to be overwritten!

\usepackage{fancyhdr}

\usepackage{lastpage}       % ``n of m'' page numbering
\usepackage{lscape}         % Makes landscape easier

\usepackage{verbatim}       % Verbatim blocks
\usepackage{listings}       % Source code listings
\usepackage{graphicx}
\usepackage{float}
\usepackage{epsfig}         % Embed encapsulated postscript
\usepackage{array}          % Array environment
\usepackage{qrcode}         % QR codes
\usepackage{enumitem}       % Required by Tom Johnson's exam question header

\usepackage{hhline}         % Horizontal lines in tables
\usepackage{siunitx}        % Correct spacing of units
\usepackage{amsmath}        % American Mathematical Society
\usepackage{amssymb}        % Maths symbols
\usepackage{amsthm}         % Theorems

\usepackage{ifthen}         % Conditional processing in tex

\usepackage[top=3cm,
            bottom=3cm,
            inner=2cm,
            outer=5cm]{geometry}

% PDF metadata + URL formatting
\usepackage[
            pdfauthor={\studentname},
            pdftitle={\svcourse, SV \svnumber},
            pdfsubject={},
            pdfkeywords={9d2547b00aba40b58fa0378774f72ee6},
            pdfproducer={},
            pdfcreator={},
            hidelinks]{hyperref}

\renewcommand{\headrulewidth}{0.4pt}
\renewcommand{\footrulewidth}{0.4pt}
\fancyheadoffset[LO,LE,RO,RE]{0pt}
\fancyfootoffset[LO,LE,RO,RE]{0pt}
\pagestyle{fancy}
\fancyhead{}
\fancyhead[LO,RE]{{\bfseries \studentname}\\\studentemail}
\fancyhead[RO,LE]{{\bfseries \svcourse, SV~\svnumber}\\\svdate\ \svtime, \svvenue}
\fancyfoot{}
\fancyfoot[LO,RE]{For: \svrname}
\fancyfoot[RO,LE]{\today\hspace{1cm}\thepage\ / \pageref{LastPage}}
\fancyfoot[C]{\qrcode[height=0.8cm]{\svuploadkey}}
\setlength{\headheight}{22.55pt}


\ifthenelse{\equal{\jkfside}{oneside}}{

 \ifthenelse{\equal{\jkfhanded}{left}}{
  % 1. Left-handed marker, one-sided printing or e-marking, use oneside and...
  \evensidemargin=\oddsidemargin
  \oddsidemargin=73pt
  \setlength{\marginparwidth}{111pt}
  \setlength{\marginparsep}{-\marginparsep}
  \addtolength{\marginparsep}{-\textwidth}
  \addtolength{\marginparsep}{-\marginparwidth}
 }{
  % 2. Right-handed marker, one-sided printing or e-marking, use oneside.
  \setlength{\marginparwidth}{111pt}
 }

}{
 % 3. Alternating margins, two-sided printing, use twoside.
}


\setlength{\parindent}{0em}
\addtolength{\parskip}{1ex}

% Exam question headings, labels and sensible layout (courtesy of Tom Johnson)
\setlist{parsep=\parskip, listparindent=\parindent}
\newcommand{\examhead}[3]{\section{#1 Paper #2 Question #3}}
\newenvironment{examquestion}[3]{
\examhead{#1}{#2}{#3}\setlist[enumerate, 1]{label=(\alph*)}\setlist[enumerate, 2]{label=(\roman*)}
\marginpar{\href{https://www.cl.cam.ac.uk/teaching/exams/pastpapers/y#1p#2q#3.pdf}{\qrcode{https://www.cl.cam.ac.uk/teaching/exams/pastpapers/y#1p#2q#3.pdf}}}
\marginpar{\footnotesize \href{https://www.cl.cam.ac.uk/teaching/exams/pastpapers/y#1p#2q#3.pdf}{https://www.cl.cam.ac.uk/\\teaching/exams/pastpapers/\\y#1p#2q#3.pdf}}
}{}


\fancyhead[RO,LE]{{\bfseries NST Maths, SV~9}\\12-02-2022\ 12:00, Video Link}

\usepackage{physics}

\begin{document}

\section*{Ordinary Differential Equations}

\begin{enumerate}
\setcounter{enumi}{8}

\item
\begin{enumerate}

\item
\begin{equation}
\begin{split}
\dv[2]{y}{x} - 5\dv{y}{x} + 6y &= 0\\
\lambda^2 - 5\lambda + 6 &= 0\\
(\lambda - 2)(\lambda - 3) &= 0\\
\lambda = 2 \vee \lambda &= 3\\
y_c &= Ae^{2x} + Be^{3x}\\
y_p &= 0 \\
y &= y_c + y_p \\
y &= Ae^{2x} + Be^{3x}\\
\end{split}
\end{equation}
\begin{equation}
\begin{split}
y(0) &= 0\\
0 &= Ae^{2\times 0} + Be^{2 \times 0}\\
0 &= A + B\\
\end{split}
\end{equation}
\begin{equation}
\begin{split}
y'(0) &= 1\\
\dv{y}{x} &= 2Ae^{2x} + 3Be^{3x} \\
\dv{y}{x}\left(0\right) &= 2Ae^{2 \times 0} + 3Be^{3 \times 0} \\
1 &= 2A + 3B \\
1 &= 2(A + B) + B \\
1 &= B \\
0 &= A + 1 \\
A &= -1 \\
y &= -e^{2x} + e^{3x} \\
\end{split}
\end{equation}

\item
\begin{equation}
\begin{split}
\left(\dv[n]{x} + n^2\right)y &= 0\\
\lambda^2 + n^2 &= 0\\
\lambda &= \pm ni\\
y_c &= P\sin nx + Q\cos nx\\
y_p &= 0 \\
y &= y_c + y_ p \\
y &= P\sin nx + Q\cos nx\\
\end{split}
\end{equation}
\begin{equation}
\begin{split}
y(0) &= 0\\
0 &= P\sin 0 + Q\cos 0\\
0 &= Q\\
\end{split}
\end{equation}
\begin{equation}
\begin{split}
y'(0) &= 1\\
\dv{y}{x}\left(0\right) &= nP\cos 0\\
1 &= nP\\
P &= \frac{1}{n}\\
y &= \frac{1}{n}\sin nx\\
\end{split}
\end{equation}

\item
\begin{equation}
\begin{split}
\left(\dv[2]{x} + 2\dv{x} + 4\right)y &= 0\\
\lambda^2 + 2\lambda + 4 &= 0\\
\lambda &= \frac{-2 \pm \sqrt{4 - 16}}{2}\\
\lambda &= -1 \pm \sqrt{3}i\\
y_c &= e^{-x}(P\sin (\sqrt{3}x) + Q \cos (\sqrt{3}x))\\
y_p &= 0 \\
y &= y_p + y_c \\
y &= e^{-x}(P\sin (\sqrt{3}x) + Q \cos (\sqrt{3}x))
\end{split}
\end{equation}
\begin{equation}
\begin{split}
y(0) &= 0\\
0 &= e^0(P\sin 0 + Q \cos 0)\\
0 &= Q \\
\end{split}
\end{equation}
\begin{equation}
\begin{split}
\dv{y}{x} &= -Pe^{-x}\sin (\sqrt{3}x) + \sqrt{3}Pe^{-x}\cos(\sqrt{3}x)\\
\dv{y}{x}\left(0\right) &= -Pe^{0}\sin 0 + e^{0}\sqrt{3}P\cos(0)\\
1 &= 0 + \sqrt{3}P\\
P &= \frac{1}{\sqrt{3}}\\
y &= \frac{1}{\sqrt{3}}e^{-x}\sin(\sqrt{3}x)\\
\end{split}
\end{equation}

\item
\begin{equation}
\begin{split}
\left(\dv[2]{x} + 9\right)y &= 18\\
\lambda^2 + 9 &= 0\\
\lambda &= \pm 3i\\
y_c &= P\sin 3x + Q \cos 3x + y_p\\
y_p &= c\\
9c &= 18\\
c &= 2\\
y &= y_c + y_p \\
y &= P\sin 3x + Q \cos 3x + 2\\
\end{split}
\end{equation}
\begin{equation}
\begin{split}
y(0) &= 0\\
0 &= P\sin 0 + Q \cos 0 + 2\\
0 &= 0 + Q + 2\\
Q &= -2\\
\end{split}
\end{equation}
\begin{equation}
\begin{split}
y'(0) &= 1 \\
\dv{y}{x} &= 3P\cos 3x - 6 \sin 3x\\
\dv{y}{x}\left(0\right) &= 3P\cos 0 - 6 \sin 0\\
1 &= 3P\\
P &= \frac{1}{3}\\
y &= \frac{1}{3}\sin 3x - 2 \cos 3x + 2\\
\end{split}
\end{equation}

\item
\begin{equation}
\begin{split}
\left(\dv[2]{x} - 3\dv{x} + 2\right)y &= e^{5x}\\
\lambda^2 - 3\lambda + 2 &= 0\\
(\lambda - 1)(\lambda - 2) &= 0\\
\lambda = 1 \vee \lambda &= 2\\
\end{split}
\end{equation}
\begin{equation}
\begin{split}
y_c &= Ae^x + Be^{2x}\\
y_p &= ke^{5x}\\
\dv{y_p}{x} &= 5ke^{5x}\\
\dv[2]{y_p}{x} &= 25ke^{5x}\\
\dv[2]{y_p}{x} - 3\dv{y_p}{x} + 2y_p &= e^{5x}\\
(25k - 15k + 2k)e^{5x} &= e^{5x}\\
12k &= 1\\
k &= \frac{1}{12}\\
y &= y_c + y_p \\
y &= Ae^x + Be^{2x} + \frac{1}{12}e^{5x}\\
\end{split}
\end{equation}
\begin{equation}
\begin{split}
y(0) &= 0\\
0 &= Ae^0 + Be^{0} + \frac{1}{12}e^{0}\\
0 &= A + B + \frac{1}{12}\\
\end{split}
\end{equation}
\begin{equation}
\begin{split}
y'(0) &= 1\\
\dv{y}{x} &= Ae^x + 2Be^{2x} + \frac{5}{12}e^{5x}\\
\dv{y}{x}\left(0\right) &= Ae^0 + 2Be^0 + \frac{5}{12}e^0\\
1 &= A + 2B + \frac{5}{12}\\
1 &= (A + B + \frac{1}{12}) + B + \frac{4}{12}\\
B &= \frac{2}{3}\\
A &= -\frac{1}{12} - B\\
A &= -\frac{3}{4}\\
y &= -\frac{3}{4}e^x + \frac{2}{3}e^{2x} + \frac{1}{12}e^{5x}\\
\end{split}
\end{equation}

\item
\begin{equation}
\begin{split}
\dv[2]{y}{x} - 2\dv{y}{x} + y &= 0\\
\lambda^2 - 2\lambda + 1 &= 0\\
\lambda &= 1\\
y_c &= (A + Bx)e^{x}\\
y_p &= 0\\
y &= y_c + y_p\\
y &= (A + Bx)e^{x}\\
\end{split}
\end{equation}

\item
The complementary function is the same as in the previous question so we can reuse the result and only 
need to work out the particular integral and the initial conditions.
\begin{equation}
\begin{split}
\dv[2]{y}{x} - 2\dv{y}{x} + y &= e^{2x} + e^{x}\\
y_p &= Pe^{2x} + Qx^2e^x\\
\dv{y_p}{x} &= 2Pe^{2x} + 2Qxe^x + Qx^2e^x\\
\dv[2]{y_p}{x} &= 4Pe^{2x} + 2Qe^x + 4Qxe^x + Qx^2e^x\\
\end{split}
\end{equation}
\begin{equation}
\begin{split}
\dv[2]{y_p}{x} - 2\dv{y_p}{x} + y &= e^{2x} + e^{x}\\
4Pe^{2x} - 4Pe^{2x} + Pe^{2x} + 2Qe^x &= e^{2x} + e^x\\
Pe^{2x} + 2Qe^x &= e^{2x} + e^x\\
P = 1 &\wedge Q = \frac{1}{2}\\
y_p &= e^{2x} + \frac{1}{2}x^2e^x\\
y &= y_c + y_p \\
y &= (A + Bx + \frac{1}{2}x^2)e^{x} + e^{2x}\\
\end{split}
\end{equation}

\end{enumerate}

\item
\begin{equation}
\begin{split}
-iR &= \frac{q}{C} + V\\
i &= -\frac{q}{RC} - \frac{V}{R}\\
-L\dv{j}{t} &= \frac{q}{C} + V\\
\dv{j}{t} &= -\frac{q}{LC} - \frac{V}{L}\\
\dv{q}{t} &= i + j\\
\dv{q}{t} &= -\frac{q}{RC} - \frac{V}{R} + j\\
\dv[2]{q}{t} &= -\frac{1}{RC}\dv{q}{t} - \frac{1}{R}\dv{V}{t} + \dv{j}{t}\\
\dv[2]{q}{t} &= -\frac{1}{RC}\dv{q}{t} - \frac{1}{R}\dv{V}{t} - \frac{q}{LC} - \frac{V}{L}\\
\dv[2]{q}{t} + \frac{1}{RC}\dv{q}{t} + \frac{1}{LC}q &= -\frac{1}{R}\dv{V}{t} - \frac{1}{L}V \text{ as required}\\
\end{split}
\end{equation}

\begin{enumerate}

\item 

Since we have initial conditions $Q = 0$ and $Q$ is not changing, we know that both $Q = 0$ and 
$\dv{Q}{t} = 0$. This means that the RHS of the equation is now 0.

\begin{equation}\label{firstpart}
\begin{split}
\dv[2]{q}{t} + \frac{1}{RC}\dv{q}{t} + \frac{1}{LC}q &= 0\\
L = 8R^2C&\\
\dv[2]{q}{t} + \frac{1}{RC}\dv{q}{t} + \frac{1}{8R^2C^2}q &= 0\\
\lambda^2 + \frac{1}{RC}\lambda + \frac{1}{8R^2C^2} &= 0\\
\end{split}
\end{equation}
\begin{equation}
\begin{split}
\lambda &= \frac{-\frac{1}{RC} \pm \sqrt{\frac{1}{R^2C^2} - \frac{4}{8R^2C^2}}}{2}\\
\lambda &= \frac{-\frac{1}{RC} \pm \sqrt{\frac{1}{2R^2C^2}}}{2}\\
\lambda &= \frac{-\frac{1}{RC} \pm \frac{1}{\sqrt{2}RC}}{2}\\
\lambda &= \frac{-2 \pm \sqrt{2}}{4RC}\\
q &= Ae^{\left(\frac{-2 + \sqrt{2}}{4RC}\right)t} + Be^{\left(\frac{-2 - \sqrt{2}}{4RC}\right)t}\\
\end{split}
\end{equation}
\begin{equation}
\begin{split}
q(0) &= Q\\
A + B &= Q\\
\dv{q}{t}\left(0\right) &= -\frac{Q}{RC}\\
\left(\frac{-2 + \sqrt{2}}{4RC}\right)A + \left(\frac{-2 - \sqrt{2}}{4RC}\right)B &= -\frac{Q}{RC}\\
\frac{-2(A + B)}{4RC} + \frac{\sqrt{2}(A - B)}{4RC} &= -\frac{Q}{RC}\\
-\frac{Q}{2RC} + \frac{2\sqrt{2}A}{4RC} - \frac{\sqrt{2}Q}{4RC} &= -\frac{Q}{RC}\\
\frac{\sqrt{2}A}{2RC} &= \frac{(\sqrt{2} - 2)Q}{4RC}\\
A &= \frac{(1 - \sqrt{2})Q}{2}\\
\end{split}
\end{equation}
\begin{equation}
\begin{split}
A + B &= Q\\
B &= Q - \frac{(1 - \sqrt{2})Q}{2}\\
B &= \frac{(1 + \sqrt{2})Q}{2}\\
\end{split}
\end{equation}
\begin{equation}
q = \frac{(1 - \sqrt{2})Q}{2}e^{\left(\frac{-2 + \sqrt{2}}{4RC}\right)t} + \frac{(1 + \sqrt{2})Q}{2}e^{\left(\frac{-2 - \sqrt{2}}{4RC}\right)t}\\
\end{equation}

The roots to this equation are distinct and real: this is strong dampening. 
The charge will decrease to 0 exponentially without ever oscillating.

\item
\begin{equation}
\begin{split}
\dv[2]{q}{t} + \frac{1}{RC}\dv{q}{t} + \frac{1}{LC}q &= 0\\
L = 4R^2C^2&\\
\dv[2]{q}{t} + \frac{1}{RC}\dv{q}{t} + \frac{1}{4R^2C^2}q &= 0\\
\lambda^2 + \frac{1}{RC}\lambda + \frac{1}{4R^2C^2} &= 0\\
\end{split}
\end{equation}
\begin{equation}
\begin{split}
\lambda &= \frac{-\frac{1}{RC} \pm \sqrt{\frac{1}{R^2C^2} - \frac{4}{4R^2C^2}}}{2}\\
\lambda &= -\frac{1}{2RC}\\
q &= (A + Bt)e^{-\frac{t}{2RC}}\\
\end{split}
\end{equation}
\begin{equation}
\begin{split}
q(0) &= Q\\
A &= Q\\
\end{split}
\end{equation}
\begin{equation}
\begin{split}
\dv{q}{t}\left(0\right) &= -\frac{Q}{RC}\\
Be^{-\frac{0}{2RC}} - \frac{Q}{2RC}e^{-\frac{t}{2RC}} &= -\frac{Q}{RC}\\
B - \frac{Q}{2RC} &= -\frac{Q}{RC}\\
B &= -\frac{Q}{2RC}\\
\end{split}
\end{equation}
\begin{equation}
q = (Q - \frac{Qt}{2RC})e^{-\frac{t}{2RC}}\\
\end{equation}

In this equation, the roots to the simultaneous are repeated: this is critial dampening. 
In this form the charge will decrease to 0 without ever increasing 
or oscillating -- however this decrease will be faster than for strong dampening.

\item
\begin{equation}
\begin{split}
\dv[2]{q}{t} + \frac{1}{RC}\dv{q}{t} + \frac{1}{LC}q &= 0\\
L = 2R^2C^2&\\
\dv[2]{q}{t} + \frac{1}{RC}\dv{q}{t} + \frac{1}{2R^2C^2}q &= 0\\
\lambda^2 + \frac{1}{RC}\lambda + \frac{1}{2R^2C^2} &= 0\\
\end{split}
\end{equation}
\begin{equation}
\begin{split}
\lambda &= \frac{-\frac{1}{RC} \pm \sqrt{\frac{1}{R^2C^2} - \frac{4}{2R^2C^2}}}{2}\\
\lambda &= -\frac{1}{2RC} \pm \frac{1}{2RC}i\\
q &= \left(A \sin{\left(\frac{t}{2RC}\right)} + B\cos{\left(\frac{t}{2RC}\right)}\right)e^{-\frac{t}{2RC}}\\
\end{split}
\end{equation}
\begin{equation}
\begin{split}
q(0) &= Q\\
B &= Q\\
\end{split}
\end{equation}
\begin{equation}
\begin{split}
\dv{q}{t}\left(0\right) &= -\frac{Q}{RC}\\
-\frac{q}{2RC} + \frac{1}{2RC}\left(A\cos\left(\frac{t}{2RC}\right) - Q\sin\left(\frac{t}{2RC}\right)\right)e^{-\frac{t}{2RC}} &= -\frac{Q}{RC}\\
-\frac{Q}{2RC} + \frac{A}{2RC} &= -\frac{Q}{RC}\\
A &= -Q\\
\end{split}
\end{equation}
\begin{equation}
q = \left(-Q \sin{\left(\frac{t}{2RC}\right)} + Q\cos{\left(\frac{t}{2RC}\right)}\right)e^{-\frac{t}{2RC}}\\
\end{equation}

The roots to the equation in this case are complex: this is weak dampening -- in this version the charge will oscillate with constant 
angular frequency between two exponentially decreasing values.

\end{enumerate}

\item
\begin{enumerate}
\item
\begin{equation}
\begin{split}
y'' - (2 + c)y' + (1 + c)y &= e^{(1 + 2c)x}\\
\lambda^2 - (2 + c)\lambda + (1 + c) &= 0\\
(\lambda - 1)(\lambda - (1 + c)) &= 0\\
\lambda = 1 \vee \lambda &= 1 + c\\
\end{split}
\end{equation}
\begin{equation}
\begin{split}
y_c &= Pe^x + Qe^{(1 + c)x}\\
y_p &= ke^{(1 + 2c)x}\\
\dv{y_p}{x} &= k(1 + 2c)e^{(1 + 2c)x}\\
\dv[2]{y_p}{x} &= k(1 + 2c)^2e^{(1 + 2c)x}\\
\end{split}
\end{equation}
\begin{equation}
\begin{split}
y_p'' + (2 + c)y_p' + (1 + c)y_p &= e^{(1 + 2c)x}\\
k((1 + 2c)^2 - (2 + c)(1 + 2c) + (1 + c))e^{(1 + 2c)x} &= e^{(1 + 2c)x}\\
k(4c^2 + 4c + 1 - 2c^2 - 5c - 2 + 1 + c) &= 1\\
2c^2k &= 1\\
k &= \frac{1}{2c^2}\\
\end{split}
\end{equation}
\begin{equation}
\begin{split}
k &= \frac{1}{2(c + 1)(3c + 2)}\\
y_p &= \frac{1}{2(c + 1)(3c + 2)}e^{(1 + 2c)x}\\
y &= y_c + y_p\\
y &= Pe^x + Qe^{(1 + c)x} + \frac{1}{2c^2}e^{(1 + 2c)x}\\
\end{split}
\end{equation}
Let:
\begin{equation}
\begin{split}
P &= A - \frac{B}{c} + \frac{1}{2c^2}\\
Q &= B\frac{1}{c} - \frac{1}{c^2}\\
\end{split}
\end{equation}
\begin{equation}
\begin{split}
y &= Ae^x - B\frac{1}{c}e^x + \frac{1}{2c^2} e^x + B\frac{1}{c}e^{(1 + c)x} - \frac{1}{c^2}e^{(1 + c)x} + \frac{1}{2c^2}e^{(1 + 2c)x}\\
y &= Ae^x + B\frac{1}{c}(e^{(1 + c)x} - e^x) + \frac{1}{2c^2}(e^{(1 + 2c)x} - 2e^{(1 + c)x} + e^x)\\
y &= Ae^x + B\frac{e^x}{c}(e^{cx} - 1) + \frac{e^x}{2c^2}(e^{2cx} - 2e^{cx} + 1) \text{ as required}\\
\end{split}
\end{equation}

\item
To find the limit as $c \rightarrow 0$, I will apply l'h\^opitals rule separately for the different parts of 
the expression.
\begin{equation}
\lim_{c \rightarrow 0} f(x, c) = Ae^x + Be^x\lim_{c\rightarrow 0}\left(\frac{(e^{cx} - 1)}{c}\right) + e^x\lim_{c \rightarrow 0}\left(\frac{(e^{2cx} - 2e^{cx} + 1)}{2c^2}\right)\\
\end{equation}
\begin{equation}
\begin{split}
\frac{(e^{cx} - 1)}{c} &= \frac{f(c)}{g(c)}\\
\lim_{c \rightarrow 0} f(c) &= e^0 - 1 = 0\\
\lim_{c \rightarrow 0} g(c) &= 0\\
\end{split}
\end{equation}
So we can apply l'h\^opitals rule to find the limit as $c \rightarrow 0$.
\begin{equation}
\begin{split}
\lim_{c \rightarrow 0} B\frac{e^x}{c}(e^{cx} - 1) &= \frac{\dv{c}(e^{cx} - 1)}{\dv{c}(c)} \\
												  &= \frac{xe^{cx}}{1} \\
												  &= xe^{cx}\\
												  &= xe^0\\
												  &= x\\
\end{split}
\end{equation}
\begin{equation}
\begin{split}
\frac{(e^{2cx} - 2e^{cx} + 1)}{2c^2} &= \frac{f(c)}{g(c)}\\
\lim_{c \rightarrow 0} f(c) &= e^0 - 2e^0 + 1 = 0\\
\lim_{c \rightarrow 0} g(c) &= 2 \cdot 0^2 = 0\\
\end{split}
\end{equation}
So we can apply l'h\^opitals rule to find the limit as $c \rightarrow 0$.
\begin{equation}
\begin{split}
\lim_{c \rightarrow 0} B\frac{e^x}{c}(e^{cx} - 1) &= \frac{\dv{c} (e^{2cx} - 2e^{cx} + 1)}{\dv{c} (2c^2)} \\
												  &= \frac{2xe^{2cx} - 2xe^{cx}}{4c}\\
\end{split}
\end{equation}
Now we must find the limit of this new function as $c \rightarrow 0$.
\begin{equation}
\begin{split}
\frac{2xe^{2cx} - 2xe^{cx}}{4c} &= \frac{f(x)}{g(x)}\\
\lim_{c \rightarrow 0} f(x) &= 2xe^0 - 2xe^0 = 2x - 2x = 0\\
\lim_{c \rightarrow 0} g(x) &= 4 \times 0 = 0\\
\end{split}
\end{equation}
So we can apply l'h\^opitals rule once more.
\begin{equation}
\begin{split}
\lim_{c \rightarrow 0} \frac{2xe^{2cx} - 2xe^{cx}}{4c} &= \frac{\dv{c}(2xe^{2cx} - 2xe^{cx})}{\dv{c}(4c)}\\
													   &= \frac{4x^2e^{2cx} - 2x^2e^{cx}}{4}\\
													   &= x^2e^{0} - \frac{1}{2}x^2e^0\\
													   &= \frac{1}{2}x^2\\
\end{split}
\end{equation}

Substituting these results back into the original expression gives:
\begin{equation}
\begin{split}
\lim_{c \rightarrow 0} f(x, c) &= Ae^x + Bxe^x + \frac{1}{2}x^2e^x\\
							   &= (A + Bx + \frac{1}{2}x^2)e^x\\
\end{split}
\end{equation}
So the solution to the differential equation when $c = 0$ is:
\begin{equation}
y = (A + Bx + \frac{1}{2}x^2)e^x\\
\end{equation}

\end{enumerate}

\item 
\begin{equation}
\begin{split}
-\frac{\eta}{2}\dv{C}{\eta} &= \dv[2]{C}{\eta}\\
\frac{1}{\dv{C}{\eta}}\dv[2]{C}{\eta} &= -\frac{\eta}{2}\\
\dv{\eta}\left(\ln\left(\dv{C}{\eta}\right)\right) &= -\frac{\eta}{2} \text{ as required}\\
\end{split}
\end{equation}
\begin{equation}
\begin{split}
\ln\left(\dv{C}{\eta}\right) &= -\int\frac{\eta}{2} d\eta\\
\ln\left(\dv{C}{\eta}\right) &= - \frac{\eta^2}{4} + c\\
\dv{C}{\eta} &= e^{-\frac{\eta^2}{4} + c}\\
\dv{C}{\eta} &= Ae^{-\frac{\eta^2}{4}}\\
\dv{C}{\eta}\left(0\right) &= Ae^0\\
\dv{C}{\eta}\left(0\right) &= A\\
C &= B + A\int^\eta_0e^{-\frac{t^2}{4}}dt\\
C(0) &= B + A\int^0_0e^{-\frac{t^2}{4}}dt\\
C(0) &= B\\
\end{split}
\end{equation}
So if $\dv{C}{\eta}\left(0\right) = A$ and $C(0) = B$ then $C = B + A\int^\eta_0e^{-\frac{t^2}{4}}dt$ as required.

\item
\begin{enumerate}

\item
\begin{equation}
\begin{split}
\dv{x}{t} &= ax\\
\frac{1}{x}\dv{x}{t} &= a\\
\ln x &= at + c\\
x &= ke^{at}\\
x(0) &= 2\\
2 &= ke^0\\
2 &= k\\
x &= 2e^{at}\\
\end{split}
\end{equation}
\begin{equation}
\begin{split}
\dv{y}{t} &= ay + bx\\
\dv{y}{t} &= ay + 2be^{at}\\
\dv{y}{t} - ay &= 2be^{at}\\
\mu(y) &= e^{\int -a dt}\\
\mu(y) &= e^{-at}\\
e^{-at}\dv{y}{t} - aye^{-at} &= 2b\\
ye^{-at} &= 2bt + C\\
y &= 2bte^{at} + Ce^{at}\\
y(0) &= 1\\
1 &= C\\
y &= 2bte^{at} + e^{at}\\
\end{split}
\end{equation}

\item
\begin{equation}
\begin{split}
\dv{x}{t} &= x - xy\\
\dv{t}{x} &= \frac{1}{x - xy}\\
\dv{y}{t} &= -y + xy\\
\dv{y}{t}\dv{t}{x} &= \frac{-y + xy}{x - xy}\\
\dv{y}{x} &= \frac{y(x - 1)}{x(1 - y)}\\
\end{split}
\end{equation}
And so the coupled differential equations can be transformed into a differential equation of the form 
$\dv{y}{x} = f(x, y)$ as required.
\begin{equation}
\begin{split}
\dv{y}{x} &= \frac{y(x - 1)}{x(1 - y)}\\
\frac{1 - y}{y}\dv{y}{x} &= \frac{x - 1}{x}\\
\left(\frac{1}{y} - 1\right)\dv{y}{x} &= \left(1 - \frac{1}{x}\right)\\
\ln y - y &= x - \ln x + c\\
e^{\ln y - y} &= e^{x - \ln x + c}\\
ye^{-y} &= e^c\frac{e^x}{x}\\
e^{-c} &= \frac{e^y}{y}\cdot \frac{e^x}{x}\\
A &= \frac{e^y}{y}\cdot \frac{e^x}{x}\\
\end{split}
\end{equation}
And so $\frac{e^y}{y}\cdot \frac{e^x}{x}$ is independent of $t$ as required.

\end{enumerate}

\end{enumerate}

\end{document}