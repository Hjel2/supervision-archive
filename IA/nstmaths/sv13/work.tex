\newcommand{\svrname}{Dr John Fawcett}
\newcommand{\jkfside}{oneside}
\newcommand{\jkfhanded}{right}

\newcommand{\studentname}{Harry Langford}
\newcommand{\studentemail}{hjel2@cam.ac.uk}

\documentclass[10pt,\jkfside,a4paper]{article}

\input{./templatenst/includes.tex}
% DO NOT add \usepackage commands here.  Place any custom commands
% into your SV work files.  Anything in the template directory is
% likely to be overwritten!

\usepackage{fancyhdr}

\usepackage{lastpage}       % ``n of m'' page numbering
\usepackage{lscape}         % Makes landscape easier

\usepackage{verbatim}       % Verbatim blocks
\usepackage{epsfig}         % Embed encapsulated postscript
\usepackage{array}          % Array environment
\usepackage[nolinks]{qrcode}         % QR codes
\usepackage{enumitem}       % Required by Tom Johnson's exam question header

\usepackage{hhline}         % Horizontal lines in tables
\usepackage{siunitx}        % Correct spacing of units
\usepackage{amsmath}        % American Mathematical Society
\usepackage{amssymb}        % Maths symbols
\usepackage{amsthm}         % Theorems

\usepackage{ifthen}         % Conditional processing in tex

\usepackage[top=3cm,
            bottom=3cm,
            inner=2cm,
            outer=5cm]{geometry}

% PDF metadata + URL formatting
\usepackage[
            pdfauthor={\studentname},
            pdftitle={\svcourse, SV \svnumber},
            pdfsubject={},
            pdfkeywords={9d2547b00aba40b58fa0378774f72ee6},
            pdfproducer={},
            pdfcreator={},
            hidelinks]{hyperref}

\renewcommand{\headrulewidth}{0.4pt}
\renewcommand{\footrulewidth}{0.4pt}
\fancyheadoffset[LO,LE,RO,RE]{0pt}
\fancyfootoffset[LO,LE,RO,RE]{0pt}
\pagestyle{fancy}
\fancyhead{}
\fancyhead[LO,RE]{{\bfseries \studentname}\\\studentemail}
\fancyhead[RO,LE]{{\bfseries \svcourse, SV~\svnumber}\\\svdate\ \svtime, \svvenue}
\fancyfoot{}
\fancyfoot[LO,RE]{For: \svrname}
\fancyfoot[RO,LE]{\today\hspace{1cm}\thepage\ / \pageref{LastPage}}
\fancyfoot[C]{\qrcode[height=0.8cm]{\svuploadkey}}
\setlength{\headheight}{22.55pt}

\ifthenelse{\equal{\jkfside}{oneside}}{

 \ifthenelse{\equal{\jkfhanded}{left}}{
  % 1. Left-handed marker, one-sided printing or e-marking, use oneside and...
  \evensidemargin=\oddsidemargin
  \oddsidemargin=73pt
  \setlength{\marginparwidth}{111pt}
  \setlength{\marginparsep}{-\marginparsep}
  \addtolength{\marginparsep}{-\textwidth}
  \addtolength{\marginparsep}{-\marginparwidth}
 }{
  % 2. Right-handed marker, one-sided printing or e-marking, use oneside.
  \setlength{\marginparwidth}{111pt}
 }

}{
 % 3. Alternating margins, two-sided printing, use twoside.
}

\setlength{\parindent}{0em}
\addtolength{\parskip}{1ex}

% Exam question headings, labels and sensible layout (courtesy of Tom Johnson)
\setlist{parsep=\parskip, listparindent=\parindent}
\newcommand{\examhead}[3]{\section{#1 Paper #2 Question #3}}
\newenvironment{examquestion}[3]{
    \examhead{#1}{#2}{#3}\setlist[enumerate, 1]{label=(\alph*)}\setlist[enumerate, 2]{label=(\roman*)}
    \marginpar{\qrcode{https://www.cl.cam.ac.uk/teaching/exams/pastpapers/y#1p#2q#3.pdf}}
    \marginpar{\footnotesize \url{https://www.cl.cam.ac.uk/teaching/exams/pastpapers/y#1p#2q#3.pdf}}
}{}



\fancyhead[RO,LE]{{\bfseries NST Maths, SV~13}\\12-03-2022\ 12:00, Video Link}

\usepackage{physics}
\usepackage{enumitem}

\begin{document}

\begin{enumerate}
\setcounter{enumi}{5}

\item For the function $f(x, y, z) = \ln (x^2 + y^2) + z$, find $\nabla f$.

\begin{equation}
\nabla f = \left(\frac{2x}{x^2 + y^2}, \frac{2y}{x^2 + y^2}, 1\right)
\end{equation}

\begin{enumerate}

\item At the point $(3, -4, 0)$ the normal to the cylinder is trivially in the direction 
$\begin{pmatrix} 3 \\ -4 \\ 0 \\ \end{pmatrix}$. 
The magnitude of this vector is $5$ and so $\hat{n}$ is $\begin{pmatrix} \frac{3}{5} \\ 
-\frac{4}{5} \\ 0 \\ \end{pmatrix}$. Using this we can work out the value of $f$ at this 
point.
\begin{equation}
\begin{split}
(\nabla f)(3, -4, 4)\cdot\begin{pmatrix}\frac{3}{5} \\ -\frac{4}{5} \\ 0 \\ \end{pmatrix} &= \begin{pmatrix}\frac{6}{9 + 16} \\ -\frac{8}{9 + 16} \\ 1\end{pmatrix} \cdot\begin{pmatrix}\frac{3}{5} \\ -\frac{4}{5} \\ 0 \\ \end{pmatrix} \\
(\nabla f)(3, -4, 4)\cdot\begin{pmatrix}\frac{3}{5} \\ -\frac{4}{5} \\ 0 \\ \end{pmatrix} &= \frac{6}{25}\times\frac{3}{5} + \frac{8}{25}\times\frac{4}{5} \\
(\nabla f)(3, -4, 4)\cdot\begin{pmatrix}\frac{3}{5} \\ -\frac{4}{5} \\ 0 \\ \end{pmatrix} &= \frac{18}{125} + \frac{32}{125} \\
(\nabla f)(3, -4, 4)\cdot\begin{pmatrix}\frac{3}{5} \\ -\frac{4}{5} \\ 0 \\ \end{pmatrix} &= \frac{2}{5} \\
\end{split}
\end{equation}

\item
\begin{equation}
\hat{\mathbf{m}} = \frac{\mathbf{m}}{|\mathbf{m}|} = \frac{\mathbf{m}}{\sqrt{5}} = \begin{pmatrix} \frac{1}{\sqrt{5}} \\ \frac{2}{\sqrt{5}} \\ 0 \\ \end{pmatrix}
\end{equation}

\begin{equation}
\begin{split}
(\nabla f)(3, -4, 4)\cdot \mathbf{m} &= \begin{pmatrix}\frac{6}{9 + 16} \\ -\frac{8}{9 + 16} \\ 1\end{pmatrix}\cdot\begin{pmatrix} \frac{1}{\sqrt{5}} \\ \frac{2}{\sqrt{5}} \\ 0 \\ \end{pmatrix} \\
(\nabla f)(3, -4, 4)\cdot \mathbf{m} &= \frac{6}{25}\times\frac{1}{\sqrt{5}} - \frac{8}{25}\times\frac{2}{\sqrt{5}} \\
(\nabla f)(3, -4, 4)\cdot \mathbf{m} &= \frac{6}{25\sqrt{5}} - \frac{16}{25\sqrt{5}} \\
(\nabla f)(3, -4, 4)\cdot \mathbf{m} &= -\frac{2\sqrt{5}}{25} \\
\end{split}
\end{equation}

\end{enumerate}

\item 
\begin{equation}
\begin{split}
f &= xz + z^2 - xy^2 \\
\nabla f &= \left(z - y^2, -2xy, x + 2z\right) \\
(\nabla f)(1, 1, 2) &= \left(1, -2, 5\right) \\
\end{split}
\end{equation}

So a normal vector to the surface at the point (1, 1, 2) is $\begin{pmatrix} 1 \\ -2 \\ 5 \\ \end{pmatrix}$.

Hence the equation of the tangent plane at this point is:
\begin{equation}
\begin{split}
\begin{pmatrix} 1 \\ -2 \\ 5 \\ \end{pmatrix}\cdot\begin{pmatrix} x \\ y \\ z \\ \end{pmatrix} &= \begin{pmatrix} 1 \\ -2 \\ 5 \\ \end{pmatrix} \cdot \begin{pmatrix} 1 \\ 1 \\ 2 \\ \end{pmatrix} \\
x - 2y + 5z &= 9 \\
\end{split}
\end{equation}

\item
\begin{equation}
\begin{split}
f &= 3x^2y\sin\left(\frac{\pi x}{2}\right) - z \\
\nabla f &= \left(6xy\sin\left(\frac{\pi x}{2}\right) + \frac{3\pi x^2y}{2}\cos\left(\frac{\pi x}{2}\right), 3x^2\sin\left(\frac{\pi x}{2}\right), -1 \right) \\
\end{split}
\end{equation}
\begin{equation}
\begin{split}
z(1, 1) &= 3\sin\left(\frac{\pi}{2}\right) \\
z(1, 1) &= 3 \\
\end{split}
\end{equation}
\begin{equation}
\begin{split}
(\nabla f)(1, 1, 3) &= \left(6, 3, -1 \right) \\
\end{split}
\end{equation}
So the equation of the plane which is tangent to the surface at the point (1, 1, 3) is:
\begin{equation}
\begin{split}
\begin{pmatrix} 6 \\ 3 \\ -1 \\ \end{pmatrix} \cdot \begin{pmatrix} x \\ y \\ z \\ \end{pmatrix} &= \begin{pmatrix} 6 \\ 3 \\ -1 \\ \end{pmatrix} \cdot \begin{pmatrix} 1 \\ 1 \\ 3 \end{pmatrix} \\
6x + 3y - z &= 6 \\
\end{split}
\end{equation}

At the point with $x = 1, y = \frac{1}{2}$, $z = \frac{3}{2}$.
\begin{equation}
\begin{split}
(\nabla f)\left(1, \frac{1}{2}, \frac{3}{2}\right) &= \left(3, 3, -1\right) \\
\end{split}
\end{equation}

Consider now $-n$. This is also a normal to the plane but facing in the increasing 
$z$ direction (I assume that we are placing the marble on the plane from above -- not 
suspending it below the plane). Now the normal is $\begin{pmatrix} -3 \\ -3 \\ 1 \\ \end{pmatrix}$. 
so if we place a marble on the plane then it will roll South-West.

\item 
Consider the substitution:
\begin{equation}
\begin{split}
x &= a\cos\theta \\
\dd{x} &= -a\sin\theta \dd{\theta} \\
y &= a\sin\theta \\
\dd{y} &= a\cos\theta \dd{\theta} \\
\end{split}
\end{equation}

\begin{equation}
\begin{split}
 & \int_\Gamma [P(x, y)\dd{x} + Q(x, y)\dd{y}] \\
=& \int_\Gamma [-x^2y\dd{x} + xy^2\dd{y}] \\
=& \int^{\pi}_{0}[a^4\cos^2\theta\sin^2\theta\dd{\theta} + a^4\cos^2\theta\sin^2\theta\dd{\theta}] \\
=& a^4\int^{\pi}_{0}2\cos^2\theta\sin^2\theta\dd{\theta} \\
=& a^4\int^{\pi}_{0}\frac{1}{2}\sin^{2}2\theta\dd{\theta} \\
=& a^4\int^{\pi}_{0}\frac{1}{4} - \frac{1}{4}\cos4\theta\dd{\theta} \\
=& a^4\left[\frac{1}{4}\theta - \frac{1}{16}\sin4\theta\right]^\pi_0 \\
=& \frac{\pi a^4}{4} \\
\end{split}
\end{equation}

\begin{equation}
\begin{split}
 & \int\int_D \left(\pdv{Q}{x} - \pdv{P}{y}\right)\dd{x}\dd{y} \\
=& \int \int y^2 + x^2\dd{x}\dd{y} \\
=& \int^{\pi}_0\int^a_0 r^2 \times r\dd{r}\dd{\theta} \\
=& \int^{\pi}_0\int^a_0 r^3\dd{r}\dd{\theta} \\
=& \int^{\pi}_0\left[\frac{1}{4}r^4\right]^a_0\dd{\theta} \\
=& \int^{\pi}_0\frac{a^4}{4}\dd{\theta} \\
=& \frac{\pi a^4}{4} \\
=& \int_\Gamma [P(x, y)\dd{x} + Q(x, y)\dd{y}] \text{ as required } \\
\end{split}
\end{equation}

\item 

\begin{equation}
\int_{\Gamma}\left[f(y)\dd{x} + x\cos{y}\dd{y}\right] = 0
\end{equation}

The integrand of an exact differential along any closed contour is equal to zero. 
So if $\int_{\Gamma}\left[f(y)\dd{x} + x\cos{y}\dd{y}\right]$ is an exact differential, 
then it will be equal to zero for all closed contours $\Gamma$.

For the integrand to be an exact differential:
\begin{equation}
\begin{split}
\left(\pdv{y}\right)f(y) &= \left(\pdv{x}\right)x\cos{y} \\
\left(\pdv{y}\right)f(y) &= \cos{y} \\
f(y) &= \int \cos{y} \dd{y} \\
f(y) &= \sin{y} + c \\
\end{split}
\end{equation}

\item 
\begin{enumerate}[label=(\roman*)]

\item 

For the purposes of this question I will assume the curve is $\textit{closed}$ since the statement is 
untrue if it is not.

If $\mathbf{F} = -\nabla(\Phi)$ for some $\Phi$, then $\int_C \mathbf{F}\cdot \dd{\mathbf{x}} = 0$ for 
all closed curves $C$.

\begin{equation}
\begin{split}
\mathbf{F} &= \mathbf{c}\times\mathbf{v} \\
\mathbf{F} &= -\mathbf{c}\left(-\dv{\mathbf{x}(t)}{t}\right) \\
\mathbf{F} &= -\left(\nabla(-\mathbf{x}(t))\right) \\
\mathbf{F} &= -\nabla\Phi \\
\end{split}
\end{equation}

This is a necessary and sufficient condition for the $\int_C \mathbf{F}\cdot \dd{\mathbf{x}}$ to be 
equal to zero for all closed curves $C$.

So the integral $\int_C \mathbf{F}\cdot \dd{\mathbf{x}} = 0$ for 
all closed curves $C$.

The integral is equal to the work done. So the work done is equal to 0 for all closed curves $C$.

\item 

\begin{enumerate}[label=(\alph*)]

\item

\begin{equation}
\begin{split}
W &= \int_C \mathbf{F} \cdot \dd{\mathbf{x}} \\
  &= \int^\pi_0 (y, -x, -1)  \cdot (-\sin t, \cos t, 1) \dd{t} \\
  &= \int^\pi_0 (\sin t, -\cos t, -1)  \cdot (-\sin t, \cos t, 1) \dd{t} \\
  &= \int^\pi_0 -\sin^2 t - \cos^2 t - 1 \dd{t} \\
  &= \int^\pi_0 -1 - 1 \dd{t} \\
  &= \int^\pi_0 -2 \dd{t} \\
  &= -2\pi \\
\end{split}
\end{equation}

\item 

\begin{equation}
\begin{split}
W &= \int_C \mathbf{F} \cdot \dd{\mathbf{x}} \\
  &= \int^\pi_0 (x, y, 0) \cdot (-\sin t, \cos t, 1) \dd{t} \\
  &= \int^\pi_0 -\sin t \cos t + \sin t \cos t \dd{t} \\
  &= \int^\pi_0 0 \dd{t} \\
  &= 0 \\
\end{split}
\end{equation}

\end{enumerate}

\end{enumerate}

\item 

A condition satisfied by all conservative vector fields $\mathbf{F} = (P(x, y), Q(x, y))$ is that 
$\mathbf{F}$ is an exact differential. So:

\begin{equation}
\left(\pdv{y}\right)P(x, y) = \left(\pdv{x}\right)Q(x, y)
\end{equation}

\begin{enumerate}

\item

\begin{equation}
P = x^2y + y, \text{   } Q = xy^2 + x 
\end{equation}
\begin{equation}
\begin{split}
\left(\pdv{y}\right)P(x, y) &= x^2 + 1 \\
\left(\pdv{x}\right)Q(x, y) &= y^2 + 1 \\
\left(\pdv{y}\right)P(x, y) &\neq \left(\pdv{x}\right)Q(x, y) \\
\end{split}
\end{equation}
So $\mathbf{F}$ is \textbf{not} a conservative vector field

\item
\begin{equation}
P = ye^{xy} + 2x + y, \text{   } Q = xe^{xy} + x \\
\end{equation}
\begin{equation}
\begin{split}
\left(\pdv{y}\right)P(x, y) &= xye^{xy} + e^{xy} + 1 \\
\left(\pdv{x}\right)Q(x, y) &= xye^{xy} + e^{xy} + 1 \\
\left(\pdv{y}\right)P(x, y) &= \left(\pdv{x}\right)Q(x, y) \\
\end{split}
\end{equation}
So $\mathbf{F}$ is a conservative vector field

\end{enumerate}

\item 
\begin{enumerate}[label=(\roman*)]

\item 

\begin{equation*}
\begin{split}
(0, 0, 0) \longrightarrow (0, 0, 1) \\
x = y = 0, z = t \\
\dd{\mathbf{x}} = (0, 0, 1) \\
\end{split}
\end{equation*}
\begin{equation}
\begin{split}
 & \int \mathbf{F} \cdot \dd{\mathbf{x}} \\
=& \int^1_0 (4x^3z + 2x, z^2 - 2y, x^4 + 2yz) \cdot (0, 0, 1) \dd{t} \\
=& \int^1_0 x^4 + 2yz\dd{t} \\
=& \int^1_0 0\dd{t} \\
=& 0 \\
\end{split}
\end{equation}

\begin{equation*}
\begin{split}
(0, 0, 1) \longrightarrow (0, 1, 1) \\
x = 0, y = t, z = 1 \\
\dd{\mathbf{x}} = (0, 1, 0) \\
\end{split}
\end{equation*}
\begin{equation}
\begin{split}
 & \int \mathbf{F} \cdot \dd{\mathbf{x}} \\
=& \int^1_0 (4x^3z + 2x, z^2 - 2y, x^4 + 2yz) \cdot (0, 1, 0) \dd{t} \\
=& \int^1_0 z^2 - 2y \dd{t} \\
=& \int^1_0 1 - 2t \dd{t} \\
=& 0 \\
\end{split}
\end{equation}

\begin{equation*}
\begin{split}
(0, 1, 1) \longrightarrow (1, 1, 1) \\
x = t, y = z = 1 \\
\dd{\mathbf{x}} = (1, 0, 0) \\
\end{split}
\end{equation*}
\begin{equation}
\begin{split}
 & \int \mathbf{F} \cdot \dd{\mathbf{x}} \\
=& \int^1_0 (4x^3z + 2x, z^2 - 2y, x^4 + 2yz) \cdot (1, 0, 0) \dd{t} \\
=& \int^1_0 4x^3z + 2x \dd{t} \\
=& \int^1_0 4t^3 + 2t \dd{t} \\
=& [t^4 + t^2]^1_0 \\
=& 2 \\
\end{split}
\end{equation}

\begin{equation}
0 + 0 + 2 = 2
\end{equation}

So the line integral along the sequence of straight line paths is 2.

\item 

\begin{equation}
\begin{split}
x = y = z = t \\
\dd{\mathbf{x}} = (1, 1, 1) \\
\end{split}
\end{equation}

\begin{equation}
\begin{split}
 & \int \mathbf{F}\cdot \dd{\mathbf{x}} \\
=& \int^1_0 (4x^3z + 2x, z^2 - 2y, x^4 + 2yz) \cdot (1, 1, 1) \dd{t} \\
=& \int^1_0 4x^3z + 2x + z^2 - 2y + x^4 + 2yz\dd{t} \\
=& \int^1_0 4t^4 + 2t + t^2 - 2t + t^4 + 2t^2\dd{t} \\
=& \int^1_0 5t^4 + 3t^2\dd{t} \\
=& [t^5 + t^3]^1_0 \\
=& 2 \\
\end{split}
\end{equation}

A function $f(x, y, z)$ such that $\mathbf{F} = \nabla f$ is:
\begin{equation}
f(x, y, z) = x^4z + x^2 + yz^2 - y^2 + c \\
\end{equation}

\end{enumerate}

\item 

To integrate $C$, I will make the substitution:
\begin{equation}
\begin{split}
x &= \cos\theta \\
y &= \sin\theta \\
z &= 0 \\
\dd{\mathbf{x}} &= (-\sin\theta, \cos\theta, 0) \dd{\theta} \\
\end{split}
\end{equation}

\begin{equation}
\begin{split}
 & \int_C \mathbf{E} \cdot \dd{\mathbf{x}} \\
=& \int^{2\pi}_0 (-ye^{-2t}, xe^{-2t}, 0) \cdot (-\sin\theta, \cos\theta, 0) \dd{\theta} \\
=& \int^{2\pi}_0 (-\sin\theta e^{-2t}, \cos\theta e^{-2t}, 0) \cdot (-\sin\theta, \cos\theta, 0) \dd{\theta} \\
=& \int^{2\pi}_0 \sin^2\theta e^{-2t} + \cos^2\theta e^{-2t} \dd{\theta} \\
=& \int^{2\pi}_0 (\sin^2\theta + \cos^2\theta)e^{-2t} \dd{\theta} \\
=& \int^{2\pi}_0 e^{-2t} \dd{\theta} \\
=& 2\pi e^{-2t} \\
\end{split}
\end{equation}

\begin{equation}
\begin{split}
 & -\dv{t}\int_S \mathbf{B} \cdot \dd{\mathbf{S}} \\
=& -\dv{t}\int_S (0, 0, e^{-2t}) \cdot \hat{n} \dd{S} \\
=& -\dv{t}\int_S (0, 0, e^{-2t}) \cdot (0, 0, 1) \dd{S} \\
=& -\dv{t}\int_S e^{-2t} \dd{S} \\
=& -\dv{t}\int^{2\pi}_0\int^1_0 re^{-2t} \dd{r}\dd{\theta} \\
=& -\dv{t}\int^{2\pi}_0\left[\frac{1}{2}r^2e^{-2t}\right]^1_0\dd{\theta} \\
=& -\dv{t}\int^{2\pi}_0 \frac{1}{2}e^{-2t} \dd{\theta} \\
=& -\dv{t} \left(2\pi \times \frac{1}{2}e^{-2t}\right) \\
=& -\dv{t} \left(\pi e^{-2t}\right) \\
=& -\left(-2\pi e^{-2t}\right) \\
=& 2\pi e^{-2t} \\
=& \int_C \mathbf{E} \cdot \dd{\mathbf{x}} \text{ as required } \\
\end{split}
\end{equation}

\end{enumerate}

\end{document}