\newcommand{\svrname}{Dr John Fawcett}
\newcommand{\jkfside}{oneside}
\newcommand{\jkfhanded}{right}

\newcommand{\studentname}{Harry Langford}
\newcommand{\studentemail}{hjel2@cam.ac.uk}

\documentclass[10pt,\jkfside,a4paper]{article}

% DO NOT add \usepackage commands here.  Place any custom commands
% into your SV work files.  Anything in the template directory is
% likely to be overwritten!

\usepackage{fancyhdr}

\usepackage{lastpage}       % ``n of m'' page numbering
\usepackage{lscape}         % Makes landscape easier

\usepackage{verbatim}       % Verbatim blocks
\usepackage{listings}       % Source code listings
\usepackage{epsfig}         % Embed encapsulated postscript
\usepackage{array}          % Array environment
\usepackage{qrcode}         % QR codes
\usepackage{enumitem}       % Required by Tom Johnson's exam question header

\usepackage{hhline}         % Horizontal lines in tables
\usepackage{siunitx}        % Correct spacing of units
\usepackage{amsmath}        % American Mathematical Society
\usepackage{amssymb}        % Maths symbols
\usepackage{amsthm}         % Theorems

\usepackage{ifthen}         % Conditional processing in tex

\usepackage[top=3cm,
            bottom=3cm,
            inner=2cm,
            outer=5cm]{geometry}

% PDF metadata + URL formatting
\usepackage[
            pdfauthor={\studentname},
            pdftitle={\svcourse, SV \svnumber},
            pdfsubject={},
            pdfkeywords={9d2547b00aba40b58fa0378774f72ee6},
            pdfproducer={},
            pdfcreator={},
            hidelinks]{hyperref}


% DO NOT add \usepackage commands here.  Place any custom commands
% into your SV work files.  Anything in the template directory is
% likely to be overwritten!

\usepackage{fancyhdr}

\usepackage{lastpage}       % ``n of m'' page numbering
\usepackage{lscape}         % Makes landscape easier

\usepackage{verbatim}       % Verbatim blocks
\usepackage{listings}       % Source code listings
\usepackage{graphicx}
\usepackage{float}
\usepackage{epsfig}         % Embed encapsulated postscript
\usepackage{array}          % Array environment
\usepackage{qrcode}         % QR codes
\usepackage{enumitem}       % Required by Tom Johnson's exam question header

\usepackage{hhline}         % Horizontal lines in tables
\usepackage{siunitx}        % Correct spacing of units
\usepackage{amsmath}        % American Mathematical Society
\usepackage{amssymb}        % Maths symbols
\usepackage{amsthm}         % Theorems

\usepackage{ifthen}         % Conditional processing in tex

\usepackage[top=3cm,
            bottom=3cm,
            inner=2cm,
            outer=5cm]{geometry}

% PDF metadata + URL formatting
\usepackage[
            pdfauthor={\studentname},
            pdftitle={\svcourse, SV \svnumber},
            pdfsubject={},
            pdfkeywords={9d2547b00aba40b58fa0378774f72ee6},
            pdfproducer={},
            pdfcreator={},
            hidelinks]{hyperref}

\renewcommand{\headrulewidth}{0.4pt}
\renewcommand{\footrulewidth}{0.4pt}
\fancyheadoffset[LO,LE,RO,RE]{0pt}
\fancyfootoffset[LO,LE,RO,RE]{0pt}
\pagestyle{fancy}
\fancyhead{}
\fancyhead[LO,RE]{{\bfseries \studentname}\\\studentemail}
\fancyhead[RO,LE]{{\bfseries \svcourse, SV~\svnumber}\\\svdate\ \svtime, \svvenue}
\fancyfoot{}
\fancyfoot[LO,RE]{For: \svrname}
\fancyfoot[RO,LE]{\today\hspace{1cm}\thepage\ / \pageref{LastPage}}
\fancyfoot[C]{\qrcode[height=0.8cm]{\svuploadkey}}
\setlength{\headheight}{22.55pt}


\ifthenelse{\equal{\jkfside}{oneside}}{

 \ifthenelse{\equal{\jkfhanded}{left}}{
  % 1. Left-handed marker, one-sided printing or e-marking, use oneside and...
  \evensidemargin=\oddsidemargin
  \oddsidemargin=73pt
  \setlength{\marginparwidth}{111pt}
  \setlength{\marginparsep}{-\marginparsep}
  \addtolength{\marginparsep}{-\textwidth}
  \addtolength{\marginparsep}{-\marginparwidth}
 }{
  % 2. Right-handed marker, one-sided printing or e-marking, use oneside.
  \setlength{\marginparwidth}{111pt}
 }

}{
 % 3. Alternating margins, two-sided printing, use twoside.
}


\setlength{\parindent}{0em}
\addtolength{\parskip}{1ex}

% Exam question headings, labels and sensible layout (courtesy of Tom Johnson)
\setlist{parsep=\parskip, listparindent=\parindent}
\newcommand{\examhead}[3]{\section{#1 Paper #2 Question #3}}
\newenvironment{examquestion}[3]{
\examhead{#1}{#2}{#3}\setlist[enumerate, 1]{label=(\alph*)}\setlist[enumerate, 2]{label=(\roman*)}
\marginpar{\href{https://www.cl.cam.ac.uk/teaching/exams/pastpapers/y#1p#2q#3.pdf}{\qrcode{https://www.cl.cam.ac.uk/teaching/exams/pastpapers/y#1p#2q#3.pdf}}}
\marginpar{\footnotesize \href{https://www.cl.cam.ac.uk/teaching/exams/pastpapers/y#1p#2q#3.pdf}{https://www.cl.cam.ac.uk/\\teaching/exams/pastpapers/\\y#1p#2q#3.pdf}}
}{}


\fancyhead[RO,LE]{{\bfseries NST Maths, SV~11}\\16-02-22\ 12:00, Video Link}

\usepackage{physics}

\begin{document}

\begin{enumerate}

\setcounter{enumi}{9}

\item 
\begin{enumerate}

\item 
\begin{equation}
\begin{split}
y \dd{x} + x \dd{y} \\
\pdv{y}\left(y\right) &= 1\\
\pdv{x}\left(x\right) &= 1\\
\end{split}
\end{equation}
Since $\pdv{y}\left(P\right) = \pdv{x}\left(Q\right)$ the differential is exact.

\begin{equation}
\begin{split}
y \dd{x} + x \dd{y} &= 0 \\
xy &= c\\
y &= \frac{c}{x}\\
\end{split}
\end{equation}

\item 
\begin{equation}
\begin{split}
y \dd{x} + x^2 \dd{y} \\
\pdv{y}\left(y\right) &= 1\\
\pdv{x}\left(x^2\right) &= 2x \neq 1\\
\end{split}
\end{equation}
$\pdv{y}\left(P\right) \neq \pdv{x}\left(Q\right)$ and so the differential is not exact.

\begin{equation}
\begin{split}
\mu(x) &= e^{\int\frac{1}{x^2}\left(\pdv{y}\left(y\right) - \pdv{x}\left(x^2\right) \right)\dd{x}}\\
	   &= e^{\int \frac{1}{x^2}\left( 1 - 2x\right) \dd{x}}\\
	   &= e^{\int \frac{1}{x^2} - \frac{2}{x} \dd{x}}\\
	   &= e^{-\frac{1}{x} - 2\ln x} \\
	   &= x^{-2}e^{-\frac{1}{x}} \\
\end{split}
\end{equation}

\begin{equation}
\begin{split}
y \dd{x} + x^2 \dd{y} &= 0 \\
x^{-2}ye^{-\frac{1}{x}} \dd{x} + e^{-\frac{1}{x}} \dd{y} &= 0 \\
ye^{-\frac{1}{x}} &= c \\
y &= ce^{\frac{1}{x}} \\
\end{split}
\end{equation}

\item 
\begin{equation}
\begin{split}
(x + y) \dd{x} + (x - y) \dd{y} \\
\pdv{y}(x + y) &= 1 \\
\pdv{x}(x - y) &= 1 \\
\end{split}
\end{equation}
$\pdv{y}\left(P\right) = \pdv{x}\left(Q\right)$ and so the differential is exact.

\begin{equation}
\begin{split}
(x + y) \dd{x} + (x - y) \dd{y} &= 0 \\
(x + y) \dd{x} + (x - y) \dd{y} &= 0 \\
\frac{1}{2}x^2 + xy - \frac{1}{2}y^2 &= c \\
\end{split}
\end{equation}

\item 
\begin{equation*}
(\cosh x \cos y + \cosh y \cos x) \dd{x} - (\sinh x \sin y - \sinh y \sin x) \dd{y} \\
\end{equation*}
\begin{equation}
\begin{split}
\pdv{y}\left(\cosh x \cos y + \cosh y \cos x\right) &= -\cosh x \sin y + \sinh y \cos x \\
\pdv{x}\left(-\sinh x \sin y + \sinh y \sin x\right) &= -\cosh x \sin y + \sinh y \cos x \\
\end{split}
\end{equation}
So $\pdv{y}\left(P\right) = \pdv{x}\left(Q\right)$ and so the differential is exact.

\begin{equation}
\begin{split}
(\cosh x \cos y + \cosh y \cos x) \dd{x} - (\sinh x \sin y - \sinh y \sin x) \dd{y} &= 0 \\
\sinh x \cos y + \cosh y \sin x &= c \\
\end{split}
\end{equation}

\item 
\begin{equation}
\begin{split}
(\cos x - \sin x) \dd{x} + (\sin x + \cos x) \dd{y} \\
\pdv{y}\left(\cos x - \sin x \right) &= 0 \\
\pdv{x}\left(\sin x + \cos x \right) &= \cos x - \sin x \\
\end{split}
\end{equation}
So $\pdv{y}\left(P\right) \neq \pdv{x}\left(Q\right)$ and so the differential is not exact.

\begin{equation}
\begin{split}
\mu(y) &= e^{\int \frac{1}{(\sin x + \cos x)} \left(\pdv{y}\left((\cos x - \sin x)\right) - \pdv{x}\left( \sin x + \cos x \right) \right) \dd{x}} \\
\mu(y) &= e^{\int \frac{1}{(\sin x + \cos x)} \left(0 - \cos x + \sin x \right) \dd{x}} \\
\mu(y) &= e^{\int 1 \dd{x}} \\
\mu(y) &= e^y \\
\end{split}
\end{equation}

\begin{equation}
\begin{split}
(\cos x - \sin x) \dd{x} + (\sin x + \cos x) \dd{y} &= 0 \\
(\cos x - \sin x)e^y \dd{x} + (\sin x + \cos x)e^y \dd{y} &= 0 \\
(\sin x + \cos x)e^y &= c \\
e^y &= \frac{c}{\sin x + \cos x} \\
y &= -\ln A(\sin x + \cos x) \\
\end{split}
\end{equation}

\item 
\begin{equation}
\begin{split}
\frac{x}{x^2 + y^2} \dd{y} - \frac{y}{x^2 + y^2} \dd{x} \\
\pdv{y}\left(-\frac{y}{x^2 + y^2}\right) &= \frac{y^2 - x^2}{x^2 + y^2} \\
\pdv{x}\left(\frac{x}{x^2 + y^2}\right) &= \frac{y^2 - x^2}{x^2 + y^2} \\
\end{split}
\end{equation}
So $\pdv{y}\left(P\right) = \pdv{x}\left(Q\right)$ and so the differential is exact.

\begin{equation}
\begin{split}
\frac{x}{x^2 + y^2} \dd{y} - \frac{y}{x^2 + y^2} \dd{x} &= 0 \\
\arctan\left(\frac{y}{x}\right) &= c \\
\frac{y}{x} &= k \\
y &= kx \\
\end{split}
\end{equation}

\end{enumerate}

\item 

\begin{enumerate}

\item 

\begin{equation}
\begin{split}
H &= U + pV \\
\dd{H} &= \dd{U} + V\dd{p} + p\dd{V} \\
\dd{H} &= (\dd{U} + p\dd{V}) + V\dd{p} \\
\dd{H} &= T\dd{S} + V\dd{p} \\
\end{split}
\end{equation}

We know this is an exact differential. So it is of the form 
\begin{equation}
\dd{H} = P \dd{S} + Q\dd{p}\\
\end{equation}
Where $\left(\pdv{P}{p}\right)_S = \left(\pdv{Q}{S}\right)_p$. Substituting in the actual coefficients 
of the equation ($P = T$, $Q = V$) gives us:
\begin{equation}
\left(\pdv{V}{S}\right)_p = \left(\pdv{T}{p}\right)_S\\
\end{equation}

As required.

\item 

\begin{equation}
\begin{split}
\dd{U} &= T\dd{S} - p\dd{V} \\
\dd{U} &= T\left(\left(\pdv{S}{p}\right)_V\dd{p} + \left(\pdv{S}{V}\right)_p\dd{V}\right) - p\dd{V} \\
\dd{U} &= T\left(\pdv{S}{p}\right)_V\dd{p} + \left(T\left(\pdv{S}{V}\right)_p- p\right)\dd{V} \\
\end{split}
\end{equation}

We know that $\dd{U}$ is an exact integral -- this implies:

\begin{equation}
\begin{split}
\pdv{p}\left(T\left(\pdv{S}{V}\right)_p - p\right) &= \pdv{V}\left(T\left(\pdv{S}{p}\right)_V\right) \\
\left(\pdv{T}{p}\right)_V\left(\pdv{S}{V}\right)_p +  T\left(\pdv[2]{S}{p}{V}\right) - 1 &= \left(\pdv{T}{V}\right)_p\left(\pdv{S}{p}\right)_V + T\left(\pdv[2]{S}{p}{V}\right) \\
\left(\pdv{S}{V}\right)_p\left(\pdv{T}{p}\right)_V - 1 &= \left(\pdv{S}{p}\right)_V\left(\pdv{T}{V}\right)_p \\
\left(\pdv{S}{V}\right)_p\left(\pdv{T}{p}\right)_V - \left(\pdv{S}{p}\right)_V\left(\pdv{T}{V}\right)_p &= 1 \\
\end{split}
\end{equation}

\end{enumerate}

\item 

\begin{equation}
\begin{split}
G &= U + Vp - ST \\
\dd{G} &=  \dd{U} + \pdv{p}\left(Vp\right)_V\dd{p} + \pdv{V}\left(Vp\right)_p\dd{V} - \pdv{S}\left(ST\right)_T\dd{S} - \pdv{T}\left(ST\right)_S\dd{T} \\
\dd{G} &= \dd{U} + V\dd{p} + p\dd{V} - T\dd{S} - S\dd{T} \\
\dd{G} &= T\dd{S} - p\dd{V} + V\dd{p} + p\dd{V} - T\dd{S} - S\dd{T} \\
\dd{G} &= V\dd{p} - S\dd{T} \\
\end{split}
\end{equation}

Since we know that $\dd{G}$ is an exact differential; we know that 
the partial derivative of $V$ with respect to $T$ is equal to the partial derivative of $-S$ with 
respect to $p$.

\begin{equation}
\begin{split}
\left(\pdv{V}{T}\right)_p &= -\left(\pdv{S}{p}\right)_T \\
\left(\pdv{S}{p}\right)_T &= -\left(\pdv{V}{T}\right)_p \\
\end{split}
\end{equation}

\item 

\begin{enumerate}

\item 

Since $p$ is a function of either $V$ and $T$ or $V$ and $S$; we 
can express $\dd{p}$ as a function of either also. We can also express 
any of the variables $V, T, S$ as a function of the other two. I will 
express $S$ as a function of $V$ and $T$.

\begin{equation}\label{firstpart}
\dd{p} = \left(\pdv{p}{V}\right)_T\dd{V} + \left(\pdv{p}{T}\right)_V\dd{T} \\
\end{equation}
\begin{equation}\label{functionfordp}
\dd{p} = \left(\pdv{p}{V}\right)_S\dd{V} + \left(\pdv{p}{S}\right)_V\dd{S} \\
\end{equation}
\begin{equation}\label{functionfordS}
\dd{S} = \left(\pdv{S}{V}\right)_T\dd{V} + \left(\pdv{S}{T}\right)_V\dd{T} \\
\end{equation}
Substitute (\ref{functionfordS}) into (\ref{functionfordp}).
\begin{equation}\label{penultimatesection}
\begin{split}
\dd{p} &= \left(\pdv{p}{V}\right)_S\dd{V} + \left(\pdv{p}{S}\right)_V\dd{S} \\
\dd{p} &= \left(\pdv{p}{V}\right)_S\dd{V} + \left(\pdv{p}{S}\right)_V\left(\pdv{S}{V}\right)_T\dd{V} + \left(\pdv{p}{S}\right)_V\left(\pdv{S}{T}\right)_V\dd{T} \\
\dd{p} &= \left(\pdv{p}{V}\right)_S\dd{V} + \left(\pdv{p}{S}\right)_V\left(\pdv{S}{V}\right)_T\dd{V} + \left(\pdv{p}{T}\right)_V\dd{T} \\
\dd{p} &= \left(\left(\pdv{p}{V}\right)_S + \left(\pdv{p}{S}\right)_V\left(\pdv{S}{V}\right)_T\right)\dd{V} + \left(\pdv{p}{T}\right)_V\dd{T} \\
\end{split}
\end{equation}
Now subtract (\ref{firstpart}) from (\ref{penultimatesection}).
\begin{equation}
\begin{split}
\dd{p} - \dd{p} &= \left(\left(\pdv{p}{V}\right)_S + \left(\pdv{p}{S}\right)_V\left(\pdv{S}{V}\right)_T\right)\dd{V} + \left(\pdv{p}{T}\right)_V\dd{T} - \left(\pdv{p}{V}\right)_T\dd{V} - \left(\pdv{p}{T}\right)_V\dd{T} \\
0 &= \left(\left(\pdv{p}{V}\right)_S - \left(\pdv{p}{V}\right)_T + \left(\pdv{p}{S}\right)_V\left(\pdv{S}{V}\right)_T\right)\dd{V} + \left(\left(\pdv{p}{T}\right)_V - \left(\pdv{p}{T}\right)_V\right)\dd{T} \\
0 &= \left(\left(\pdv{p}{V}\right)_S - \left(\pdv{p}{V}\right)_T + \left(\pdv{p}{S}\right)_V\left(\pdv{S}{V}\right)_T\right)\dd{V} \\
\end{split}
\end{equation}
So the coefficients of $\dd{V}$ must be equal to zero.
\begin{equation}
\begin{split}
0 &= \left(\pdv{p}{V}\right)_S - \left(\pdv{p}{V}\right)_T + \left(\pdv{p}{S}\right)_V\left(\pdv{S}{V}\right)_T \\
\left(\pdv{p}{V}\right)_T - \left(\pdv{p}{V}\right)_S &= \left(\pdv{p}{S}\right)_V\left(\pdv{S}{V}\right)_T \\
\left(\pdv{p}{V}\right)_T - \left(\pdv{p}{V}\right)_S &= \frac{\left(\pdv{S}{V}\right)_T}{\left(\pdv{S}{p}\right)_V} \\
\end{split}
\end{equation}

\item 

\begin{equation}
\begin{split}
T\dd{S} &= \dd{U} + p\dd{V} \\
T\dd{S} &= \left(\pdv{U}{T}\right)_V\dd{T} + \left(\left(\pdv{U}{V}\right)_T + p\right)\dd{V} \\
\dd{S} &= \left(\frac{1}{T}\left(\pdv{U}{T}\right)_V\right)\dd{T} + \left(\frac{1}{T}\left(\pdv{U}{V}\right)_T + \frac{p}{T}\right)\dd{V} \\
\end{split}
\end{equation}
$\dd{S}$ is an exact differential. So the right hand side must also be an exact differential.
\begin{equation}\label{hardpart}
\begin{split}
\frac{1}{T}\left(\pdv[2]{U}{T}{V}\right) &= -\frac{1}{T^2}\left(\pdv{U}{V}\right)_T + \frac{1}{T}\left(\pdv[2]{U}{T}{V}\right) - \frac{p}{T^2} + \frac{1}{T}\left(\pdv{p}{T}\right)_V \\
0 &= -\frac{1}{T^2}\left(\pdv{U}{V}\right)_T - \frac{p}{T^2} + \frac{1}{T}\left(\pdv{p}{T}\right)_V \\
\frac{T}{p}\left(\pdv{p}{T}\right)_V &= \frac{1}{p}\left(\pdv{U}{V}\right)_T + 1 \\
\end{split}
\end{equation}

We can form another relation:
\begin{equation}
\begin{split}
\dd{U} &= T\dd{S} - p\dd{V} \\
\dd{U} &= T\left(\pdv{S}{T}\right)_V\dd{T} + \left(T\left(\pdv{S}{V}\right)_T - p\right)\dd{V} \\
\end{split}
\end{equation}
Since $\dd{U}$ is an exact differential, we also know the right-hand-side of the equation is exact.
\begin{equation}
\begin{split}
T\left(\pdv[2]{S}{T}{V}\right) &= T\left(\pdv[2]{S}{T}{V}\right) + \left(\pdv{S}{V}\right)_T - \left(\pdv{p}{T}\right)_V \\
0 &= \left(\pdv{S}{V}\right)_T - \left(\pdv{p}{T}\right)_V \\
\left(\pdv{p}{T}\right)_V &= \left(\pdv{S}{V}\right)_T \\
\end{split}
\end{equation}
Substituting this into (\ref{hardpart}) gives:
\begin{equation}
\begin{split}
\frac{T}{p}\left(\pdv{p}{T}\right)_V &= \frac{1}{p}\left(\pdv{U}{V}\right)_T + 1 \\
\frac{T}{p}\left(\pdv{S}{V}\right)_T &= \frac{1}{p}\left(\pdv{U}{V}\right)_T + 1 \\
\end{split}
\end{equation}

Returning to the original equation gives:
\begin{equation}
\begin{split}
\dd{U} &= T\dd{S} - p\dd{V} \Longrightarrow \\
\left(\pdv{U}{S}\right)_V &= T \\
\end{split}
\end{equation}

Consider now:
\begin{equation}
\begin{split}
 & \left(\pdv{\ln p}{\ln V}\right)_T - \left(\pdv{\ln p}{\ln V}\right)_S \\
=& \frac{V}{p}\left(\left(\pdv{p}{V}\right)_T - \left(\pdv{p}{V}\right)_S \right) \\
=& \frac{V\left(\pdv{S}{V}\right)_T}{p\left(\pdv{S}{p}\right)_V} \\
=& V\frac{\frac{T}{p}\left(\pdv{S}{V}\right)_T}{T\left(\pdv{S}{p}\right)_V} \\
=& V\frac{\left(\frac{1}{p}\left(\pdv{U}{V}\right)_T + 1\right)}{\left(\pdv{U}{S}\right)_V\left(\pdv{S}{p}\right)_V} \\
=& V\frac{\left(\frac{1}{p}\left(\pdv{U}{V}\right)_T + 1\right)}{\left(\pdv{U}{p}\right)_V} \\
=& V\frac{\left(\frac{1}{p}\left(\pdv{U}{V}\right)_T + 1\right)}{\left(\pdv{U}{T}\right)_V\left(\pdv{T}{p}\right)_V} \\
=& V\left(\pdv{p}{T}\right)_V\frac{\left(\frac{1}{p}\left(\pdv{U}{V}\right)_T + 1\right)}{\left(\pdv{U}{T}\right)_V} \\
=& \left(\pdv{(pV)}{T}\right)_V\frac{\left(\frac{1}{p}\left(\pdv{U}{V}\right)_T + 1\right)}{\left(\pdv{U}{T}\right)_V} \\
\end{split}
\end{equation}

\item 
Note that since $pV^\gamma$ depends only on $S$:

\begin{equation}
\begin{split}
\left(\pdv{\ln p}{\ln V}\right)_S = 0 \\
\left(\pdv{\ln p}{\ln V}\right)_T = \dv{\ln p}{\ln V} \\
\end{split}
\end{equation}

\begin{equation}
\begin{split}
U &= C_vT \\
\left(\pdv{U}{T}\right)_V &= C_v \\
\left(\pdv{U}{V}\right)_T &= 0 \\
pV &= RT \\
\left(\pdv{(pV)}{T}\right)_V &= R \\
\end{split}
\end{equation}

\newpage

In the below equation: $k$ is the constant of integration and $A$ is another constant which is 
equal to $k^{\frac{R}{C_v}}$.
\begin{equation}
\begin{split}
\left(\pdv{\ln p}{\ln V}\right)_T - \left(\pdv{\ln p}{\ln V}\right)_S &= \left(\pdv{(pV)}{T}\right)_V\frac{\left(\frac{1}{p}\left(\pdv{U}{V}\right)_T + 1\right)}{\left(\pdv{U}{T}\right)_V} \\
\dv{\ln p}{\ln V} - 0 &= R\frac{(0 + 1)}{C_v} \\
\dv{\ln p}{\ln V} &= \frac{R}{C_v} \\
\ln p &= \frac{R}{C_v}\ln kV \\
\ln p &= \ln k^{\frac{R}{C_v}}V^{-\frac{R}{C_v}} \\
p &= AV^{\frac{R}{C_v}} \\
pV^{-\frac{R}{C_v}} &= A \\
\end{split}
\end{equation}
\begin{equation}
\gamma = -\frac{R}{C_v}
\end{equation}

\item 
\begin{equation}
\begin{split}
\gamma &= -\frac{R}{C_v} \\
\gamma &= -\frac{R}{\frac{3}{2}R} \\
\gamma &= -\frac{2}{3} \\
\end{split}
\end{equation}

\end{enumerate}

\end{enumerate}

\end{document}