\newcommand{\svrname}{Dr John Fawcett}
\newcommand{\jkfside}{oneside}
\newcommand{\jkfhanded}{right}

\newcommand{\studentname}{Harry Langford}
\newcommand{\studentemail}{hjel2@cam.ac.uk}

\documentclass[10pt,\jkfside,a4paper]{article}

\input{./templatenst/includes.tex}
% DO NOT add \usepackage commands here.  Place any custom commands
% into your SV work files.  Anything in the template directory is
% likely to be overwritten!

\usepackage{fancyhdr}

\usepackage{lastpage}       % ``n of m'' page numbering
\usepackage{lscape}         % Makes landscape easier

\usepackage{verbatim}       % Verbatim blocks
\usepackage{epsfig}         % Embed encapsulated postscript
\usepackage{array}          % Array environment
\usepackage[nolinks]{qrcode}         % QR codes
\usepackage{enumitem}       % Required by Tom Johnson's exam question header

\usepackage{hhline}         % Horizontal lines in tables
\usepackage{siunitx}        % Correct spacing of units
\usepackage{amsmath}        % American Mathematical Society
\usepackage{amssymb}        % Maths symbols
\usepackage{amsthm}         % Theorems

\usepackage{ifthen}         % Conditional processing in tex

\usepackage[top=3cm,
            bottom=3cm,
            inner=2cm,
            outer=5cm]{geometry}

% PDF metadata + URL formatting
\usepackage[
            pdfauthor={\studentname},
            pdftitle={\svcourse, SV \svnumber},
            pdfsubject={},
            pdfkeywords={9d2547b00aba40b58fa0378774f72ee6},
            pdfproducer={},
            pdfcreator={},
            hidelinks]{hyperref}

\renewcommand{\headrulewidth}{0.4pt}
\renewcommand{\footrulewidth}{0.4pt}
\fancyheadoffset[LO,LE,RO,RE]{0pt}
\fancyfootoffset[LO,LE,RO,RE]{0pt}
\pagestyle{fancy}
\fancyhead{}
\fancyhead[LO,RE]{{\bfseries \studentname}\\\studentemail}
\fancyhead[RO,LE]{{\bfseries \svcourse, SV~\svnumber}\\\svdate\ \svtime, \svvenue}
\fancyfoot{}
\fancyfoot[LO,RE]{For: \svrname}
\fancyfoot[RO,LE]{\today\hspace{1cm}\thepage\ / \pageref{LastPage}}
\fancyfoot[C]{\qrcode[height=0.8cm]{\svuploadkey}}
\setlength{\headheight}{22.55pt}

\ifthenelse{\equal{\jkfside}{oneside}}{

 \ifthenelse{\equal{\jkfhanded}{left}}{
  % 1. Left-handed marker, one-sided printing or e-marking, use oneside and...
  \evensidemargin=\oddsidemargin
  \oddsidemargin=73pt
  \setlength{\marginparwidth}{111pt}
  \setlength{\marginparsep}{-\marginparsep}
  \addtolength{\marginparsep}{-\textwidth}
  \addtolength{\marginparsep}{-\marginparwidth}
 }{
  % 2. Right-handed marker, one-sided printing or e-marking, use oneside.
  \setlength{\marginparwidth}{111pt}
 }

}{
 % 3. Alternating margins, two-sided printing, use twoside.
}

\setlength{\parindent}{0em}
\addtolength{\parskip}{1ex}

% Exam question headings, labels and sensible layout (courtesy of Tom Johnson)
\setlist{parsep=\parskip, listparindent=\parindent}
\newcommand{\examhead}[3]{\section{#1 Paper #2 Question #3}}
\newenvironment{examquestion}[3]{
    \examhead{#1}{#2}{#3}\setlist[enumerate, 1]{label=(\alph*)}\setlist[enumerate, 2]{label=(\roman*)}
    \marginpar{\qrcode{https://www.cl.cam.ac.uk/teaching/exams/pastpapers/y#1p#2q#3.pdf}}
    \marginpar{\footnotesize \url{https://www.cl.cam.ac.uk/teaching/exams/pastpapers/y#1p#2q#3.pdf}}
}{}



\fancyhead[RO,LE]{{\bfseries NST Maths, SV~10}\\19/02/2022\ 12:00, Video Link}

\usepackage{physics}
\usepackage{tikz}

\begin{document}

\begin{enumerate}

\setcounter{enumi}{3}

\item 
\begin{equation}
\begin{split}
\pdv{f}{x} &= \pdv{x}\left(x(y^2 + 2y - 1)\right)\\
		   &= y^2 + 2y - 1\\
\pdv{f}{y} &= \pdv{y}\left(x(y^2 + 2y - 1)\right)\\
		   &= 2xy + 2x\\
\end{split}
\end{equation}
\begin{equation}
\nabla (f) = (y^2 + 2y - 1, 2xy + 2x)\\
\end{equation}

\begin{enumerate}

\item At $(-1, 0)$, $\nabla (f) = (-1, -2)$

\begin{center}
\begin{tikzpicture}
\draw[help lines, color=gray!30, dashed] (-2, -2) grid (2, 2);
\draw[->] (-2,0)--(2,0) node[right]{$x$};
\draw[->] (0,-2)--(0,2) node[above]{$y$};
\draw [-stealth] (0.5, 1) -- (-0.5, -1);
\end{tikzpicture}
\end{center}

\item At $(1, 0)$, $\nabla (f) = (-1, 2)$

\begin{center}
\begin{tikzpicture}
\draw[help lines, color=gray!30, dashed] (-2, -2) grid (2, 2);
\draw[->] (-2,0)--(2,0) node[right]{$x$};
\draw[->] (0,-2)--(0,2) node[above]{$y$};
\draw [-stealth] (0.5, -1) -- (-0.5, 1);
\end{tikzpicture}
\end{center}

\item At $(-1, 1)$, $\nabla (f) = (2, -4)$

\begin{center}
\begin{tikzpicture}
\draw[help lines, color=gray!30, dashed] (-2, -2) grid (2, 2);
\draw[->] (-2,0)--(2,0) node[right]{$x$};
\draw[->] (0,-2)--(0,2) node[above]{$y$};
\draw [-stealth] (-0.5, 1) -- (0.5, -1);
\end{tikzpicture}
\end{center}

\item At $(-1, 1)$, $\nabla (f) = (2, 4)$

\begin{center}
\begin{tikzpicture}
\draw[help lines, color=gray!30, dashed] (-2, -2) grid (2, 2);
\draw[->] (-2,0)--(2,0) node[right]{$x$};
\draw[->] (0,-2)--(0,2) node[above]{$y$};
\draw [-stealth] (-0.5, -1) -- (0.5, 1);
\end{tikzpicture}
\end{center}

\end{enumerate}

\item 
\begin{equation}
\begin{split}
T &= 2\pi\left(\frac{\ell}{g}\right)^{\frac{1}{2}}\\
\ln T &= \ln \left(2\pi\left(\frac{\ell}{g}\right)^{\frac{1}{2}}\right)\\
\ln T &= \ln 2\pi + \frac{1}{2} \ln \left(\frac{\ell}{g}\right)\\
\ln T &= \ln 2\pi + \frac{1}{2} \ln \ell - \frac{1}{2}\ln g\\
\frac{\dd T}{T} &= \frac{\dd \ell}{2\ell} - \frac{\dd g}{2g}\\
\frac{\dd g}{g} &= \frac{\dd \ell}{\ell} - \frac{2\dd T}{T}\\
\end{split}
\end{equation}

\begin{enumerate}

\item
\begin{equation}
\begin{split}
\frac{\dd g}{g} &= \frac{\dd \ell}{\ell} - \frac{2\dd T}{T}\\
\frac{\dd g}{g} &= 0.001 + 0\\
\frac{\dd g}{g} &= 0.001\\
\end{split}
\end{equation}
So a $0.1\%$ error in the measurement of $\ell$ will result in a $0.1\%$ error in $g$.

\item 
\begin{equation}
\begin{split}
\frac{\dd g}{g} &= \frac{\dd \ell}{\ell} - \frac{2\dd T}{T}\\
\frac{\dd g}{g} &= 0 + 0.002\\
\frac{\dd g}{g} &= 0.002\\
\end{split}
\end{equation}
So a $0.1\%$ error in the measurement of $T$ will result in a $0.2\%$ error in $g$.

\end{enumerate}

\item
\begin{enumerate}

Here are some formulae I will use for the following question:
\begin{equation}
\begin{split}
x &= r \cos \phi \\
\left(\pdv{x}{r}\right)_\phi &= \cos \phi \\
\left(\pdv{x}{\phi}\right)_r &= -r \sin \phi \\
\end{split}
\end{equation}

\begin{equation}
\begin{split}
y &= r \sin \phi \\
\left(\pdv{y}{r}\right)_\phi &= \sin \phi \\
\left(\pdv{y}{\phi}\right)_r &= r \cos \phi \\
\end{split}
\end{equation}

\begin{equation}
\begin{split}
r^2 &= x^2 + y^2 \\
2r\left(\pdv{r}{x}\right)_y &= 2x \\
\left(\pdv{r}{x}\right)_y &= \frac{2r \cos\phi}{2r} \\
\left(\pdv{r}{x}\right)_y &= \cos\phi \\
2r\left(\pdv{r}{y}\right)_x &= 2y \\
\left(\pdv{r}{y}\right)_x &= \frac{2r \sin\phi}{2r} \\
\left(\pdv{r}{y}\right)_x &= \sin\phi \\
\end{split}
\end{equation}

\begin{equation}
\begin{split}
\phi &= \arctan\left(\frac{x}{y}\right) \\
\left(\pdv{\phi}{x}\right)_y &= \frac{y}{x^2 + y^2} \\
\left(\pdv{\phi}{x}\right)_y &= \frac{r\sin \phi}{r^2} \\
\left(\pdv{\phi}{x}\right)_y &= \frac{\sin \phi}{r} \\
\left(\pdv{\phi}{y}\right)_x &= -\frac{x}{x^2 + y^2} \\
\left(\pdv{\phi}{y}\right)_x &= -\frac{r\cos \phi}{r^2} \\
\left(\pdv{\phi}{y}\right)_x &= -\frac{\cos \phi}{r} \\
\end{split}
\end{equation}

\item Derivation of expressions by differentiating with the chain rule:

\begin{equation}
\begin{split}
f(x, y) &= e^{-xy} \\
\left(\pdv{f}{x}\right)_y &= -ye^{-xy} \\
\end{split}
\end{equation}

\begin{equation}
\begin{split}
f(x, y) &= e^{-xy} \\
\left(\pdv{f}{y}\right)_x &= -xe^{-xy} \\
\end{split}
\end{equation}

\begin{equation}
\begin{split}
f(x, y) &= e^{-xy} \\
\left(\pdv{f}{r}\right)_\phi &= - xe^{-xy}\left(\pdv{y}{r}\right)_\phi - ye^{-xy}\left(\pdv{x}{r}\right)_\phi \\
\left(\pdv{f}{r}\right)_\phi &= - xe^{-xy}\sin \phi - ye^{-xy}\cos \phi \\
\left(\pdv{f}{r}\right)_\phi &= - r \cos \phi \sin \phi e^{-r^2 \sin \phi \cos \phi} - r \cos \phi \sin \phi e^{-r^2 \sin \phi \cos \phi} \\
\left(\pdv{f}{r}\right)_\phi &= - 2r \cos \phi \sin \phi e^{-r^2 \sin \phi \cos \phi}\\
\left(\pdv{f}{r}\right)_\phi &= - r \sin 2\phi e^{-\frac{1}{2} r^2 \sin 2\phi}\\
\end{split}
\end{equation}

\begin{equation}
\begin{split}
f(x, y) &= e^{-xy} \\
\left(\pdv{f}{\phi}\right)_r &= - xe^{-xy}\left(\pdv{y}{\phi}\right)_r - ye^{-xy}\left(\pdv{x}{\phi}\right)_r \\
\left(\pdv{f}{\phi}\right)_r &= - r \cos \phi e^{-\frac{1}{2}r^2 \sin 2 \phi } r \cos \phi + r \sin \phi e^{-\frac{1}{2}r^2 \sin 2 \phi} r \sin \phi \\
\left(\pdv{f}{\phi}\right)_r &= - r^2 \cos^2 \phi e^{-\frac{1}{2}r^2 \sin 2 \phi } + r^2 \sin^2 \phi e^{-\frac{1}{2}r^2 \sin 2 \phi}\\
\left(\pdv{f}{\phi}\right)_r &= - r^2 \cos 2 \phi e^{-\frac{1}{2}r^2 \sin 2 \phi }\\
\end{split}
\end{equation}

\item Derivation of expressions by expression in polar coordinates followed by differentiation.

\begin{equation}
\begin{split}
f(r, \phi) &= e^{-\frac{1}{2}r^2\sin 2 \phi} \\
\left(\pdv{f}{x}\right)_y &= \left(\pdv{f}{r}\right)_\phi \left(\pdv{r}{x}\right)_y - \frac{\sin\phi}{r} \left(\pdv{f}{\phi}\right)_r \left(\pdv{\phi}{x}\right)_y \\
\left(\pdv{f}{x}\right)_y &= \left(- r \sin 2 \phi e^{-\frac{1}{2}^2\sin 2 \phi}\right)\cos\phi - \left(- r^2 \cos 2 \phi e^{-\frac{1}{2}r^2 \sin 2\phi}\right)\left(\frac{\sin \phi}{r}\right) \\
\left(\pdv{f}{x}\right)_y &= \left(r \cos 2 \phi \sin \phi - r \sin 2 \phi \cos \phi\right)e^{-\frac{1}{2}^2\sin 2 \phi} \\
\left(\pdv{f}{x}\right)_y &= r\left(2 \cos^2 \phi \sin \phi - \sin \phi - 2 \sin \phi \cos^2 \phi\right)e^{-\frac{1}{2}^2\sin 2 \phi} \\
\left(\pdv{f}{x}\right)_y &= - r\sin \phi e^{-\frac{1}{2}^2\sin 2 \phi} \\
\left(\pdv{f}{x}\right)_y &= - y e^{-xy} \\
\end{split}
\end{equation}

\begin{equation}
\begin{split}
f(r, \phi) &= e^{-\frac{1}{2}r^2 \sin 2\phi} \\
\left(\pdv{f}{y}\right)_x &= \left(\pdv{f}{r}\right)_\phi \left(\pdv{r}{x}\right)_y + \left(\pdv{f}{\phi}\right)_r \left(\pdv{\phi}{x}\right)_y \\
\left(\pdv{f}{y}\right)_x &= \left(- r \sin 2 \phi e^{-\frac{1}{2}r^2 \sin 2 \phi}\right) \sin \phi + \left(- r^2 \cos 2 \phi e^{-\frac{1}{2}r^2\sin 2 \phi}\right)\left(\frac{cos\phi}{r}\right) \\
\left(\pdv{f}{y}\right)_x &= \left(- r \sin 2 \phi \sin \phi - r \cos 2 \phi \cos \phi \right)e^{-\frac{1}{2}r^2 \sin 2 \phi} \\
\left(\pdv{f}{y}\right)_x &= -r\left(\sin 2 \phi \sin \phi + \cos 2 \phi \cos \phi \right)e^{-\frac{1}{2}r^2 \sin 2 \phi} \\
\left(\pdv{f}{y}\right)_x &= -r\left(2\sin^ \phi \cos \phi + \cos \phi - 2 \sin^2 \phi \cos \phi \right)e^{-\frac{1}{2}r^2 \sin 2 \phi} \\
\left(\pdv{f}{y}\right)_x &= -r\cos \phi e^{-\frac{1}{2}r^2 \sin 2 \phi} \\
\left(\pdv{f}{y}\right)_x &= -x e^{-xy} \\
\end{split}
\end{equation}

\begin{equation}
\begin{split}
f(r, \phi) &= e^{-\frac{1}{2}r^2 \sin 2 \phi} \\
\left(\pdv{f}{r}\right)_\phi &= -r \sin 2 \phi e^{-\frac{1}{2}r^2 \sin 2 \phi} \\
\left(\pdv{f}{r}\right)_\phi &= -r \sin 2 \phi e^{-\frac{1}{2}r^2 \sin 2 \phi} \\
\end{split}
\end{equation}

\begin{equation}
\begin{split}
f(r, \phi) &= e^{-\frac{1}{2}r^2 \sin 2 \phi} \\
\left(\pdv{f}{\phi}\right)_r &= -r \sin 2 \phi e^{-\frac{1}{2}r^2 \sin 2 \phi} \\
\left(\pdv{f}{\phi}\right)_r &= -r^2 \cos 2 \phi e^{-\frac{1}{2}r^2 \sin 2 \phi} \\
\end{split}
\end{equation}

\end{enumerate}

\item
\begin{equation}
\begin{split}
xyz + x^3 + y^4 + z^5 &= 0 \\
yz\left(\pdv{x}{y}\right)_z + xz + 3x^2\left(\pdv{x}{y}\right)_z + 4y^3 &= 0 \\
(yz + 3x^2)\left(\pdv{x}{y}\right)_z &= - xz - 4y^3 \\
\left(\pdv{x}{y}\right)_z &= -\frac{xz + 4y^3}{yz + 3x^2} \\
\end{split}
\end{equation}

\begin{equation}
\begin{split}
xyz + x^3 + y^4 + z^5 &= 0 \\
xz\left(\pdv{y}{z}\right)_x + xy + 4y^3\left(\pdv{y}{z}\right)_x + 5z^4 &= 0 \\
(xz + 4y^3)\left(\pdv{y}{z}\right)_x &= -xy - 5z^4 \\
\left(\pdv{y}{z}\right)_x &= -\frac{xy + 5z^4}{xz + 4y^3} \\
\end{split}
\end{equation}

\begin{equation}
\begin{split}
xyz + x^3 + y^4 + z^5 &= 0 \\
xy\left(\pdv{z}{x}\right)_y + yz + 3x^2 + 5z^4 \left(\pdv{z}{x}\right)_y &= 0 \\
(xy + 5z^4)\left(\pdv{z}{x}\right)_y &= -yz - 3x^2 \\
\left(\pdv{z}{x}\right)_y &= -\frac{yz + 3x^2}{xy + 5z^4} \\
\end{split}
\end{equation}

\begin{equation}
\begin{split}
 & \left(\pdv{x}{y}\right)_z \times \left(\pdv{y}{z}\right)_x \times \left(\pdv{z}{x}\right)_y \\
=& -\frac{xz + 4y^3}{yz + 3x^2} \times -\frac{xy + 5z^4}{xz + 4y^3} \times -\frac{yz + 3x^2}{xy + 5z^4} \\
=& -\frac{(xz + 4y^3)(xy + 5z^4)(yz + 3x^2)}{(yz + 3x^2)(xz + 4y^3)(xy + 5z^4)} \\
=& -\frac{(xz + 4y^3)(xy + 5z^4)(yz + 3x^2)}{(xz + 4y^3)(xy + 5z^4)(yz + 3x^2)} \\
=& -1 \\
\end{split}
\end{equation}

\item

\begin{equation}
\begin{split}
\left(p + \frac{a}{V^2}\right)(V - b) &= RT \\
p &= \frac{RT}{V - b} - \frac{a}{V^2}\\
\left(\pdv{p}{V}\right)_T &= -\frac{RT}{(V - b)^2} + \frac{2a}{V^3} \\
\left(\pdv{p}{V}\right)_T &= -\frac{RTV^3}{V^3(V - b)^2} + \frac{2a(V - b)^2}{V^3(V - b)^2} \\
\left(\pdv{p}{V}\right)_T &= \frac{- RTV^3 + 2a(V - b)^2}{V^3(V - b)^2} \\
\end{split}
\end{equation}

\begin{equation}
\begin{split}
\left(p + \frac{a}{V^2}\right)(V - b) &= RT \\
\left(\left(p + \frac{a}{V^2}\right)-\frac{2a}{V^3}(V - b)\right)\left(\pdv{V}{T}\right)_p &= R \\
\left(\frac{RT}{V - b} - \frac{2a}{V^2} + \frac{2ab}{V^3} \right)\left(\pdv{V}{T}\right)_p &= R \\
(RTV^3 - 2aV(V - b) + 2ab(V - b) )\left(\pdv{V}{T}\right)_p &= RV^3(V - b) \\
(RTV^3 - 2a(V - b)^2)\left(\pdv{V}{T}\right)_p &= RV^3(V - b) \\
\left(\pdv{V}{T}\right)_p &= \frac{RV^3(V - b)}{RTV^3 - 2a(V - b)^2} \\
\end{split}
\end{equation}

\begin{equation}
\begin{split}
RT &= \left(p + \frac{a}{V^2}\right)(V - b) \\
R\left(\pdv{T}{p}\right)_V &= (V - b) \\
\left(\pdv{T}{p}\right)_V &= \frac{(V - b)}{R} \\
\end{split}
\end{equation}

\begin{equation}
\begin{split}
 & \left(\pdv{p}{V}\right)_T \cdot \left(\pdv{V}{T}\right)_p \cdot \left(\pdv{T}{p}\right)_V \\
=& \frac{- RTV^3 + 2a(V - b)^2}{V^3(V - b)^2} \cdot \frac{RV^3(V - b)}{RTV^3 - 2a(V - b)^2} \times \frac{(V - b)}{R} \\
=& \frac{RV^3(V - b)^2(-RTV^3 + 2a(V - b)^2)}{RV^3(V - b)^2(RTV^3 - 2a(V - b)^2)} \\
=& -1 \\
\end{split}
\end{equation}

\item If $u$ and $v$ are rotated axis of $x$ and $y$ then for some constant angle $\theta$:
\begin{equation}
\begin{split}
\text{Let } x &= u\cos\theta + v\sin\theta \\
\left(\pdv{x}{u}\right)_v &= \cos\theta \\
\left(\pdv{x}{v}\right)_u &= \sin\theta \\
\text{Let } y &= u\sin\theta - v\cos\theta \\
\left(\pdv{y}{u}\right)_v &= \sin\theta \\
\left(\pdv{y}{v}\right)_u &= -\cos\theta \\
\end{split}
\end{equation}
\begin{equation}
\begin{split}
\left(\pdv{f}{u}\right)_v &= \left(\pdv{f}{x}\right)_y\left(\pdv{x}{u}\right)_v + \left(\pdv{f}{y}\right)_x\left(\pdv{y}{u}\right)_v \\
\left(\pdv[2]{f}{u}\right)_v &= \left(\pdv[2]{f}{x}\right)_y\left(\pdv{x}{u}\right)^2_v + \left(\pdv{f}{x}\right)_y\left(\pdv[2]{x}{u}\right)_v + \left(\pdv[2]{f}{y}\right)_x\left(\pdv{y}{u}\right)^2_v + \left(\pdv{f}{y}\right)_x\left(\pdv[2]{y}{u}\right)_v\\
\left(\pdv[2]{f}{u}\right)_v &= \left(\pdv[2]{f}{x}\right)_y\cos^2\theta + 0\left(\pdv{f}{x}\right)_y + \left(\pdv[2]{f}{y}\right)_x\sin^2\theta + 0\left(\pdv{f}{y}\right)_x\\
\left(\pdv[2]{f}{u}\right)_v &= \left(\pdv[2]{f}{x}\right)_y\cos^2\theta + \left(\pdv[2]{f}{y}\right)_x\sin^2\theta \\
\end{split}
\end{equation}
\begin{equation}
\begin{split}
\left(\pdv{f}{v}\right)_u &= \left(\pdv{f}{x}\right)_y\left(\pdv{x}{v}\right)_u + \left(\pdv{f}{y}\right)_x\left(\pdv{y}{v}\right)_u \\
\left(\pdv[2]{f}{v}\right)_u &= \left(\pdv[2]{f}{x}\right)_y\left(\pdv{x}{v}\right)^2_u + \left(\pdv{f}{x}\right)_y\left(\pdv[2]{x}{v}\right)_u + \left(\pdv[2]{f}{y}\right)_x\left(\pdv{y}{v}\right)^2_u + \left(\pdv{f}{y}\right)_x\left(\pdv[2]{y}{v}\right)_u\\
\left(\pdv[2]{f}{v}\right)_u &= \left(\pdv[2]{f}{x}\right)_y\sin^2\theta + 0\left(\pdv{f}{x}\right)_y + \left(\pdv[2]{f}{y}\right)_x(-\cos\theta)^2 + 0\left(\pdv{f}{y}\right)_x\\
\left(\pdv[2]{f}{v}\right)_u &= \left(\pdv[2]{f}{x}\right)_y\sin^2\theta + \left(\pdv[2]{f}{y}\right)_x\cos^2\theta \\
\end{split}
\end{equation}
\begin{equation}
\begin{split}
\left(\pdv[2]{f}{u}\right)_v + \left(\pdv[2]{f}{v}\right)_u &= \left(\pdv[2]{f}{x}\right)_y\cos^2\theta + \left(\pdv[2]{f}{y}\right)_x\sin^2\theta + \left(\pdv[2]{f}{x}\right)_y\sin^2\theta + \left(\pdv[2]{f}{y}\right)_x\cos^2\theta \\
\left(\pdv[2]{f}{u}\right)_v + \left(\pdv[2]{f}{v}\right)_u &= \left(\pdv[2]{f}{x}\right)_y\left(\sin^2\theta + \cos^2\theta\right) + \left(\pdv[2]{f}{y}\right)_x\left(\sin^2\theta + \cos^2\theta\right) \\
\left(\pdv[2]{f}{u}\right)_v + \left(\pdv[2]{f}{v}\right)_u &= \left(\pdv[2]{f}{x}\right)_y + \left(\pdv[2]{f}{y}\right)_x \text{ as required}\\
\end{split}
\end{equation}

\end{enumerate}

\end{document}