\newcommand{\svrname}{Dr John Fawcett}
\newcommand{\jkfside}{oneside}
\newcommand{\jkfhanded}{right}

\newcommand{\studentname}{Harry Langford}
\newcommand{\studentemail}{hjel2@cam.ac.uk}

\documentclass[10pt,\jkfside,a4paper]{article}

\input{./template/includes.tex}
% DO NOT add \usepackage commands here.  Place any custom commands
% into your SV work files.  Anything in the template directory is
% likely to be overwritten!

\usepackage{fancyhdr}

\usepackage{lastpage}       % ``n of m'' page numbering
\usepackage{lscape}         % Makes landscape easier

\usepackage{verbatim}       % Verbatim blocks
\usepackage{epsfig}         % Embed encapsulated postscript
\usepackage{array}          % Array environment
\usepackage[nolinks]{qrcode}         % QR codes
\usepackage{enumitem}       % Required by Tom Johnson's exam question header

\usepackage{hhline}         % Horizontal lines in tables
\usepackage{siunitx}        % Correct spacing of units
\usepackage{amsmath}        % American Mathematical Society
\usepackage{amssymb}        % Maths symbols
\usepackage{amsthm}         % Theorems

\usepackage{ifthen}         % Conditional processing in tex

\usepackage[top=3cm,
            bottom=3cm,
            inner=2cm,
            outer=5cm]{geometry}

% PDF metadata + URL formatting
\usepackage[
            pdfauthor={\studentname},
            pdftitle={\svcourse, SV \svnumber},
            pdfsubject={},
            pdfkeywords={9d2547b00aba40b58fa0378774f72ee6},
            pdfproducer={},
            pdfcreator={},
            hidelinks]{hyperref}

\renewcommand{\headrulewidth}{0.4pt}
\renewcommand{\footrulewidth}{0.4pt}
\fancyheadoffset[LO,LE,RO,RE]{0pt}
\fancyfootoffset[LO,LE,RO,RE]{0pt}
\pagestyle{fancy}
\fancyhead{}
\fancyhead[LO,RE]{{\bfseries \studentname}\\\studentemail}
\fancyhead[RO,LE]{{\bfseries \svcourse, SV~\svnumber}\\\svdate\ \svtime, \svvenue}
\fancyfoot{}
\fancyfoot[LO,RE]{For: \svrname}
\fancyfoot[RO,LE]{\today\hspace{1cm}\thepage\ / \pageref{LastPage}}
\fancyfoot[C]{\qrcode[height=0.8cm]{\svuploadkey}}
\setlength{\headheight}{22.55pt}

\ifthenelse{\equal{\jkfside}{oneside}}{

 \ifthenelse{\equal{\jkfhanded}{left}}{
  % 1. Left-handed marker, one-sided printing or e-marking, use oneside and...
  \evensidemargin=\oddsidemargin
  \oddsidemargin=73pt
  \setlength{\marginparwidth}{111pt}
  \setlength{\marginparsep}{-\marginparsep}
  \addtolength{\marginparsep}{-\textwidth}
  \addtolength{\marginparsep}{-\marginparwidth}
 }{
  % 2. Right-handed marker, one-sided printing or e-marking, use oneside.
  \setlength{\marginparwidth}{111pt}
 }

}{
 % 3. Alternating margins, two-sided printing, use twoside.
}

\setlength{\parindent}{0em}
\addtolength{\parskip}{1ex}

% Exam question headings, labels and sensible layout (courtesy of Tom Johnson)
\setlist{parsep=\parskip, listparindent=\parindent}
\newcommand{\examhead}[3]{\section{#1 Paper #2 Question #3}}
\newenvironment{examquestion}[3]{
    \examhead{#1}{#2}{#3}\setlist[enumerate, 1]{label=(\alph*)}\setlist[enumerate, 2]{label=(\roman*)}
    \marginpar{\qrcode{https://www.cl.cam.ac.uk/teaching/exams/pastpapers/y#1p#2q#3.pdf}}
    \marginpar{\footnotesize \url{https://www.cl.cam.ac.uk/teaching/exams/pastpapers/y#1p#2q#3.pdf}}
}{}



\usepackage{physics}

\begin{document}

Taking the \textbf{differential} (not the derivative) gives:
\begin{equation}
\dd{\ln y} = \frac{\dd{y}}{y} \\
\end{equation}
This is the differential of $\ln y$ -- not with respect to anything but just taking the differential.

To do S5 we will take differentials of this.
\begin{equation}
\begin{split}
f(x, y, z) &= xyz + x^3 + y^4 + z^5 \\
\dd{f} &= \dd{xyz} + \dd{x^3} + \dd{y^4} + \dd{z^5} \\
\dd{f} &= yz\dd{x} + xz\dd{y} + xy\dd{z} + 3x^2\dd{x} + 4y^3\dd{y} + 5z^4\dd{z} \\
\dd{f} &= (yz + 3x^2)\dd{x} + (xz + 4y^3)\dd{y} + (xy + 5z^4)\dd{z} \\
\end{split}
\end{equation}

The partial derivative with some constant kept constant is equal to the normal derivative with the 
other variable kept constant. It is the exact same. Dividing the differential can give easier methods 
for finding this:

\begin{equation}
\begin{split}
\left(\pdv{x}{y}\right)_y &\equiv \left(\dv{x}{y}\right)_z \\
0 &= (yz + 3x^2)\dv{x}{y} + (xz + 4y^3) + (xy + 5z^4)\left(\dv{z}{y}\right)_z \\
0 &= (yz + 3x^2)\dv{x}{y} + (xz + 4y^3) \\
\left(\dv{x}{y}\right)_z &= -\frac{xz + 4y^3}{yz + 3x^2} \\
\left(\pdv{x}{y}\right)_z &= -\frac{xz + 4y^3}{yz + 3x^2} \\
\end{split}
\end{equation}
Note that $\dd{z}{y}$ with $z$ held constant is 0.

Consider a function $f = f(x, y, z) = 0$.

\begin{equation}
\begin{split}
\dd{f} &= \left(\pdv{f}{x}\right)_{y,z}\dd{x} + \left(\pdv{f}{y}\right)_{x,z}\dd{y} + \left(\pdv{f}{z}\right)_{x,y}\dd{z} \\
0 &= \left(\pdv{f}{x}\right)_{y,z}\dd{x} + \left(\pdv{f}{y}\right)_{x,z}\dd{y} + \left(\pdv{f}{z}\right)_{x,y}\dd{z} \\
0 &= \left(\pdv{f}{x}\right)_{y,z}\left(\pdv{x}{y}\right)_z + \left(\pdv{f}{y}\right)_{x,z}\left(\dd{y}{y}\right)_{z} + \left(\pdv{f}{z}\right)_{x,y}\left(\pdv{z}{y}\right)_z \\
0 &= \left(\pdv{f}{x}\right)_{y,z}\left(\pdv{x}{y}\right)_z + \left(\pdv{f}{y}\right)_{x,z} \times 1 + \left(\pdv{f}{z}\right)_{x,y}\times 0 \\
-\left(\pdv{f}{y}\right)_{x,z} &= \left(\pdv{f}{x}\right)_{y,z}\left(\pdv{x}{y}\right)_z \\
\left(\pdv{x}{y}\right)_z &= -\frac{\left(\pdv{f}{y}\right)_{x,z}}{\left(\pdv{f}{x}\right)_{y,z}} \\
\end{split}
\end{equation}
For the other two functions:
\begin{equation}
\begin{split}
\left(\pdv{y}{z}\right)_x &= -\frac{\left(\pdv{f}{z}\right)_{x,y}}{\left(\pdv{f}{y}\right)_{x,z}} \\
\left(\pdv{z}{x}\right)_y &= -\frac{\left(\pdv{f}{x}\right)_{y,z}}{\left(\pdv{f}{z}\right)_{x,y}} \\
\end{split}
\end{equation}

Repeating this for other functions gives similar results. All the denominators and numerators of the functions 
cancel out. This gives:
\begin{equation}
\left(\pdv{x}{y}\right)_x\cdot\left(\pdv{y}{z}\right)_x\cdot\left(\pdv{z}{x}\right)_y = -1 \\
\end{equation}

This generalises to functions with higher numbers of variables: if you multiply the partial derivatives of a 
$n$ variable function together then it is equal to $(-1)^n$.\\
IE for any arbitrary function $f$:
\begin{equation}
f = f(x_1, x_2, x_3, x_4, x_5) \\
\end{equation}
\begin{equation}
\begin{split}
\left(\pdv{x_1}{x_2}\right)_{x_3, x_4, x_5} \times \left(\pdv{x_2}{x_3}\right)_{x_1, x_4, x_5} \times \left(\pdv{x_3}{x_4}\right)_{x_1, x_2, x_5} \times \left(\pdv{x_4}{x_5}\right)_{x_1, x_2, x_3} \times \left(\pdv{x_5}{x_1}\right)_{x_2, x_3, x_4} &= (-1)^5 \\
\left(\pdv{x_1}{x_2}\right)_{x_3, x_4, x_5} \times \left(\pdv{x_2}{x_3}\right)_{x_1, x_4, x_5} \times \left(\pdv{x_3}{x_4}\right)_{x_1, x_2, x_5} \times \left(\pdv{x_4}{x_5}\right)_{x_1, x_2, x_3} \times \left(\pdv{x_5}{x_1}\right)_{x_2, x_3, x_4} &= -1 \\
\end{split}
\end{equation}

Also note that when you partially derive a $n$ variable function then you need to keep $n - 2$ variables constant. 

Ie if $f = f(x_1\dots x_n)$

Then the partial derivative of $x_1$ with respect to $x_2$ is:
\begin{equation}
\left(\pdv{x_1}{x_2}\right)_{x_3\dots x_n}
\end{equation}

Once you have the first derivative of a function then you can use the operator method to find the second 
derivative very quickly as shown below.

\begin{equation}
\begin{split}
x &= u\cos\theta - v\sin\theta \\
y &= u\sin\theta + v\cos\theta \\
\end{split}
\end{equation}
\begin{equation}
\begin{split}
\pdv{f}{u} = \left(\pdv{u}\right) f &= \pdv{f}{x}\pdv{x}{u} + \pdv{f}{y}\pdv{y}{u} \\
\pdv{f}{u} = \left(\pdv{u}\right) f &= \pdv{f}{x}\cos\theta + \pdv{f}{y}\sin\theta \\
\pdv{f}{v} \left(\pdv{v}\right) f &= \pdv{f}{x}\pdv{u}{v} + \pdv{f}{y}\pdv{y}{u} \\
\pdv{f}{v} = \left(\pdv{v}\right) f &= -\pdv{f}{x}\sin\theta + \pdv{f}{y}\cos\theta \\
\end{split}
\end{equation}

So we know what the differential operator with respect to $u$ does to a function. This means 
we can re-apply this to $\pdv{f}{u}$ immediately giving us a function for $\pdv[2]{f}{u}$ 
without further calculation.

\begin{equation}
\begin{split}
\pdv[2]{f}{u} = \left(\pdv{u}\right)\pdv{f}{u} &= \pdv[2]{f}{x}\cos^2\theta + \pdv[2]{f}{y}\sin^2\theta + 2\pdv[2]{f}{x}{y}\sin\theta\cos\theta \\
\pdv[2]{f}{v} = \left(\pdv{v}\right)\pdv{f}{v} &= \pdv[2]{f}{x}\sin^2\theta + \pdv[2]{f}{y}\cos^2\theta - 2\pdv[2]{f}{x}{y}\sin\theta\cos\theta \\
\end{split}
\end{equation}

\begin{equation}
\begin{split}
\pdv[2]{f}{u} + \pdv[2]{f}{v} &= (\sin^2\theta + \cos^2\theta)\pdv[2]{f}{x} + (\sin^2\theta + \cos^2\theta)\pdv[2]{f}{y} \\
\pdv[2]{f}{u} + \pdv[2]{f}{v} &= \pdv[2]{f}{x} + \pdv[2]{f}{y} \\
\end{split}
\end{equation}

\end{document}