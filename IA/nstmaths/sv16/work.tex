\documentclass[10pt,oneside,a4paper]{article}
\usepackage{amsfonts}
\usepackage{mathtools}
\usepackage{physics}

% DO NOT add \usepackage commands here.  Place any custom commands
% into your SV work files.  Anything in the template directory is
% likely to be overwritten!

\usepackage{fancyhdr}

\usepackage{lastpage}       % ``n of m'' page numbering
\usepackage{lscape}         % Makes landscape easier

\usepackage{verbatim}       % Verbatim blocks
\usepackage{listings}       % Source code listings
\usepackage{graphicx}
\usepackage{float}
\usepackage{epsfig}         % Embed encapsulated postscript
\usepackage{array}          % Array environment
\usepackage{qrcode}         % QR codes
\usepackage{enumitem}       % Required by Tom Johnson's exam question header

\usepackage{hhline}         % Horizontal lines in tables
\usepackage{siunitx}        % Correct spacing of units
\usepackage{amsmath}        % American Mathematical Society
\usepackage{amssymb}        % Maths symbols
\usepackage{amsthm}         % Theorems

\usepackage{ifthen}         % Conditional processing in tex

\usepackage[top=3cm,
            bottom=3cm,
            inner=2cm,
            outer=5cm]{geometry}

% PDF metadata + URL formatting
\usepackage[
            pdfauthor={\studentname},
            pdftitle={\svcourse, SV \svnumber},
            pdfsubject={},
            pdfkeywords={9d2547b00aba40b58fa0378774f72ee6},
            pdfproducer={},
            pdfcreator={},
            hidelinks]{hyperref}

\renewcommand{\headrulewidth}{0.4pt}
\renewcommand{\footrulewidth}{0.4pt}
\fancyheadoffset[LO,LE,RO,RE]{0pt}
\fancyfootoffset[LO,LE,RO,RE]{0pt}
\pagestyle{fancy}
\fancyhead{}
\fancyhead[LO,RE]{{\bfseries \studentname}\\\studentemail}
\fancyhead[RO,LE]{{\bfseries \svcourse, SV~\svnumber}\\\svdate\ \svtime, \svvenue}
\fancyfoot{}
\fancyfoot[LO,RE]{For: \svrname}
\fancyfoot[RO,LE]{\today\hspace{1cm}\thepage\ / \pageref{LastPage}}
\fancyfoot[C]{\qrcode[height=0.8cm]{\svuploadkey}}
\setlength{\headheight}{22.55pt}


\ifthenelse{\equal{\jkfside}{oneside}}{

 \ifthenelse{\equal{\jkfhanded}{left}}{
  % 1. Left-handed marker, one-sided printing or e-marking, use oneside and...
  \evensidemargin=\oddsidemargin
  \oddsidemargin=73pt
  \setlength{\marginparwidth}{111pt}
  \setlength{\marginparsep}{-\marginparsep}
  \addtolength{\marginparsep}{-\textwidth}
  \addtolength{\marginparsep}{-\marginparwidth}
 }{
  % 2. Right-handed marker, one-sided printing or e-marking, use oneside.
  \setlength{\marginparwidth}{111pt}
 }

}{
 % 3. Alternating margins, two-sided printing, use twoside.
}


\setlength{\parindent}{0em}
\addtolength{\parskip}{1ex}

% Exam question headings, labels and sensible layout (courtesy of Tom Johnson)
\setlist{parsep=\parskip, listparindent=\parindent}
\newcommand{\examhead}[3]{\section{#1 Paper #2 Question #3}}
\newenvironment{examquestion}[3]{
\examhead{#1}{#2}{#3}\setlist[enumerate, 1]{label=(\alph*)}\setlist[enumerate, 2]{label=(\roman*)}
\marginpar{\href{https://www.cl.cam.ac.uk/teaching/exams/pastpapers/y#1p#2q#3.pdf}{\qrcode{https://www.cl.cam.ac.uk/teaching/exams/pastpapers/y#1p#2q#3.pdf}}}
\marginpar{\footnotesize \href{https://www.cl.cam.ac.uk/teaching/exams/pastpapers/y#1p#2q#3.pdf}{https://www.cl.cam.ac.uk/\\teaching/exams/pastpapers/\\y#1p#2q#3.pdf}}
}{}


\fancyhead[RO,LE]{{\bfseries NST Maths, SV~16}\\ 21-05-2022 12:00, Video Link}

\begin{document}

\begin{enumerate}

\setcounter{enumi}{20}

\item
By the definition of determinant:
\[
\begin{split}
\forall \mathbf{M}, j. |\mathbf{M}| &= \sum_{i=1} \mathbf{M}_{j, i} C_{j, i} \Longrightarrow \\
|\mathbf{A}_i| &= \sum_{j = 1} (\mathbf{A}_i)_{j, i} C_{j, i} \\
&= \sum_{j = 1} \mathbf{y}_i C_{j, i}
\end{split}
\]

\[
\begin{split}
\mathbf{x}_i 
&= \sum_{j = 1} \left( \mathbf{A}^{-1} \right)_{ij}\mathbf{y}_j \\
&= \sum_{j = 1} \frac{C_{j, i}}{|\mathbf{A}|} \mathbf{y}_j \\
&= \sum_{j = 1} \frac{\mathbf{y}_j C_{j, i}}{|\mathbf{A}|} \\
&= \frac{|\mathbf{A}_i|}{|\mathbf{A}|} \\
\end{split}
\]

There are two interpretations of the last question, I will describe both:

\begin{itemize}

\item ``Comment on the cases where there is no inverse and where the inverse is not unique.''

There is no inverse if and only if $\det \mathbf{A} = 0$.

If there is an inverse, then it is unique.

\item ``Comment on the cases where there is no solution to the equation and where the solution to the equation is not
 unique.

There is no solution to the equation if and only if the determinant of $\mathbf{A}$ is zero and $\mathbf{y} $
contains values which cause the equations to become inconsistent.

$\mathbf{A}\mathbf{x} = \mathbf{y} $ has multiple solutions if and only if the determinant of $\mathbf{A}$ is zero and
$\mathbf{y} $ contains values which do not cause equations to become inconsistent.

\end{itemize}

\item 

\[
\begin{split}
\begin{pmatrix}
1 & 1 & a \\
3 & 4 & 2 + 3a \\
-1 & 1 & 1 \\
\end{pmatrix}
\begin{pmatrix}
x \\
y \\
z \\
\end{pmatrix}
&= 
\begin{pmatrix}
1 \\
5 \\
b \\
\end{pmatrix} \\
\end{split}
\]

\[
\begin{split}
|\mathbf{A}| &= 1\begin{vmatrix} 4 & 2 + 3a \\ 1 & 1 \\ \end{vmatrix} - 1 \begin{vmatrix} 3 & 2 + 3a \\ -1 & 1 \\ \end{vmatrix} + a\begin{vmatrix} 3 & 4 \\ -1 & 1 \\ \end{vmatrix} \\
             &= (2 - 3a) - (5 + 3a) + 7a \\
             &= a - 3 \\
\end{split}
\]

Unique solutions do not exist for $a = 3$.

\[
\begin{split}
x + y + 3z &= 1 \\
x - x + y + y + 3z + z &= b + 1 \\
2y + 4z &= b + 1 \\
\end{split}
\]
\[
\begin{split}
3x + 4y + 11z &= 5 \\
3x - 3x + 4y - 3y + 11z - 9z &= 5 - 3 \\
y + 2z &= 2 \\
2y + 4z &= 4 \\
\end{split}
\]
So for the equation to have any solutions;
\[
b + 1 = 4 \Longrightarrow b = 3
\]

So there are many solutions for $b = 3$ and there are no solutions for $b \neq 3$.

\item 

\[
\mathbf{A} =
\begin{pmatrix}
1 & 4 & -6 \\
0 & 4 & 0 \\
0 & 0 & 9 \\
\end{pmatrix}
\]

\item

\[
\mathbf{R} =
\begin{pmatrix}
0 & -1 & 0 \\
1 & 0 &  0 \\
0 & 0 & 1 \\
\end{pmatrix}
\]

\item

The eigenvalues of a reflection matrix are the coefficients of the unit normal to the plane in which the reflection
matrix reflects.

Two of the eigenvectors lie in the plane and the other is the normal to the plane. They form a basis where the plane
lies in two axis and the third eigenvector is the distance from the plane.
They are orthogonal.

\item

\begin{enumerate}[label=(\roman*)]

\item

\[
\begin{split}
\begin{vmatrix}
5 - \lambda & 1 & 1 \\
1 & 3 - \lambda & 1 \\
1 & 1 & 3 - \lambda \\
\end{vmatrix}
&= \mathbf{0} \\
(5 - \lambda)((3 - \lambda)^2 - 1) - (3 - \lambda - 1) + (1 - (3 - \lambda)) &= 0 \\
(5 - \lambda)(\lambda^2 - 6\lambda + 8) + \lambda - 2 + \lambda - 2 &= 0 \\
\lambda^3 - 11\lambda^2 + 36\lambda - 36\lambda &= 0 \\
(\lambda - 2)(\lambda - 3)(\lambda - 6) &= 0 \\
\end{split}
\]

The eigenvalues are $2, 3, 6$. So all the eigenvalues are all real.

\item

The sum of the eigenvalues is $2 + 3 + 6 = 11$. $Tr(\mathbf{M}) = 5 + 3 + 3 = 11$. So the sum of the eigenvalues is
equal to the trace of the matrix $\mathbf{M}$.

\item

\[
\begin{split}
|\mathbf{M}| &=
\begin{vmatrix}
5 & 1 & 1 \\
1 & 3 & 1 \\
1 & 1 & 3 \\
\end{vmatrix} \\
&=
5 (9 - 1) - 1(3 - 1) + 1(1 - 3) \\
&= 40 - 2 - 2 \\
&= 36 \\
\end{split}
\]

The product of the eigenvalues is $2 \times 3 \times 6 = 36$. So the product of the eigenvalues is equal to the
determinant of the matrix.

\item
The eigenvector for $\lambda = 2$:
\[
\begin{split}
(\mathbf{M} - 2\mathbf{I})\mathbf{e}_1 &= \mathbf{0} \\
\begin{pmatrix}
3 & 1 & 1 \\
1 & 1 & 1 \\
1 & 1 & 1 \\
\end{pmatrix}
\mathbf{e}_1 &= \mathbf{0} \\
\end{split}
\]
\begin{align*}
3\mathbf{v}_1 + \mathbf{v}_2 + \mathbf{v}_3 &= 0 & \mathbf{v}_1 + \mathbf{v}_2 + \mathbf{v}_3 &= 0 \\
2\mathbf{v}_1 &= 0 & 2\mathbf{v}_2 + 2\mathbf{v}_3 &= 0 \\
\mathbf{v}_1 &= 0 & \mathbf{v}_2 &= -\mathbf{v}_3 \\
\end{align*}
Setting $\mathbf{v}_2 = \frac{1}{\sqrt {2}}$ gives the normalised eigenvector $\mathbf{e}_1$.
\[
\mathbf{e}_1 =
\frac{1}{\sqrt {2}}\begin{pmatrix}
0 \\ 1 \\ -1 \\
\end{pmatrix}
\]

The eigenvector for $\lambda = 3$:
\[
\begin{split}
(\mathbf{M} - 3\mathbf{I})\mathbf{e}_2 &= \mathbf{0} \\
\begin{pmatrix}
2 & 1 & 1 \\
1 & 0 & 1 \\
1 & 1 & 0 \\
\end{pmatrix}
\mathbf{e}_2 &= \mathbf{0} \\
\end{split}
\]
\begin{align*}
\mathbf{v}_1 &= -\mathbf{v}_3 & \mathbf{v}_1 &= -\mathbf{v}_2 \\
\end{align*}
Setting $\mathbf{v}_1 = \frac{1}{\sqrt {3}}$ gives the normalised eigenvector:
\[
\mathbf{e}_2 =
\frac{1}{\sqrt {3}}
\begin{pmatrix}
1 \\ -1 \\ -1 \\
\end{pmatrix}
\]

The eigenvector for $\lambda = 6$:
\[
\begin{split}
(\mathbf{M} - 6\mathbf{I})\mathbf{e}_3 &= \mathbf{0} \\
\begin{pmatrix}
-1 & 1 & 1 \\
1 & -3 & 1 \\
1 & 1 & -3 \\
\end{pmatrix}
\mathbf{e}_3 &= \mathbf{0} \\
\end{split}
\]
\begin{align*}
-v_1 + v_2 + v_3 &= 0 & v_1 -3v_2 + v_3 &= 0 & v_1 + v_2 - 3v_3 &= 0 \\
-2v_2 + 2v_3 &= 0 & 4v_1 - 8v_3 &= 0 \\
v_2 &= v_3 & v_1 &= 2v_3 \\
\end{align*}
Setting $v_3 = \frac{1}{\sqrt {6}}$ gives the normalised eigenvector:
\[
\mathbf{e}_3 =
\frac{1}{\sqrt {6}}
\begin{pmatrix}
2 \\ 1 \\ 1
\end{pmatrix}
\]

\[
\mathbf{e}_1 \cdot \mathbf{e}_2 = \mathbf{e}_1 \cdot \mathbf{e}_3 = \mathbf{e}_2 \cdot \mathbf{e}_3 = 0
\]
So the eigenvectors are orthogonal.

\end{enumerate}

\item

\[
\begin{split}
|\mathbf{A} - \lambda\mathbf{I}| &= 0 \\
\begin{vmatrix}
4 - \lambda & -2 & 0 \\
-2 & 3 - \lambda & -2 \\
0 & -2 & 2 - \lambda \\
\end{vmatrix} &= 0 \\
(4 - \lambda)((3 - \lambda)(2 - \lambda) - 4) - 4(2 - \lambda) &= 0 \\
(4 - \lambda)(\lambda^2 - 5 \lambda + 6 - 4) - 8 + 4\lambda &= 0 \\
(4 - \lambda)(\lambda^2 - 5\lambda + 2) - 8 + 4\lambda &= 0 \\
\lambda^3 - 5\lambda^2 + 2\lambda - 4\lambda^2 + 20\lambda - 8 + 8 - 4\lambda &= 0 \\
\lambda^3 - 9\lambda^2 + 18\lambda &= 0 \\
\lambda(\lambda^2 - 9\lambda + 18) &= 0 \\
\lambda(\lambda - 3)(\lambda - 6) &= 0 \\
\end{split}
\]

For $\lambda = 0$:
\begin{gather*}
\begin{pmatrix}
4 & -2 & 0 \\
-2 & 3 & -2 \\
0 & -2 & 2 \\
\end{pmatrix}
\mathbf{v}
=
\mathbf{0} \\
2v_1 = v_2 \wedge v_2 = v_3 \\
\end{gather*}
Setting $v_1 = \frac{1}{3}$ gives the normalised eigenvector:
\[
\mathbf{e}_1 = \frac{1}{3}\begin{pmatrix}
1 \\ 2 \\ 2 \\
\end{pmatrix}
\]

For $\lambda = 3$:
\begin{gather*}
\begin{pmatrix}
1 & -2 & 0 \\
-2 & 0 & -2 \\
0 & -2 & -1 \\
\end{pmatrix}
\mathbf{v} = \mathbf{0} \\
v_1 = 2v_2 \wedge v_1 = -v_3 \\
\end{gather*}
Setting $v_2 = \frac{1}{3} $ gives the normalised eigenvector:
\[
\mathbf{e}_2 = \frac{1}{3}\begin{pmatrix}
2 \\ 1 \\ -2 \\
\end{pmatrix}
\]

For $\lambda = 6$:
\begin{gather*}
\begin{pmatrix}
-2 & -2 & 0 \\
-2 & -3 & -2 \\
0 & -2 & -4 \\
\end{pmatrix}
\mathbf{v} = \mathbf{0} \\
v_1 = -v_2 \wedge v_2 = -2v_3 \\
\end{gather*}
Setting $v_3 =\frac{1}{3} $ gives the normalised eigenvector:
\[
\mathbf{e}_3 =\frac{1}{3} \begin{pmatrix}
2 \\ -2 \\ 1 \\
\end{pmatrix}
\]

\[
\begin{split}
\mathbf{y} &= \mathbf{y}\cdot \mathbf{e}_1 \mathbf{e}_1 + \mathbf{y}\cdot \mathbf{e}_2 \mathbf{e}_2 +
\mathbf{y}\cdot \mathbf{e}_3 \mathbf{e}_3 \\
&= 0\mathbf{e}_1 + 3\mathbf{e}_2 + 6\mathbf{e}_3 \\
\end{split}
\]
So the coordinates of $y$ in the basis $(\mathbf{e}_1, \mathbf{e}_2, \mathbf{e}_3)$ are $(0, 3, 6)$.

\[
\begin{split}
\mathbf{Ax} &= \mathbf{y} \\
\mathbf{Ax} &= 3\mathbf{e}_2 + 6\mathbf{e}_3 \\
\mathbf{Ax} &= \mathbf{A}(\mathbf{e_2} + \mathbf{e}_3) \\
\end{split}
\]
So $\mathbf{x} = \mathbf{e}_2 + \mathbf{e}_3 $ is a solution to the equation.

This is not unique since $\forall k. \mathbf{A}k\mathbf{e}_1 = 0 \times k \times \mathbf{e}_1 = \mathbf{0}$.
So $\forall k. \mathbf{x} = k\mathbf{e}_1 + \mathbf{e}_2 + \mathbf{e}_3$ is a solution to the equation.

\item

\[
\begin{split}
\left( \mathbf{Ax}  \right)_{ij} &= \sum^{}_{j=1} a_{ij}x_j \\
&= \sum^{}_{j=1, k=1} \lambda e_{ik}e^T_{kj}x_j \\
&= \sum^{}_{j=1, k=1} \lambda e_{ik}e_{jk}x_j \\
&= \sum^{}_{j=1} \lambda e_{i}e_j x_j \\
&= \lambda \mathbf{e} \cdot \mathbf{x} \mathbf{e} \\
\end{split}
\]

The definition of an eigenvalue and eigenvector of $\mathbf{A}$ is that $\mathbf{A}\mathbf{e} = \lambda\mathbf{e} $.
Since $ \mathbf{A}\mathbf{e} = \lambda \mathbf{e} $, and
$\mathbf{A}\mathbf{e} = \lambda \left( \mathbf{e} \cdot \mathbf{e} \right) \mathbf{e} = \lambda\mathbf{e}$, $\lambda$
 is an eigenvalue of
$\mathbf{A}$ with eigenvector $\mathbf{e}$ as required. Note that all eigenvectors are orthogonal. So for all the
other eigenvectors of $\mathbf{A}$, $\mathbf{A}\mathbf{e}_i = 0$. So the eigenvalue of all the other eigenvectors
must be zero.

Hence $\mathbf{A}$ has eigenvalue $\lambda$ with eigenvector $\mathbf{e}$ and all other eigenvalues
are zero. As required.

% thinking about using the result above to show that the vector formed by the mehtod they describe must have
% eigenvectors and eigenvalues as B. We can then state that since they're aboeu to uniquely determine the matrix B,
% this matrix must be B.

Let
\[
\mathbf{C} \triangleq \lambda_1 \mathbf{e}_1 \mathbf{e}^T_1 + \lambda_2 \mathbf{e}_2 \mathbf{e}^T_2 \dots + \lambda_n
\mathbf{e}_n \mathbf{e}^T_n
\]
Since eigenvectors are orthonormal, $\mathbf{e}_i\mathbf{e}_j = \delta_{ij}$.
Matrix multiplication and addition is distributive. So from the result with $\mathbf{A}$:
\[
\forall i. \mathbf{C}\mathbf{e}_i = \lambda\mathbf{e}_i
\]
Hence the eigenvectors and eigenvalues of $\mathbf{C}$ are $\lambda_1, \lambda_2, \dots, \lambda_n $ and
$\mathbf{e}_1, \mathbf{e}_2, \dots \mathbf{e}_n $.
Since eigenvectors and eigenvalues uniquely determine a matrix and $\mathbf{C} $ has the same eigenvectors and
eigenvalues as $\mathbf{B} $, $\mathbf{B} = \mathbf{C}$. So:
\[
\mathbf{B} = \lambda_1 \mathbf{e}_1 \mathbf{e}^T_1 + \lambda_2 \mathbf{e}_2 \mathbf{e}^T_2 \dots + \lambda_n
\mathbf{e}_n \mathbf{e}^T_n
\]
As required.

\item

\[
\begin{pmatrix}
x & y & z \\
\end{pmatrix}
\begin{pmatrix}
1 & 8 \\
8 & -11 \\
\end{pmatrix}
\begin{pmatrix}
x \\ y \\ z \\
\end{pmatrix}
=
1
\]

\[
\begin{split}
\begin{vmatrix}
1 - \lambda & 8 \\
8 & -11 - \lambda \\
\end{vmatrix}
&=
(\lambda - 1)(\lambda + 11) - 64 \\
&= \lambda^2 + 10\lambda - 75 \\
&= (\lambda - 5)(\lambda + 15)
\end{split}
\]
So the eigenvalues of $\mathbf{A}$ are $\lambda_1 = 5$ and $\lambda_2 = -15$.

For $\lambda_1$:
\[
\begin{pmatrix}
-4 & 8 \\
8 & -16 \\
\end{pmatrix}
\begin{pmatrix}
a \\ b \\
\end{pmatrix}
=
\mathbf{0} \\
\]
\[
a = 2b
\]
Setting $b = \frac{1}{\sqrt {5}}$ gives a normalised eigenvector:
\[
\mathbf{e}_1 =
\frac{1}{\sqrt {5}}
\begin{pmatrix}
2 \\ 1 \\
\end{pmatrix}
\]

For $\lambda_2$:
\[
\begin{pmatrix}
16 & 8 \\
8 & 4 \\
\end{pmatrix}
=
\mathbf{0}
\]
\[
2a = -b
\]
Setting $a = \frac{1}{\sqrt {5}}$ gives a normalised eigenvector:
\[
\mathbf{e}_2 =
\frac{1}{\sqrt {5}}
\begin{pmatrix}
1 \\ -2 \\
\end{pmatrix}
\]

\begin{align*}
x &= \begin{pmatrix}
1 \\ 0 \\
\end{pmatrix}\cdot \mathbf{e}_1 \mathbf{e}_1 +
\begin{pmatrix}
1 \\ 0 \\
\end{pmatrix}\cdot \mathbf{e}_2 \mathbf{e}_2
&
y &= \begin{pmatrix}
0 \\ 1 \\
\end{pmatrix}\cdot \mathbf{e}_1 \mathbf{e}_1 +
\begin{pmatrix}
0 \\ 1 \\
\end{pmatrix}\cdot \mathbf{e}_2 \mathbf{e}_2 \\
x &= \frac{1}{\sqrt {5}}(2\mathbf{e}_1 + \mathbf{e}_2) & y &= \frac{1}{\sqrt {5}}(\mathbf{e}_1 - 2\mathbf{e}_2)
\end{align*}

\[
\begin{split}
x^2 + 16xy - 11y^2 &= 1 \\
x(x + 16y) - 11y^2 &= 1 \\
\frac{1}{5}(2\mathbf{e}_1 + \mathbf{e}_2)(2\mathbf{e}_1 + \mathbf{e}_2 + 16(\mathbf{e}_1 - 2\mathbf{e}_2)) - 11
(\mathbf{e}_1 - 2\mathbf{e}_2)^2) &= 1 \\
(2\mathbf{e}_1 + \mathbf{e}_2)(18\mathbf{e}_1 - 31\mathbf{e}_2) - 11\mathbf{e}_1^2 +
44\mathbf{e}_1\mathbf{e}_2 - 44\mathbf{e}_2^2 &= 5 \\
36\mathbf{e}_1 - 44\mathbf{e}_1\mathbf{e}_2 - 31\mathbf{e}_2^2 - 11\mathbf{e}_1^2 +
44\mathbf{e}_1\mathbf{e}_2 - 44\mathbf{e}_2^2 &= 5 \\
25\mathbf{e}_1^2 - 75\mathbf{e}_2^2 &= 5 \\
5\mathbf{e}_1^2 - 15\mathbf{e}_2^2 &= 1 \\
\end{split}
\]

\item

\[
\begin{split}
f &= x^3 + xy + y^2 \\
\pdv{f}{x} &= 3x^2 + y \\
\pdv{f}{y} &= x + 2y \\
\pdv[2]{f}{x}{y} &= 1 \\
\pdv[2]{f}{x} &= 6x \\
\pdv[2]{f}{y} &= 2 \\
\end{split}
\]

At a stationary point, $\pdv{f}{x} = \pdv{f}{y} = 0$.
\begin{align}
x + 2y &= 0 & 3x^2 + y &= 0 \\
x + 2y &= 0 & y &= -3x^2 \\
x - 6x^2 &= 0 \\
x(6x - 1) &= 0 \\
\end{align}
\[
x = 0 \vee x = \frac{1}{6}
\]
At $x = 0$:
\[
\begin{split}
x + 2y &= 0 \\
2y &= 0 \\
y &= 0 \\
\end{split}
\]
At $x = \frac{1}{6}$:
\[
\begin{split}
x + 2y &= 0 \\
\frac{1}{6} + 2y &= 0 \\
y &= -\frac{1}{12} \\
\end{split}
\]
So the stationary points of the function $f$ are $\left( 0, 0 \right)$ and $\left( \frac{1}{6}, -\frac{1}{12} \right)$.

The Hessian for the function $f$ at the point $(0, 0)$ is:
\[
\mathbf{H}
=
\begin{pmatrix}
0 & 1 \\
1 & 2 \\
\end{pmatrix}
\]
\[
\begin{split}
|\mathbf{H} - \lambda\mathbf{I}| &= 0 \\
\lambda(\lambda - 2) - 1 &= 0 \\
\lambda^2 - 2\lambda - 1 &= 0 \\
\lambda &= 1 \pm \sqrt {2} \\
\end{split}
\]
Since the eigenvectors have opposite signs, the point $(0, 0)$ must be a stationary point.

The Hessian for the function $f$ at the point $\left( \frac{1}{6}, -\frac{1}{12} \right)$ is:
\[
\mathbf{H}
=
\begin{pmatrix}
1 & 1 \\
1 & 2 \\
\end{pmatrix}
\]
\[
\begin{split}
|\mathbf{H} - \lambda\mathbf{I}| &= 0 \\
(\lambda - 1)(\lambda - 2) - 1 &= 0 \\
\lambda^2 - 3\lambda + 1 &= 0 \\
\lambda &= \frac{3 \pm \sqrt {5}}{2} \\
\end{split}
\]
Since all eigenvalues at this point are positive, the function must have a minimum at this point.

\item

The following matrix is orthogonal if and only if the dot product of every pair of the columns is zero.

\begin{align*}
c_1 \cdot c_2 &= \frac{1}{2}(1 + 1 - \sqrt {2^2}) & c_1 \cdot c_3 &= \frac{1}{2}(\sqrt {2} - \sqrt {2} + 0) & c_2
\cdot c_3 &= \frac{1}{2}(\sqrt {2} - \sqrt {2}) \\
&= 0 & &= 0 & &= 0 \\
\end{align*}

If $\mathbf{x}$ is aligned with this axis, then $f(\mathbf{x}) = \mathbf{x}$.

\[
\begin{split}
|\mathbf{O} - \lambda\mathbf{I}| &=
\frac{1}{8}
\begin{vmatrix}
1 - 2\lambda & 1 & \sqrt {2} \\
1 & 1 - 2\lambda & -\sqrt {2} \\
-\sqrt {2} & \sqrt {2} & -2\lambda \\
\end{vmatrix} \\
&= \frac{1}{8}\left((1 - 2\lambda)((1 - 2\lambda)(-2\lambda) + 2) - 1(-2\lambda - 2) + \sqrt {2}\left(\sqrt {2} + \sqrt
{2}(1 - 2\lambda)\right)\right) \\
&= \frac{1}{8} ((1 - 2\lambda)(4\lambda^2 - 2\lambda + 2) + (2\lambda + 2) + (4 - 4\lambda)) \\
&= \frac{1}{8}(-8\lambda^3 + 8\lambda^2 - 6\lambda + 2 + 2\lambda + 2 + 4 - 4\lambda) \\
&= \frac{1}{8}(-8\lambda^3 + 8\lambda^2 - 8\lambda + 8) \\
&= -(\lambda^3 - \lambda^2 + \lambda - 1) \\
&= -(\lambda - 1)(\lambda^2 + 1) \\
\end{split}
\]
Setting $\lambda = 1$ gives $|\mathbf{O} - \lambda\mathbf{I}| = 0$. So $1$ is an eigenvalue of $\mathbf{O}$.

The axis of rotation is the eigenvector corresponding to eigenvalue 1.

\[
\frac{1}{2}
\begin{pmatrix}
-1 & 1 & \sqrt {2} \\
1 & -1 & -\sqrt {2} \\
\sqrt {2} & \sqrt {2} & -2 \\
\end{pmatrix}
\mathbf{v} = \mathbf{0} \\
\]
\begin{align*}
v_1 - v_2 - \sqrt {2}v_3 &= 0 & \sqrt {2}v_1 + \sqrt {2}v_2 - 2v_3 &= 0 \\
2\sqrt {2}v_2 &= 0 & 2\sqrt {2}v_1 - 4v_3 &= 0 \\
v_2 &= 0 & v_1 &= \sqrt {2} v_3 &= 0 \\
\end{align*}

Setting $v_3 = \frac{1}{\sqrt {3}}$ gives the normalised eigenvector:
\[
\mathbf{e} =
\frac{1}{\sqrt {3}}
\begin{pmatrix}
\sqrt {2} \\ 0 \\ 1 \\
\end{pmatrix}
\]
$\mathbf{e}$ is the axis of rotation.

If vectors lie in the 2D subspace orthogonal to the axis of rotation, then the
coordinate of the axis of rotation is zero. Consider now only the 2D subspace which
is orthogonal to the axis of rotation with $0\mathbf{e}_1$. A vector lying in
this subspace will experience a 2D rotation of $\theta$ \textit{within this subspace}.

So the matrix has the same effect as a 2D rotation within the subspace given by:
\[
\mathbf{R} =
\begin{pmatrix}
\cos\theta & -\sin\theta \\
\sin\theta & \cos\theta \\
\end{pmatrix}
\]

\[
\mathbf{R} =
\begin{pmatrix}
\cos \theta & -\sin \theta \\
\sin \theta & \cos \theta \\
\end{pmatrix}
\]
For all eigenvalues of $\mathbf{R}$, $|\mathbf{R} - \lambda \mathbf{I}|$.

\[
\begin{split}
|\mathbf{R} - \lambda\mathbf{I}| &= 0 \\
\begin{vmatrix}
\cos\theta - \lambda & -\sin\theta \\
\sin\theta & \cos\theta - \lambda \\
\end{vmatrix}
&= 0 \\
(\cos\theta - \lambda)(\cos\theta - \lambda) - (\sin\theta)(-\sin\theta) &= 0 \\
\cos^2\theta - 2\lambda \cos\theta + \lambda^2 + \sin^2\theta &= 0 \\
\lambda^2 - 2\lambda \cos\theta + 1 &= 0 \\
\end{split}
\]
\[
\begin{split}
\lambda &= \frac{2\cos\theta \pm \sqrt {4\cos^2\theta - 4}}{2} \\
&= \cos\theta \pm \sqrt {\cos^2\theta - 1} \\
&= \cos\theta \pm \sqrt {-\sin^2\theta} \\
&= \cos\theta \pm i\sin\theta \\
\end{split}
\]
\begin{align*}
\lambda &= \cos\theta + i\sin\theta & & \vee \ \ & \lambda &= \cos -\theta + i\sin -\theta \\
\lambda &= e^{i\theta} & & \vee \ \ & \lambda &= e^{-i\theta} \\
\end{align*}

If $\mathbf{O}$ is a rotation matrix about an axis, then in this coordinate system,
$\mathbf{O}$ is a standard 3D rotation matrix in the form of one of the below matrices:

\begin{align}
\begin{pmatrix}
1 & 0 & 0 \\
0 & \cos\theta & -\sin\theta \\
0 & \sin\theta & \cos\theta \\
\end{pmatrix}
& &
\begin{pmatrix}
\cos\theta & 0 & -\sin\theta \\
0 & 1 & 0 \\
\sin\theta & 0 & \cos\theta \\
\end{pmatrix}
& &
\begin{pmatrix}
\cos\theta & -\sin\theta & 0\\
\sin\theta & \cos\theta  & 0\\
0 & 0 & 1 \\
\end{pmatrix}
\end{align}

Note that the trace of all of these matrices is $1 + 2\cos\theta$. So the
trace of $\mathbf{O}$ is $1 + 2\cos\theta$.

The trace of $\mathbf{O}$ is 1. Defining $0 \leq \theta < 2\pi$ gives:
\[
\begin{split}
1 &= 1 + 2\cos\theta \\
0 &= 2\cos\theta \\
0 &= \cos\theta \\
\end{split}
\]
\[
\theta = \frac{\pi}{2} \vee \theta = \frac{3\pi}{2}
\]

So $\theta$ is a rotation of $\frac{\pi}{2}$ or $\frac{3\pi}{2} $.

\dag Presumably which depends on which direction you define your rotation to
go -- these calculations didn't differentiate between clockwise and anticlockwise
and so we would expect two results $\theta_1, \theta_2$ where $\theta_1 = 2\pi - \theta_2$.

\end{enumerate}

\end{document}
