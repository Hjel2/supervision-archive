\newcommand{\svrname}{Dr John Fawcett}
\newcommand{\jkfside}{oneside}
\newcommand{\jkfhanded}{right}

\newcommand{\studentname}{Harry Langford}
\newcommand{\studentemail}{hjel2@cam.ac.uk}

\documentclass[10pt,\jkfside,a4paper]{article}

% DO NOT add \usepackage commands here.  Place any custom commands
% into your SV work files.  Anything in the template directory is
% likely to be overwritten!

\usepackage{fancyhdr}

\usepackage{lastpage}       % ``n of m'' page numbering
\usepackage{lscape}         % Makes landscape easier

\usepackage{verbatim}       % Verbatim blocks
\usepackage{listings}       % Source code listings
\usepackage{epsfig}         % Embed encapsulated postscript
\usepackage{array}          % Array environment
\usepackage{qrcode}         % QR codes
\usepackage{enumitem}       % Required by Tom Johnson's exam question header

\usepackage{hhline}         % Horizontal lines in tables
\usepackage{siunitx}        % Correct spacing of units
\usepackage{amsmath}        % American Mathematical Society
\usepackage{amssymb}        % Maths symbols
\usepackage{amsthm}         % Theorems

\usepackage{ifthen}         % Conditional processing in tex

\usepackage[top=3cm,
            bottom=3cm,
            inner=2cm,
            outer=5cm]{geometry}

% PDF metadata + URL formatting
\usepackage[
            pdfauthor={\studentname},
            pdftitle={\svcourse, SV \svnumber},
            pdfsubject={},
            pdfkeywords={9d2547b00aba40b58fa0378774f72ee6},
            pdfproducer={},
            pdfcreator={},
            hidelinks]{hyperref}


% DO NOT add \usepackage commands here.  Place any custom commands
% into your SV work files.  Anything in the template directory is
% likely to be overwritten!

\usepackage{fancyhdr}

\usepackage{lastpage}       % ``n of m'' page numbering
\usepackage{lscape}         % Makes landscape easier

\usepackage{verbatim}       % Verbatim blocks
\usepackage{listings}       % Source code listings
\usepackage{graphicx}
\usepackage{float}
\usepackage{epsfig}         % Embed encapsulated postscript
\usepackage{array}          % Array environment
\usepackage{qrcode}         % QR codes
\usepackage{enumitem}       % Required by Tom Johnson's exam question header

\usepackage{hhline}         % Horizontal lines in tables
\usepackage{siunitx}        % Correct spacing of units
\usepackage{amsmath}        % American Mathematical Society
\usepackage{amssymb}        % Maths symbols
\usepackage{amsthm}         % Theorems

\usepackage{ifthen}         % Conditional processing in tex

\usepackage[top=3cm,
            bottom=3cm,
            inner=2cm,
            outer=5cm]{geometry}

% PDF metadata + URL formatting
\usepackage[
            pdfauthor={\studentname},
            pdftitle={\svcourse, SV \svnumber},
            pdfsubject={},
            pdfkeywords={9d2547b00aba40b58fa0378774f72ee6},
            pdfproducer={},
            pdfcreator={},
            hidelinks]{hyperref}

\renewcommand{\headrulewidth}{0.4pt}
\renewcommand{\footrulewidth}{0.4pt}
\fancyheadoffset[LO,LE,RO,RE]{0pt}
\fancyfootoffset[LO,LE,RO,RE]{0pt}
\pagestyle{fancy}
\fancyhead{}
\fancyhead[LO,RE]{{\bfseries \studentname}\\\studentemail}
\fancyhead[RO,LE]{{\bfseries \svcourse, SV~\svnumber}\\\svdate\ \svtime, \svvenue}
\fancyfoot{}
\fancyfoot[LO,RE]{For: \svrname}
\fancyfoot[RO,LE]{\today\hspace{1cm}\thepage\ / \pageref{LastPage}}
\fancyfoot[C]{\qrcode[height=0.8cm]{\svuploadkey}}
\setlength{\headheight}{22.55pt}


\ifthenelse{\equal{\jkfside}{oneside}}{

 \ifthenelse{\equal{\jkfhanded}{left}}{
  % 1. Left-handed marker, one-sided printing or e-marking, use oneside and...
  \evensidemargin=\oddsidemargin
  \oddsidemargin=73pt
  \setlength{\marginparwidth}{111pt}
  \setlength{\marginparsep}{-\marginparsep}
  \addtolength{\marginparsep}{-\textwidth}
  \addtolength{\marginparsep}{-\marginparwidth}
 }{
  % 2. Right-handed marker, one-sided printing or e-marking, use oneside.
  \setlength{\marginparwidth}{111pt}
 }

}{
 % 3. Alternating margins, two-sided printing, use twoside.
}


\setlength{\parindent}{0em}
\addtolength{\parskip}{1ex}

% Exam question headings, labels and sensible layout (courtesy of Tom Johnson)
\setlist{parsep=\parskip, listparindent=\parindent}
\newcommand{\examhead}[3]{\section{#1 Paper #2 Question #3}}
\newenvironment{examquestion}[3]{
\examhead{#1}{#2}{#3}\setlist[enumerate, 1]{label=(\alph*)}\setlist[enumerate, 2]{label=(\roman*)}
\marginpar{\href{https://www.cl.cam.ac.uk/teaching/exams/pastpapers/y#1p#2q#3.pdf}{\qrcode{https://www.cl.cam.ac.uk/teaching/exams/pastpapers/y#1p#2q#3.pdf}}}
\marginpar{\footnotesize \href{https://www.cl.cam.ac.uk/teaching/exams/pastpapers/y#1p#2q#3.pdf}{https://www.cl.cam.ac.uk/\\teaching/exams/pastpapers/\\y#1p#2q#3.pdf}}
}{}


\fancyhead[RO,LE]{{\bfseries NST Maths, SV~14}\\19-03-2022\ 12:00, Video Link}

\usepackage{physics}
\usepackage{enumitem}

\begin{document}

\begin{enumerate}
\setcounter{enumi}{14}

\item
\begin{enumerate}

\item
\[
\begin{split}
 & \int_S \mathbf{F}\cdot \dd{\mathbf{S}} \\
=& \int_S \frac{1}{a} \begin{pmatrix} \alpha x^3 \\ \beta y^3 \\ \gamma z^3 \\ \end{pmatrix} \cdot \begin{pmatrix} x \\ y \\ z \\ \end{pmatrix}\dd{S} \\
=& \frac{1}{a}\int_S \alpha x^4 + \beta y^4 + \gamma z^4 \dd{S} \\
=& \frac{(\alpha + \beta + \gamma)}{a} \int_S z^4 \dd{S} \text{ by symmetry}\\
=& \frac{(\alpha + \beta + \gamma)}{a} \int^{2\pi}_0\int^\pi_0 a^2 z^4\sin\theta \dd{\theta}\dd{\phi} \\
=& a(\alpha + \beta + \gamma) \int^{2\pi}_0\int^\pi_0 a^4 \cos^4 \theta \sin\theta \dd{\theta}\dd{\phi} \\
=& a^5(\alpha + \beta + \gamma) \int^{2\pi}_0\dd{\phi}\left[-\frac{1}{5}\cos^5\theta\right]^\pi_0 \\
=& a^5(\alpha + \beta + \gamma) \times 2\pi \times \frac{2}{5} \\
=& \frac{4\pi a^5 (\alpha + \beta + \gamma)}{5}
\end{split}
\]

\item 

The integral over the curved surface of the cylinder:

\[
\begin{split}
 & \int_S \mathbf{F} \cdot \dd{\mathbf{S}} \\
=& \int^{h}_{-h}\int^{2\pi}_0 
\begin{pmatrix}
\alpha x^3 \\ \beta y^3 \\ \gamma z^3 \\
\end{pmatrix} \cdot 
\frac{1}{\sqrt{4x^2 + 4y^2}}
\begin{pmatrix}
2x \\ 2y \\ 0 \\
\end{pmatrix}
a\dd{\theta}\dd{z} \\
=& \int^{h}_{-h}\int^{2\pi}_0 
\begin{pmatrix}
\alpha x^3 \\ \beta y^3 \\ \gamma z^3 \\
\end{pmatrix} \cdot 
\frac{2a}{\sqrt{4a^2}}
\begin{pmatrix}
x \\ y \\ 0 \\
\end{pmatrix}
\dd{\theta}\dd{z} \\
=& \int^{2h}_0\dd{z}\int^{2\pi}_0 
\begin{pmatrix}
\alpha x^3 \\ \beta y^3 \\ \gamma z^3 \\
\end{pmatrix} \cdot 
\begin{pmatrix}
x \\ y \\ 0 \\
\end{pmatrix}
\dd{\theta}\\
=& 2h\int^{2\pi}_0 \alpha x^4 + \beta y^4 \dd{\theta}\\
=& (\alpha + \beta)2h \int^{2\pi}_0 x^4 \dd{\theta} \text{ by rotational symmetry} \\
=& (\alpha + \beta)2a^4h \int^{2\pi}_0 \cos^4\theta \dd{\theta} \\
=& (\alpha + \beta)2a^4h \int^{2\pi}_0 \frac{1}{8}\cos4\theta + \frac{1}{2}\cos2\theta + \frac{3}{8} \dd{\theta} \\
=& (\alpha + \beta)2a^4h \left[ \frac{1}{32}\sin4\theta + \frac{1}{4}\sin2\theta + \frac{3}{8}\theta \right]^{2\pi}_0 \\
=& (\alpha + \beta)2a^4h \times \frac{3\pi}{4} \\
=& (\alpha + \beta)\frac{3\pi a^4h}{2} \\
\end{split}
\]
\newpage
The integral over the ends of the cylinder:

\[
\begin{split}
 & \int_S \mathbf{F} \cdot \dd{\mathbf{S}} \\
=& \int^{2\pi}_0 \begin{pmatrix} \alpha x^3 \\ \beta y^3 \\ \gamma h^3 \\ \end{pmatrix} \cdot \begin{pmatrix} 0 \\ 0 \\ 1 \\ \end{pmatrix} a\dd{\theta} + \int^{2\pi}_0 \begin{pmatrix} \alpha x^3 \\ \beta y^3 \\ -\gamma h^3 \\ \end{pmatrix} \cdot \begin{pmatrix} 0 \\ 0 \\ -1 \\ \end{pmatrix} a\dd{\theta} \\
=& \int^{2\pi}_0 a\gamma h^3 \dd{\theta} + \int^{2\pi}_0 a\gamma h^3 \dd{\theta} \\
=& 2a\gamma h^3\int^{2\pi}_0 \dd{\theta} \\
=& 4a\gamma \pi h^3 \\
\end{split}
\]

So the total integral over the surface of the cylinder is equal to
\[(\alpha + \beta)\frac{3\pi a^4h}{2} + 4a\gamma\pi h^3 = ah\pi(\frac{3}{2}(\alpha + \beta)a^3 + 4\gamma h^2)\]

\end{enumerate}

\item
\begin{enumerate}

\item

Note that I am fully aware of the divergence theorem -- however 18.(a) implied that we should 
not use it for this part of the question.

Consider the opposite faces of the cube. These have the same limits of integration and 
so I will integrate them together and then sum the results at the end to get the total 
integral over the surface of the cube.

\[
\mathbf{F} = \begin{pmatrix} x^2 + ay^2 \\ 3xy \\ 6z \\ \end{pmatrix}
\]

Integrating over the faces which are perpendicular to the $x$ axis gives:

\[
\begin{split}
 & \int^1_0\int^1_0 \begin{pmatrix} 1^2 + ay^2 \\ 3(1)y \\ 6z \\ \end{pmatrix} \cdot \begin{pmatrix} 1 \\ 0 \\ 0 \\ \end{pmatrix} +  \begin{pmatrix} (0)^2 + ay^2 \\ 3(-1)y \\ 6z \\ \end{pmatrix} \cdot \begin{pmatrix} -1 \\ 0 \\ 0 \\ \end{pmatrix} \dd{y}\dd{x} \\
=& \int^1_0\int^1_0 ay^2 + 1 - ay^2 \dd{y}\dd{z} \\
=& \int^1_0\int^1_0 1 \dd{y}\dd{z} \\
=& 1 \\
\end{split}
\]

Integrating over the faces which are perpendicular to the $y$ axis gives:

\[
\begin{split}
 & \int^1_0\int^1_0 \begin{pmatrix} x^2 + a(1)^2 \\ 3x(1) \\ 6z \\ \end{pmatrix} \cdot \begin{pmatrix} 0 \\ 1 \\ 0 \\ \end{pmatrix} +  \begin{pmatrix} x^2 + a(0)^2 \\ 3x(0) \\ 6z \\ \end{pmatrix} \cdot \begin{pmatrix} 0 \\ -1 \\ 0 \\ \end{pmatrix} \dd{x}\dd{z} \\
=& \int^1_0\int^1_0 3x \dd{x}\dd{z} \\
=& \frac{3}{2}
\end{split}
\]

Integrating over the faces which are perpendicular to the $z$ axis gives:

\[
\begin{split}
 & \int^1_0\int^1_0 \begin{pmatrix} x^2 + ay^2 \\ 3xy \\ 6(1) \\ \end{pmatrix} \cdot \begin{pmatrix} 0 \\ 0 \\ 1 \\ \end{pmatrix} + \begin{pmatrix} x^2 + ay^2 \\ 3xy \\ 6(0) \\ \end{pmatrix} \cdot \begin{pmatrix} 0 \\ 0 \\ -1 \\ \end{pmatrix} \dd{x}\dd{y} \\
=& \int^1_0\int^1_0 6 \dd{x}\dd{y} \\
=& 6 \\
\end{split}
\]

So the sum of the integrals over the faces of the cube is equal to:
\[
1 + \frac{3}{2} + 6 = 8\frac{1}{2}
\]

\item 

\[
\begin{split}
 & \iiint f \dd{x}\dd{y}\dd{z} \\
=& \int^1_0 bx + 6 \dd{x} \int^1_0\dd{y}\int^1_0\dd{z} \\
=& \left[\frac{1}{2}bx^2 + 6x\right]^1_0\times 1 \times 1 \\\
=& \frac{1}{2}b + 6 \\
\end{split}
\]

These integrals have the same value whenever:
\[
\begin{split}
\frac{1}{2} b + 6 &= 8\frac{1}{2} \\
\frac{1}{2}b &= \frac{5}{2} \\
b &= 5 \\
\end{split}
\]
So these two integrals have the same value when $b = 5$; equality is not dependant on the value of $a$.

\end{enumerate}

\item 
\[
\begin{split}
 & \int_S \mathbf{u}\cdot \mathbf{n}\dd{S} \\
=& \int_S \frac{1}{r} \times \frac{\mathcal{Q}}{4\pi \epsilon_0 r^3} \mathbf{x}\cdot\mathbf{x} \dd{S} \\
=& \frac{\mathcal{Q}}{4\pi \epsilon_0 r^4}\int_S \mathbf{x}^2 \dd{S} \\
=& \frac{\mathcal{Q}}{4\pi \epsilon_0 r^4}\int_S |\mathbf{x}|^2 \dd{S} \\
=& \frac{\mathcal{Q}}{4\pi \epsilon_0 r^4}\int_S r^2 \dd{S} \\
=& \frac{\mathcal{Q}}{4\pi \epsilon_0 r^2}\int_S 1 \dd{S} \\
=& \frac{\mathcal{Q}}{4\pi \epsilon_0 r^2} \times 4\pi r^2 \\
=& \frac{\mathcal{Q}}{\epsilon_0} \text{ as required}\\
\end{split}
\]

\item

\begin{enumerate}

\item 

Using the divergence theorem:
\[
\begin{split}
\int_S \mathbf{F} \cdot \dd{S} &= \int_V (\nabla \cdot \mathbf{F}) \dd{V} \\
&= \int^1_0\int^1_0\int^1_0 2x + 3x + 6 \dd{x}\dd{y}\dd{z} \\
&= \int^1_0 5x + 6 \dd{x}\int^1_0\dd{y}\int^1_0\dd{z} \\
&= \frac{5}{2} + 6 \\
&= 8\frac{1}{2} \\
\end{split}
\]

\item 

\[
\begin{split}
 & \nabla \times \mathbf{E} \\
=& \left(0 - 0, 0 - 0, e^{-2t} + e^{-2t} \right) \\
=& \left(0, 0, 2e^{-2t} \right) \\
\end{split}
\]

\[
\begin{split}
 & -\pdv{\mathbf{B}}{t} \\
=& -\left(0, 0, -2e^{-2t}\right) \\
=& \left(0, 0, 2e^{-2t}\right) \\
\end{split}
\]

So $\nabla \times \mathbf{E} = -\pdv{\mathbf{B}}{t}$ as required.

Taking Stokes Theorem and substituting $\mathbf{E} = \mathbf{F}$ gives:
\[
\begin{split}
\int_C \mathbf{F} \cdot \dd{\mathbf{x}} &= \int_S (\nabla \times \mathbf{F}) \cdot \dd{S} \\
\int_C \mathbf{E} \cdot \dd{\mathbf{x}} &= \int_S (\nabla \times \mathbf{E}) \cdot \dd{S} \\
\int_C \mathbf{E} \cdot \dd{\mathbf{x}} &= -\int_S \left(\pdv{\mathbf{B}}{t}\right) \cdot \dd{S} \\
\int_C \mathbf{E} \cdot \dd{\mathbf{x}} &= -\dv{t}\int_S \mathbf{B} \cdot \dd{S} \\
\end{split}
\]
Which gives the required result.

\end{enumerate}

\item 

The integral of the curl of any vector field over the volume of an object is equal 
to the integral of that field over the surface of the object.

\[
\int_S \mathbf{F} \cdot \dd{S} = \int_V (\nabla \cdot \mathbf{F}) \dd{V}
\]

Note that the surface can be closed by adding the circle $x^2 + y^2 = 1, z = 0$.

So by Stoke's Theorem:

\[
\int_S \mathbf{F}\cdot \mathbf{n} \dd{S} = \int_V \nabla \times \mathbf{F} \dd{V} - \int_{S_c} \mathbf{F} \cdot \mathbf{n}_c \dd{S}
\]

Where $S_c$ is the surface of the circle and $n_c$ is the normal to the circle.

Using the Divergence Theorem:
\[
\begin{split}
 & \int_V (\nabla \times \mathbf{F}) \dd{V} \\
=& \int^1_0\int^{2\pi}_0\int^{1-z}_{0} (\nabla \times \mathbf{F}) \dd{r}\dd{\theta}\dd{z} \\
=& \int^1_0\int^{2\pi}_0\int^{1-z}_{0} 3x^2 + 3y^2 + 6z \dd{r}\dd{\theta}\dd{z} \\
=& \int^{2\pi}_0\dd{\theta}\int^1_0\int^{1-z}_{0} 3r^2 + 6z \dd{r}\dd{z} \\
=& 2\pi\int^1_0 3(1-z)^2 + 6z \dd{z} \\
=& 2\pi\int^1_0 3z^2 + 3 \dd{z} \\
=& 2\pi\left[z^3 + 3z\right]^1_0 \\
=& 8\pi \\
\end{split}
\]

\[
\begin{split}
 & \int^{2\pi}_0 \begin{pmatrix} x^3 + 3y + 0^2 \\ y^3 \\ x^2 + y^2 + 3\times 0^2 \\ \end{pmatrix} \cdot \begin{pmatrix} 0 \\ 0 \\ -1 \\ \end{pmatrix} \dd{S} \\
=& \int^{2\pi}_0 -(x^2 + y^2) \dd{S} \\
=& \int^{2\pi}_0 -1 \dd{S} \\
=& -2\pi \\
\end{split}
\]

So the integral across the surface is equal to:
\[
8\pi - -2\pi = 10\pi
\]

\item 

Stoke's Theorem states that:

\[
\int_S (\nabla \times \mathbf{F}) \cdot \dd{\mathbf{S}} = \int_C \mathbf{F} \cdot \dd{\mathbf{x}}
\]

I will integrate across the surface of the hemisphere, and then across the bounding curve and 
show that Stoke's Theorem holds.

\[
\begin{split}
 & \int_S (\nabla \times \mathbf{A}) \cdot \dd{\mathbf{S}} \\
=& \int^{2\pi}_{0}\int^{\frac{\pi}{2}}_0 \begin{pmatrix} 0 - 0 \\ 0 - 0 \\ -1 - 1 \end{pmatrix} \cdot \begin{pmatrix} x \\ y \\ z \\ \end{pmatrix} \sin\theta \dd{\theta}\dd{\phi} \\
=& \int^{2\pi}_{0}\dd{\phi}\int^{\frac{\pi}{2}}_0 -2z \sin\theta \dd{\theta} \\
=& \int^{2\pi}_{0}\dd{\phi}\int^{\frac{\pi}{2}}_0 -2 \sin\theta\cos\theta \dd{\theta} \\
=& 2\pi\left[\cos^2\theta\right]^{\frac{\pi}{2}}_0 \\
=& -2\pi \\
\end{split}
\]

\[
\begin{split}
 & \int_C \mathbf{F} \cdot \dd{\mathbf{x}} \\
=& \int^{2\pi}_0\begin{pmatrix} \sin\theta \\ -\cos\theta \\ 0 \\ \end{pmatrix} \cdot \begin{pmatrix} -\sin\theta \\ \cos\theta \\ 0 \\ \end{pmatrix} \dd{\theta} \\
=& \int^{2\pi}_0 -\sin^2\theta - \cos^2\theta \dd{\theta} \\
=& \int^{2\pi}_0 -1 \dd{\theta} \\
=& -2\pi \\
\end{split}
\]

So for the hemispherical surface $r = 1, z \geq 0$ and the vector field $\mathbf{A}(\mathbf{x}) = (y, -x, z)$, 
Stoke's Theorem holds.

\end{enumerate}

\end{document}