\newcommand{\svrname}{Mr Jakub Perlin}
\newcommand{\jkfside}{oneside}
\newcommand{\jkfhanded}{right}

\newcommand{\studentname}{Harry Langford}
\newcommand{\studentemail}{hjel2@cam.ac.uk}

\documentclass[10pt,\jkfside,a4paper]{article}

\newcommand{\svcourse}{CST Part IA: Introduction to Probability}
\newcommand{\svnumber}{1}
\newcommand{\svvenue}{Churchill, Room TBD}
\newcommand{\svdate}{2022-05-14}
\newcommand{\svtime}{11:00}
\newcommand{\svuploadkey}{PO5ogKIM8KQA22FZS8IAf8gxA8XKi19jxIBVHIfFZ+3GCBXuNUXS9lVN6bNYjxM/}

\newcommand{\svrname}{Mr Matthew Ireland}
\newcommand{\jkfside}{twoside}
\newcommand{\jkfhanded}{right}

\newcommand{\studentname}{Harry Langford}
\newcommand{\studentemail}{hjel2@cam.ac.uk}

\input{../../template/includes.tex}
% DO NOT add \usepackage commands here.  Place any custom commands
% into your SV work files.  Anything in the template directory is
% likely to be overwritten!

\usepackage{fancyhdr}

\usepackage{lastpage}       % ``n of m'' page numbering
\usepackage{lscape}         % Makes landscape easier

\usepackage{verbatim}       % Verbatim blocks
\usepackage{epsfig}         % Embed encapsulated postscript
\usepackage{array}          % Array environment
\usepackage[nolinks]{qrcode}         % QR codes
\usepackage{enumitem}       % Required by Tom Johnson's exam question header

\usepackage{hhline}         % Horizontal lines in tables
\usepackage{siunitx}        % Correct spacing of units
\usepackage{amsmath}        % American Mathematical Society
\usepackage{amssymb}        % Maths symbols
\usepackage{amsthm}         % Theorems

\usepackage{ifthen}         % Conditional processing in tex

\usepackage[top=3cm,
            bottom=3cm,
            inner=2cm,
            outer=5cm]{geometry}

% PDF metadata + URL formatting
\usepackage[
            pdfauthor={\studentname},
            pdftitle={\svcourse, SV \svnumber},
            pdfsubject={},
            pdfkeywords={9d2547b00aba40b58fa0378774f72ee6},
            pdfproducer={},
            pdfcreator={},
            hidelinks]{hyperref}

\renewcommand{\headrulewidth}{0.4pt}
\renewcommand{\footrulewidth}{0.4pt}
\fancyheadoffset[LO,LE,RO,RE]{0pt}
\fancyfootoffset[LO,LE,RO,RE]{0pt}
\pagestyle{fancy}
\fancyhead{}
\fancyhead[LO,RE]{{\bfseries \studentname}\\\studentemail}
\fancyhead[RO,LE]{{\bfseries \svcourse, SV~\svnumber}\\\svdate\ \svtime, \svvenue}
\fancyfoot{}
\fancyfoot[LO,RE]{For: \svrname}
\fancyfoot[RO,LE]{\today\hspace{1cm}\thepage\ / \pageref{LastPage}}
\fancyfoot[C]{\qrcode[height=0.8cm]{\svuploadkey}}
\setlength{\headheight}{22.55pt}

\ifthenelse{\equal{\jkfside}{oneside}}{

 \ifthenelse{\equal{\jkfhanded}{left}}{
  % 1. Left-handed marker, one-sided printing or e-marking, use oneside and...
  \evensidemargin=\oddsidemargin
  \oddsidemargin=73pt
  \setlength{\marginparwidth}{111pt}
  \setlength{\marginparsep}{-\marginparsep}
  \addtolength{\marginparsep}{-\textwidth}
  \addtolength{\marginparsep}{-\marginparwidth}
 }{
  % 2. Right-handed marker, one-sided printing or e-marking, use oneside.
  \setlength{\marginparwidth}{111pt}
 }

}{
 % 3. Alternating margins, two-sided printing, use twoside.
}

\setlength{\parindent}{0em}
\addtolength{\parskip}{1ex}

% Exam question headings, labels and sensible layout (courtesy of Tom Johnson)
\setlist{parsep=\parskip, listparindent=\parindent}
\newcommand{\examhead}[3]{\section{#1 Paper #2 Question #3}}
\newenvironment{examquestion}[3]{
    \examhead{#1}{#2}{#3}\setlist[enumerate, 1]{label=(\alph*)}\setlist[enumerate, 2]{label=(\roman*)}
    \marginpar{\qrcode{https://www.cl.cam.ac.uk/teaching/exams/pastpapers/y#1p#2q#3.pdf}}
    \marginpar{\footnotesize \url{https://www.cl.cam.ac.uk/teaching/exams/pastpapers/y#1p#2q#3.pdf}}
}{}



\usepackage{oz}

\begin{document}

\section*{6 On relations}

\subsection*{6.1 Basic exercises}

\begin{enumerate}

\item Let $A = \{1, 2, 3, 4\}$, $B = \{a, b, c, d\}$ and $C = \{x, y, z\}$.\\
Let $R = \{(1, a), (2, d), (3, a), (3, b), (3, d)\}: A \pfun B$\\
and $S = \{(b, x), (b, y), (c, y), (d, z)\}: B \pfun C$.

Draw the internal diagrams of the relations. What is the composition $S \circ R: A \pfun C$?



\item Prove that relational composition is associative and has the identity relation as the neutral element.



\item For a relation $R: A \pfun B$, let its opposite or dual relation $R^{\text{op}}: B \pfun A$ be defined by:
\begin{equation}
b R^{\text{op}} a \Longleftrightarrow a R b
\end{equation}

For $R, S: A \pfun B$ and $T: B \pfun C$, prove that:

\begin{enumerate}

\item $R \subseteq S \Longrightarrow R^{\text{op}} \subseteq S^{\text{op}}$



\item $(R \cap S)^{\text{op}} = R^{\text{op}} \cap S^{\text{op}}$



\item $(R \cup S)^{\text{op}} = R^{\text{op}} \cup S^{\text{op}}$



\item $(T \circ S)^{\text{op}} = S^{\text{op}} \circ T^{\text{op}}$



\end{enumerate}

\end{enumerate}

\subsection*{6.2 Core exercises}

\begin{enumerate}

\item Let $R, R' \subseteq A \times B$ and $S, S' \subseteq B \times C$ be two pairs of relations 
and assume $R \subseteq R'$ and $S \subseteq S'$.\\
Prove that $S \circ R \subseteq S' \circ R'$.



\item Let $\mathcal{F} \subseteq \mathcal{P}(A \times B)$ and $\mathcal{G} \subseteq \mathcal{P}(B \times C)$ be two 
collections of relations from $A$ to $B$ and from $B$ to $C$, respectively Prove that
\begin{equation}
(\bigcup \mathcal{G})\circ(\bigcup \mathcal{F}) = \bigcup \{S \circ R | R \in \mathcal{F}, S \in \mathcal{G}\}: A \pfun C
\end{equation}
Recall that the notation $\{ S \circ R: A \pfun C | R \in \mathcal{F}, S \in \mathcal{G}\}$ is a common syntactic sugar 
for the formal definition $\{T \in \mathcal{P}(A \times C)| \exists R \in \mathcal{F}: \exists S \in \mathcal{G}: T = S \circ R\}$. Hence,
\begin{equation}
T \in \{S \circ R \in A \pfun C | R \in \mathcal{F}, S \in \mathcal{G}\} \Longleftrightarrow \exists R \in \mathcal{F}: \exists S \in \mathcal{G}: T = S \circ R
\end{equation}
What happens in the case of big intersections?



\item Suppose $R$ is a relation on a set $A$. Prove that

\begin{enumerate}

\item $R$ is reflexive iff id$_A \subseteq R$



\item $R$ is symmetric iff $R = R^{\text{op}}$



\item $R$ is transitive iff $R \circ R \subseteq R$



\item $R$ is antisymmetric iff $R \cap R^{\text{op}} \subseteq \text{id}_A$



\end{enumerate}

\item Let $R$ be an arbitrary relation on a set $A$, for example, representing an undirected graph.
We are interested in constructing the smallest transitive relation (graph) containing $R$, called the 
transitive closure of $R$.

\begin{enumerate}

\item We define the family of relations which are transitive supersets of $R$:
\begin{equation}
\mathcal{T}_R \triangleq \{Q: A \pfun A | R \subseteq Q \text{ and } Q \text{ is transitive}\}
\end{equation}
$R$ is not necessarily going to be an element of this family, as it might not be transitive.
However $R$ is a lower bound for $\mathcal{T}_R$, as it is a subset of every element of the family.

Prove that the set $\bigcap \mathcal{T}_R$ is the transitive closure for $R$.



\item $\bigcap \mathcal{T}_R$ is the intersection of an infinite number of relations so it's difficult 
to compute the transitive closure this way. A better approach is to start with $R$, and keep adding the 
missing connections until we get a transitive graph. This can be done by repeatedly composing $R$ with 
itself: after $n$ compositions, all paths of length $n$ in the graph represented by $R$ will have a 
transitive connection between their endpoints.

Prove that the (at least once) iterated composition $R^{\circ +} \triangleq R \circ R^{\circ}*$ is the 
transitive closure for $R$, i.e. it coincides with the greatest lower bound of $\mathcal{T}_R: R^{\circ +} = \bigcap \mathcal{T}_R$.

Hint: show that $R^{\circ +}$ is both an element and a lower bound of $\mathcal{T}_R$.



\end{enumerate}

\end{enumerate}

\section*{7 On partial functions}

\subsection*{7.1 Basic exercises}

\begin{enumerate}

\item Let $A_2 = \{1, 2\}$ and $A_3 = \{a, b, c\}$. List the element of the sets PFun$(A_i, A_j)$ for 
$i, j \in \{2, 3\}$.

Hint: there may be quite a few, so you can think of ways of characterising all of them without giving 
an explicit listing.



\item Prove that a relation $R: A \pfun B$ is a partial function iff $R \circ R^{\text{op}} \subseteq \text{id}_B$.



\item Prove that the identity relation is a partial function, and that the composition of partial functions is a partial function.



\end{enumerate}

\subsection*{7.2 Core exercises}

\begin{enumerate}

\item Show that (PFun$(A, B), \subseteq$) is a partial order. What is its least element, if it exists?



\item Let $\mathcal{F} \subseteq \text{PFun}(A, B)$ be a non-empty collection of partial functions from $A$ to $B$.

\begin{enumerate}

\item Show that $\bigcap \mathcal{F}$ is a partial function.



\item Show that $\bigcup \mathcal{F}$ need not be a partial function by defining two partial functions $f, g: A \rightharpoonup B$ 
such that $f \cup g: A \pfun B$ is a non-functional relation.



\item Let $h: A \rightharpoonup B$ be a partial function. Show that if every element of $\mathcal{F}$ is below $h$ then $\bigcup \mathcal{F}$ is 
a partial function.



\end{enumerate}

\end{enumerate}

\section*{8 On functions}

\subsection*{8.1 Basic exercises}

\begin{enumerate}

\item Let $A_2 = \{1, 2\}$ and $A_3 = \{a, b, c\}$. List the elements of the sets 
Fun$(A_i, A_j)$ for $i, j \in \{2, 3\}$.



\item Prove that the identity partial function is a function, and the composition of functions 
yields a function



\item Prove or disprove that (Fun$(A, B), \subseteq$) is a partial order.



\item Find endofunctions $f, g: A \rightarrow A$ such that $f \circ g \neq g \circ f$.



\end{enumerate}

\subsection*{8.2 Core exercises}

\begin{enumerate}

\item A relation $R: A \pfun B$ is said to be total if $\forall a \in A, \exists  \in B: a R b$. Prove 
that this is equivalent to id$_A \subseteq R^{\text{op}} \circ R$. Conclude that a relation $R: A \pfun B$ 
is a function iff $R \circ R^{text{op}} \subseteq \text{id}_B$ and id$_A \subseteq R^{\text{op}} \circ R$.



\item Let $\chi: \mathcal{P}(U) \rightarrow (U \Rightarrow [2])$ be the function mapping subsets 
$S \subseteq U$ to their characteristi functions $\chi_S: U \rightarrow [2]$.

\begin{enumerate}

\item Prove that for all $x \in U$,

\begin{itemize}

\item $\chi_{A \cup B}(x) = (\chi_A(x) \vee \chi_B(x)) = \text{max}(\chi_A(x), \chi_B(x))$

\item $\chi_{A \cap B}(x) = (\chi_A(x) \wedge \chi_B(x)) = \text{min}(\chi_A(x), \chi_B(x))$

\item $\chi_{A^{\mathsf{c}}}(x) = ¬(\chi_A(x)) = (1 - \chi_A(x))$

\end{itemize}

\item For what construction $A ? B$ on sets $A$ and $B$ does it hold that 
\begin{equation}
\chi_{A ? B}(x) = (\chi_A(x) \oplus \chi_B(x)) = (\chi_A(x) +_2 \chi_B(x))
\end{equation}



\end{enumerate}

\end{enumerate}

\subsection*{8.3 Optional advanced exercise}

Consider a set $A$ together with an element $a \in A$ and an endofunction $f: A \rightarrow A$.

Say that a relation $R: \mathbb{N} \pfun A$ is $(a, f)$-closed whenever
\begin{equation}
R(0, a) \text{ and } \forall n \in \mathbb{N}, x \in A: R(n, x) \Rightarrow R(n + 1, f(x))
\end{equation}
Define the relation $F: \mathbb{N} \pfun A$ as
\begin{equation}
F \triangleq \bigcap\{R: \mathbb{N} \pfun A | R \text{ is } (a, f)\text{-closed}\}
\end{equation}

\begin{enumerate}

\item Prove that $F$ is $(a, f$-closed.



\item Prove that $F$ is total, that is: $\forall n \in \mathbb{N}: \exists y \in A: F(n, y)$.



\item Prove that $F$ is a function $\mathbb{N} \rightarrow A$, that is: $\forall n \in \mathbb{N} \exists ! y \in A: F(n, y)$.

Hint: Proceed by induction. Observe that, in view of the previous item, to show that $\exists !y \in A: F(k, y)$ it suffices to 
exibit an $(a, f)$-closed relation $R_k$ such that $\exists ! y \in A: R_k(k, y)$. (Why?)

For instance, as the relation $R_0 = \{(m, y) \in \mathbb{N} \times A | m = 0 \Rightarrow y = a\}$ is $(a, f)$-closed one 
has that $F(0, y) \Rightarrow R_0(0, y) \Rightarrow y = a$.



\item Show that if $h$ is a function $\mathbb{N} \rightarrow A$ with $h(0) = a$ and $\forall n \in \mathbb{N}: h(n + 1) = f(h(n))$ then 
$h = F$.

Thus, for every set $A$ together with an element $a \in A$ and an endofunction $f: A \rightarrow A$ there exists a unique function 
$F: \mathbb{N} \rightarrow A$, typically said to be inductively defined, satisfying the recurrence relation
\begin{equation}
F(n) = 
\begin{cases}
a & \text{for } n = 0\\
f(F(n - 1)) & \text{for } n \geq 1\\
\end{cases}
\end{equation}

\end{enumerate}

\end{document}