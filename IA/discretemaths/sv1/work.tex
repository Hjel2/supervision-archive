\newcommand{\svrname}{Mr Jakub Perlin}
\newcommand{\jkfside}{oneside}
\newcommand{\jkfhanded}{right}

\newcommand{\studentname}{Harry Langford}
\newcommand{\studentemail}{hjel2@cam.ac.uk}

\documentclass[10pt,\jkfside,a4paper]{article}

\newcommand{\svcourse}{CST Part IA: Software Engineering and Security}
\newcommand{\svnumber}{1}
\newcommand{\svvenue}{Microsoft Teams}
\newcommand{\svdate}{2022-05-11}
\newcommand{\svtime}{15:00}
\newcommand{\svuploadkey}{CBd13xmL7PC1zqhNIoLdTiYUBnxZhzRAtJxv/ytRdM1r7qIfwMsxeVwM/pPcIo8l}

\newcommand{\svrname}{Dr Sam Ainsworth}
\newcommand{\jkfside}{oneside}
\newcommand{\jkfhanded}{yes}

\newcommand{\studentname}{Harry Langford}
\newcommand{\studentemail}{hjel2@cam.ac.uk}

% DO NOT add \usepackage commands here.  Place any custom commands
% into your SV work files.  Anything in the template directory is
% likely to be overwritten!

\usepackage{fancyhdr}

\usepackage{lastpage}       % ``n of m'' page numbering
\usepackage{lscape}         % Makes landscape easier

\usepackage{verbatim}       % Verbatim blocks
\usepackage{listings}       % Source code listings
\usepackage{epsfig}         % Embed encapsulated postscript
\usepackage{array}          % Array environment
\usepackage{qrcode}         % QR codes
\usepackage{enumitem}       % Required by Tom Johnson's exam question header

\usepackage{hhline}         % Horizontal lines in tables
\usepackage{siunitx}        % Correct spacing of units
\usepackage{amsmath}        % American Mathematical Society
\usepackage{amssymb}        % Maths symbols
\usepackage{amsthm}         % Theorems

\usepackage{ifthen}         % Conditional processing in tex

\usepackage[top=3cm,
            bottom=3cm,
            inner=2cm,
            outer=5cm]{geometry}

% PDF metadata + URL formatting
\usepackage[
            pdfauthor={\studentname},
            pdftitle={\svcourse, SV \svnumber},
            pdfsubject={},
            pdfkeywords={9d2547b00aba40b58fa0378774f72ee6},
            pdfproducer={},
            pdfcreator={},
            hidelinks]{hyperref}


% DO NOT add \usepackage commands here.  Place any custom commands
% into your SV work files.  Anything in the template directory is
% likely to be overwritten!

\usepackage{fancyhdr}

\usepackage{lastpage}       % ``n of m'' page numbering
\usepackage{lscape}         % Makes landscape easier

\usepackage{verbatim}       % Verbatim blocks
\usepackage{listings}       % Source code listings
\usepackage{graphicx}
\usepackage{float}
\usepackage{epsfig}         % Embed encapsulated postscript
\usepackage{array}          % Array environment
\usepackage{qrcode}         % QR codes
\usepackage{enumitem}       % Required by Tom Johnson's exam question header

\usepackage{hhline}         % Horizontal lines in tables
\usepackage{siunitx}        % Correct spacing of units
\usepackage{amsmath}        % American Mathematical Society
\usepackage{amssymb}        % Maths symbols
\usepackage{amsthm}         % Theorems

\usepackage{ifthen}         % Conditional processing in tex

\usepackage[top=3cm,
            bottom=3cm,
            inner=2cm,
            outer=5cm]{geometry}

% PDF metadata + URL formatting
\usepackage[
            pdfauthor={\studentname},
            pdftitle={\svcourse, SV \svnumber},
            pdfsubject={},
            pdfkeywords={9d2547b00aba40b58fa0378774f72ee6},
            pdfproducer={},
            pdfcreator={},
            hidelinks]{hyperref}

\renewcommand{\headrulewidth}{0.4pt}
\renewcommand{\footrulewidth}{0.4pt}
\fancyheadoffset[LO,LE,RO,RE]{0pt}
\fancyfootoffset[LO,LE,RO,RE]{0pt}
\pagestyle{fancy}
\fancyhead{}
\fancyhead[LO,RE]{{\bfseries \studentname}\\\studentemail}
\fancyhead[RO,LE]{{\bfseries \svcourse, SV~\svnumber}\\\svdate\ \svtime, \svvenue}
\fancyfoot{}
\fancyfoot[LO,RE]{For: \svrname}
\fancyfoot[RO,LE]{\today\hspace{1cm}\thepage\ / \pageref{LastPage}}
\fancyfoot[C]{\qrcode[height=0.8cm]{\svuploadkey}}
\setlength{\headheight}{22.55pt}


\ifthenelse{\equal{\jkfside}{oneside}}{

 \ifthenelse{\equal{\jkfhanded}{left}}{
  % 1. Left-handed marker, one-sided printing or e-marking, use oneside and...
  \evensidemargin=\oddsidemargin
  \oddsidemargin=73pt
  \setlength{\marginparwidth}{111pt}
  \setlength{\marginparsep}{-\marginparsep}
  \addtolength{\marginparsep}{-\textwidth}
  \addtolength{\marginparsep}{-\marginparwidth}
 }{
  % 2. Right-handed marker, one-sided printing or e-marking, use oneside.
  \setlength{\marginparwidth}{111pt}
 }

}{
 % 3. Alternating margins, two-sided printing, use twoside.
}


\setlength{\parindent}{0em}
\addtolength{\parskip}{1ex}

% Exam question headings, labels and sensible layout (courtesy of Tom Johnson)
\setlist{parsep=\parskip, listparindent=\parindent}
\newcommand{\examhead}[3]{\section{#1 Paper #2 Question #3}}
\newenvironment{examquestion}[3]{
\examhead{#1}{#2}{#3}\setlist[enumerate, 1]{label=(\alph*)}\setlist[enumerate, 2]{label=(\roman*)}
\marginpar{\href{https://www.cl.cam.ac.uk/teaching/exams/pastpapers/y#1p#2q#3.pdf}{\qrcode{https://www.cl.cam.ac.uk/teaching/exams/pastpapers/y#1p#2q#3.pdf}}}
\marginpar{\footnotesize \href{https://www.cl.cam.ac.uk/teaching/exams/pastpapers/y#1p#2q#3.pdf}{https://www.cl.cam.ac.uk/\\teaching/exams/pastpapers/\\y#1p#2q#3.pdf}}
}{}


\usepackage[utf8]{inputenc}
\usepackage{mathtools}

\begin{document}

\section{On Proofs}

\subsection{Basic Exercises}

\begin{enumerate}

\item Suppose $n$ is a natural number larger than 2, and $n$ is not a prime number. 
Then $n \cdot 2 + 13$ is not a prime number.

Disproof by counterexample:\\
Let $n = 8$. \\
Then $n \cdot 2 + 13 = 29$. \\
But 29 is prime.
So the statement is disproved.

\item If $x^2 + y = 13$ and $y \neq 4$ then $x \neq 3$.

This statement is logically equivalent to the contrapositive: if $x = 3$ then 
$y = 4$ or $x^2 + y \neq 13$.
This is proved below.
\begin{equation}
\begin{split}
x &= 3\\
x^2 + y &= 13\\
3^2 + y &= 13\\
9 + y &= 13\\
y &= 4\\
\end{split}
\end{equation}
So either the $y = 4$ or $x^2 + y \neq 13$ as required.

\item For an integer $n$, $n^2$ is even if and only if $n$ is even.

If:\\
Assume $n$ is even. So $n$ can be written in the form $2\cdot k$ for some $k$. 
\begin{equation}
\begin{split}
n &= 2\cdot k\\
n^2 &= 4\cdot k^2\\ 
&= 2(2\cdot k^2)
\end{split}
\end{equation}
This is an even number of the form $2\cdot i$ where $i = 2\cdot k^2$.\\
So if $n$ is even; then $n^2$ is even.

Only if:\\
If $n^2$ is even then $n$ is even. This is logically equivalent to the 
contrapositive: if $n$ is odd then $n^2$ is odd.\\
Assume $n$ is odd. So $n$ can be written in the form $2\cdot k + 1$ for some $k$.
\begin{equation}
\begin{split}
n &= 2\cdot k + 1\\
n^2 &= (2\cdot k + 1)\cdot(2\cdot k + 1)\\
&= 4\cdot k^2 + 4\cdot k + 1\\
&= 2(2\cdot k^2 + 2\cdot k) + 1\\
\end{split}
\end{equation}
This is an odd number of the form $2\cdot j + 1$ where $j=2\cdot k^2 + 2\cdot k$.\\
So if $n$ is odd; then $n^2$ is odd. As required.

\item For all real numbers $x$ and $y$ there is a real number $z$ such that 
$x + z = y - z$.

\begin{equation}
\begin{split}
x + z &= y - z\\
2\cdot z &= y - x\\
\therefore z &= \frac{y - x}{2}\\
\end{split}
\end{equation}

Since the set of reals is closed under both addition and division and $x, y \in \mathbb{R}$: $\frac{y - x}{2}\in\mathbb{R}$. 
Hence $z\in\mathbb{R}$ and the statement is proved.

\item For all real numbers $x$ and $y$ there is an integer $z$ such that 
$x + z = y - z$.

Disproof by counterexample:\\
Let $y = x + 1$.
\begin{equation}
\begin{split}
x + z &= y - z\\
x + z &= x + 1 - z\\
2\cdot z &= 1\\
z &= \frac{1}{2}\\
\end{split}
\end{equation}
In this case: $z$ is not an integer and so the statement is disproved.

\item The sum of two rational numbers is a rational number.
Let $a = \frac{x}{y}$. Let $b = \frac{p}{q}$.
\begin{equation}
\begin{split}
a + b &= \frac{x}{y} + \frac{p}{q}\\
a + b &= \frac{q\cdot x}{q\cdot y} + \frac{p\cdot y}{q\cdot y}\\
a + b &= \frac{p\cdot y + q\cdot x}{q\cdot y}\\
\end{split}
\end{equation}
This is a rational number of the form $\frac{s}{t}$ where $s=p\cdot y 
+ q\cdot x$ and $t = q\cdot y$. So the sum of two rational numbers is a rational 
number -- as required.

\item For every real number $x$, if $x \neq 2$ then there is a unique real 
number y such that $\frac{2\cdot y}{y + 1} = x$.
\begin{equation}
\begin{split}
x &= \frac{2\cdot y}{y + 1}\\
x\cdot y + x &= 2\cdot y\\
x &= y\cdot (2 - x)\\
\frac{x}{2 - x} &= y\\
\end{split}
\end{equation}

Since $(\frac{x}{2 - x})$ is defined for all $x\neq 2$: there exists a $y$ for 
all $x\neq 2$.

Now we only need to prove that $y$ is unique for all x.

I will prove this by contradiction. Let $f(x) = \frac{x}{2 - x}$. Assume that 
there exists an $x_0$ and an 
$x_1$ such that $f(x_0) = f(x_1)$.
\begin{equation}
\begin{split}
f(x_0) &= f(x_1)\\
\frac{x_0}{2 - x_0} &= \frac{x_1}{2 - x_1}\\
2\cdot x_0 - x_0\cdot x_1 &= 2\cdot x_1 - x_0 \cdot x_1\\
2\cdot x_0 &= 2\cdot x_1\\
x_0 &= x_1\\
\therefore (f(x_0)= f(x_1))&\Longrightarrow (x_0 = x_1) \text{ so f is an injective function}\\
\end{split}
\end{equation}

Since $(\frac{x}{2 - x})$ is an injective function: $y$ is unique.

Hence the statement is proved.

\item For all integers $m$ and $n$, if $m\cdot n$ is even, then either m is 
even or n is even.

This statement is logically equivalent to the contrapositive:\\
If both $m$ and $n$ are odd then $m\cdot n$ is odd.\\
Let $m = 2\cdot i + 1$ and $n = 2\cdot j + 1$.
\begin{equation}
\begin{split}
m\cdot n &= (2\cdot i + 1)\cdot (2\cdot j + 1)\\
m\cdot n &= 4\cdot i\cdot j + 2\cdot i + 2\cdot j + 1\\
m\cdot n &= 2\cdot(2\cdot i\cdot j + i + j) + 1\\
\end{split}
\end{equation}
This is an odd number of the form $2\cdot k + 1$ where $k = 2\cdot i\cdot j + i + j$. 
So the contrapositive is proved and hence the statement is proved -- as required.

\end{enumerate}

\subsection{Core Exercises}

\begin{enumerate}

\item Characterise those integers $d$ and $n$ such that:

\begin{enumerate}

\item $0|n$

$n = 0$

\item $d|0$

$d \in \mathbb{Z}$

\end{enumerate}

\item Let $k$, $m$, $n$ be integers with $k$ positive. Show that:
\begin{equation}
\begin{split}
(k\cdot m)|(k\cdot n) &\Longleftrightarrow m|n\\
\end{split}
\end{equation}

\begin{center}

($\Longrightarrow$)

\begin{equation}
\begin{split}
(k\cdot m)&|(k\cdot n)\\
k\cdot m \cdot i &= k\cdot n \text{ for some $i$}\\
m \cdot i &= n\\
\therefore m &| n \text{ as required}
\end{split}
\end{equation}

($\Longleftarrow$)

\begin{equation}
\begin{split}
m&|n\\
m\cdot i &= n\\
k\cdot m\cdot i &= k\cdot n\\
(k\cdot m) \cdot i &= (k\cdot n)\\
\therefore (k\cdot m)&|(k\cdot n)\text{ as required}\\
\end{split}
\end{equation}

And so the statement is proved.
\end{center}

\item Prove or disprove that: For all natural numbers $n$, $2|2^n$.

$n$ is a natural number. So $n \geqslant 1$. So $n - 1 \geqslant 0$.\\
Hence $2^{n - 1} \in \mathbb{Z}^+$.\\
\begin{equation}
\begin{split}
2 \cdot (2^{n - 1}) &= 2^n\\
2^{(n-1)} \in \mathbb{Z}^+\\
\therefore 2&|2^n\text{ as required}
\end{split}
\end{equation}
Hence $2|2^n$.

The submission said this statement was true and was based on the \textbf{wrong} 
belief that $0 \not\in \mathbb{N}$. The below proof is correct taking $0\in \mathbb{N}$.

Disproof by counter example: Let $n$ = 0.
\begin{equation}
\begin{split}
2^0 &= 1\\
2 \not| \text{ } 1\\
\end{split}
\end{equation}
So the statement is disproved.

\item Show that for all integers $l$, $m$, $n$,
\begin{equation}
\begin{split}
l|m \wedge m|n \Longrightarrow l|n\\
\end{split}
\end{equation}
\begin{equation}
\begin{split}
a\cdot l &= m\\
b\cdot m &= n\\
a\cdot (b \cdot l) &= n\\
(a\cdot b)\cdot l &= n\\
\therefore l &| n\\
\end{split}
\end{equation}

\item Find a counterexample to the statement: For all positive integers $k$, $m$, $n$, 
\begin{equation}
\begin{split}
(m|k \wedge n|k) \Longrightarrow (m\cdot n)|k\\
\end{split}
\end{equation}

Let $m = 4$, $n = 6$ and $k = 12$.\\
$4 | 12 \wedge 6 | 12$\\
So $m | k \wedge n | k$\\
But $24 \not| \text{ }12$.\\
Hence this is a counterexample to the statement so the statement is disproved.

\item Prove that for all integers $d$, $k$, $l$, $m$, $n$,

\begin{enumerate}

\item $d|m \wedge d|n \Longrightarrow d|(m + n)$

\begin{equation}
\begin{split}
d &| m \\
i\cdot d &= m\\
d &| n\\
j\cdot d &= n\\
i\cdot d + j\cdot d &= m + n\\
(i + j)\cdot d &= (m + n)\\
\therefore d &| (m + n)\\
\end{split}
\end{equation}
So the statement is proved as required.

\item $d|m \Longrightarrow d|k\cdot m$

\begin{equation}
\begin{split}
d &| m\\
i\cdot d &= m\\
k\cdot i\cdot d &= k\cdot m\\
(k\cdot i)\cdot d &= k\cdot m\\
\therefore d&|(k\cdot m) \text{ as required}
\end{split}
\end{equation}

\item $d|m \wedge d|n \Longrightarrow d|(k\cdot m + l\cdot n)$

From part (b): $d|m \Longrightarrow d|(k\cdot m)$.\\
So $d|m \wedge d|n \Longrightarrow d|(k\cdot m) \wedge (l\cdot n)$.

From part (a): $d|m \wedge d|n \Longrightarrow d|(m+n)$.\\
So $d|(k\cdot m) \wedge d|(l\cdot n) \Longrightarrow d|(k\cdot m + l\cdot n)$ as required.

\end{enumerate}

\item Prove that for all integers $n$,
\begin{equation}
\begin{split}
30 | n \Longleftrightarrow (2|n \wedge 3|n \wedge 5|n)\\
\end{split}
\end{equation}

If:

\begin{equation}
\begin{split}
30|n\\
30\cdot k &= n\\
2\cdot (15\cdot k) &= n\\
\therefore 2|n \text{ as required}\\
3\cdot (10\cdot k) &= n\\
\therefore 3|n \text{ as required}\\
5\cdot (6\cdot k) &= n\\
\therefore 5|n \text{ as required}\\
\end{split}
\end{equation}

Only if:

If $a | c$ and $b | c$ and $b$ and $c$ are coprime: then $a\cdot b | c$.

Since $2$, $3$ and $5$ are all coprime:\\
\begin{equation}
\begin{split}
2 | n \wedge 3 | n \wedge 5 | n 
&\Longrightarrow (2\cdot 3\cdot 5)|n\\
&\Longrightarrow 30|n \text{ as required}\\
\end{split}
\end{equation}

\item Show that for all integers $m$ and $n$,
\begin{equation}
\begin{split}
(m|n \wedge n|m)\Longrightarrow (m=n\cup m=-n)\\
\end{split}
\end{equation}

\begin{equation}\label{mndivideeachother1}
\begin{split}
m &| n\\
k\cdot m &= n\\
\end{split}
\end{equation}

\begin{equation}\label{mndivideeachother2}
\begin{split}
n &| m\\
c\cdot n &= m\\
\end{split}
\end{equation}

Combining (\ref{mndivideeachother1}) and (\ref{mndivideeachother1}) gives:

\begin{equation}\label{mndivideeachother2}
\begin{split}
k\cdot c\cdot n &= n\\
k\cdot c &= 1\\
c &= \frac{1}{k}\\
\end{split}
\end{equation}

However, since both $c$ and $k$ are integers, this means that either 
$(c=1 \wedge k=1) \cup (c=-1 \wedge k=-1)$. \\
So $(n = m) \cup (n = -m)$ as required.

\item Prove or disprove that: For all positive integers $k$, $m$, $n$,
\begin{equation}
\begin{split}
k|(m \cdot n) \Longrightarrow k|m \cup k|n\\
\end{split}
\end{equation}

Disproof by counterexample:

Let $k = 6$, $m = 3$ and $n = 4$.\\
$6|12$ so $k|(m\cdot n)$.\\
However, $6\not|\text{ }3$ and $6\not|\text{ }4$.\\
So the statement is disproved by a counterexample.

\item Let $P(m)$ be a statement for $m$ ranging over the natural numbers, 
and consider the following derived statemets (with $n$ also ranging over the 
natural numbers):
\begin{equation}
\begin{split}
P^\#(n)\triangleq \forall k \in \mathbb{N}. 0 \leqslant k \leqslant n \Longrightarrow P(k)\\
\end{split}
\end{equation}

\begin{enumerate}

\item Show that, for all natural numbers $\ell$, $P^\#(\ell)\Longrightarrow P(\ell)$\\

\begin{equation}\label{phashnimpliesp}
\begin{split}
P^\#(n)&\triangleq \forall k \in \mathbb{N}. 0 \leqslant k \leqslant n \Longrightarrow P(k)\\
P^\#(n)&= (\forall k \in \mathbb{N}. 0\leqslant k \leqslant (n - 1) \Longrightarrow P(k)) \wedge P(n)\\
P^\#(n)&= P^\#(n - 1)\wedge P(n)\\
\therefore P^\#(n) &\Longrightarrow P(n)\text{ as required}\\
\end{split}
\end{equation}

\item Exhibit a concrete statement $P(m)$ and a specific natural number $n$ for which 
the following statement does \textit{not} hold:
\begin{equation}
\begin{split}
P(n) \Longrightarrow P^\#(n)\\
\end{split}
\end{equation}

Let $P(n) \triangleq (\exists k \in \mathbb{N}. n = 2\cdot k)$.

If $n = 2$ the the statement above does not hold (since $P(n)$ is true but $P^\#(n)$ is 
not true.

\item Prove the following:

\begin{itemize}

\item $P^\#(0) \Longleftrightarrow P(0)$

\begin{equation}
\begin{split}
P^\#(n)&\triangleq \forall k \in \mathbb{N}. 0 \leqslant k \leqslant n \Longrightarrow P(k)\\
\therefore P^\#(0)&\triangleq \forall k \in \mathbb{N}. 0 \leqslant k \leqslant 0 \Longrightarrow P(k)\\
P^\#(0)&\triangleq P(0)\\
\end{split}
\end{equation}

So $P^\#(0)$ is equivalent to $P(0)$.\\
Hence $P^\#(0) \Longleftrightarrow P(0)$ as required.

\item $\forall n \in \mathbb{N}. (P^\#(n) \Longrightarrow P\#(n+1)) 
\Longleftrightarrow (P^\#(n) \Longrightarrow P(n+1))$\\

($\Longrightarrow$)

\begin{equation}
\begin{split}
&\text{ } P^\#(n) \Longrightarrow P^\#(n + 1)\\
=&\text{ } P^\#(n) \Longrightarrow P^\#(n + 1)\Longrightarrow P(n + 1)\text{ using (\ref{phashnimpliesp})}\\
=&\text{ } P^\#(n) \Longrightarrow P(n+1) \text{ as required}\\
\end{split}
\end{equation}

($\Longleftarrow$)

\begin{center}
\begin{equation}\label{phashtop}
\begin{split}
P^\#(n+1)&\triangleq \forall k \in \mathbb{N}. 0 \leqslant k \leqslant n+1 \Longrightarrow P(k)\\
P^\#(n+1)&= \forall k \in \mathbb{N}. 0 \leqslant k < n \Longrightarrow P(k) \wedge P(n+1)\\
\therefore P^\#(n + 1) &= P^\#(n) \wedge P(n + 1)\\
\end{split}
\end{equation}

\begin{equation}
\begin{split}
& \text{ }P^\#(n) \Longrightarrow P(n+1)\\
=& \text{ }P^\#(n) \Longrightarrow (P^\#(n) \wedge P(n+1))\\
=& \text{ }P^\#(n) \Longrightarrow P^\#(n + 1)\text{ as required using (\ref{phashtop})}\\
\end{split}
\end{equation}
\end{center}

\item $(\forall m \in \mathbb{N}. P^\#(m))\Longleftrightarrow (\forall m  \in\mathbb{N}.P(m))$

\begin{center}
($\Longrightarrow$)

\begin{equation}
\begin{split}
P^\#(n) &\Longrightarrow P(n)\text{ using \ref{phashnimpliesp}}\\
\therefore (\forall m \in \mathbb{N}. P^\#(m)) &\Longrightarrow (\forall m \in \mathbb{N}. P(m)) \text{ as required}\\
\end{split}
\end{equation}

($\Longleftarrow$)

\begin{equation}
\begin{split}
&\forall m \in\mathbb{N}.P(m)\\
\therefore\text{ } &\forall m, k \in \mathbb{N}.\text{ } 0 \leqslant k \leqslant m \Longrightarrow P(m)\\
\therefore\text{ } & \forall m \in \mathbb{N} P^\#(m)\\
\text{Since $m$ is arbitrary: }&\forall m \in \mathbb{N}. P^\#(m)\text{ as required}\\
\end{split}
\end{equation}

\begin{equation}
\begin{split}
\end{split}
\end{equation}

\end{center}

\end{itemize}

\end{enumerate}

\end{enumerate}

\subsection{Optional Exercises}

\begin{enumerate}

\item A series of questions about the properties and relationships of triangular and 
square numbers (adapted from David Burton).

\begin{itemize}

\item A natural number is said to be \textit{triangular} if it is of the form $\Sigma^k_{i=0}i=0+1+...+k$, for 
some natural $k$. For example, the first three triangular numbers are $t_0=0$, $t_1=1$ and $t_2=3$.

Find the next three triangular numbers $t_3$, $t_4$ and $t_5$.

$t_3 = 6$, $t_4 = 10$, $t_5 = 15$

\item Find a formula for the $k^{th}$ triangular number $t_k$.

$t_k = \frac{k}{2}\cdot(k + 1)$

\item A natural number is said to be \textit{square} if it is of the form $k^2$ for some natural 
number $k$.

Show that $n$ is triangular iff $8\cdot n + 1$ is a square. (Plutarch, circ. 100BC)

If:

Let $n$ be a number such that $8\cdot n + 1$ is a square number.\\
Let $k^2 = 8\cdot n + 1$\\
Since $8\cdot n + 1$ is a number of the form $2\cdot i + 1$ where $i = (4\cdot n)$; 
$8\cdot n + 1$ is odd. As $8\cdot n + 1$ is odd: $k$ must be odd.\\
So $k = 2\cdot j + 1$ for some $j$.\\
\begin{equation}
\begin{split}
8\cdot n + 1 &= (2\cdot j + 1)^2\\
8\cdot n + 1 &= 4\cdot j^2 + 4\cdot j + 1\\
8\cdot n 	 &= 4\cdot j^2 + 4\cdot j\\
n			 &= \frac{1}{2}(j^2 + j)\\
n			 &= \frac{j}{2}(j + 1) \text{ as required}\\
\end{split}
\end{equation}

Only if:

Let n be a triangle number. So $n = \frac{k}{2}\cdot(k + 1)$ for some k.
\begin{equation}
\begin{split}
8\cdot n + 1 &= 8\cdot \frac{k}{2}\cdot(k + 1) + 1\\
			 &= 4\cdot k\cdot (k + 1) + 1\\
			 &= 4\cdot k^2 + 4\cdot k + 1\\
			 &= (2\cdot k + 1)^2\\
\end{split}
\end{equation}
So if $n$ is a trangle number then $8\cdot n + 1$ is a square number.

Hence $n$ is triangular iff $8\cdot n + 1$ is a square number.

\item Show that the sum of every two consecutive triangular numbers is a square. (Nicomachus, circ. 100BC)

\begin{equation}
\begin{split}
t_k + t_{k + 1} &= \frac{k}{2}\cdot (k + 1) + \frac{k + 1}{2}\cdot(k + 2)\\
				&= \frac{k + 1}{2}\cdot k + \frac{k + 1}{2}\cdot(k + 2)\\
				&= \frac{k + 1}{2}\cdot(2\cdot k + 2)\\
				&= (k + 1)\cdot (k + 1)\\
				&= (k + 1)^2\\
\end{split}
\end{equation}
So the sum of two consecutive triangular numbers is square. As required.

\item Show that, for all natural numbers $n$, if $n$ is triangular, then so are $9\cdot n + 1$, $25\cdot n + 3$, 
$49\cdot n + 6$ and $81\cdot n + 10$. (Euler, 1775)

$n$ is triangular. So $n = \frac{k}{2}\cdot(k + 1)$ for some k.
\begin{equation}
\begin{split}
9\cdot n + 1 &= 9\cdot \frac{k}{2}\cdot(k + 1) + 1\\
			 &=	\frac{9\cdot k^2}{2} + \frac{9\cdot k}{2} + 1\\
			 &= \frac{1}{2}\cdot (9\cdot k^2 + 9\cdot k + 2)\\
			 &=	\frac{1}{2}\cdot (3\cdot k + 1)\cdot (3\cdot k + 2)\\
			 &= \frac{3\cdot k + 1}{2}\cdot ((3\cdot k + 1) + 1)\\
\end{split}
\end{equation}
So if $n$ is a triangular number then so is $9\cdot n + 1$.

\begin{equation}
\begin{split}
25\cdot n + 3 &= 25\cdot \frac{k}{2}\cdot(k + 1) + 3\\
			  &= \frac{25\cdot k^2}{2} + \frac{25\cdot k}{2} + 3\\
			  &= \frac{1}{2}\cdot (25\cdot k^2 + 25\cdot k + 6)\\
			  &= \frac{1}{2}\cdot (5\cdot k + 2)\cdot (5\cdot k + 3)\\
			  &= \frac{5\cdot k + 2}{2}\cdot ((5\cdot k + 2) + 1)\\
\end{split}
\end{equation}
So if $n$ is a triangular number then so is $25\cdot n + 3$.

\begin{equation}
\begin{split}
49\cdot n + 6 &= 49\cdot \frac{k}{2}\cdot(k + 1) + 6\\
		 	  &= \frac{49\cdot k^2}{2} + \frac{49\cdot k}{2} + 6\\
			  &= \frac{1}{2}\cdot (49\cdot k^2 + 49\cdot k + 12)\\
			  &= \frac{1}{2}\cdot (7\cdot k + 3)\cdot (7\cdot k + 4)\\
			  &= \frac{7\cdot k + 3}{2}\cdot ((7\cdot k + 3) + 1)\\
\end{split}
\end{equation}
So if $n$ is a triangular number then so is $49\cdot n + 6$.

\begin{equation}
\begin{split}
81\cdot n + 10 &= 81\cdot \frac{k}{2}\cdot(k + 1) + 10\\
			   &= \frac{81\cdot k^2}{2} + \frac{81\cdot k}{2} + 10\\
			   &= \frac{1}{2}\cdot (81\cdot k^2 + 81\cdot k + 20)\\
			   &= \frac{1}{2}\cdot (9\cdot k + 4)\cdot (9\cdot k + 5)\\
			   &= \frac{9\cdot k + 4}{2}\cdot ((9\cdot k + 4) + 1)\\
\end{split}
\end{equation}
So if $n$ is a triangular number then so is $81\cdot n + 10$.

Hence the statement is proved.

\item Prove the generalisation: For all $n$ and $k$ natural numbers, there exists a natural number 
$q$ such that $(2\cdot n + 1)^2\cdot t_k+t_n=t_q$. (Jordan 1991, attributed to Euler)

\begin{equation}
\begin{split}
&(2\cdot n + 1)^2\cdot t_k+t_n\\
=& (2\cdot n + 1)^2\cdot \frac{k}{2}\cdot(k + 1) + \frac{n}{2}\cdot(n + 1)\\
=& (4\cdot n^2 + 4\cdot n + 1)\cdot \frac{k}{2}\cdot(k + 1) + \frac{n}{2}\cdot(n + 1)\\
=& \frac{1}{2}((4\cdot n^2\cdot k + 4\cdot n \cdot k + k)\cdot(k + 1) + n^2 + n))\\
=& \frac{1}{2}(4\cdot n^2\cdot k^2 + 4\cdot n \cdot k^2 + k^2 + 4\cdot n^2\cdot k + 4\cdot n \cdot k + k + n^2 + n)\\
=& \frac{1}{2}(2\cdot n \cdot k + n + k)\cdot((2\cdot n \cdot k + n + k) + 1)\\
=& \frac{(2\cdot n \cdot k + n + k)}{2}\cdot((2\cdot n \cdot k + n + k) + 1)\\
=& \frac{q}{2}\cdot(q + 1) \text{ where } q = 2\cdot n \cdot k + n + k\\
\end{split}
\end{equation}
So for each $n$ and $k$, there exists an integer $q$ such that $(2\cdot n + 1)^2\cdot t_k+t_n=t_q$ as required.

\end{itemize}

\item Let $P(x)$ be a predicate on a variable $x$ and let $Q$ be a statement not mentioning $x$. Show that 
the following equivalence holds:
\begin{equation}
\begin{split}
((\exists x. P(x))\Longrightarrow Q)\Longleftrightarrow(\forall x.(P(x)\Longrightarrow Q))\\
\end{split}
\end{equation}

($\Longrightarrow$)

$Q$ is independent of $x$. Since $P(x)$ is dependent only on $x$ and $Q$ is independent 
of $x$; $Q$ is independent of $P(x)$.

So if there exists a single case such that $(P(x) \Longrightarrow Q)$, then Q is always true 
(since Q is independent of P(x)).\\
So $(\forall P(x) \Longrightarrow Q)$. As required.

($\Longleftarrow$)

Since $(\forall x.(P(x)\Longrightarrow Q))$, $P(x)\Longrightarrow Q$ for at least one $x$. 
So $((\exists x. P(x))\Longrightarrow Q)$ as required.

\end{enumerate}

\end{document}