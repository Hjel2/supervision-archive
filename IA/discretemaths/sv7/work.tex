\newcommand{\svrname}{Mr Jakub Perlin}
\newcommand{\jkfside}{oneside}
\newcommand{\jkfhanded}{right}

\newcommand{\studentname}{Harry Langford}
\newcommand{\studentemail}{hjel2@cam.ac.uk}

\documentclass[10pt,\jkfside,a4paper]{article}

\newcommand{\svcourse}{CST Part IA: Software Engineering and Security}
\newcommand{\svnumber}{1}
\newcommand{\svvenue}{Microsoft Teams}
\newcommand{\svdate}{2022-05-11}
\newcommand{\svtime}{15:00}
\newcommand{\svuploadkey}{CBd13xmL7PC1zqhNIoLdTiYUBnxZhzRAtJxv/ytRdM1r7qIfwMsxeVwM/pPcIo8l}

\newcommand{\svrname}{Dr Sam Ainsworth}
\newcommand{\jkfside}{oneside}
\newcommand{\jkfhanded}{yes}

\newcommand{\studentname}{Harry Langford}
\newcommand{\studentemail}{hjel2@cam.ac.uk}

% DO NOT add \usepackage commands here.  Place any custom commands
% into your SV work files.  Anything in the template directory is
% likely to be overwritten!

\usepackage{fancyhdr}

\usepackage{lastpage}       % ``n of m'' page numbering
\usepackage{lscape}         % Makes landscape easier

\usepackage{verbatim}       % Verbatim blocks
\usepackage{listings}       % Source code listings
\usepackage{epsfig}         % Embed encapsulated postscript
\usepackage{array}          % Array environment
\usepackage{qrcode}         % QR codes
\usepackage{enumitem}       % Required by Tom Johnson's exam question header

\usepackage{hhline}         % Horizontal lines in tables
\usepackage{siunitx}        % Correct spacing of units
\usepackage{amsmath}        % American Mathematical Society
\usepackage{amssymb}        % Maths symbols
\usepackage{amsthm}         % Theorems

\usepackage{ifthen}         % Conditional processing in tex

\usepackage[top=3cm,
            bottom=3cm,
            inner=2cm,
            outer=5cm]{geometry}

% PDF metadata + URL formatting
\usepackage[
            pdfauthor={\studentname},
            pdftitle={\svcourse, SV \svnumber},
            pdfsubject={},
            pdfkeywords={9d2547b00aba40b58fa0378774f72ee6},
            pdfproducer={},
            pdfcreator={},
            hidelinks]{hyperref}


% DO NOT add \usepackage commands here.  Place any custom commands
% into your SV work files.  Anything in the template directory is
% likely to be overwritten!

\usepackage{fancyhdr}

\usepackage{lastpage}       % ``n of m'' page numbering
\usepackage{lscape}         % Makes landscape easier

\usepackage{verbatim}       % Verbatim blocks
\usepackage{listings}       % Source code listings
\usepackage{graphicx}
\usepackage{float}
\usepackage{epsfig}         % Embed encapsulated postscript
\usepackage{array}          % Array environment
\usepackage{qrcode}         % QR codes
\usepackage{enumitem}       % Required by Tom Johnson's exam question header

\usepackage{hhline}         % Horizontal lines in tables
\usepackage{siunitx}        % Correct spacing of units
\usepackage{amsmath}        % American Mathematical Society
\usepackage{amssymb}        % Maths symbols
\usepackage{amsthm}         % Theorems

\usepackage{ifthen}         % Conditional processing in tex

\usepackage[top=3cm,
            bottom=3cm,
            inner=2cm,
            outer=5cm]{geometry}

% PDF metadata + URL formatting
\usepackage[
            pdfauthor={\studentname},
            pdftitle={\svcourse, SV \svnumber},
            pdfsubject={},
            pdfkeywords={9d2547b00aba40b58fa0378774f72ee6},
            pdfproducer={},
            pdfcreator={},
            hidelinks]{hyperref}

\renewcommand{\headrulewidth}{0.4pt}
\renewcommand{\footrulewidth}{0.4pt}
\fancyheadoffset[LO,LE,RO,RE]{0pt}
\fancyfootoffset[LO,LE,RO,RE]{0pt}
\pagestyle{fancy}
\fancyhead{}
\fancyhead[LO,RE]{{\bfseries \studentname}\\\studentemail}
\fancyhead[RO,LE]{{\bfseries \svcourse, SV~\svnumber}\\\svdate\ \svtime, \svvenue}
\fancyfoot{}
\fancyfoot[LO,RE]{For: \svrname}
\fancyfoot[RO,LE]{\today\hspace{1cm}\thepage\ / \pageref{LastPage}}
\fancyfoot[C]{\qrcode[height=0.8cm]{\svuploadkey}}
\setlength{\headheight}{22.55pt}


\ifthenelse{\equal{\jkfside}{oneside}}{

 \ifthenelse{\equal{\jkfhanded}{left}}{
  % 1. Left-handed marker, one-sided printing or e-marking, use oneside and...
  \evensidemargin=\oddsidemargin
  \oddsidemargin=73pt
  \setlength{\marginparwidth}{111pt}
  \setlength{\marginparsep}{-\marginparsep}
  \addtolength{\marginparsep}{-\textwidth}
  \addtolength{\marginparsep}{-\marginparwidth}
 }{
  % 2. Right-handed marker, one-sided printing or e-marking, use oneside.
  \setlength{\marginparwidth}{111pt}
 }

}{
 % 3. Alternating margins, two-sided printing, use twoside.
}


\setlength{\parindent}{0em}
\addtolength{\parskip}{1ex}

% Exam question headings, labels and sensible layout (courtesy of Tom Johnson)
\setlist{parsep=\parskip, listparindent=\parindent}
\newcommand{\examhead}[3]{\section{#1 Paper #2 Question #3}}
\newenvironment{examquestion}[3]{
\examhead{#1}{#2}{#3}\setlist[enumerate, 1]{label=(\alph*)}\setlist[enumerate, 2]{label=(\roman*)}
\marginpar{\href{https://www.cl.cam.ac.uk/teaching/exams/pastpapers/y#1p#2q#3.pdf}{\qrcode{https://www.cl.cam.ac.uk/teaching/exams/pastpapers/y#1p#2q#3.pdf}}}
\marginpar{\footnotesize \href{https://www.cl.cam.ac.uk/teaching/exams/pastpapers/y#1p#2q#3.pdf}{https://www.cl.cam.ac.uk/\\teaching/exams/pastpapers/\\y#1p#2q#3.pdf}}
}{}


\begin{document}

\section*{9 On bijections}

\subsection*{9.1 Basic exercises}

\begin{enumerate}

\item 

\begin{enumerate}

\item Define a function that has (i) none, (ii) exactly one, and (iii) more than one retraction.

\begin{enumerate}[label=(\roman*)]

\item



\item



\item



\end{enumerate}

\item Define a function that has (i) none, (ii) exactly one, and (iii) more than one section.

\begin{enumerate}[label=(\roman*)]

\item 



\item



\item



\end{enumerate}

\end{enumerate}

\item Let $n$ be an integer

\begin{enumerate}

\item How many sections are there for the absolute-value map $x \mapsto |x|: [-n \dots n] \rightarrow [0 \dots n]$?



\item How many retractions are there for the exponential map $x \mapsto 2^x: [0 \dots n] \rightarrow [0 \dots 2^n]$?



\end{enumerate}

\item Give an example of two sets $A$ and $B$ and a function $f: A \rightarrow B$ such that $f$ has a retraction 
but no section. Explain how you know that $f$ has these properties.



\item Prove that the identity function is a bijection and that the composition of bijections is a bijection.



\item For $F: A \rightarrow B$, prove that if there are $g, h: B \rightarrow A$ such that $g \circ f = \text{id}_A$ and 
$f \circ h = \text{id}_B$ then $g = h$. Conclude as a corollary that, whenever it exists, the inverse of a function is unique.



\end{enumerate}

\subsection*{9.2 Core exercises}

\begin{enumerate}

\item We say that two functions $s: A \rightarrow B$ and $r: B \rightarrow A$ are a section-retraction pair whenever 
$r \circ s = \text{id}_A$; and that a function $e: B \rightarrow B$ is an idempotent whenever $e \circ e = e$. This question 
demonstrates that section-retraction pairs and idempotents are closely connected: any section-retraction pair 
gives rise to an idempotent function, and any idempotent function can be split into a section-retraction pair.

\begin{enumerate}

\item Let $f: C \rightarrow D$ and $g: D \rightarrow C$ be functions such that $f \circ g \circ f = f$.

\begin{enumerate}[label=(\roman*)]

\item Can you conclude that $f \circ g$ is idempotent? What about $g \circ f$? Justify your answers.



\item Define a map $g'$ using $f$ and $g$ that satisfies both
\begin{equation}
f \circ g' \circ f = f \text{ and } g' \circ f' \circ g' = g'
\end{equation}



\item Show that if $s: A \rightarrow B$ and $r: B \rightarrow A$ are a section-retraction pair then 
the composite $s \circ r: B \rightarrow B$ is idempotent.



\item Show that for every idempotent $e: B \rightarrow B$ there exists a set $A$ (called a retract of $B$) and 
a section-retraction pair $s: A \rightarrow B$ and $r: B \rightarrow A$ such that $s \circ r = e$.



\end{enumerate}

\end{enumerate}

\end{enumerate}

\section*{10 On equivalence relations}

\subsection*{10.1 Basic exercises}

\begin{enumerate}

\item Prove that the isomorphic relation $\cong$ between sets is an equivalence relation.



\item Prove that the identity relation id$_A$ on a set $A$ is an equivalence relation, and that $A/\text{id}_A \cong A$.



\item Show that, for a positive integer $m$, the relation $\equiv_m on \mathbb{Z}$ given by 
\begin{equation}
x \equiv_m y \Longleftrightarrow x \equiv y (\text{mod } m)
\end{equation}
is an equivalence relation. What are the equivalence classes of this relation?



\item Show that the relation $\equiv$ on $\mathbb{Z} \times \mathbb{Z}^+$ given by
\begin{equation}
(a, b) \equiv (x, y) \Longleftrightarrow a \cdot y = x \cdot b
\end{equation}
is an equivalence relation. What are the equivalence classes of this relation?



\end{enumerate}

\subsection*{10.2 Core exercises}

\begin{enumerate}

\item Let $E_1$ and $E_2$ be two equivalence relations on a set $A$. Either prove or disprove 
the following statements

\begin{enumerate}

\item $E_1 \cup E_2$ is an equivalence relation on $A$.



\item $E_1 \cap E_2$ is an equivalence relation on $A$.



\end{enumerate}

\item For an equivalence relation $E$ on a set $A$, show that $[a_1]_E = [a_2]_E$ iff $a_1 E a_2$, 
where 
\begin{equation}
[a]_E = \{x \in A | x E a\}.
\end{equation}



\item For a function $f: A \rightarrow B$ define a relation $\equiv_f$ on $A$ by the rule: 
for all $a, a' \in A$,
\begin{equation}
a \equiv_f a' \Longleftrightarrow f(a) = f(a')
\end{equation}

\item Show that for every function $f: A \rightarrow B$, the relation $\equiv_f$ is an equivalence 
relation on $A$.



\item Prove that every equivalence relation $E$ in a set $A$ is equal to $\equiv_q$, where $q: A \twoheadrightarrow A/E$ 
is the quotient function $q(a) = [a]_E$.



\item Prove that for every surjection $f: A \twoheadrightarrow B$,
\begin{equation}
B \cong (A/ \equiv_f)
\end{equation}

\end{enumerate}

\end{document}