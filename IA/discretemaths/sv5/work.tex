\newcommand{\svrname}{Mr Jakub Perlin}
\newcommand{\jkfside}{oneside}
\newcommand{\jkfhanded}{right}

\newcommand{\studentname}{Harry Langford}
\newcommand{\studentemail}{hjel2@cam.ac.uk}

\documentclass[10pt,\jkfside,a4paper]{article}

\newcommand{\svcourse}{CST Part IA: Software Engineering and Security}
\newcommand{\svnumber}{1}
\newcommand{\svvenue}{Microsoft Teams}
\newcommand{\svdate}{2022-05-11}
\newcommand{\svtime}{15:00}
\newcommand{\svuploadkey}{CBd13xmL7PC1zqhNIoLdTiYUBnxZhzRAtJxv/ytRdM1r7qIfwMsxeVwM/pPcIo8l}

\newcommand{\svrname}{Dr Sam Ainsworth}
\newcommand{\jkfside}{oneside}
\newcommand{\jkfhanded}{yes}

\newcommand{\studentname}{Harry Langford}
\newcommand{\studentemail}{hjel2@cam.ac.uk}

% DO NOT add \usepackage commands here.  Place any custom commands
% into your SV work files.  Anything in the template directory is
% likely to be overwritten!

\usepackage{fancyhdr}

\usepackage{lastpage}       % ``n of m'' page numbering
\usepackage{lscape}         % Makes landscape easier

\usepackage{verbatim}       % Verbatim blocks
\usepackage{listings}       % Source code listings
\usepackage{epsfig}         % Embed encapsulated postscript
\usepackage{array}          % Array environment
\usepackage{qrcode}         % QR codes
\usepackage{enumitem}       % Required by Tom Johnson's exam question header

\usepackage{hhline}         % Horizontal lines in tables
\usepackage{siunitx}        % Correct spacing of units
\usepackage{amsmath}        % American Mathematical Society
\usepackage{amssymb}        % Maths symbols
\usepackage{amsthm}         % Theorems

\usepackage{ifthen}         % Conditional processing in tex

\usepackage[top=3cm,
            bottom=3cm,
            inner=2cm,
            outer=5cm]{geometry}

% PDF metadata + URL formatting
\usepackage[
            pdfauthor={\studentname},
            pdftitle={\svcourse, SV \svnumber},
            pdfsubject={},
            pdfkeywords={9d2547b00aba40b58fa0378774f72ee6},
            pdfproducer={},
            pdfcreator={},
            hidelinks]{hyperref}


% DO NOT add \usepackage commands here.  Place any custom commands
% into your SV work files.  Anything in the template directory is
% likely to be overwritten!

\usepackage{fancyhdr}

\usepackage{lastpage}       % ``n of m'' page numbering
\usepackage{lscape}         % Makes landscape easier

\usepackage{verbatim}       % Verbatim blocks
\usepackage{listings}       % Source code listings
\usepackage{graphicx}
\usepackage{float}
\usepackage{epsfig}         % Embed encapsulated postscript
\usepackage{array}          % Array environment
\usepackage{qrcode}         % QR codes
\usepackage{enumitem}       % Required by Tom Johnson's exam question header

\usepackage{hhline}         % Horizontal lines in tables
\usepackage{siunitx}        % Correct spacing of units
\usepackage{amsmath}        % American Mathematical Society
\usepackage{amssymb}        % Maths symbols
\usepackage{amsthm}         % Theorems

\usepackage{ifthen}         % Conditional processing in tex

\usepackage[top=3cm,
            bottom=3cm,
            inner=2cm,
            outer=5cm]{geometry}

% PDF metadata + URL formatting
\usepackage[
            pdfauthor={\studentname},
            pdftitle={\svcourse, SV \svnumber},
            pdfsubject={},
            pdfkeywords={9d2547b00aba40b58fa0378774f72ee6},
            pdfproducer={},
            pdfcreator={},
            hidelinks]{hyperref}

\renewcommand{\headrulewidth}{0.4pt}
\renewcommand{\footrulewidth}{0.4pt}
\fancyheadoffset[LO,LE,RO,RE]{0pt}
\fancyfootoffset[LO,LE,RO,RE]{0pt}
\pagestyle{fancy}
\fancyhead{}
\fancyhead[LO,RE]{{\bfseries \studentname}\\\studentemail}
\fancyhead[RO,LE]{{\bfseries \svcourse, SV~\svnumber}\\\svdate\ \svtime, \svvenue}
\fancyfoot{}
\fancyfoot[LO,RE]{For: \svrname}
\fancyfoot[RO,LE]{\today\hspace{1cm}\thepage\ / \pageref{LastPage}}
\fancyfoot[C]{\qrcode[height=0.8cm]{\svuploadkey}}
\setlength{\headheight}{22.55pt}


\ifthenelse{\equal{\jkfside}{oneside}}{

 \ifthenelse{\equal{\jkfhanded}{left}}{
  % 1. Left-handed marker, one-sided printing or e-marking, use oneside and...
  \evensidemargin=\oddsidemargin
  \oddsidemargin=73pt
  \setlength{\marginparwidth}{111pt}
  \setlength{\marginparsep}{-\marginparsep}
  \addtolength{\marginparsep}{-\textwidth}
  \addtolength{\marginparsep}{-\marginparwidth}
 }{
  % 2. Right-handed marker, one-sided printing or e-marking, use oneside.
  \setlength{\marginparwidth}{111pt}
 }

}{
 % 3. Alternating margins, two-sided printing, use twoside.
}


\setlength{\parindent}{0em}
\addtolength{\parskip}{1ex}

% Exam question headings, labels and sensible layout (courtesy of Tom Johnson)
\setlist{parsep=\parskip, listparindent=\parindent}
\newcommand{\examhead}[3]{\section{#1 Paper #2 Question #3}}
\newenvironment{examquestion}[3]{
\examhead{#1}{#2}{#3}\setlist[enumerate, 1]{label=(\alph*)}\setlist[enumerate, 2]{label=(\roman*)}
\marginpar{\href{https://www.cl.cam.ac.uk/teaching/exams/pastpapers/y#1p#2q#3.pdf}{\qrcode{https://www.cl.cam.ac.uk/teaching/exams/pastpapers/y#1p#2q#3.pdf}}}
\marginpar{\footnotesize \href{https://www.cl.cam.ac.uk/teaching/exams/pastpapers/y#1p#2q#3.pdf}{https://www.cl.cam.ac.uk/\\teaching/exams/pastpapers/\\y#1p#2q#3.pdf}}
}{}


\begin{document}

\section*{5. On sets}

\subsection*{5.1 Basic exercises}

\begin{enumerate}

\item Prove that $\subseteq$ is a partial order, that is, it is:

\begin{enumerate}

\item reflexive: $\forall$ sets $A$, $A \subseteq A$

We shall prove that every element in $A$ is also in $A$.

\begin{equation}
\begin{split}
\forall a \in A: a &\in A \Longleftrightarrow\\
A &\subseteq A \text{ as required}\\
\end{split}
\end{equation}

\item transistive: $\forall$ sets $A$, $B$, $C$. $(A \subseteq B \wedge B \subseteq C) \Longrightarrow A \subseteq C$

We shall prove that every element in $A$ must be in $B$. Since every element in $B$ is in $C$: every element in $A$ is also in $C$.

Assume $A \subseteq B \wedge B \subseteq C$
\begin{equation}
\begin{split}
\text{by assumption: } A &\subseteq B \Longleftrightarrow\\
\forall a \in A: a &\in B\\
\text{by assumption: } B &\subseteq C \Longleftrightarrow\\
\forall b \in B: b &\in C\\
\therefore \forall a \in A: a &\in B \Longrightarrow\\
\forall a \in A: a &\in C \Longrightarrow\\
A &\subseteq C\\
\end{split}
\end{equation}

\item antisymmetric: $\forall$ sets $A$, $B$. $(A \subseteq B \wedge B \subseteq A) \Longleftrightarrow A = B$

I shall prove that every $a \in A$ is also in $B$ and every $b \in B$ is also in $A$. This implies that $A$ and 
$B$ contain the same elements and hence are the same set.

\begin{equation}
\begin{split}
A &\subseteq B \Longleftrightarrow\\
\forall a \in A: a &\in B\\
B &\subseteq A \Longleftrightarrow\\
\forall b \in B: b &\in A\\
\forall a \in A: a \in B &\wedge \forall b \in B: b \in A \Longleftrightarrow\\
A &= B\\
\end{split}
\end{equation}

\end{enumerate}

\item Prove the following statements:

\begin{enumerate}

\item $\forall$ sets $A$. $\emptyset \subseteq A$

By definition if $S$ is a set:
\begin{equation}
S \subseteq A \Longleftrightarrow \forall s \in S: s \in A\\
\end{equation}
For $\emptyset$ this is vacuously true.
\begin{equation}
\begin{split}
(\emptyset \subseteq A &\Longleftrightarrow \forall s \in \emptyset: s \in A)\Longleftrightarrow\\
(\emptyset \subseteq A &\Longleftrightarrow \text{true})\Longleftrightarrow\\
\emptyset &\subseteq A\text{ as required}\\
\end{split}
\end{equation}

\item $\forall$ sets $A$. $(\forall x: x \notin A) \Longleftrightarrow A = \emptyset$

TODO

\end{enumerate}

\item Find the union, and intersection of: 

\begin{enumerate}

\item $\{1, 2, 3, 4, 5\}$ and $\{-1, 1, 3, 5, 7\}$

\begin{equation}
\{1, 2, 3, 4, 5\} \cup \{-1, 1, 3, 5, 7\} = \{-1, 1, 2, 3, 4, 5, 7\}\\
\end{equation}

\begin{equation}
\{1, 2, 3, 4, 5\} \cap \{-1, 1, 3, 5, 7\} = \{1, 3, 5\}\\
\end{equation}

\item $\{x \in \mathbb{R}| x > 7\}$ and $\{ x \in \mathbb{N}: x > 5\}$

\begin{equation}
\begin{split}
 & \{x \in \mathbb{R}: x > 7\} \cup \{ x \in \mathbb{N}: x > 5\}\\
=& \{x \in \mathbb{R}: x > 7 \vee x \in \{6, 7\}\}\\
\end{split}
\end{equation}

\begin{equation}
\begin{split}
 & \{x \in \mathbb{R}: x > 7\} \cap \{ x \in \mathbb{N}: x > 5\}\\
=& \{x \in \mathbb{N}: x > 7\}\\
\end{split}
\end{equation}

\end{enumerate}

\item Find the Cartesian product and disjoint union of $\{1, 2, 3, 4, 5\}$ and $\{-1, 1, 3, 5, 7\}$.

The Cartesian product of two sets $S$ and $T$ is $\{x: \forall s \in S, \forall t \in T: x = (s, t)\}$

For the sets $\{1, 2, 3, 4, 5\}$ and $\{-1, 1, 3, 5, 7\}$ this is equal to:
\begin{equation}
\begin{split}
&\{(1, -1), (1, 1), (1, 3), (1, 5), (1, 7), (2, -1), (2, 1), (2, 3), (2, 5), (2, 7), (3, -1), (3, 1), (3, 3), \\&(3, 5), (3, 7), (4, -1), (4, 1), (4, 3), (4, 5), (4, 7), (5, -1), (5, 1), (5, 3), (5, 5), (5, 7)\}\\
\end{split}
\end{equation}

\item Let $I = \{2, 3, 4, 5\}$ and for each $i \in I$, let $A_i = \{i, i + 1, i - 1, 2\cdot i\}$.

\begin{enumerate}

\item List the elements of all sets $A_i$ for $i \in I$

\begin{equation}
\begin{split}
A_2 &= \{1, 2, 3, 4\}\\
A_3 &= \{2, 3, 4, 6\}\\
A_4 &= \{3, 4, 5, 8\}\\
A_5 &= \{4, 5, 6, 10\}\\
\end{split}
\end{equation}

\item Let $\{A_i| i \in I\}$ stand for $\{A_2, A_3, A_4, A_5\}$. Find $\bigcup\{A_i| i \in I\}$ and $\bigcap\{A_i| i \in I\}$.

\begin{equation}
\begin{split}
\bigcup\{A_i: i \in I\}&\{1, 2, 3, 4, 5, 6, 8, 10\}\\
\end{split}
\end{equation}

\begin{equation}
\begin{split}
\bigcap\{A_i:i \in I\}&\{4\}\\
\end{split}
\end{equation}

\end{enumerate}

\item Let $U$ be a set. For all $A, B \in \mathcal{P}(A)$, prove that:

\begin{enumerate}

\item $A^\mathsf{c} = B \Longleftrightarrow (A \cup B = U \wedge A \cap B = \emptyset)$



\item Double complement elimination: $(A^\mathsf{c})^\mathsf{c} = A$



\item The De-Morgan laws: $(A \cup B)^\mathsf{c} = A^\mathsf{c} \cap B^\mathsf{c}$ and $(A \cap B)^\mathsf{c} = A^\mathsf{c}\cup B^\mathsf{c}$



\end{enumerate}

\end{enumerate}

\subsection*{5.2 Core exercises}

\begin{enumerate}

\item Prove that for all sets $U$ and subsets $A, B \subseteq U$:

\begin{enumerate}

\item $\forall X: A \subseteq X \wedge B \subseteq X \Longleftrightarrow (A \cup B) \subseteq X)$



\item $\forall Y: Y \subseteq A \wedge Y \subseteq B \Longleftrightarrow Y \subseteq (A \cap B)$



\end{enumerate}

\item Either prove or disprove that, for all sets $A$ and $B$,

\begin{enumerate}

\item $A \subseteq B \Longrightarrow \mathcal{P}(A) \subseteq \mathcal{P}(B)$



\item $\mathcal{P}(A \cup B) \subseteq \mathcal{P}(A) \cup \mathcal{P}(B)$



\item $\mathcal{P}(A) \cup \mathcal{P}(B) \subseteq \mathcal{P}(A \cup B)$



\item $\mathcal{P}(A \cap B) \subseteq \mathcal{P}(A) \cap \mathcal{P}(B)$



\item $\mathcal{P}(A) \cap \mathcal{P}(B) \subseteq \mathcal{P}(A \cap B)$



\end{enumerate}

\item Let $U$ be a set. For all $A, B \in \mathcal{P}(U)$ prove that the following statements are equivalent.

\begin{enumerate}
\item $A \cup B = B$

\item $A \subseteq B$

\item $A \cap B = A$

\item $B^\mathsf{c} \subseteq A^\mathsf{c}$




\end{enumerate}

\item For sets $A, B, C, D$, prove or disprove at least three of the following statements:

\begin{enumerate}

\item $(A \subseteq C \wedge B \subseteq D) \Longrightarrow A \times B \subseteq C \times D$



\item $(A \cup C) \times (B \cup D) \subseteq (A \times B) \cup (C \times D)$



\item $(A \times C) \cup (B \times D) \subseteq (A \cup B) \times (C \cup D)$



\item $A \times (B \cup C) \subseteq (A \times B) \cup (A \times C)$



\item $(A \times B) \cup (A \times D) \subseteq A \times (B \cup D)$



\end{enumerate}

\item For sets $A, B, C, D$, prove or disprove at least three of the following statements:

\begin{enumerate}

\item $(A \subseteq C \wedge B \subseteq D) \Longrightarrow A \uplus B \subseteq C \uplus D$



\item $(A \cup B) \uplus C \subseteq (A \uplus C) \cup (B \uplus C)$



\item $(A \uplus C) \cup (B \uplus C) \subseteq (A \cup B) \uplus C$



\item $(A \cap B) \uplus C \subseteq (A \uplus C) \cap (B \uplus C)$



\item $(A \uplus C) \cap (B \uplus C) \subseteq (A \cap B) \uplus C$



\end{enumerate}

\item Prove the following properties of the big unions and intersections of a family of sets $\mathcal{F}\subseteq \mathcal{P}(A)$:

\begin{enumerate}

\item $\forall U \subseteq A: (\forall X \in \mathcal{F}: X \subseteq U) \Longleftrightarrow \bigcup \mathcal{F}\subseteq U$



\item $\forall L \subseteq A: (\forall X \in \mathcal{F}: L \subseteq X) \Longleftrightarrow L \subseteq \bigcap \mathcal{F}$



\end{enumerate}

\item Let $A$ be a set.

\begin{enumerate}

\item For a family $\mathcal{F} \subseteq \mathcal{P}(A)$, let $\mathcal{U} \triangleq \{ U \subseteq A| \forall S \in \mathcal{F}: S \subseteq U\}$. 
Prove that $\bigcup\mathcal{F} = \bigcap\mathcal{U}$.



\item Analogously, define the family $\mathcal{L}\subseteq \mathcal{P}(A)$ such that $\bigcap \mathcal{F} = \bigcup \mathcal{L}$. Also prove this statement.



\end{enumerate}

\end{enumerate}

\subsection*{5.3 Optional advanced exercises}

\begin{enumerate}

\item Prove that for all families of sets $\mathcal{F}_1$ and $\mathcal{F}_2$

\begin{equation}
(\bigcup\mathcal{F}_1)\cup(\bigcup\mathcal{F}_2) = \bigcup(\mathcal{F}_1\cup\mathcal{F}_2)\\
\end{equation}

State and prove tha analogous property for intersections of non-empty families of sets.

\item For a set $U$, prove that $(\mathcal{P}(U), \subseteq, \cup, \cap, U, \emptyset, (\cdot)^\mathsf{c})$ is a Boolean algebra.

\end{enumerate}

\end{document}