\newcommand{\svrname}{Mr Jakub Perlin}
\newcommand{\jkfside}{oneside}
\newcommand{\jkfhanded}{right}

\newcommand{\studentname}{Harry Langford}
\newcommand{\studentemail}{hjel2@cam.ac.uk}

\documentclass[10pt,\jkfside,a4paper]{article}

\newcommand{\svcourse}{CST Part IA: Software Engineering and Security}
\newcommand{\svnumber}{1}
\newcommand{\svvenue}{Microsoft Teams}
\newcommand{\svdate}{2022-05-11}
\newcommand{\svtime}{15:00}
\newcommand{\svuploadkey}{CBd13xmL7PC1zqhNIoLdTiYUBnxZhzRAtJxv/ytRdM1r7qIfwMsxeVwM/pPcIo8l}

\newcommand{\svrname}{Dr Sam Ainsworth}
\newcommand{\jkfside}{oneside}
\newcommand{\jkfhanded}{yes}

\newcommand{\studentname}{Harry Langford}
\newcommand{\studentemail}{hjel2@cam.ac.uk}

% DO NOT add \usepackage commands here.  Place any custom commands
% into your SV work files.  Anything in the template directory is
% likely to be overwritten!

\usepackage{fancyhdr}

\usepackage{lastpage}       % ``n of m'' page numbering
\usepackage{lscape}         % Makes landscape easier

\usepackage{verbatim}       % Verbatim blocks
\usepackage{listings}       % Source code listings
\usepackage{epsfig}         % Embed encapsulated postscript
\usepackage{array}          % Array environment
\usepackage{qrcode}         % QR codes
\usepackage{enumitem}       % Required by Tom Johnson's exam question header

\usepackage{hhline}         % Horizontal lines in tables
\usepackage{siunitx}        % Correct spacing of units
\usepackage{amsmath}        % American Mathematical Society
\usepackage{amssymb}        % Maths symbols
\usepackage{amsthm}         % Theorems

\usepackage{ifthen}         % Conditional processing in tex

\usepackage[top=3cm,
            bottom=3cm,
            inner=2cm,
            outer=5cm]{geometry}

% PDF metadata + URL formatting
\usepackage[
            pdfauthor={\studentname},
            pdftitle={\svcourse, SV \svnumber},
            pdfsubject={},
            pdfkeywords={9d2547b00aba40b58fa0378774f72ee6},
            pdfproducer={},
            pdfcreator={},
            hidelinks]{hyperref}


% DO NOT add \usepackage commands here.  Place any custom commands
% into your SV work files.  Anything in the template directory is
% likely to be overwritten!

\usepackage{fancyhdr}

\usepackage{lastpage}       % ``n of m'' page numbering
\usepackage{lscape}         % Makes landscape easier

\usepackage{verbatim}       % Verbatim blocks
\usepackage{listings}       % Source code listings
\usepackage{graphicx}
\usepackage{float}
\usepackage{epsfig}         % Embed encapsulated postscript
\usepackage{array}          % Array environment
\usepackage{qrcode}         % QR codes
\usepackage{enumitem}       % Required by Tom Johnson's exam question header

\usepackage{hhline}         % Horizontal lines in tables
\usepackage{siunitx}        % Correct spacing of units
\usepackage{amsmath}        % American Mathematical Society
\usepackage{amssymb}        % Maths symbols
\usepackage{amsthm}         % Theorems

\usepackage{ifthen}         % Conditional processing in tex

\usepackage[top=3cm,
            bottom=3cm,
            inner=2cm,
            outer=5cm]{geometry}

% PDF metadata + URL formatting
\usepackage[
            pdfauthor={\studentname},
            pdftitle={\svcourse, SV \svnumber},
            pdfsubject={},
            pdfkeywords={9d2547b00aba40b58fa0378774f72ee6},
            pdfproducer={},
            pdfcreator={},
            hidelinks]{hyperref}

\renewcommand{\headrulewidth}{0.4pt}
\renewcommand{\footrulewidth}{0.4pt}
\fancyheadoffset[LO,LE,RO,RE]{0pt}
\fancyfootoffset[LO,LE,RO,RE]{0pt}
\pagestyle{fancy}
\fancyhead{}
\fancyhead[LO,RE]{{\bfseries \studentname}\\\studentemail}
\fancyhead[RO,LE]{{\bfseries \svcourse, SV~\svnumber}\\\svdate\ \svtime, \svvenue}
\fancyfoot{}
\fancyfoot[LO,RE]{For: \svrname}
\fancyfoot[RO,LE]{\today\hspace{1cm}\thepage\ / \pageref{LastPage}}
\fancyfoot[C]{\qrcode[height=0.8cm]{\svuploadkey}}
\setlength{\headheight}{22.55pt}


\ifthenelse{\equal{\jkfside}{oneside}}{

 \ifthenelse{\equal{\jkfhanded}{left}}{
  % 1. Left-handed marker, one-sided printing or e-marking, use oneside and...
  \evensidemargin=\oddsidemargin
  \oddsidemargin=73pt
  \setlength{\marginparwidth}{111pt}
  \setlength{\marginparsep}{-\marginparsep}
  \addtolength{\marginparsep}{-\textwidth}
  \addtolength{\marginparsep}{-\marginparwidth}
 }{
  % 2. Right-handed marker, one-sided printing or e-marking, use oneside.
  \setlength{\marginparwidth}{111pt}
 }

}{
 % 3. Alternating margins, two-sided printing, use twoside.
}


\setlength{\parindent}{0em}
\addtolength{\parskip}{1ex}

% Exam question headings, labels and sensible layout (courtesy of Tom Johnson)
\setlist{parsep=\parskip, listparindent=\parindent}
\newcommand{\examhead}[3]{\section{#1 Paper #2 Question #3}}
\newenvironment{examquestion}[3]{
\examhead{#1}{#2}{#3}\setlist[enumerate, 1]{label=(\alph*)}\setlist[enumerate, 2]{label=(\roman*)}
\marginpar{\href{https://www.cl.cam.ac.uk/teaching/exams/pastpapers/y#1p#2q#3.pdf}{\qrcode{https://www.cl.cam.ac.uk/teaching/exams/pastpapers/y#1p#2q#3.pdf}}}
\marginpar{\footnotesize \href{https://www.cl.cam.ac.uk/teaching/exams/pastpapers/y#1p#2q#3.pdf}{https://www.cl.cam.ac.uk/\\teaching/exams/pastpapers/\\y#1p#2q#3.pdf}}
}{}


\begin{document}

\section*{3 More on numbers}

\subsection*{3.1 Basic exercises}

\begin{enumerate}

\setcounter{enumi}{1}
\item Find the gcd of 21212121 and 12121212.

Using Euclid's Algorithm:
\begin{equation}
\begin{split}
\text{gcd}(21212121, 12121212) &= \text{gcd}(12121212, 9090909)\\
							   &= \text{gcd}(9090909, 3030303)\\
							   &= 3030303\\
\end{split}
\end{equation}

\item Prove that for all positive integers $m$ and $n$, and integers $k$ and $l$, 
\begin{equation}
\begin{split}
\text{gcd}(m,n)&|(k\cdot m + l\cdot n)\\
\end{split}
\end{equation}
\begin{equation}\label{313a}
\begin{split}
\forall m, n \in \mathbb{Z}^+: \text{gcd}(m,n)&| n\Longleftrightarrow\\
\forall m, n \in \mathbb{Z}^+: \exists a \in \mathbb{Z}: a \cdot \text{gcd}(m,n) &= m\\
\forall m, n \in \mathbb{Z}^+: \exists a \in \mathbb{Z}: \forall k \in \mathbb{Z}: (a\cdot k)\cdot \text{gcd}(m,n) &= k\cdot m\\
\end{split}
\end{equation}
\begin{equation}\label{313b}
\begin{split}
\forall m, n \in \mathbb{Z}^+: \text{gcd}(m,n)&| n\Longleftrightarrow\\
\forall m, n \in \mathbb{Z}^+: \exists b \in \mathbb{Z}: b \cdot \text{gcd}(m,n) &= n\Longleftrightarrow\\
\forall m, n \in \mathbb{Z}^+: \exists b \in \mathbb{Z}: \forall l \in \mathbb{Z}: (b\cdot l)\cdot \text{gcd}(m,n) &= l\cdot n\\
\end{split}
\end{equation}
\begin{center}
Adding (\ref{313a}) and (\ref{313b}) gives:
\end{center}
\begin{equation}
\begin{split}
\forall m, n \in \mathbb{Z}^+: \exists a, b \in \mathbb{Z}: \forall k, l \in \mathbb{Z}: (a\cdot k)\cdot \text{gcd}(m,n) + (b\cdot l) \cdot \text{gcd}(m,n) &= k\cdot m + l\cdot n \Longleftrightarrow\\
\forall m, n \in \mathbb{Z}^+: \exists a, b \in \mathbb{Z}: \forall k, l \in \mathbb{Z}: (a\cdot k + b\cdot l) \cdot \text{gcd}(m,n) &= k\cdot m + l\cdot n \Longrightarrow\\
\forall m, n \in \mathbb{Z}^+: \forall k, l \in \mathbb{Z}: \text{gcd}(m,n) &| k\cdot m + l\cdot n\\
\end{split}
\end{equation}


\item Find integers $x$ and $y$ such that $x\cdot 30 + y\cdot 22 = \text{gcd}(30,22)$. 
Now find integers $x'$ and $y'$ with $0 \leq y' < 30$ such that $x' \cdot 30 + y'\cdot 22 = \text{gcd}(30, 22)$

gcd(30, 22) = 2\\
$x = 3$ and $y = -4$:
\begin{equation}
\begin{split}
 & x \cdot 30 + y \cdot 22\\
=& 90 - 88\\
=& 2\\
=& \text{gcd}(30,22)\\
\end{split}
\end{equation}

$y=11$ and $x = -8$
\begin{equation}
\begin{split}
 & x\cdot 30 + y\cdot 22\\
=& -8 \cdot 30 + 11 \cdot 22\\
=& -240 + 242\\
=& 2\\
=& \text{gcd}(30,22)\\
\end{split}
\end{equation}

\item Prove that for all positive integers $n$ and primes $p$, if $n^2 \equiv 1(\text{mod } p)$ 
then either $n\equiv 1(\text{mod } p)$ or $n\equiv-1(\text{mod } p)$.

\begin{equation}
\begin{split}
n^2 &\equiv 1 (\text{mod } p)\Longleftrightarrow\\
n^2 - 1 &\equiv 0 (\text{mod } p)\Longleftrightarrow\\
p &| n^2 - 1\Longleftrightarrow\\
p &| (n - 1)(n + 1)\Longleftrightarrow\\
\text{Since $p$ is prime: }p &| (n - 1) \vee p | (n + 1)\Longleftrightarrow\\
(n - 1) &\equiv 0 (\text{mod } p) \vee (n + 1) \equiv 0 (\text{mod } p)\Longleftrightarrow\\
n &\equiv 1 (\text{mod } p) \vee n \equiv -1 (\text{mod } p)\text{ as required}\\
\end{split}
\end{equation}

\end{enumerate}

\subsection*{3.2 Core exercises}

\begin{enumerate}

\item Prove that for all positive integers $m$ and $n$, $\text{gcd}(m, n) = m \text{ iff } m|n$.

($\Longrightarrow$)
\begin{equation}
\begin{split}
\text{Assume: }\text{gcd}(m, n) &= m\Longrightarrow\\
\forall m, n \in \mathbb{Z}: \text{gcd}(m, n) &| n\Longrightarrow\\
m &| n\text{ as required}\\
\end{split}
\end{equation}
($\Longleftarrow$)
\begin{equation}
\begin{split}
m &| n\\
\forall m, n \in \mathbb{Z}: \text{gcd}(m, n) &| m\\
\forall m, n \in \mathbb{Z}: \text{gcd}(m, n) &| m \wedge m | n\Longrightarrow\\
m &| \text{gcd}(m, n)\\
\forall m, n \in \mathbb{Z}: m | \text{gcd}(m, n) &\wedge \text{gcd}(m, n) | m \Longleftrightarrow\\
\text{gcd}(m, n) &= m\\
\end{split}
\end{equation}

\item Let $m$ and $n$ be positive integers with $\text{gcd}(m, n) = 1$. Prove that for every natural number 
$k$,
\begin{equation*}
\begin{split}
m | k &\wedge n | k \Longleftrightarrow m\cdot n | k\\
\end{split}
\end{equation*}

($\Longrightarrow$)
\begin{equation}
\begin{split}
m &| k \wedge n | k \Longleftrightarrow\\
\frac{m\cdot n}{\text{gcd}(m, n)} &| k\Longleftrightarrow\\
\frac{m\cdot n}{1} &| k\Longleftrightarrow\\
m\cdot n &| k\text{ as required}\\
\end{split}
\end{equation}

($\Longleftarrow$)
\begin{equation}
\begin{split}
m\cdot n &| k\Longleftrightarrow\\
\exists c \in \mathbb{Z}: c\cdot m \cdot n &= k\Longleftrightarrow\\
\exists c \in \mathbb{Z}: (c\cdot m)\cdot n &= k \wedge (c\cdot n)\cdot m = k\Longleftrightarrow\\
n &| k \wedge m | k\text{ as required}\\
\end{split}
\end{equation}

\item Prove that for all positive integers $a$, $b$, $c$, if $\text{gcd}(a,c)=1$ then $\text{gcd}(a\cdot b,c) = \text{gcd}(b,c))$.

\begin{equation}\label{gcdabcgcdac}
\begin{split}
 & \text{gcd}(a\cdot b, c)\\
=& \text{gcd}(\text{gcd}(a, c) \cdot b, c)\\
=& \text{gcd}(1 \cdot b, c)\\
=& \text{gcd}(b, c)\text{ as required}\\
\end{split}
\end{equation}

\item Prove that for all positive integers $m$ and $n$, and integers $i$ and $j$:

\begin{equation}
\begin{split}
n\cdot i &\equiv n\cdot j(\text{mod } m) \Longleftrightarrow i\equiv j(\text{mod }\frac{m}{\text{gcd}(m,n)})\\
\end{split}
\end{equation}

($\Longrightarrow$)
\begin{equation}
\begin{split}
n\cdot i &\equiv n\cdot j (\text{mod } m)\Longleftrightarrow\\
\frac{n}{\text{gcd}(m, n)}\cdot i &\equiv \frac{n}{\text{gcd}(m, n)}\cdot j (\text{mod } \frac{m}{\text{gcd}(m, n)})\Longrightarrow\\
\text{since $\frac{n}{\text{gcd}(m, n)}$ is coprime with}&\text{ $\frac{m}{\text{gcd}(m, n)}$, it must have a multiplicative inverse in $\mathbb{Z}_{\frac{m}{\text{gcd}(m, n)}}$}\Longrightarrow\\
\frac{n}{\text{gcd}(m, n)} \cdot \left[\frac{n}{\text{gcd}(m, n)}\right]^{-1}_m \cdot i &\equiv \frac{n}{\text{gcd}(m, n)} \cdot \left[\frac{n}{\text{gcd}(m, n)}\right]^{-1}_m \cdot j (\text{mod } \frac{m}{\text{gcd}(m, n)})\Longleftrightarrow\\
i &\equiv j (\text{mod }\frac{m}{\text{gcd}(m, n)}) \text{ as required}\\
\end{split}
\end{equation}

($\Longleftarrow$)
\begin{equation}
\begin{split}
i &\equiv j (\text{mod } \frac{m}{\text{gcd}(m, n)}\Longrightarrow\\
\text{gcd}(m, n)\cdot i &\equiv \text{gcd}(m, n)\cdot j (\text{mod } m)\Longrightarrow\\
\frac{n}{\text{gcd}(m, n)}\cdot \text{gcd}(m, n) i &\equiv \frac{n}{\text{gcd}(m, n)} \cdot \text{gcd}(m, n) \cdot j (\text{mod } m)\Longrightarrow\\
n\cdot i &\equiv n\cdot j (\text{mod } m)\text{ as required}\\
\end{split}
\end{equation}


\item Prove that for all positive integers $m$, $n$, $p$, $q$ such that $\text{gcd}(m,n) = \text{gcd}(p,q) = 1$, if $q\cdot m = p\cdot n$ 
then $m = p$ and $n = q$.

\begin{equation}
\begin{split}
q\cdot m = p\cdot n \wedge \text{gcd}(m, n) &= 1 \wedge \text{gcd}(p, q) = 1\Longleftrightarrow\\
m &| p \wedge q | n\\
\exists i, j \in \mathbb{Z}: i \cdot m &= p \wedge j \cdot q = n\Longleftrightarrow\\
\exists i, j \in \mathbb{Z}: i \cdot j \cdot q \cdot m &= p\cdot n\Longleftrightarrow\\
\exists i, j \in \mathbb{Z}: i \cdot j \cdot q \cdot m &= q\cdot m\Longleftrightarrow\\
\exists i, j \in \mathbb{Z}: i, j &= 1\Longleftrightarrow\\
p &= m \wedge n = q \text{ as required}\\
\end{split}
\end{equation}

\item Prove that for all positive integers $a$ and $b$, $\text{gcd}(13\cdot a + 8\cdot b, 5\cdot a + 3\cdot b)=\text{gcd}(a,b)$.

Using Euclid's algorithm:
\begin{equation}
\begin{split}
 & gcd(13\cdot a + 8\cdot b, 5\cdot a + 3\cdot b)\\
=& gcd(5\cdot a + 3\cdot b, 3\cdot a + 2\cdot b)\\
=& gcd(3\cdot a + 2\cdot b, 2\cdot a + b)\\
=& gcd(2\cdot a + b, a + b)\\
=& gcd(a + b, a)\\
=& gcd(a, b)\text{ as required}\\
\end{split}
\end{equation}

\item Let $n$ be an integers

\begin{enumerate}

\setcounter{enumii}{2}

\item Conclude that if $p$ is a prime number greater than 3, then $p^2 - 1$ is divisible by 24.

Take an arbitrary prime numbers $p>3$.\\
Since $p$ is prime and $p\neq 3$: $3\not|p \Longrightarrow p^2 \equiv 1 (\text{mod } 3)$ from part (a)\\
All prime numbers except 2 are odd. $p > 3 \Longrightarrow p \neq 2 \Longrightarrow p^2 \equiv 1 (\text{mod } 8)$ from part (b)\\

\begin{equation}
\begin{split}
p^2 \equiv 1 (\text{mod } 3) &\wedge p ^2 \equiv 1 (\text{mod } 8)\Longleftrightarrow\\
p^2 - 1 \equiv 0 (\text{mod } 3) &\wedge p ^2 - 1 \equiv 0 (\text{mod } 8)\Longleftrightarrow\\
\exists i, j \in \mathbb{Z}: p &= 3\cdot i \wedge p = 8\cdot j\Longleftrightarrow\\
\exists i, j \in \mathbb{Z}: p^2 - 1 &= 9\cdot (8\cdot j) - 8\cdot (3\cdot i)\Longleftrightarrow\\
\exists i, j \in \mathbb{Z}: p^2 - 1 &= 24 \cdot (3\cdot j - i)\Longleftrightarrow\\
p^2 - 1 &\equiv 0 (\text{mod } 24)\Longleftrightarrow\\
p^2 &\equiv 1 (\text{mod } 24)\\
\end{split}
\end{equation}

\end{enumerate}

\item Prove that $n^{13} \equiv n(\text{mod } 10)$ for all integers $n$.

\begin{equation}
\begin{split}
\text{Using Fermat's}&\text{ Little Theorem}:\\
n^2 &\equiv n (\text{mod } 2)\Longleftrightarrow\\
n^{12} &\equiv n^6 (\text{mod } 2)\Longleftrightarrow\\
n^{12} &\equiv n^3 (\text{mod } 2)\Longleftrightarrow\\
n^{13} &\equiv n^4 (\text{mod } 2)\Longleftrightarrow\\
n^{13} &\equiv n (\text{mod } 2)\\
n^{13} - n &\equiv 0 (\text{mod } 2)\\
\end{split}
\end{equation}
\begin{equation}
\begin{split}
\text{Using Fermat's}&\text{ Little Theorem}:\\
n^5 &\equiv n (\text{mod } 5)\Longleftrightarrow\\
n^{10} &\equiv n^2 (\text{mod } 5)\Longleftrightarrow\\
n^{13} &\equiv n^5 (\text{mod } 5)\Longleftrightarrow\\
n^{13} &\equiv n (\text{mod } 5)\Longleftrightarrow\\
n^{13} - n &\equiv 0 (\text{mod } 5)\\
\end{split}
\end{equation}
\begin{equation}
\begin{split}
n^{13} - n = 0 (\text{mod } 2) &\wedge n^{13} - n = 0 (\text{mod } 5)\Longleftrightarrow\\
\exists i, j \in \mathbb{Z}: 2\cdot i = n^{13} - n &\wedge 5\cdot j = n^{13} - n \Longleftrightarrow\\
\exists i, j \in \mathbb{Z}: n^{13} - n &= 5\cdot(2\cdot i) - 4\cdot(5\cdot j)\Longleftrightarrow\\
\exists i, j \in \mathbb{Z}: n^{13} - n &= 10\cdot (i - 2\cdot j)\Longleftrightarrow\\
n^{13} - n &\equiv 0 (\text{mod } 10)\Longleftrightarrow\\
n^{13} &\equiv n (\text{mod } 10)\text{ as required}\\
\end{split}
\end{equation}

\item Prove that for all positive integers $l$, $m$ and $n$, if $\text{gcd}(l,m\cdot n) = 1$ then 
$\text{gcd}(l,m)=1$ and $\text{gcd}(l,n)=1$.

This is equivalent to the contrapositive:\\
If $\text{gcd}(l, m) \neq 1 \vee \text{gcd}(l, n) \neq 1$ then $\text{gcd}(l, m \cdot n) \neq 1$

Let $i = \text{gcd}(l, m)$ and $j = \text{gcd}(l, n)$.
\begin{equation}
\begin{split}
i | l \wedge i &| m\Longleftrightarrow\\
i | l \wedge i &| m\cdot n\Longleftrightarrow\\
\exists k \in \mathbb{Z}: \text{gcd}(l, m\cdot n) &= k\cdot i\Longleftrightarrow\\
(i \neq 1 &\Longrightarrow \text{gcd}(l, m\cdot n) \neq 1)\\
(\text{gcd}(l, m) \neq 1 &\Longrightarrow \text{gcd}(l, m\cdot n) \neq 1)\\
\end{split}
\end{equation}
\begin{equation}
\begin{split}
j | l \wedge j &| n\Longleftrightarrow\\
j | l \wedge j &| m\cdot n\Longleftrightarrow\\
\exists k \in \mathbb{Z}: \text{gcd}(l, m\cdot n) &= k\cdot j\Longleftrightarrow\\
(j \neq 1 &\Longrightarrow \text{gcd}(l, m\cdot n) \neq 1)\Longleftrightarrow\\
(\text{gcd}(l, n) \neq 1 &\Longrightarrow \text{gcd}(l, m\cdot n) \neq 1)\\
\end{split}
\end{equation}
So $\text{gcd}(l, n) \neq 1 \vee \text{gcd}(l, m) \neq 1 \Longrightarrow \text{gcd}(l, m\cdot n) \neq 1$ as required. \\
Since the contrapositive is true, the original statement must be true.

\item Solve the following congruences:

\begin{enumerate}

\item $77\cdot x \equiv 11 (\text{mod } 40)$

\begin{equation}
\begin{split}
77\cdot x &\equiv 11 (\text{mod } 40)\Longleftrightarrow\\
-3\cdot x &\equiv -29 (\text{mod } 40)\Longleftrightarrow\\
3\cdot x &\equiv 29 (\text{mod } 40)\Longleftrightarrow\\
\exists k \in \mathbb{Z}: 3\cdot x &= 29 + 40\cdot k\\
\text{By inspection } 3 &| 29 + 40\cdot 1\Longleftrightarrow\\
3 &| 69\Longleftrightarrow\\
x &\equiv \frac{69}{3}(\text{mod } 40)\Longleftrightarrow\\
x &\equiv 23(\text{mod } 40)\\
\end{split}
\end{equation}

\item $12\cdot y \equiv 30 (\text{mod } 54)$

\begin{equation}
\begin{split}
12\cdot y &\equiv 30(\text{mod } 54)\Longleftrightarrow\\
\exists k \in \mathbb{Z}: 12\cdot y &= 30 + 54\cdot k\\
\text{By inspection } 12 &| 30 + 54 \Longleftrightarrow\\
12 &| 30 + 54 \Longleftrightarrow\\
y &\equiv \frac{84}{12}(\text{mod } 54)\Longleftrightarrow\\
y &\equiv 7(\text{mod } 54)\\
\end{split}
\end{equation}

\item $13 \equiv z (\text{mod } 21) \wedge 3 \cdot z \equiv 2 (\text{mod } 17)$

\begin{equation}
\begin{split}
13 &\equiv z(\text{mod } 21) \wedge 3\cdot z \equiv 2(\text{mod } 17)\Longleftrightarrow\\
\exists k \in \mathbb{Z}: z &= 13 + k\cdot 21 \wedge 3\cdot z \equiv 2(\text{mod } 17)\Longleftrightarrow\\
\end{split}
\end{equation}
Substitute in $z = 13 + k\cdot 21$ into$ 3\cdot z \equiv 2(\text{mod } 17)$
\begin{equation}
\begin{split}
\exists k \in \mathbb{Z}: 3\cdot 13 + 63\cdot k &\equiv 2(\text{mod } 17)\Longleftrightarrow\\
63\cdot k &\equiv 2 - 39(\text{mod } 17)\Longleftrightarrow\\
12\cdot k &\equiv 14(\text{mod } 17)\Longleftrightarrow\\
\text{By inspection } 12\cdot 4 &\equiv 14(\text{mod } 17))\Longleftrightarrow\\
k &\equiv 4(\text{mod } 17)\\
z &= 13 + 4\cdot 21 (\text{mod } 17)\Longleftrightarrow\\
z &= 13 + 16 (\text{mod } 17)\Longleftrightarrow\\
z &= 12 (\text{mod } 17)\\
\end{split}
\end{equation}

\end{enumerate}

\item What is the multiplicative inverse of (a) 2 in $\mathbb{Z}_{7}$, (b) 7 in $\mathbb{Z}_{40}$ 
and (c) 13 in $\mathbb{Z}_{23}$?

\begin{enumerate}

\item 4 by inspection

\item 23 by inspection

\item 16 by inspection

\end{enumerate}

\item Prove that $[22^{12001}]_{175}$ has a multiplicative inverse in $\mathbb{Z}_{175}$

\begin{equation}
\begin{split}
22^{12001} &= 22\cdot (22^4)^{3000}\Longleftrightarrow\\
22^{12001} &\equiv 22\cdot 1 (\text{mod } 5)\Longleftrightarrow\\
22^{12001} - 22 &\equiv 0 (\text{mod } 5)\\
\end{split}
\end{equation}
\begin{equation}
\begin{split}
22^{12001} &= 22\cdot (22^6)^{2000}\Longleftrightarrow\\
22^{12001} &\equiv 22\cdot 1 (\text{mod } 7)\Longleftrightarrow\\
22^{12001} - 22 &\equiv 0 (\text{mod } 7)\\
\end{split}
\end{equation}
\begin{equation}
\begin{split}
22^{12001} - 22 &\equiv 0 (\text{mod } 5) \wedge 22^{12001} - 22 \equiv 0 (\text{mod } 7)\Longleftrightarrow\\
\exists i, j \in \mathbb{Z}: 5 \cdot i &= 22^{12001} - 22 \wedge 7 \cdot j = 22^{12001} - 22\Longleftrightarrow\\
\exists i, j \in \mathbb{Z}: 22^{12001} - 22 &\equiv 15\cdot(7\cdot j) - 14\cdot(5\cdot i)\Longleftrightarrow\\
\exists i, j \in \mathbb{Z}: 22^{12001} - 22 &\equiv 35\cdot(5\cdot j - 2\cdot i)\Longleftrightarrow\\
22^{12001} - &22 \equiv 0 (\text{mod } 35)\Longleftrightarrow\\
\exists k \in \{0,1,2,3,4\}: 22^{12001} - 22 &\equiv 35\cdot k (\text{mod } 175)\Longleftrightarrow\\
\exists k \in \{0,1,2,3,4\}: 22^{12001} &\equiv 35\cdot k + 22 (\text{mod } 175)\\
\forall k \in \{0,1,2,3,4\}: 35\cdot k + &22 \text{ is coprime to 175} \Longleftrightarrow\\
\forall k \in \{0,1,2,3,4\}: 35\cdot k + &22 \text{ has a multiplicative inverse in }\mathbb{Z}_m \Longleftrightarrow\\
22^{12001} &\text{ has a multiplicative inverse in }\text{Z}_m\\
\end{split}
\end{equation}

\end{enumerate}

\subsection*{3.3 Optional exercises}

\begin{enumerate}

\item Let $a$ and $b$ be natural numbers such that $a^2 | b \cdot (b + a)$. Prove that $a|b$.

This is the same as the contrapositive $a\not|\text{ } b\Longrightarrow a^2 \not|\text{ } b\cdot (b + a)$:
\begin{equation}\label{abbplusa1}
\begin{split}
a&\not|\text{ } b\Longleftrightarrow\\
\forall i \in \mathbb{Z}: i\cdot a &\neq b\Longleftrightarrow\\
\forall i \in \mathbb{Z}: i\cdot a^2 &\neq a\cdot b\\
\end{split}
\end{equation}
\begin{equation}\label{abbplusa2}
\begin{split}
a&\not|\text{ } b\Longleftrightarrow\\
a^2&\not|\text{ } b^2\Longleftrightarrow\\
\forall j \in \mathbb{Z}: j\cdot a^2 &\neq b^2\\
\end{split}
\end{equation}
\begin{center}
Combining (\ref{abbplusa1}) and (\ref{abbplusa2}) gives:
\end{center}
\begin{equation}\label{abbplusa3}
\begin{split}
\forall i, j \in \mathbb{Z}: i\cdot a^2 + j\cdot a^2 &\neq a\cdot b + b^2\Longleftrightarrow\\
\forall k \in \mathbb{Z}: k\cdot a^2 &\neq b\cdot(b + a)\Longleftrightarrow\\
a^2&\not|\text{ } b \cdot (b + a)\text{ as required}\\
\end{split}
\end{equation}
Since we have proved the contrapositive; we have proved the original statement.

\item Prove the converse of (1.3.1): For all natural numbers $n$ and $s$, if there exists a natural 
number $q$ such that $(2\cdot n + 1)^2\cdot s + t_n = t_q$, then $s$ is a triangular number.

\begin{equation}
\begin{split}
(2\cdot n + 1)^2\cdot s + \frac{n}{2}(n + 1) &= \frac{q}{2}(q + 1) \Longleftrightarrow\\
(2\cdot n + 1)^2\cdot s &= \frac{q}{2}(q + 1) - \frac{n}{2}(n + 1) \Longleftrightarrow\\
2 \cdot (2\cdot n + 1)^2\cdot s &= q^2 + q - n^2 - n \Longleftrightarrow\\
2\cdot s &= \frac{(q - n)\cdot(q + n + 1)}{(2\cdot n + 1)^2}\Longleftrightarrow\\
2\cdot s &= \frac{q - n}{2\cdot n + 1} \cdot \frac{q + n + 1}{2 \cdot n + 1}\Longleftrightarrow\\
s &= \frac{1}{2}\cdot\frac{q - n}{2\cdot n + 1}\cdot\left(\frac{q - n}{2\cdot n + 1} + 1\right)\\
\end{split}
\end{equation}
\begin{equation}
\begin{split}
s \in \mathbb{Z}\Longleftrightarrow\\
\frac{1}{2}\cdot\frac{q - n}{2\cdot n + 1}\cdot\left(\frac{q - n}{2\cdot n + 1} + 1\right)\in \mathbb{Z} \Longleftrightarrow\\
\frac{q - n}{2\cdot n + 1}\cdot\left(\frac{q - n}{2\cdot n + 1} + 1\right)\in \mathbb{Z}\\
\end{split}
\end{equation}
To prove that this is a triangle number, we must prove that $\frac{q - n}{2\cdot n + 1} \in \mathbb{Z}$. 
I will do this by contradiction. Assume $\exists k \in \mathbb{Q}: k\cdot (k + 1) \in \mathbb{Z}$.
\begin{equation}
\begin{split}
\exists k \in \mathbb{Q}: &k\cdot(k + 1) \in \mathbb{Z} \Longleftrightarrow\\
\exists a, b \in\ \mathbb{Z}: &\frac{b}{a}\cdot \frac{b + a}{a} \in \mathbb{Z}\Longleftrightarrow\\
\exists a, b \in \mathbb{Z}: &\frac{b\cdot (b + a)}{a^2} \in \mathbb{Z}:\Longleftrightarrow\\
a^2 &| b\cdot(b + a)\Longrightarrow\\
a &| b \text{ from (\ref{abbplusa3})}\Longleftrightarrow\\
\frac{b}{a} \in \mathbb{Z}\\
\end{split}
\end{equation}
However this contradicts our original assumption that $\frac{b}{a}\in \mathbb{Q}$. So this cannot be 
true and hence $k\cdot (k + 1) \in \mathbb{Z} \Longrightarrow k \in \mathbb{Z}$.\\
Since we know that $\cdot\frac{q - n}{2\cdot n + 1}\cdot\left(\frac{q - n}{2\cdot n + 1} + 1\right) \in \mathbb{Z}$, 
we also know that $\frac{q - n}{2\cdot n + 1} \in \mathbb{Z}$.\\
This proves that $s$ is a triangular number ($t_{\frac{q - n}{2\cdot n + 1}}$) -- as required.

\item Informally justify the correctness of the following alternative algorithm for computing the gcd of 
two positive integers:

\begin{verbatim}
let rec gcd0(m, n) = if m = n then m
							  else let p = min m n
								   and q = max m n
								    in gcd0(p, q - p)
\end{verbatim}

Proof by Loop Invariant:\\
Case $m = n$.
If $m = n$, then gcd$(m, n) = m$.
In this case, the algorithm terminates and returns $m$. So the algorithm is correct in this case.

Case $m > n$.\label{gcd0case2}
If $m > n$, then the algorithm calls itself on $n, m - n$. \\
$m - n < m$ so the problem has been reduced in size.\\
$m - n > 0$ and gcd$(m, n) = \text{gcd}(n, m - n$ for all $m, n$. So the result of the algorithm is still the same.

Case $m < n$:
Same argument as ($m > n$) with $m$ and $n$ reversed.

Since for every case the end result of the algorithm is unchanged and the algorithm terminates in every case; it must calculate 
the gcd$(m, n)$ correctly. Hence the algorithm is correct.

\end{enumerate}

\end{document}