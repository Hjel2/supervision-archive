\newcommand{\svrname}{Mr Jakub Perlin}
\newcommand{\jkfside}{oneside}
\newcommand{\jkfhanded}{right}

\newcommand{\studentname}{Harry Langford}
\newcommand{\studentemail}{hjel2@cam.ac.uk}

\documentclass[10pt,\jkfside,a4paper]{article}

\newcommand{\svcourse}{CST Part IA: Software Engineering and Security}
\newcommand{\svnumber}{1}
\newcommand{\svvenue}{Microsoft Teams}
\newcommand{\svdate}{2022-05-11}
\newcommand{\svtime}{15:00}
\newcommand{\svuploadkey}{CBd13xmL7PC1zqhNIoLdTiYUBnxZhzRAtJxv/ytRdM1r7qIfwMsxeVwM/pPcIo8l}

\newcommand{\svrname}{Dr Sam Ainsworth}
\newcommand{\jkfside}{oneside}
\newcommand{\jkfhanded}{yes}

\newcommand{\studentname}{Harry Langford}
\newcommand{\studentemail}{hjel2@cam.ac.uk}

% DO NOT add \usepackage commands here.  Place any custom commands
% into your SV work files.  Anything in the template directory is
% likely to be overwritten!

\usepackage{fancyhdr}

\usepackage{lastpage}       % ``n of m'' page numbering
\usepackage{lscape}         % Makes landscape easier

\usepackage{verbatim}       % Verbatim blocks
\usepackage{listings}       % Source code listings
\usepackage{epsfig}         % Embed encapsulated postscript
\usepackage{array}          % Array environment
\usepackage{qrcode}         % QR codes
\usepackage{enumitem}       % Required by Tom Johnson's exam question header

\usepackage{hhline}         % Horizontal lines in tables
\usepackage{siunitx}        % Correct spacing of units
\usepackage{amsmath}        % American Mathematical Society
\usepackage{amssymb}        % Maths symbols
\usepackage{amsthm}         % Theorems

\usepackage{ifthen}         % Conditional processing in tex

\usepackage[top=3cm,
            bottom=3cm,
            inner=2cm,
            outer=5cm]{geometry}

% PDF metadata + URL formatting
\usepackage[
            pdfauthor={\studentname},
            pdftitle={\svcourse, SV \svnumber},
            pdfsubject={},
            pdfkeywords={9d2547b00aba40b58fa0378774f72ee6},
            pdfproducer={},
            pdfcreator={},
            hidelinks]{hyperref}


% DO NOT add \usepackage commands here.  Place any custom commands
% into your SV work files.  Anything in the template directory is
% likely to be overwritten!

\usepackage{fancyhdr}

\usepackage{lastpage}       % ``n of m'' page numbering
\usepackage{lscape}         % Makes landscape easier

\usepackage{verbatim}       % Verbatim blocks
\usepackage{listings}       % Source code listings
\usepackage{graphicx}
\usepackage{float}
\usepackage{epsfig}         % Embed encapsulated postscript
\usepackage{array}          % Array environment
\usepackage{qrcode}         % QR codes
\usepackage{enumitem}       % Required by Tom Johnson's exam question header

\usepackage{hhline}         % Horizontal lines in tables
\usepackage{siunitx}        % Correct spacing of units
\usepackage{amsmath}        % American Mathematical Society
\usepackage{amssymb}        % Maths symbols
\usepackage{amsthm}         % Theorems

\usepackage{ifthen}         % Conditional processing in tex

\usepackage[top=3cm,
            bottom=3cm,
            inner=2cm,
            outer=5cm]{geometry}

% PDF metadata + URL formatting
\usepackage[
            pdfauthor={\studentname},
            pdftitle={\svcourse, SV \svnumber},
            pdfsubject={},
            pdfkeywords={9d2547b00aba40b58fa0378774f72ee6},
            pdfproducer={},
            pdfcreator={},
            hidelinks]{hyperref}

\renewcommand{\headrulewidth}{0.4pt}
\renewcommand{\footrulewidth}{0.4pt}
\fancyheadoffset[LO,LE,RO,RE]{0pt}
\fancyfootoffset[LO,LE,RO,RE]{0pt}
\pagestyle{fancy}
\fancyhead{}
\fancyhead[LO,RE]{{\bfseries \studentname}\\\studentemail}
\fancyhead[RO,LE]{{\bfseries \svcourse, SV~\svnumber}\\\svdate\ \svtime, \svvenue}
\fancyfoot{}
\fancyfoot[LO,RE]{For: \svrname}
\fancyfoot[RO,LE]{\today\hspace{1cm}\thepage\ / \pageref{LastPage}}
\fancyfoot[C]{\qrcode[height=0.8cm]{\svuploadkey}}
\setlength{\headheight}{22.55pt}


\ifthenelse{\equal{\jkfside}{oneside}}{

 \ifthenelse{\equal{\jkfhanded}{left}}{
  % 1. Left-handed marker, one-sided printing or e-marking, use oneside and...
  \evensidemargin=\oddsidemargin
  \oddsidemargin=73pt
  \setlength{\marginparwidth}{111pt}
  \setlength{\marginparsep}{-\marginparsep}
  \addtolength{\marginparsep}{-\textwidth}
  \addtolength{\marginparsep}{-\marginparwidth}
 }{
  % 2. Right-handed marker, one-sided printing or e-marking, use oneside.
  \setlength{\marginparwidth}{111pt}
 }

}{
 % 3. Alternating margins, two-sided printing, use twoside.
}


\setlength{\parindent}{0em}
\addtolength{\parskip}{1ex}

% Exam question headings, labels and sensible layout (courtesy of Tom Johnson)
\setlist{parsep=\parskip, listparindent=\parindent}
\newcommand{\examhead}[3]{\section{#1 Paper #2 Question #3}}
\newenvironment{examquestion}[3]{
\examhead{#1}{#2}{#3}\setlist[enumerate, 1]{label=(\alph*)}\setlist[enumerate, 2]{label=(\roman*)}
\marginpar{\href{https://www.cl.cam.ac.uk/teaching/exams/pastpapers/y#1p#2q#3.pdf}{\qrcode{https://www.cl.cam.ac.uk/teaching/exams/pastpapers/y#1p#2q#3.pdf}}}
\marginpar{\footnotesize \href{https://www.cl.cam.ac.uk/teaching/exams/pastpapers/y#1p#2q#3.pdf}{https://www.cl.cam.ac.uk/\\teaching/exams/pastpapers/\\y#1p#2q#3.pdf}}
}{}



\usepackage{enumitem}

\begin{document}

\section*{2 On numbers}

\subsection*{2.1 Basic exercises}

\begin{enumerate}

\item Let $i$, $j$ be integers and let $m$, $n$ be positive integers. Show that:

\begin{enumerate}

\item $i \equiv i (\text{mod } m)$

\begin{equation}\label{211a}
\begin{split}
m&|0\Longleftrightarrow\\
m&|(i - i)\Longleftrightarrow\\
i - i &\equiv 0(\text{mod } m)\Longleftrightarrow\\
i &\equiv i(\text{mod } m)\text{ as required}\\
\end{split}
\end{equation}

\item $i \equiv j (\text{mod } m) \Longrightarrow j \equiv i (\text{mod } m)$

\begin{equation}\label{211b}
\begin{split}
i &\equiv j (\text{mod } m) \Longleftrightarrow\\
\exists k \in \mathbb{Z}: i &\equiv j + k\cdot m \Longleftrightarrow\\
\exists k \in \mathbb{Z}: j &\equiv i - k\cdot m \Longleftrightarrow\\
j &\equiv i(\text{mod } m)\text{ as required}\\
\end{split}
\end{equation}

\item $i \equiv j (\text{mod } m) \wedge j \equiv k (\text{mod } m) \Longrightarrow i \equiv k (\text{mod } m)$

\begin{equation}\label{211ca}
\begin{split}
i &\equiv j (\text{mod } m) \Longleftrightarrow\\
j &\equiv i (\text{mod } m)\text{ using (\ref{211b})}\\
\end{split}
\end{equation}
\begin{equation}\label{211cb}
\begin{split}
j &\equiv k(\text{mod } m)\\
\end{split}
\end{equation}
\begin{center}
Combining (\ref{211ca})and (\ref{211cb}) gives:
\end{center}
\begin{equation}
\begin{split}
i &\equiv k(\text{mod } m) \Longleftrightarrow\\
\end{split}
\end{equation}

\end{enumerate}

\item Prove that for all integers $i$, $j$, $k$, $l$, $m$, $n$ with $m$ positive and $n$ nonnegative,

\begin{enumerate}

\item $i \equiv j (\text{mod } m) \wedge k \equiv l (\text{mod } m) \Longrightarrow i + k \equiv j + l (\text{mod } m)$

\begin{equation}\label{212ba}
\begin{split}
i &\equiv j (\text{mod } m) \Longleftrightarrow\\
\text{ } \exists a \in \mathbb{Z}: i &= j + a\cdot m \\
\end{split}
\end{equation}
\begin{equation}\label{212bb}
\begin{split}
k &\equiv l (\text{mod } m) \Longleftrightarrow\\
\exists b \in \mathbb{Z}: k &= l + b\cdot m\\
\end{split}
\end{equation}
\begin{center}
Adding (\ref{212ba}) and (\ref{212bb}) gives:
\end{center}
\begin{equation}\label{212bc}
\begin{split}
\exists a, b \in \mathbb{Z}: i + k &= j + a\cdot m + l + b\cdot m \Longleftrightarrow\\
\exists a, b \in \mathbb{Z}: i + k &= j + l + (a + b)\cdot m \Longleftrightarrow\\
i + k &\equiv j + l (\text{mod } m)\\
\end{split}
\end{equation}

\item $i \equiv j (\text{mod } m) \wedge k \equiv l (\text{mod } m) \Longrightarrow i \cdot k \equiv j\cdot l (\text{mod } m)$

\begin{equation}\label{2a}
\begin{split}
i &\equiv j (\text{mod } m) \Longleftrightarrow\\
(a) \exists p \in \mathbb{Z}: i &= j + p \cdot m\\
k &\equiv l(\text{mod } m) \Longleftrightarrow\\
(b) \exists q \in \mathbb{Z}: k &= l + q \cdot m\\
\text{Combining (a)}&\text{ and (b) gives:}\\
\exists p, q \in \mathbb{Z}: i \cdot k &= (j + p\cdot m)\cdot(l + q\cdot m) \Longleftrightarrow\\
\exists p, q \in \mathbb{Z}: i \cdot k &= j \cdot l + j \cdot q \cdot m + l\cdot p \cdot m + p \cdot q \cdot m \cdot m \Longleftrightarrow\\
\exists p, q \in \mathbb{Z}: i \cdot k &= j\cdot l + (j\cdot q + l\cdot p + p\cdot q \cdot m) \cdot m \Longleftrightarrow\\
i\cdot k &= j\cdot l (\text{mod } m)\\
\end{split}
\end{equation}

\item $i \equiv j (\text{mod } m) \Longrightarrow i^n \equiv j^n (\text{mod } m)$

Proof by induction:

At $n=0$:\\
\begin{equation}
\begin{split}
\forall m \in \mathbb{Z}: 1 &\equiv 1(\text{mod } m)\Longleftrightarrow\\
\forall m \in \mathbb{Z}: i^0 &\equiv j^0(\text{mod } m)\\
\end{split}
\end{equation}

So the statement is true for $n = 0$.

Assume that the statement also holds true for $n = k$.
\begin{equation}\label{212ca}
\begin{split}
\text{By assumption: } i^k &\equiv j^k(\text{mod } m)\\
\end{split}
\end{equation}
\begin{equation}\label{212cb}
\begin{split}
\text{By assumption: } i &\equiv j(\text{mod } m)\\
\end{split}
\end{equation}
\begin{center}
Using (\ref{2a}) we can combine (\ref{212ca}) and (\ref{212cb})\\
\end{center}
\begin{equation}\label{212c}
\begin{split}
i^k \cdot i &\equiv j^k \cdot j(\text{mod } m)\Longleftrightarrow\\
i^{k+1} &\equiv j^{k+1}(\text{mod } m)\\
\end{split}
\end{equation}

So if the statement holds for $n=k$, then it also holds for $n=k+1$. Since the 
statement is true for $n=0$; by induction it must also be true for all $n\in\mathbb{N}$.

\end{enumerate}

\item Prove that for all natural numbers $k$, $l$ and positive integers $m$,

\begin{enumerate}

\item $\text{rem}(k \cdot m + l, m) = \text{rem}(l, m)$

\begin{equation}
\begin{split}
l &= l(\text{mod }m)\Longleftrightarrow\\
k\cdot m + l &= l(\text{mod } m)\Longleftrightarrow\\
\text{rem}(k\cdot m + l, m) &= \text{rem}(l,m)\text{ as required}\\
\end{split}
\end{equation}

\item $\text{rem}(k + l, m) = \text{rem}(\text{rem}(k,m) + l, m)$

\begin{equation}
\begin{split}
k + l &= k + l (\text{mod } m)\Longleftrightarrow\\
k + l &= \text{rem}(k,m) + l (\text{mod } m)\Longleftrightarrow\\
\text{rem}(k + l, m) &= \text{rem}(\text{rem}(k,m) + l, m)\text{ as required}\
\end{split}
\end{equation}

\item $\text{rem}(k \cdot l, m) = \text{rem}(k \cdot \text{rem}(l, m), m)$

\begin{equation}
\begin{split}
k\cdot l &= k\cdot l(\text{mod }m)\Longleftrightarrow\\
k \cdot l &= k\cdot \text{rem}(l, m) (\text{mod } m)\Longleftrightarrow\\
\text{rem}(k \cdot l, m) &\neq \text{rem}(k \cdot \text{rem}(l, m), m)\text{ as required}\\
\end{split}
\end{equation}

\end{enumerate}

\item Let $m$ be a positive integer.

\begin{enumerate}

\item Prove the associativity of the addition and multiplication operations in $\mathbb{Z}_m$; that is:
\begin{equation}
\forall i, j, k \in \mathbb{Z}_m. (i+_mj)+_mk=i+_m(j+_mk)\text{ and }(i\cdot_mj)\cdot_mk=i\cdot_m(j\cdot_mk)\\
\end{equation}

Proof of the associativity of the addition operation in $\mathbb{Z}_m$:
\begin{equation}
\begin{split}
\forall i, j, k \in \mathbb{Z}_m: s &= (i+_mj)+_mk\Longleftrightarrow\\
\forall i, j, k \in \mathbb{Z}_m: s &\equiv (i + j) + k (\text{mod } m)\Longleftrightarrow\\
\forall i, j, k \in \mathbb{Z}_m: s &\equiv i + j + k (\text{mod } m)\Longleftrightarrow\\
\forall i, j, k \in \mathbb{Z}_m: s &\equiv i + (j + k) (\text{mod } m) \Longleftrightarrow\\
\therefore &\forall i, j, k \in \mathbb{Z}_m. (i+_mj)+_mk=i+_m(j+_mk)\text{ as required}
\end{split}
\end{equation}

Proof of the associativity of the multiplication operation in $\mathbb{Z}_m$:
\begin{equation}
\begin{split}
\forall i, j, k \in \mathbb{Z}_m: p &= (i\cdot_mj)\cdot_mk\Longleftrightarrow\\
\forall i, j, k \in \mathbb{Z}_m: p &\equiv (i\cdot j)\cdot k (\text{mod } m)\Longleftrightarrow\\
\forall i, j, k \in \mathbb{Z}_m: p &\equiv i\cdot j \cdot k (\text{mod } m)\Longleftrightarrow\\
\forall i, j, k \in \mathbb{Z}_m: p &\equiv i\cdot (j \cdot k) (\text{mod } m)\Longleftrightarrow\\
\therefore \forall i, j, k \in \mathbb{Z}_m:(i\cdot_mj)\cdot_mk &= i\cdot_m(j\cdot_mk)\text{ as required}\\
\end{split}
\end{equation}

\item Prove that the additive inverse of $k$ in $\mathbb{Z}_m$ is $[-k]_m$.

\begin{equation}
\begin{split}
[-k]_m &= -k + m\Longleftrightarrow\\
k + [-k]_m &\equiv k - k + m (\text{mod } m)\Longleftrightarrow\\
k + [-k]_m &\equiv m (\text{mod } m)\Longleftrightarrow\\
k + [-k]_m &\equiv 0 (\text{mod } m)\\
\end{split}
\end{equation}

Since $k + [-k]_m \equiv 0 (\text{mod } m)$; $[-k]_m$ is the additive inverse of $k$ in $\mathbb{Z}_m$ as required.

\end{enumerate}

\end{enumerate}

\subsection*{2.2 Core exercises}

\begin{enumerate}

\item Find an integer $i$, natural numbers $k$, $l$ and a positive integer $m$ for which $k \equiv l (\text{mod } m)$ 
holds while $i^k \equiv i^l(\text{mod } m)$ does not.

\begin{center}
$i = 2$, $k = 1$, $l = 4$, $m = 3$
\begin{equation}
\begin{split}
1 &\equiv 4 (\text{mod } 3)\Longrightarrow\\
k &\equiv l (\text{mod } m)\\
\end{split}
\end{equation}
\begin{equation}
\begin{split}
2 &\not\equiv 1 (\text{mod } 3)\Longleftrightarrow\\
2^1 &\not\equiv 2^4 (\text{mod } 3)\Longrightarrow\\
i^k &\not\equiv i^l (\text{mod } m)\\
\end{split}
\end{equation}
\end{center}

\item Formalise and prove the following statement: A natural number is a multiple of 3 iff so is the number 
obtained by summing its digits. Do the same for analogous criterion for multiples of 9 and a similar condition 
for multiples of 11.

\begin{center}
Let $a_i$ be the $i^{th}$ digit of $n \in \mathbb{Z}$.
\begin{equation}
\begin{split}
n &\equiv \sum^\infty_{i=0}a_i\cdot 10^i (\text{mod } 3) \Longleftrightarrow\\
n &\equiv \sum^\infty_{i=0}a_i + a_i\cdot(10^i - 1) (\text{mod } 3) \Longleftrightarrow\\
\vspace{0.5em}
\forall i \geq 1: 10^i - 1 &= \sum^{i-1}_{j=0}9\cdot 10^j\Longleftrightarrow\\
\forall i \geq 1: 10^i - 1 &\equiv 9\cdot \sum^{i-1}_{j=0}\cdot 10^j (\text{mod } 3)\Longleftrightarrow\\
\forall i \geq 1: 10^i - 1 &\equiv 0 (\text{mod } 3)\\
\text{for } i = 0: 10^i - 1 &= 0 \Longleftrightarrow\\
10^i - 1 &\equiv 0 (\text{mod } 3)\\
\vspace{0.5em}
\therefore n &\equiv \sum^\infty_{i=0}a_i (\text{mod } 3)\\
n &\equiv 0 (\text{mod } 3) \Longleftrightarrow 3|n\\
\therefore \sum^\infty_{i=0}a_i& \equiv 0 (\text{mod } 3) \Longleftrightarrow 3|n
\end{split}
\end{equation}

Let $a_i$ be the $i^{th}$ digit of $n \in \mathbb{Z}$.
\begin{equation}
\begin{split}
n &\equiv \sum^\infty_{i=0}a_i\cdot 10^i (\text{mod } 9) \Longleftrightarrow\\
n &\equiv \sum^\infty_{i=0}a_i + a_i\cdot(10^i - 1) (\text{mod } 9) \Longleftrightarrow\\
\vspace{0.5em}
\forall i \geq 1: 10^i - 1 &= \sum^{i-1}_{j=0}9\cdot 10^j\Longleftrightarrow\\
\forall i \geq 1: 10^i - 1 &\equiv 9\cdot \sum^{i-1}_{j=0}\cdot 10^j (\text{mod } 9)\Longleftrightarrow\\
\forall i \geq 1: 10^i - 1 &\equiv 0 (\text{mod } 9)\\
\text{for } i = 0: 10^i - 1 &= 0 \Longleftrightarrow\\
10^i - 1 &\equiv 0 (\text{mod } 9)\\
\vspace{0.5em}
\therefore n &\equiv \sum^\infty_{i=0}a_i (\text{mod } 9)\\
n &\equiv 0 (\text{mod } 9) \Longleftrightarrow 9|n\\
\therefore \sum^\infty_{i=0}a_i& \equiv 0 (\text{mod } 9) \Longleftrightarrow 9|n
\end{split}
\end{equation}

Let $a_i$ be the $i^{th}$ digit of $n \in \mathbb{Z}$.

TOTO: correct the below. It is wrong. Counterexample: $11\not|999$
\begin{equation}
\begin{split}
n &\equiv \sum^\infty_{i=0}a_i\cdot 10^i (\text{mod } 11) \Longleftrightarrow\\
n &\equiv \sum^\infty_{i=0}a_i + a_i\cdot(10^i - 1) (\text{mod } 11) \Longleftrightarrow\\
\end{split}
\end{equation}
\begin{center}
Since $10^i - 1\equiv 0 (\text{mod } 11)$ for $i \geq 1$:
\end{center}
\begin{equation}
\begin{split}
n &\equiv \sum^\infty_{i=0}a_i (\text{mod } 11)\\
n &\equiv 0 (\text{mod } 11) \Longleftrightarrow 11|n\\
\therefore \sum^\infty_{i=0}a_i& \equiv 0 (\text{mod } 11) \Longleftrightarrow 11|n
\end{split}
\end{equation}
\end{center}

\item Show that for every integer $n$, the remainder when $n^2$ is divided by 4 is either 0 or 1.

This can be divided into two cases: $n$ is even or $n$ is odd:

\begin{center}
$n$ is even:
\begin{equation}
\begin{split}
\exists k \in\mathbb{Z}: n &= 2\cdot k\\
\therefore \exists k \in \mathbb{Z}: n &= 2\cdot k(\text{mod } 4)\\
n^2 &= 4\cdot k^2(\text{mod } 4)\\
n^2 &= 0(\text{mod } 4)\\
\therefore\text{ $n^2$}&\text{ divided by 4 is 0.}\\
\end{split}
\end{equation}
So if $n$ is even; the remainder when $n^2$ is divided by 4 is 0.

$n$ i odd:
\begin{equation}
\begin{split}
\exists k \in\mathbb{Z}: n &= 2\cdot k + 1\\
\therefore \exists k \in \mathbb{Z}: n &= 2\cdot k + 1(\text{mod } 4)\\
n^2 &= 4\cdot k^2 + 4\cdot k + 1(\text{mod } 4)\\
n^2 &= 1(\text{mod } 4)\\
\end{split}
\end{equation}
So if $n$ is odd; the remainder when $n^2$ is divided by 4 is 1.

Since every integer $n$ is either even or odd; the remainder when $n$ is divided by 4 is either 0 or 1.
\end{center}

\item What are $\text{rem}(55^2, 79)$, $\text{rem}(23^2, 79)$, $\text{rem}(23\cdot 55, 79)$ and $\text{rem}(55^{78}, 79)$?

\begin{equation}
\begin{split}
&\text{rem}(55^2, 79)\\
=&\text{rem}(3025, 79)\\
=&23
\end{split}
\end{equation}
\begin{equation}
\begin{split}
&\text{rem}(23^2, 79)\\
=&\text{rem}(529, 79)\\
=& 55
\end{split}
\end{equation}
\begin{equation}
\begin{split}
&\text{rem}(23 \cdot 55, 79)\\
=&\text{rem}(1265, 79)\\
=& 1
\end{split}
\end{equation}
\begin{equation}
\begin{split}
&\text{rem}(55^{78}, 79)\\
=& 1\text{ using Fermat's Little Theorem}\\
\end{split}
\end{equation}

\item Calculate that $2^{153} \equiv 53(\text{mod } 153)$. At first sight this seems to contradict Fermat's Little Theorem, 
why isn't this the case though? \textit{Hint}: Simplify the problem by applying known congruences to subexpressions.

This does not contradict Fermat's Little Theorem since 153 is not prime and Fermat's Little Theorem only applies to primes.

\begin{equation}
\begin{split}
153 &= 128 + 16 + 8 + 1\Longleftrightarrow\\
2^128 \cdot 2^16 \cdot 2^8 \cdot 2^1 &\equiv 2^{153}(\text{mod } 153)\\
\vspace{1em}
2^1 &\equiv 2(\text{mod } 153)\Longleftrightarrow\\
2^2 &\equiv 4(\text{mod } 153)\Longleftrightarrow\\
2^4 &\equiv 16(\text{mod } 153)\Longleftrightarrow\\
2^8 &\equiv 103(\text{mod } 153)\Longleftrightarrow\\
2^{16} &\equiv 52(\text{mod } 153)\Longleftrightarrow\\
2^{32} &\equiv 103(\text{mod } 153)\Longleftrightarrow\\
2^{64} &\equiv 52(\text{mod } 153)\Longleftrightarrow\\
2^{128} &\equiv 103(\text{mod } 153)\Longleftrightarrow\\
\vspace{1em}
2^{153}&\equiv 103\cdot 52 \cdot 103 \cdot 2\Longleftrightarrow\\
2^{153}&\equiv 52 \cdot 52 \cdot 2\Longleftrightarrow\\
2^{153}&\equiv 103 \cdot 2\Longleftrightarrow\\
2^{153}&\equiv 206 \Longleftrightarrow\\
2^{153}&\equiv 53 \Longleftrightarrow\\
\end{split}
\end{equation}

\item Calculate the addition and multiplication tables, and the additive and multiplicative inverse tables for 
$\mathbb{Z}_3$, $\mathbb{Z}_6$ and $\mathbb{Z}_7$.

Additive table for $\mathbb{Z}_3$
\begin{tabular}{c|c c c}
& 0 & 1 & 2\\
\hline
0 & 0 & 1 & 2\\
1 & 1 & 2 & 0\\
2 & 2 & 0 & 1\\
\end{tabular}

Additive inverse table for $\mathbb{Z}_3$
\begin{tabular}{c|c c c}
 & 0 & 1 & 2\\
inverse & 0 & 2 & 1\\
\end{tabular}

Multiplication table for $\mathbb{Z}_3$
\begin{tabular}{c|c c c}
& 0 & 1 & 2\\
\hline
0 & 0 & 0 & 0\\
1 & 0 & 1 & 2\\
2 & 0 & 2 & 1\\
\end{tabular}

Multiplicative inverse table for $\mathbb{Z}_3$
\begin{tabular}{c|c c c}
number & 0 & 1 & 2\\
inverse & & 1 & 2\\
\end{tabular}

Additive table for $\mathbb{Z}_6$
\begin{tabular}{c|c c c c c c}
& 0 & 1 & 2 & 3 & 4 & 5\\
\hline
0 & 0 & 1 & 2 & 3 & 4 & 5\\
1 & 1 & 2 & 3 & 4 & 5 & 0\\
2 & 2 & 3 & 4 & 5 & 0 & 1\\
3 & 3 & 4 & 5 & 0 & 1 & 2\\
4 & 4 & 5 & 0 & 1 & 2 & 3\\
5 & 5 & 0 & 1 & 2 & 3 & 4\\
\end{tabular}

Additive inverse table for $\mathbb{Z}_6$
\begin{tabular}{c|c c c c c c}
number & 0 & 1 & 2 & 3 & 4 & 5\\
inverse & 0 & 5 & 4 & 3 & 2 & 1\\
\end{tabular}

Multiplication table for $\mathbb{Z}_6$
\begin{tabular}{c|c c c c c c}
& 0 & 1 & 2 & 3 & 4 & 5\\
\hline
0 & 0 & 0 & 0 & 0 & 0 & 0\\
1 & 0 & 1 & 2 & 3 & 4 & 5\\
2 & 0 & 2 & 4 & 0 & 2 & 4\\
3 & 0 & 3 & 0 & 3 & 0 & 3\\
4 & 0 & 4 & 2 & 0 & 4 & 2\\
5 & 0 & 5 & 4 & 3 & 2 & 1\\
\end{tabular}

Multiplicative inverse table for $\mathbb{Z}_6$
\begin{tabular}{c|c c c c c c}
number & 0 & 1 & 2 & 3 & 4 & 5\\
inverse & & 1 & & & & 5\\
\end{tabular}

Additive table for $\mathbb{Z}_7$
\begin{tabular}{c|c c c c c c c}
& 0 & 1 & 2 & 3 & 4 & 5 & 6\\
\hline
0 & 0 & 1 & 2 & 3 & 4 & 5 & 6\\
1 & 1 & 2 & 3 & 4 & 5 & 6 & 0\\
2 & 2 & 3 & 4 & 5 & 6 & 0 & 1\\
3 & 3 & 4 & 5 & 6 & 0 & 1 & 2\\
4 & 4 & 5 & 6 & 0 & 1 & 2 & 3\\
5 & 5 & 6 & 0 & 1 & 2 & 3 & 4\\
6 & 6 & 0 & 1 & 2 & 3 & 4 & 5\\
\end{tabular}

Additive inverse table for $\mathbb{Z}_7$
\begin{tabular}{c|c c c c c c c}
number & 0 & 1 & 2 & 3 & 4 & 5 & 6\\
inverse & 0 & 6 & 5 & 4 & 3 & 2 & 1\\
\end{tabular}

Multiplication table for $\mathbb{Z}_7$
\begin{tabular}{c|c c c c c c c}
& 0 & 1 & 2 & 3 & 4 & 5 & 6\\
\hline
0 & 0 & 0 & 0 & 0 & 0 & 0 & 0\\
1 & 0 & 1 & 2 & 3 & 4 & 5 & 6\\
2 & 0 & 2 & 4 & 6 & 1 & 3 & 5\\
3 & 0 & 3 & 6 & 2 & 5 & 1 & 4\\
4 & 0 & 4 & 1 & 5 & 2 & 6 & 3\\
5 & 0 & 5 & 3 & 1 & 6 & 4 & 2\\
6 & 0 & 6 & 5 & 4 & 3 & 2 & 1\\
\end{tabular}

Multiplicative inverse table for $\mathbb{Z}_7$
\begin{tabular}{c|c c c c c c c}
number & 0 & 1 & 2 & 3 & 4 & 5 & 6\\
inverse & & 1 & 4 & 5 & 2 & 3 & 6\\
\end{tabular}

\item Let $i$ and $n$ be positive integers and let $p$ be a prime. Show that if $n \equiv 1 (\text{mod } p - 1)$ then 
$i^n \equiv i(\text{mod } p)$ for all $i$ not multiple of $p$.

If $i$ is not a multiple of $p$ then we can use Fermat's Little Theorem:
\begin{equation}
\begin{split}
n &\equiv 1(\text{mod } p - 1)\Longleftrightarrow\\
\exists k\in\mathbb{Z}: n &= 1 + (p - 1)\cdot k \Longleftrightarrow\\
\exists k \in \mathbb{Z}: i^n &\equiv i^{1 + (p-1)\cdot k}(\text{mod } p)\Longleftrightarrow\\
\exists k\in\mathbb{Z}: i^n &\equiv i \cdot (i^{p - 1})^k (\text{mod } p)\Longleftrightarrow\\
\text{ using Fermat's Little Theorem: } \exists k\in\mathbb{Z}: i^n &\equiv i \cdot 1^k (\text{mod } p) \Longleftrightarrow\\
i^n &\equiv i \cdot 1 (\text{mod } p) \Longleftrightarrow\\
i^n &\equiv i (\text{mod } p)\text{ as required}\\
\end{split}
\end{equation}

\item Prove that $n^3\equiv n (\text{mod } 6)$ for all integers $n$.

\begin{equation}
\begin{split}
n^3 - n &= (n-1)\cdot n\cdot (n+1)\\
\forall n \in \mathbb{Z}: 2|(n - 1)\cdot n\cdot (n+1) &\wedge 3|(n-1)\cdot n\cdot (n+1)\Longleftrightarrow\\
\forall n \in \mathbb{Z}: \exists i, j \in \mathbb{Z}: (n-1)\cdot n\cdot(n+1) &= 2\cdot i \wedge (n-1)\cdot n\cdot(n+1) = 3\cdot j\\
\forall n \in \mathbb{Z}: 3\cdot(n-1)\cdot n\cdot(n+1) - 2\cdot (n-1)\cdot n\cdot(n+1) &= 3\cdot(2\cdot i) -2\cdot (3\cdot j)\Longleftrightarrow\\
\forall n \in \mathbb{Z}: (n-1)\cdot n\cdot(n+1) &= 6\cdot (i - j)\Longleftrightarrow\\
\forall n \in \mathbb{Z}: (n - 1)n(n+1) &\equiv 0(\text{mod } 6) \Longleftrightarrow\\
\forall n \in \mathbb{Z}: n^3 - n &\equiv 0(\text{mod } 6) \Longleftrightarrow\\
\forall n \in \mathbb{Z}: n^3 &\equiv n(\text{mod } 6) \text{ as required}\\
\end{split}
\end{equation}

\item Prove that $n^7\equiv n(\text{mod } 42)$ for all integers $n$.
\vspace{-1em}
\begin{center}
\begin{equation}
\begin{split}
\forall n \in \mathbb{Z}: n^7 - n &= (n-1)\cdot n\cdot(n+1)\cdot(n^2-n+1)\cdot(n^2+n+1)\Longrightarrow\\
\forall n \in \mathbb{Z}: \exists k\in\mathbb{Z}: n^7 - n &= k\cdot n\cdot(n+1)\Longrightarrow\\
\forall n \in \mathbb{Z}: 2&|(n^7-n)\Longleftrightarrow\\
\forall n \in \mathbb{Z}: n^7 - n &\equiv 0 (\text{mod } 2)\\
\vspace{1em}
\forall n \in \mathbb{Z}: n^7 - n &= (n-1)n(n+1)(n^2-n+1)(n^2+n+1)\Longrightarrow\\
\forall n \in \mathbb{Z}: \exists k \in \mathbb{Z}: n^7 - n &= k\cdot (n-1) \cdot n \cdot (n + 1)\Longrightarrow\\
\forall n \in \mathbb{Z}: 3&|(n^7-n)\\
\forall n \in \mathbb{Z}: n^7 - n &\equiv 0 (\text{mod } 3)\\
\vspace{1em}
\forall n \in \mathbb{Z}: n^7 &\equiv n (\text{mod } 7) \text{ using Fermat's Little Theorem}\Longleftrightarrow\\
\forall n \in \mathbb{Z}: n^7 - n &\equiv 0(\text{mod } 7)\\
\end{split}
\end{equation}
\vspace{0.5em}
\begin{equation}
\begin{split}
\forall n \in \mathbb{Z}: (n^7 - n) \equiv 0(\text{mod } 2) \wedge (n^7 - n) &\equiv 0(\text{mod } 3) \wedge (n^7-n)\equiv0(\text{mod } 7) \Longleftrightarrow\\
\forall n \in \mathbb{Z}: \exists i, j, k \in \mathbb{Z}: (n^7 - n) &= 2\cdot i \wedge (n^7 - n) = 3\cdot j \wedge (n^7 - n) = 7\cdot k\Longrightarrow\\
\forall n \in \mathbb{Z}: 21\cdot (n^7 - n) - 14\cdot (n^7 - n) - 6\cdot (n^7 - n) &= 21\cdot (2\cdot i) - 14\cdot(3\cdot j) - 6\cdot (7\cdot k) \Longleftrightarrow\\
\forall n \in \mathbb{Z}: n^7 - n &= 42\cdot(i - j - k)\Longleftrightarrow\\
\forall n \in \mathbb{Z}: n^7 - n &= 0 (\text{mod } 42)\Longleftrightarrow\\
\forall n \in \mathbb{Z}: n^7 &= n (\text{mod } 42) \text{ as required}\\
\end{split}
\end{equation}
\end{center}

\end{enumerate}

\subsection*{2.3 Optional exercises}

\begin{enumerate}

\item Prove that for all integers $n$, there exist natural numbers $j$ and $j$ such that 
$n=i^2-j^2$ iff $n\equiv 0 (\text{mod } 4)$ or $n\equiv 1 (\text{mod } 4)$ or $n\equiv 3 (\text{mod } 4)$.


($\Longrightarrow$)

Assume $\exists i, j \in \mathbb{N}: n = i^2 - j^2$\\
The difference between $i$ and $j$ can either be even or odd.\\
So either $\exists k \in \mathbb{Z}: i = j + 2\cdot k \vee \exists k \in \mathbb{Z}: i = j + 2\cdot k + 1$.

\begin{equation}
\begin{split}
\exists k \in \mathbb{Z}: i &= j + 2\cdot k\Longleftrightarrow\\
\exists k \in \mathbb{Z}: n &= (j + 2\cdot k)^2 - j^2\Longleftrightarrow\\
\exists k \in \mathbb{Z}: n &= j^2 + 4\cdot k \cdot j + 4\cdot k^2 - j^2 \Longleftrightarrow\\
\exists k \in \mathbb{Z}: n &= 4\cdot (k\cdot j + k^2)\Longleftrightarrow\\
n &\equiv 0(\text{mod } 4)\\
\end{split}
\end{equation}
\begin{equation}
\begin{split}
\exists j, k \in \mathbb{Z}: i &= j + 2\cdot k + 1\Longleftrightarrow\\
\exists j, k \in \mathbb{Z}: n &= (j + 2\cdot k + 1)^2 - j^2\Longleftrightarrow\\
\exists j, k \in \mathbb{Z}: n &= j^2 + 2\cdot j \cdot (2\cdot k + 1) + (2\cdot k + 1)^2 - j^2\Longleftrightarrow\\
\exists j, k \in \mathbb{Z}: n &= 4\cdot j\cdot k + 2\cdot j + 4\cdot k^2 + 4\cdot k + 1\Longleftrightarrow\\
\exists j, k \in \mathbb{Z}: n &= 2\cdot j + 1 + 4\cdot (j\cdot k + k^2 + k)\Longleftrightarrow\\
\exists j \in \mathbb{Z}: n &\equiv 2\cdot j + 1(\text{mod } 4)\Longleftrightarrow\\
n &\equiv 1(\text{mod } 4)\vee n \equiv 3(\text{mod } 4)\\
\end{split}
\end{equation}
\begin{center}
$\therefore \exists i, j \in \mathbb{N}: n = i^i - j^2 \Longrightarrow n \equiv 0(\text{mod } 4) \vee n \equiv 1(\text{mod } 4) \vee n \equiv 3(\text{mod } 4)$
\end{center}
\vspace{0.2em}
($\Longleftarrow$)
\begin{equation}
\begin{split}
n &\equiv 0 (\text{mod } 4)\Longleftrightarrow\\
\exists k \in \mathbb{Z}: n &= 4\cdot k\\
\text{Let } i &= k + 1 \text{ and } j = k - 1\\
& i^2 - j^2\\
=& (k + 1)^2 - (k - 1)^2\\
=& k^2 + 2\cdot k + 1 - k^2 + 2\cdot k - 1\\
=& 4\cdot k\\
=& n\\
\therefore n &\equiv 0(\text{mod } 4) \Longrightarrow \exists i, j \in \mathbb{Z}: n= i^2 - j^2\\
\end{split}
\end{equation}

\begin{equation}
\begin{split}
n &\equiv 1 (\text{mod } 4) \Longleftrightarrow\\
\exists k \in \mathbb{Z}: n &= 4\cdot k + 1\\
\text{Let } i &= 2\cdot k + 1 \text{ and } j = 2\cdot k\\
& i^2 - j^2\\
=& (2\cdot k + 1)^2 - (2\cdot k)^2\\
=& 4\cdot k^2 + 4\cdot k + 1 - 4\cdot k^2\\
=& 4\cdot k + 1\\
=& n\\
\therefore n &\equiv 1(\text{mod } 4) \Longrightarrow \exists i, j \in \mathbb{Z}: n= i^2 - j^2\\
\end{split}
\end{equation}

\begin{equation}
\begin{split}
n &\equiv 3(\text{mod } 4) \Longleftrightarrow\\
\exists k \in \mathbb{Z}: n &= 3 + 4\cdot k\\
\text{Let } i &= 2\cdot k + 2 \text{ and } j = 2\cdot k + 1\\
& i^2 - j^2\\
=& (2\cdot k + 2)^2 - (2\cdot k + 1)^2\\
=& 4\cdot k^2 + 8\cdot k + 4 - 4\cdot k^2 - 4\cdot k - 1\\
=& 4\cdot k + 3\\
=& n\\
\therefore n &\equiv 3(\text{mod } 4) \Longrightarrow \exists i, j \in \mathbb{Z}: n=i^2 - j^2\\
\end{split}
\end{equation}
\begin{center}
$\therefore \exists i, j \in \mathbb{N}: n = i^i - j^2 \Longleftarrow n \equiv 0(\text{mod } 4) \vee n \equiv 1 (\text{mod } 4) \vee n \equiv 3 (\text{mod } 4)$

$\therefore \exists i, j \in \mathbb{N}: n = i^i - j^2 \Longleftrightarrow n \equiv 0(\text{mod } 4) \vee n \equiv  1 (\text{mod } 4) \vee n \equiv 3 (\text{mod } 4)$
\end{center}

\item A \textit{decimal} (respectively \textit{binary}) \textit{repunit} is a natural number whose decimal 
(respectively binary) representation consists solely of 1's.

\begin{enumerate}

\item What are the first three decimal repunits? And the first three binary ones?

The first three decimal repunits are $1_{10}$, $11_{10}$ and $111_{10}$.\\
The first three binary repunits are $1_2$ ($1_{10}$), $11_2$ ($3_{10}$) and $111_2$ ($7_{10}$).\\

\item Show that no decimal repunit strictly greater than 1 is a square, and that the same holds 
for binary repunits. Is this the case for every base?

Show that there is no number which squares to end in $11_{10}$.
\begin{center}
Proof by contradiction. \\
Assume there is a decimal repunit $r > 1$ that is a square.
\begin{equation}
\begin{split}
\text{Assume: }\exists k \in \mathbb{Z}: k^2 &= r\\
\exists k \in \mathbb{Z}: k^2 &= r\Longrightarrow\\
\exists k \in \mathbb{Z}: k^2 &\equiv 11(\text{mod } 100)\Longrightarrow\\
k^2 \equiv 1(\text{mod } 10) &\Longleftrightarrow\\
\text{By ins}&\text{pection of the multiplication table of }\mathbb{Z}_m\\
\exists i \in \mathbb{Z}: k = 10\cdot i + 1 \vee k = 10\cdot i + 9\\
\end{split}
\end{equation}
Case 1: $k = 10\cdot i + 1$
\begin{equation}
\begin{split}
(10\cdot i + 1)^2 &\equiv 11(\text{mod } 100)\Longleftrightarrow\\
100\cdot i^2 + 20\cdot i + 1 &\equiv 11(\text{mod } 100)\Longleftrightarrow\\
20\cdot i &\equiv 10 (\text{mod } 100)\Longleftrightarrow\\
2\cdot i &\equiv 1(\text{mod } 10)\\
\text{Which}&\text{ is false, because by inspection}\\
\nexists i \in \mathbb{Z}: 2\cdot i &\equiv 1(\text{mod } 10)\\
\end{split}
\end{equation}
However, this contradicts the original assumption that $\exists i \in \mathbb{Z}: (10\cdot i + 1)^2 = r$.

Case 2: $k = 10\cdot i + 9$
\begin{equation}
\begin{split}
(10\cdot i + 9)^2 &\equiv 11(\text{mod } 100)\Longleftrightarrow\\
100\cdot i^2 + 20\cdot i + 81 &\equiv 11(\text{mod } 100)\Longleftrightarrow\\
20\cdot i &\equiv 30(\text{mod } 100)\Longleftrightarrow\\
2\cdot i &\equiv 3(\text{mod } 10)\\
\text{Which}&\text{ is false because by inspection}\\
\nexists i \in \mathbb{Z}: 2\cdot i &\equiv 3(\text{mod } 10)\\
\end{split}
\end{equation}
However, this contradicts the original assumption that $\exists i \in \mathbb{Z}: (10\cdot i + 9)^2 = r$.

So $\nexists k \in \mathbb{Z}: k^2 = r$ for any repunit $r > 1$. As required.

Assume that there is an integer $k$ such that $k^2 = r$ for some binary repunit.
\begin{equation}
\begin{split}
\exists k \in \mathbb{Z}: k^2 &= r\Longleftrightarrow\\
\exists n \in \mathbb{Z}: (2\cdot n)^2 &= r \vee (2\cdot n + 1)^2 = r\\
\end{split}
\end{equation}
Case 1: $\exists n \in \mathbb{Z}: (2\cdot n)^2 = r$.
\begin{equation}
\begin{split}
\exists n \in \mathbb{Z}: (2\cdot n)^2 &= r\Longrightarrow\\
\exists n \in \mathbb{Z}: 4\cdot n^2 &\equiv 3(\text{mod } 4)\Longleftrightarrow\\
0 &\equiv 3 (\text{mod } 4)\\
\end{split}
\end{equation}
However, this is not true. So $\nexists n \in \mathbb{Z}: (2\cdot n)^2 = r$

Case 2: $\exists n \in \mathbb{Z}: (2\cdot n + 1)^2 = r$
\begin{equation}
\begin{split}
\exists n \in \mathbb{Z}: (2\cdot n + 1)^2 &= r\Longrightarrow\\
\exists n \in \mathbb{Z}: (2\cdot n + 1)^2 &\equiv 3(\text{mod } 4)\Longleftrightarrow\\
\exists n \in \mathbb{Z}: 4\cdot n^2 + 4\cdot n + 1 &\equiv 3(\text{mod } 4)\Longrightarrow\\
1 &\equiv 3(\text{mod } 4)\\
\end{split}
\end{equation}
However, this is not true. So $\nexists n \in \mathbb{Z}: (2\cdot n + 1)^2 = r$.
\end{center}
Since all numbers are even or odd and $r$ cannot be the square of an even number or an odd number: 
$r$ cannot be the square of any number -- hence $r$ cannot be a square number. 
Since $r$ was arbitrary this proves that there are no binary repunits that are square numbers.

This is not the case for every base: consider base $k^2 - 1$ for some number k.\\
In base $k^2 - 1$: $k^2 = 11_{k^2 - 1}$.

\end{enumerate}

\end{enumerate}

\end{document}
