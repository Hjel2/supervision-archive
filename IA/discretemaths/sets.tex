\newcommand{\svcourse}{CST Part IA: Introduction to Probability}
\newcommand{\svnumber}{1}
\newcommand{\svvenue}{Churchill, Room TBD}
\newcommand{\svdate}{2022-05-14}
\newcommand{\svtime}{11:00}
\newcommand{\svuploadkey}{PO5ogKIM8KQA22FZS8IAf8gxA8XKi19jxIBVHIfFZ+3GCBXuNUXS9lVN6bNYjxM/}

\newcommand{\svrname}{Mr Matthew Ireland}
\newcommand{\jkfside}{twoside}
\newcommand{\jkfhanded}{right}

\newcommand{\studentname}{Harry Langford}
\newcommand{\studentemail}{hjel2@cam.ac.uk}


\documentclass[10pt,\jkfside,a4paper]{article}

\input{../../template/includes.tex}
% DO NOT add \usepackage commands here.  Place any custom commands
% into your SV work files.  Anything in the template directory is
% likely to be overwritten!

\usepackage{fancyhdr}

\usepackage{lastpage}       % ``n of m'' page numbering
\usepackage{lscape}         % Makes landscape easier

\usepackage{verbatim}       % Verbatim blocks
\usepackage{epsfig}         % Embed encapsulated postscript
\usepackage{array}          % Array environment
\usepackage[nolinks]{qrcode}         % QR codes
\usepackage{enumitem}       % Required by Tom Johnson's exam question header

\usepackage{hhline}         % Horizontal lines in tables
\usepackage{siunitx}        % Correct spacing of units
\usepackage{amsmath}        % American Mathematical Society
\usepackage{amssymb}        % Maths symbols
\usepackage{amsthm}         % Theorems

\usepackage{ifthen}         % Conditional processing in tex

\usepackage[top=3cm,
            bottom=3cm,
            inner=2cm,
            outer=5cm]{geometry}

% PDF metadata + URL formatting
\usepackage[
            pdfauthor={\studentname},
            pdftitle={\svcourse, SV \svnumber},
            pdfsubject={},
            pdfkeywords={9d2547b00aba40b58fa0378774f72ee6},
            pdfproducer={},
            pdfcreator={},
            hidelinks]{hyperref}

\renewcommand{\headrulewidth}{0.4pt}
\renewcommand{\footrulewidth}{0.4pt}
\fancyheadoffset[LO,LE,RO,RE]{0pt}
\fancyfootoffset[LO,LE,RO,RE]{0pt}
\pagestyle{fancy}
\fancyhead{}
\fancyhead[LO,RE]{{\bfseries \studentname}\\\studentemail}
\fancyhead[RO,LE]{{\bfseries \svcourse, SV~\svnumber}\\\svdate\ \svtime, \svvenue}
\fancyfoot{}
\fancyfoot[LO,RE]{For: \svrname}
\fancyfoot[RO,LE]{\today\hspace{1cm}\thepage\ / \pageref{LastPage}}
\fancyfoot[C]{\qrcode[height=0.8cm]{\svuploadkey}}
\setlength{\headheight}{22.55pt}

\ifthenelse{\equal{\jkfside}{oneside}}{

 \ifthenelse{\equal{\jkfhanded}{left}}{
  % 1. Left-handed marker, one-sided printing or e-marking, use oneside and...
  \evensidemargin=\oddsidemargin
  \oddsidemargin=73pt
  \setlength{\marginparwidth}{111pt}
  \setlength{\marginparsep}{-\marginparsep}
  \addtolength{\marginparsep}{-\textwidth}
  \addtolength{\marginparsep}{-\marginparwidth}
 }{
  % 2. Right-handed marker, one-sided printing or e-marking, use oneside.
  \setlength{\marginparwidth}{111pt}
 }

}{
 % 3. Alternating margins, two-sided printing, use twoside.
}

\setlength{\parindent}{0em}
\addtolength{\parskip}{1ex}

% Exam question headings, labels and sensible layout (courtesy of Tom Johnson)
\setlist{parsep=\parskip, listparindent=\parindent}
\newcommand{\examhead}[3]{\section{#1 Paper #2 Question #3}}
\newenvironment{examquestion}[3]{
    \examhead{#1}{#2}{#3}\setlist[enumerate, 1]{label=(\alph*)}\setlist[enumerate, 2]{label=(\roman*)}
    \marginpar{\qrcode{https://www.cl.cam.ac.uk/teaching/exams/pastpapers/y#1p#2q#3.pdf}}
    \marginpar{\footnotesize \url{https://www.cl.cam.ac.uk/teaching/exams/pastpapers/y#1p#2q#3.pdf}}
}{}



\usepackage{multicol}

\begin{document}

\section*{5. On sets}

\subsection*{5.1 Basic exercises}

\begin{enumerate}

\item Prove that $\subseteq$ is a partial order, that is, it is:

\begin{enumerate}

\item reflexive: $\forall$ sets $A$, $A \subseteq A$

I shall prove that every element in $A$ is also in $A$.

\begin{equation}
\begin{split}
\forall a \in A: a &\in A \Longleftrightarrow\\
A &\subseteq A\\
\end{split}
\end{equation}

\item transistive: $\forall$ sets $A$, $B$, $C$. $(A \subseteq B \wedge B \subseteq C) \Longrightarrow A \subseteq C$

Assume $A \subseteq B \wedge B \subseteq C$
\begin{equation}
\begin{split}
\text{Take an arbitrary } a &\in A \\
\text{By assumption } a & \in A \Longrightarrow a \in B\\
\text{By assumption } a &\in B \Longrightarrow a \in C\\
\text{So: } \forall a &\in A: a \in C \Longleftrightarrow\\
A &\subseteq C\\
\end{split}
\end{equation}

\item antisymmetric: $\forall$ sets $A$, $B$. $(A \subseteq B \wedge B \subseteq A) \Longleftrightarrow A = B$

$\subseteq$ is antisymmetric. 

So $A \subseteq B \wedge B \subseteq A \Longleftrightarrow A = B$ as required.

\end{enumerate}

\item Prove the following statements:

\begin{enumerate}

\item $\forall$ sets $A$. $\emptyset \subseteq A$

By definition if $S$ is a set:
\begin{equation}
S \subseteq A \Longleftrightarrow \forall s \in S: s \in A\\
\end{equation}
For $S = \emptyset$ this is vacuously true.
\begin{equation}
\begin{split}
(\emptyset \subseteq A &\Longleftrightarrow \forall s \in \emptyset: s \in A)\Longleftrightarrow\\
(\emptyset \subseteq A &\Longleftrightarrow \text{true})\Longleftrightarrow\\
\emptyset &\subseteq A\text{ as required}\\
\end{split}
\end{equation}

\item $\forall$ sets $A$. $(\forall x: x \notin A) \Longleftrightarrow A = \emptyset$

By definition $\forall x: x \notin \emptyset$. 

\begin{equation}
\begin{split}
A &= \emptyset \wedge \forall x: x \notin \emptyset \text{ by definition} \Longleftrightarrow\\
\forall x:& x \notin A \text{ as required}\\
\end{split}
\end{equation}

\end{enumerate}

\item Find the union, and intersection of: 

\begin{enumerate}

\item $\{1, 2, 3, 4, 5\}$ and $\{-1, 1, 3, 5, 7\}$

\begin{equation}
\begin{split}
  &\{1, 2, 3, 4, 5\} \cup \{-1, 1, 3, 5, 7\}\\
= &\{-1, 1, 2, 3, 4, 5, 7\}\\
\end{split}
\end{equation}
\begin{equation}
\begin{split}
  &\{1, 2, 3, 4, 5\} \cap \{-1, 1, 3, 5, 7\}\\
= &\{1, 3, 5\}\\
\end{split}
\end{equation}

\item $\{x \in \mathbb{R}| x > 7\}$ and $\{ x \in \mathbb{N}: x > 5\}$

\begin{equation}
\begin{split}
 & \{x \in \mathbb{R}: x > 7\} \cup \{ x \in \mathbb{N}: x > 5\}\\
=& \{x \in \mathbb{R}: x > 7 \vee x \in \{6, 7\}\}\\
\end{split}
\end{equation}
\begin{equation}
\begin{split}
 & \{x \in \mathbb{R}: x > 7\} \cap \{ x \in \mathbb{N}: x > 5\}\\
=& \{x \in \mathbb{N}: x > 7\}\\
\end{split}
\end{equation}

\end{enumerate}

\item Find the Cartesian product and disjoint union of $\{1, 2, 3, 4, 5\}$ and $\{-1, 1, 3, 5, 7\}$.

The Cartesian product of two sets $S$ and $T$ is $\{x: \forall s \in S, \forall t \in T: x = (s, t)\}$

For the sets $\{1, 2, 3, 4, 5\}$ and $\{-1, 1, 3, 5, 7\}$ this is equal to:
\begin{equation}
\begin{split}
&\{(1, -1), (1, 1), (1, 3), (1, 5), (1, 7), (2, -1), (2, 1), (2, 3), (2, 5), (2, 7), (3, -1), (3, 1), (3, 3), \\&(3, 5), (3, 7), (4, -1), (4, 1), (4, 3), (4, 5), (4, 7), (5, -1), (5, 1), (5, 3), (5, 5), (5, 7)\}\\
\end{split}
\end{equation}

\item Let $I = \{2, 3, 4, 5\}$ and for each $i \in I$, let $A_i = \{i, i + 1, i - 1, 2\cdot i\}$.

\begin{enumerate}

\item List the elements of all sets $A_i$ for $i \in I$

\begin{equation}
\begin{split}
A_2 &= \{1, 2, 3, 4\}\\
A_3 &= \{2, 3, 4, 6\}\\
A_4 &= \{3, 4, 5, 8\}\\
A_5 &= \{4, 5, 6, 10\}\\
\end{split}
\end{equation}

\item Let $\{A_i| i \in I\}$ stand for $\{A_2, A_3, A_4, A_5\}$. Find $\bigcup\{A_i| i \in I\}$ and $\bigcap\{A_i| i \in I\}$.

\begin{equation}
\bigcup\{A_i: i \in I\} = \{1, 2, 3, 4, 5, 6, 8, 10\}\\
\end{equation}
\begin{equation}
\bigcap\{A_i:i \in I\} = \{4\}\\
\end{equation}

\end{enumerate}

\item Let $U$ be a set. For all $A, B \in \mathcal{P}(U)$, prove that:

\begin{enumerate}

\item $A^\mathsf{c} = B \Longleftrightarrow (A \cup B = U \wedge A \cap B = \emptyset)$

\begin{equation}
\begin{split}
A^\mathsf{c} &= B \Longleftrightarrow\\
(\forall b \in U: b \notin A &\Longleftrightarrow b \in B) \Longleftrightarrow\\
A \cup B &= \forall u \in U: u \in A \vee u \notin A \Longleftrightarrow\\
A \cup B &= U\\
\end{split}
\end{equation}

\item Double complement elimination: $(A^\mathsf{c})^\mathsf{c} = A$

\begin{equation}
\begin{split}
A^{\mathsf{c}} &\triangleq \{u | u \in U \wedge u \notin A\}\\
(A^{\mathsf{c}})^{\mathsf{c}} &= \{u' | u' \in U \wedge u' \notin \{u | u \in U \wedge u \notin A\}\}\\
							  &= \{u' | u' \in U \wedge \overline{(u' \in U \wedge u' \notin A)}\}\\
							  &= \{u' | u' \in U \wedge (u' \notin U \vee u' \in A)\}\\
							  &= \{u' | (u' \in U \wedge u' \notin U) \vee (u' \in U \wedge u' \in A)\}\\
							  &= \{u' | u' \in U \wedge u' \in A\}\\
							  &= \{u' | u' \in A\} \text{ (Since } A \subseteq U: u' \in A \Longrightarrow u' \in U)\\
							  &= A\\
\end{split}
\end{equation}

\item The De-Morgan laws: $(A \cup B)^\mathsf{c} = A^\mathsf{c} \cap B^\mathsf{c}$ and $(A \cap B)^\mathsf{c} = A^\mathsf{c}\cup B^\mathsf{c}$

\begin{equation}
\begin{split}
(A \cup B)^{\mathsf{c}} &= \{x | x \notin A \cup B\} \\
						&= \{x | x \notin A \wedge x \notin B\}\\
						&= \{x | x \notin A\} \cap \{x | x \notin B\}\\
						&= A^{\mathsf{c}} \cap B^{\mathsf{c}}
\end{split}
\end{equation}

\begin{equation}
\begin{split}
(A \cap B)^{\mathsf{c}} &= \{x | x \notin A \cap B\}\\
						&= \{x | x \notin A \vee x \notin B\}\\
						&= \{x | x \notin A\} \cup \{x | x\notin B\}\\
						&= A^{\mathsf{c}} \cap B^{\mathsf{c}}\\
\end{split}
\end{equation}

\end{enumerate}

\end{enumerate}

\subsection*{5.2 Core exercises}

\begin{enumerate}

\item Prove that for all sets $U$ and subsets $A, B \subseteq U$:

\begin{enumerate}

\item $\forall X: A \subseteq X \wedge B \subseteq X \Longleftrightarrow (A \cup B) \subseteq X$

\begin{equation}
\begin{split}
\forall X : A &\subseteq X \wedge B \subseteq X \Longleftrightarrow\\
\forall a \in A : a &\in X \wedge \forall b \in B : b \in X \Longleftrightarrow\\
\forall x \in A \cup B : x &\in X \Longleftrightarrow\\
A \cup B &\subseteq X\\
\end{split}
\end{equation}

\item $\forall Y: Y \subseteq A \wedge Y \subseteq B \Longleftrightarrow Y \subseteq (A \cap B)$

\begin{equation}
\begin{split}
\forall Y : Y &\subseteq A \wedge Y \subseteq B \Longleftrightarrow\\
\forall y \in Y : y &\in A \wedge y \in B \Longleftrightarrow\\
\forall y \in Y : y &\in A \cap B \Longleftrightarrow\\
Y &\subseteq A \cap B\\
\end{split}
\end{equation}

\end{enumerate}

\item Either prove or disprove that, for all sets $A$ and $B$,

\begin{enumerate}

\item $A \subseteq B \Longrightarrow \mathcal{P}(A) \subseteq \mathcal{P}(B)$

\begin{equation}
\begin{split}
\mathcal{P}(A) &\triangleq \{a | a \subseteq A\} \\
A &\subseteq B \Longleftrightarrow\\
(\forall a: (a \subseteq A) &\Longrightarrow (a \subseteq B)) \Longleftrightarrow\\
\{a | a \subseteq A\} &\subseteq \{b | b \subseteq B\} \Longleftrightarrow\\
\mathcal{P}(A) &\subseteq \mathcal{P}(B)\\
\end{split}
\end{equation}

\item $\mathcal{P}(A \cup B) \subseteq \mathcal{P}(A) \cup \mathcal{P}(B)$

Disproof by counter-example:
\begin{equation}
\begin{split}
\text{Let } A &= \{0\} \wedge B = \{1\} \Longleftrightarrow\\
\mathcal{P}(A \cup B) &= \mathcal{P}(\{0, 1\}) = \{\emptyset, \{0\}, \{1\}, \{0, 1\}\} \wedge\\
\mathcal{P}(A) \cup \mathcal{P}(B) &= \{\emptyset, \{0\}\} \cup \{\emptyset, \{1\}\} = \{\emptyset, \{0\}, \{1\}\}\\
\text{In this case: } \mathcal{P}(A \cup B) &\nsubseteq \mathcal{P}(A) \cup \mathcal{P}(B)\\
\end{split}
\end{equation}

\item $\mathcal{P}(A) \cup \mathcal{P}(B) \subseteq \mathcal{P}(A \cup B)$

\begin{equation}
\begin{split}
\mathcal{P}(S) &\triangleq \{s | s \subseteq S\} \Longrightarrow\\
\mathcal{P}(A) \cup \mathcal{P}(B) &= \{x | x \subseteq A \vee x \subseteq B\}\\
\mathcal{P}(A \cup B) &= \{x | x \subseteq A \cup B\}\\
\forall x: x \subseteq A \vee x \subseteq B &\Longrightarrow x \subseteq A \cup B \Longrightarrow\\
\{x | x \subseteq A \vee x \subseteq B\} &\subseteq \{x | x \subseteq A \cup B\} \Longleftrightarrow\\
\mathcal{P}(A) \cup \mathcal{P}(B) &\subseteq \mathcal{P}(A \cup B)\\
\end{split}
\end{equation}

\item $\mathcal{P}(A \cap B) \subseteq \mathcal{P}(A) \cap \mathcal{P}(B)$

\begin{equation}\label{powersetunions}
\begin{split}
\mathcal{P}(A) \cup \mathcal{P}(B) &= \{a | a \subseteq A \} \cup \{b | b \subseteq B\}\\
								   &= \{a | a \subseteq A \wedge a \subseteq B\}\\
								   &= \{a | a \subseteq A \cap B\}\\
								   &= \mathcal{P}(A \cap B)\\
\end{split}
\end{equation}
\begin{equation}
\begin{split}
\text{From (\ref{powersetunions}): } \mathcal{P}(A) \cup \mathcal{P}(B) &= \mathcal{P}(A \cap B) \Longleftrightarrow\\
\mathcal{P}(A \cap B) &\subseteq \mathcal{P}(A) \cap \mathcal{P}(B)\\
\end{split}
\end{equation}

\item $\mathcal{P}(A) \cap \mathcal{P}(B) \subseteq \mathcal{P}(A \cap B)$

\begin{equation}
\begin{split}
\text{From (\ref{powersetunions}): } \mathcal{P}(A) \cup \mathcal{P}(B) &= \mathcal{P}(A \cap B) \Longleftrightarrow\\
\mathcal{P}(A) \cap \mathcal{P}(B) &\subseteq \mathcal{P}(A \cap B)\\
\end{split}
\end{equation}

\end{enumerate}

\item Let $U$ be a set. For all $A, B \in \mathcal{P}(U)$ prove that the following statements are equivalent.

\begin{multicols}{4}

\begin{enumerate}

\item $A \cup B = B$

\item $A \subseteq B$

\item $A \cap B = A$

\item $B^\mathsf{c} \subseteq A^\mathsf{c}$

\end{enumerate}

\end{multicols}

We will show that all the statements are equivalent to $A \subseteq B$.

\begin{equation}
\begin{split}
A \cup B &= B \Longleftrightarrow\\
(\forall x \in A \cup B \Longleftrightarrow x &\in B) \Longleftrightarrow\\
(\forall x : (x \in A \vee x &\in B) \Longleftrightarrow x \in B) \Longleftrightarrow\\
\forall x : x &\in A \Longrightarrow x \in B \Longleftrightarrow\\
\forall x \in A: x &\in B \Longleftrightarrow\\
A &\subseteq B\\
\end{split}
\end{equation}

\begin{equation}
\text{Trivially: } A \subseteq B \Longleftrightarrow A \subseteq B\\
\end{equation}

\begin{equation}
\begin{split}
A \cap B &= A \Longleftrightarrow\\
\forall x (x \in A \cap B \Longrightarrow x &\in A) \Longleftrightarrow\\
\forall x : ((x \in A \wedge x &\in B) \Longleftrightarrow x \in A) \Longleftrightarrow\\
\forall x: (x \in A &\Longrightarrow x \in B) \Longleftrightarrow\\
A &\subseteq B\\
\end{split}
\end{equation}

\begin{equation}
\begin{split}
B^{\mathsf{c}} &\subseteq A^{\mathsf{c}} \Longleftrightarrow\\
\forall x \notin B &\Longrightarrow x \notin A \Longleftrightarrow\\
\forall x \in A &\Longrightarrow x \in B \Longleftrightarrow\\
A &\subseteq B\\
\end{split}
\end{equation}

\item For sets $A, B, C, D$, prove or disprove at least three of the following statements:

\begin{enumerate}

\item $(A \subseteq C \wedge B \subseteq D) \Longrightarrow (A \times B \subseteq C \times D)$

\begin{equation}
\begin{split}
\text{Assume: } A \subseteq C &\wedge B \subseteq D \Longleftrightarrow\\
				(a \in A &\Longrightarrow a \in C) \wedge (b \in B \Longrightarrow b \in D)\\
A \times B &= \{s| \exists a \in A \wedge \exists b \in B. s = (a, b)\} \Longrightarrow\\
A \times B &\subseteq \{s| \exists a \in B \wedge \exists b \in D. s = (a, b)\}\Longleftrightarrow\\
A \times B &\subseteq C \times D\\
\end{split}
\end{equation}

\item $(A \cup C) \times (B \cup D) \subseteq (A \times B) \cup (C \times D)$

Proof by counterexample:
\begin{equation}
\begin{split}
\text{Let: } A = \emptyset, B = \{1\}&, C = \{2\}, D = \{3\}\\
\text{So: } (A \cup C) \times (B \cup D) &= \{2\} \times \{1, 3\} \\
										 &= \{(2, 1), (2, 3)\}\\
\text{And: } (A \times B) \cup (C \times D) &= (\emptyset \times \{1\}) \cup (\{2\} \times \{3\})\\
											&= \emptyset \cup \{(2, 3)\}\\
											&= \{(2, 3)\}\\
\text{So in this case: } (A \cup C) &\times (B \cup D) \nsubseteq (A \times B) \cup (C \times D)\\
\end{split}
\end{equation}

\item $(A \times C) \cup (B \times D) \subseteq (A \cup B) \times (C \cup D)$

I will prove distributivity of $\times$ and $\cup$ to use in this and subsequent proofs.
\begin{equation}\label{distributivity}
\begin{split}
A \times (B \cup C) =& \{s| \exists a \in A, \exists x \in B \cup C. s = (a, x)\}\\
=& \{s| \exists a \in A, \exists x. (x \in B \cup x \in C). s = (a, x)\}\\
=& \{(\exists a \in A \wedge \exists x \in B) \cup (\exists a \in A \wedge \exists x \in C). s = (a, x)\}\\
=& \{(\exists a \in A \wedge \exists x \in B). s = (a, x)\} \cup \{(\exists a \in A \wedge \exists x \in C). s = (a, x)\}\\
=& (A \times B) \cup (A \times C)\\
\end{split}
\end{equation}

\begin{equation}
\begin{split}
(A \times C) \cup (B \times D) &\subseteq (A \times C) \cup (A \times D) \cup (B \times C) \cup (B \times D)\\
							   &\subseteq (A \times (C \cup D)) \cup (B \times (C \cup D))\\
							   &\subseteq (A \cup B) \times (C \cup D)\text{ as required}\\
\end{split}
\end{equation}

\item $A \times (B \cup C) \subseteq (A \times B) \cup (A \times C)$

\begin{equation}
\begin{split}
A \times (B \cup C) &= (A \times B) \cup (A \times C) \Longrightarrow \text{ using (\ref{distributivity})} \\ 
A \times (B \cup C) &\subseteq (A \times B) \cup (A \times C) \text{ using the antisymmetry of $\subseteq$}\\
\end{split}
\end{equation}

\item $(A \times B) \cup (A \times D) \subseteq A \times (B \cup D)$

\begin{equation}
\begin{split}
(A \times B) \cup (A \times D) &= A \times (B \cup D) \Longrightarrow \text{ using (\ref{distributivity})} \\ 
(A \times B) \cup (A \times D) &\subseteq A \times (B \cup D) \text{ using the antisymmetry of $\subseteq$}\\
\end{split}
\end{equation}

\end{enumerate}

\item For sets $A, B, C, D$, prove or disprove at least three of the following statements:

\begin{enumerate}

\item $(A \subseteq C \wedge B \subseteq D) \Longrightarrow A \uplus B \subseteq C \uplus D$

\begin{equation}
\begin{split}
\text{Assume: } A \subseteq C &\wedge B \subseteq D \Longrightarrow\\
a \in A \Longrightarrow a \in C &\wedge b \in B \Longrightarrow b \in D\\
x \in (A \uplus B) &\Longleftrightarrow (\exists a \in A. x = (1, a)) \vee (\exists b \in B. x = (2, b)) \Longrightarrow\\
x \in (A \uplus B) &\Longrightarrow (\exists a \in C. x = (1, a)) \vee (\exists b \in D. x = (2, b)) \Longleftrightarrow\\
x \in (A \uplus B) &\Longrightarrow x \in (C \uplus D) \Longleftrightarrow\\
A \uplus B &\subseteq C \uplus D\\
\end{split}
\end{equation}

\item $(A \cup B) \uplus C \subseteq (A \uplus C) \cup (B \uplus C)$

I will prove the distributivity of $\uplus$ and $\cup$.
\begin{equation}\label{upluscup}
\begin{split}
x \in (A \cup B) \uplus C \Longleftrightarrow& (\exists a \in A \cup B. x = (1, a)) \vee (\exists c \in C. x = (2, c)) \Longleftrightarrow\\
x \in (A \cup B) \uplus C \Longleftrightarrow& (\exists a \in A. x = (1, a)) \vee (\exists b \in B. x = (1, b)) \vee (\exists c \in C. x = (2, c)) \Longleftrightarrow\\
x \in (A \cup B) \uplus C \Longleftrightarrow& ((\exists a \in A. x = (1, a)) \vee (\exists c \in C. x = (2, c))) \vee \\
						 &((\exists b \in B. x = (1, b)) \vee (\exists c \in C. x = (2, c))) \Longleftrightarrow\\
x \in (A \cup B) \uplus C \Longleftrightarrow& x \in (A \uplus C) \vee x \in (B \uplus C)\Longleftrightarrow\\
x \in (A \cup B) \uplus C \Longleftrightarrow& x \in ((A \uplus C) \cup (B \uplus C)) \Longleftrightarrow\\
(A \cup B) \uplus C =& (A \uplus C) \cup (B \uplus C)\\
\end{split}
\end{equation}

\begin{equation}
\begin{split}
(A \cup B) \uplus C &= (A \uplus C) \cup (B \uplus C) \text{ using (\ref{upluscup})}\Longleftrightarrow\\
(A \cup B) \uplus C &\subseteq (A \uplus C) \cup (B \uplus C) \text{ using the antisymmetry of $\subseteq$}\\
\end{split}
\end{equation}

\item $(A \uplus C) \cup (B \uplus C) \subseteq (A \cup B) \uplus C$

\begin{equation}
\begin{split}
(A \uplus C) \cup (B \uplus C) &= (A \cup B) \uplus C \text{ using (\ref{upluscup})}\Longleftrightarrow\\
(A \uplus C) \cup (B \uplus C) &\subseteq (A \cup B) \uplus C \text{ using the antisymmetry of $\subseteq$}\\
\end{split}
\end{equation}

\item $(A \cap B) \uplus C \subseteq (A \uplus C) \cap (B \uplus C)$

I will prove the distributivity of $\uplus$ and $\cap$.
\begin{equation}\label{upluscap}
\begin{split}
x \in (A \cap B) \uplus C \Longleftrightarrow& (\exists a \in A \cap B. x = (1, a)) \vee (\exists c \in C. x = (2, c)) \Longleftrightarrow\\
x \in (A \cap B) \uplus C \Longleftrightarrow& ((\exists a \in A. x = (1, a)) \wedge (\exists b \in B. x = (1, b))) \vee (\exists c \in C. x = (2, c)) \Longleftrightarrow\\
x \in (A \cap B) \uplus C \Longleftrightarrow& ((\exists a \in A. x = (1, a)) \vee (\exists c \in C. x = (2, c))) \wedge \\
											 & ((\exists b \in B. x = (1, b)) \vee (\exists c \in C. x = (2, c))) \Longleftrightarrow\\
x \in (A \cap B) \uplus C \Longleftrightarrow& x \in (A \uplus C) \cap (B \uplus C) \Longleftrightarrow\\
(A \cap B) \uplus C &= (A \uplus C) \cap (B \uplus C)\\
\end{split}
\end{equation}

\begin{equation}
\begin{split}
(A \cap B) \uplus C &= (A \uplus C) \cap (B \uplus C) \text{ using (\ref{upluscap})} \Longleftrightarrow\\
(A \cap B) \uplus C &\subseteq (A \uplus C) \cap (B \uplus C) \text{ using the antisymmetry of $\subseteq$}\\
\end{split}
\end{equation}

\item $(A \uplus C) \cap (B \uplus C) \subseteq (A \cap B) \uplus C$

\begin{equation}
\begin{split}
(A \uplus C) \cap (B \uplus C) &= (A \cap B) \uplus C \text{ using (\ref{upluscap})} \Longleftrightarrow\\
(A \uplus C) \cap (B \uplus C) &\subseteq (A \cap B) \uplus C \text{ using the antisymmetry of $\subseteq$}\\
\end{split}
\end{equation}

\end{enumerate}

\item Prove the following properties of the big unions and intersections of a family of sets $\mathcal{F}\subseteq \mathcal{P}(A)$:

\begin{enumerate}

\item $\forall U \subseteq A. (\forall X \in \mathcal{F}. X \subseteq U) \Longleftrightarrow \bigcup \mathcal{F}\subseteq U$

\begin{equation}\label{q6part1}
\begin{split}
\forall U \subseteq A.& \bigcup \mathcal{F} \subseteq U \Longleftrightarrow\\
\forall U \subseteq A.& \nexists X \in \mathcal{F}. X \nsubseteq U \Longleftrightarrow\\
\forall U \subseteq A.& \forall X \in \mathcal{F}. X \subseteq U\\
\end{split}
\end{equation}

\item $\forall L \subseteq A. (\forall X \in \mathcal{F}. L \subseteq X) \Longleftrightarrow L \subseteq \bigcap \mathcal{F}$

\begin{equation}\label{q6part2}
\begin{split}
\forall U \subseteq A.& L \subseteq \mathcal{F} \Longleftrightarrow\\
\forall U \subseteq A.& \nexists X \in \mathcal{F}. L \nsubseteq X \Longleftrightarrow\\
\forall U \subseteq A.& \forall X \in \mathcal{F}. L \subseteq X\\
\end{split}
\end{equation}

\end{enumerate}

\item Let $A$ be a set.

\begin{enumerate}

\item For a family $\mathcal{F} \subseteq \mathcal{P}(A)$, let $\mathcal{U} \triangleq \{ U \subseteq A| \forall S \in \mathcal{F}. S \subseteq U\}$. 
Prove that $\bigcup\mathcal{F} = \bigcap\mathcal{U}$.

\begin{equation}
\begin{split}
\mathcal{U} &\triangleq \{ U \subseteq A| \forall S \in \mathcal{F}. S \subseteq U\} \Longleftrightarrow\\
\mathcal{U} &= \{ U \subseteq A| \bigcup \mathcal{F} \subseteq U\} \text{ using (\ref{q6part1})}\Longleftrightarrow\\
\bigcup \mathcal{F} &\subseteq \bigcup \mathcal{F} \Longrightarrow \bigcup \mathcal{F} \in \mathcal{U} \Longleftrightarrow\\
\forall U \in \mathcal{U}. &\bigcup \mathcal{F} \subseteq U \wedge \mathcal{F} \in U \Longleftrightarrow\\
\bigcap \mathcal{U} &= \bigcup \mathcal{F}\\
\end{split}
\end{equation}

\item Analogously, define the family $\mathcal{L}\subseteq \mathcal{P}(A)$ such that $\bigcap \mathcal{F} = \bigcup \mathcal{L}$. Also prove this statement.

\begin{equation*}
\mathcal{L} \triangleq \{ L \subseteq A | \forall S \in \mathcal{F} L \subseteq S \}
\end{equation*}

\begin{equation}
\begin{split}
\mathcal{L} &\triangleq \{ L \subseteq A | \forall S \in \mathcal{F} L \subseteq S \} \Longleftrightarrow\\
\mathcal{L} &= \{ L \subseteq A | L \subseteq \bigcup \mathcal{F} \} \text{ using (\ref{q6part2})} \Longleftrightarrow\\
\bigcup \mathcal{F} \in \mathcal{L} &\wedge \forall L \in \mathcal{L}: L \subseteq \mathcal{F} \Longleftrightarrow\\
\bigcup \mathcal{L} &= \bigcap \mathcal{F}\\
\end{split}
\end{equation}

\end{enumerate}

\end{enumerate}

\subsection*{5.3 Optional advanced exercises}

\begin{enumerate}

\item Prove that for all families of sets $\mathcal{F}_1$ and $\mathcal{F}_2$

\begin{equation}
(\bigcup\mathcal{F}_1)\cup(\bigcup\mathcal{F}_2) = \bigcup(\mathcal{F}_1\cup\mathcal{F}_2)\\
\end{equation}

\begin{equation}
\begin{split}
(\bigcup\mathcal{F}_1) \cup (\bigcup\mathcal{F}_2) &= \{x| \exists S_1 \in \mathcal{F}_1. x \in S_1\} \cup \{x| \exists S_2 \in \mathcal{F}_2. x \in S_2\} \Longleftrightarrow\\
(\bigcup\mathcal{F}_1) \cup (\bigcup\mathcal{F}_2) &= \{x| \exists S_1 \in \mathcal{F}_1 \vee \exists S_2 \in \mathcal{F}_2. x \in S_1 \vee x \in S_2\}\Longleftrightarrow\\
(\bigcup\mathcal{F}_1) \cup (\bigcup\mathcal{F}_2) &= \{x| \exists S \in \mathcal{F}_1 \cup \mathcal{F}_2\. x \in S\} \Longleftrightarrow\\
(\bigcup\mathcal{F}_1) \cup (\bigcup\mathcal{F}_2) &= \bigcup(\mathcal{F}_1 \cup \mathcal{F}_2)\\
\end{split}
\end{equation}

State and prove the analogous property for intersections of non-empty families of sets.

\begin{equation*}
(\bigcap\mathcal{F}_1) \cap (\bigcap\mathcal{F}_2) = \bigcap(\mathcal{F}_1 \cap \mathcal{F}_2)
\end{equation*}

\begin{equation}
\begin{split}
(\bigcap\mathcal{F}_1) \cap (\bigcap\mathcal{F}_2) &= \{x| \forall S_1 \in \mathcal{F}_1. x \in S_1\} \cap \{x | \forall S_2 \in \mathcal{F}_2. x \in S_2\}\Longleftrightarrow\\
(\bigcap\mathcal{F}_1) \cap (\bigcap\mathcal{F}_2) &= \{x| \forall S_1 \in \mathcal{F}_1, \forall S_2 \in \mathcal{F}_2. x \in S_1 \wedge x \in S_2\}\Longleftrightarrow\\
(\bigcap\mathcal{F}_1) \cap (\bigcap\mathcal{F}_2) &= \{x| \forall S \in (\mathcal{F}_1 \cap \mathcal{F}_2). x \in S\}\Longleftrightarrow\\
(\bigcap\mathcal{F}_1) \cap (\bigcap\mathcal{F}_2) &= \bigcap(\mathcal{F}_1 \cap \mathcal{F}_2)\\
\end{split}
\end{equation}

\item For a set $U$, prove that $(\mathcal{P}(U), \subseteq, \cup, \cap, U, \emptyset, (\cdot)^\mathsf{c})$ is a Boolean algebra.



\end{enumerate}

\end{document}