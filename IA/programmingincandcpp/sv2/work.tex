\newcommand{\svcourse}{CST Part IA: Introduction to Probability}
\newcommand{\svnumber}{1}
\newcommand{\svvenue}{Churchill, Room TBD}
\newcommand{\svdate}{2022-05-14}
\newcommand{\svtime}{11:00}
\newcommand{\svuploadkey}{PO5ogKIM8KQA22FZS8IAf8gxA8XKi19jxIBVHIfFZ+3GCBXuNUXS9lVN6bNYjxM/}

\newcommand{\svrname}{Mr Matthew Ireland}
\newcommand{\jkfside}{twoside}
\newcommand{\jkfhanded}{right}

\newcommand{\studentname}{Harry Langford}
\newcommand{\studentemail}{hjel2@cam.ac.uk}


\documentclass[10pt,\jkfside,a4paper]{article}

% DO NOT add \usepackage commands here.  Place any custom commands
% into your SV work files.  Anything in the template directory is
% likely to be overwritten!

\usepackage{fancyhdr}

\usepackage{lastpage}       % ``n of m'' page numbering
\usepackage{lscape}         % Makes landscape easier

\usepackage{verbatim}       % Verbatim blocks
\usepackage{epsfig}         % Embed encapsulated postscript
\usepackage{array}          % Array environment
\usepackage[nolinks]{qrcode}         % QR codes
\usepackage{enumitem}       % Required by Tom Johnson's exam question header

\usepackage{hhline}         % Horizontal lines in tables
\usepackage{siunitx}        % Correct spacing of units
\usepackage{amsmath}        % American Mathematical Society
\usepackage{amssymb}        % Maths symbols
\usepackage{amsthm}         % Theorems

\usepackage{ifthen}         % Conditional processing in tex

\usepackage[top=3cm,
            bottom=3cm,
            inner=2cm,
            outer=5cm]{geometry}

% PDF metadata + URL formatting
\usepackage[
            pdfauthor={\studentname},
            pdftitle={\svcourse, SV \svnumber},
            pdfsubject={},
            pdfkeywords={9d2547b00aba40b58fa0378774f72ee6},
            pdfproducer={},
            pdfcreator={},
            hidelinks]{hyperref}

\renewcommand{\headrulewidth}{0.4pt}
\renewcommand{\footrulewidth}{0.4pt}
\fancyheadoffset[LO,LE,RO,RE]{0pt}
\fancyfootoffset[LO,LE,RO,RE]{0pt}
\pagestyle{fancy}
\fancyhead{}
\fancyhead[LO,RE]{{\bfseries \studentname}\\\studentemail}
\fancyhead[RO,LE]{{\bfseries \svcourse, SV~\svnumber}\\\svdate\ \svtime, \svvenue}
\fancyfoot{}
\fancyfoot[LO,RE]{For: \svrname}
\fancyfoot[RO,LE]{\today\hspace{1cm}\thepage\ / \pageref{LastPage}}
\fancyfoot[C]{\qrcode[height=0.8cm]{\svuploadkey}}
\setlength{\headheight}{22.55pt}

\ifthenelse{\equal{\jkfside}{oneside}}{

 \ifthenelse{\equal{\jkfhanded}{left}}{
  % 1. Left-handed marker, one-sided printing or e-marking, use oneside and...
  \evensidemargin=\oddsidemargin
  \oddsidemargin=73pt
  \setlength{\marginparwidth}{111pt}
  \setlength{\marginparsep}{-\marginparsep}
  \addtolength{\marginparsep}{-\textwidth}
  \addtolength{\marginparsep}{-\marginparwidth}
 }{
  % 2. Right-handed marker, one-sided printing or e-marking, use oneside.
  \setlength{\marginparwidth}{111pt}
 }

}{
 % 3. Alternating margins, two-sided printing, use twoside.
}

\setlength{\parindent}{0em}
\addtolength{\parskip}{1ex}

% Exam question headings, labels and sensible layout (courtesy of Tom Johnson)
\setlist{parsep=\parskip, listparindent=\parindent}
\newcommand{\examhead}[3]{\section{#1 Paper #2 Question #3}}
\newenvironment{examquestion}[3]{
    \examhead{#1}{#2}{#3}\setlist[enumerate, 1]{label=(\alph*)}\setlist[enumerate, 2]{label=(\roman*)}
    \marginpar{\qrcode{https://www.cl.cam.ac.uk/teaching/exams/pastpapers/y#1p#2q#3.pdf}}
    \marginpar{\footnotesize \url{https://www.cl.cam.ac.uk/teaching/exams/pastpapers/y#1p#2q#3.pdf}}
}{}



\begin{document}

\begin{examquestion}{2008}{3}{3}

A hardware engineer stores a FIFO queue of bits in an int on a platform with
32-bit ints and 8-bit chars using the following C++ class:

\begin{lstlisting}[language=C++]

class BitQueue{
	int valid_bits;	// the number of valid bits held in the queue
	int queue;	// least significant bit is most recent bit added
public:
	BitQueue(): valid_bits(0), queue(0) {}
	void push(int val, int bsize);
	int pop(int bsize);
	int size();
}

\end{lstlisting}

\begin{enumerate}[label=(\alph*)]

\item Write an implementation of BitQueue::size, which should return the
number of bits currently held in queue.

\begin{lstlisting}[language=C++]

int size(){
	return valid_bits;
}

\end{lstlisting}

\item Write an implementation of BitQueue::push, which places the bsize
least significant bits from val onto queue and updates valid.bits. An
exception should be thrown in cases where data would otherwise be lost.

% TODO add an exception!

\begin{lstlisting}[language=C++]

void push(int val, int bsize){
	int j = 1;
	for (int i = 0; i < bsize; i++){
		queue |= (j & val) << valid_bits;
		j *= 2;
	}
	valid_bits += bsize;
}

\end{lstlisting}

\item Write an implementation of BitQueue::pop, which takes bsize bits from
queue, provides them as the bsize least significant bits in the return
value, and updates valid.bits. An exception should be thrown when any
requested data is unavailable.

% TODO add an exception!

\begin{lstlisting}[language=C++]

int pop(int bsize){
	int ret = 0;
	int j = 1;
	for (int i = 0; i < bsize; i++){
		ret |= (j & queue);
		j *= 2;
	}
	queue >>= bsize;
	valid_bits -= bsize;
}

\end{lstlisting}

\item The hardware engineer has built a communication device together with a
C++ library function send to transmit data with the following declaration.

\begin{lstlisting}[language=C++]

void send(char);

\end{lstlisting}

Use the BitQueue class to write a C++ definition for:

\begin{lstlisting}[language=C++]

void sendmst(const char* msg);

\end{lstlisting}

Each of the characters in msg should be encoded, in index order, using the
following binary codes: `a'=0, `b'=10, `c'=1100 and `d'=1101. All other
characters should be ignored. Successive binary codes should be bit-packed
together and the code 111 should be used to denote the end of the message.
Chunks of 8-bits should be sent using the send function and any remaining
bits at the end of a message should be padded with zeros. For example,
executing sendmst(``abcd'') should call the send function twice, with the
binary values 01011001 followed by 10111100.

% TODO

\end{enumerate}

\end{examquestion}

\begin{examquestion}{2009}{3}{1}

Explain all of the following C or C++ features. You may use a short
fragment of code to complement your explanation.

\begin{enumerate}[label=(\alph*)]

\item The declaration of a C++ class illustrating constructor, variable and
method.

% TODO

\item The use of a virtual destructor

% TODO

\item The difference between malloc() and free(); and new and delete

% TODO

\item Overloading an operator

% TODO

\item Pointer arithmetic

% TODO

\item Catching and throwing exceptions including the passing of a
user-defined structure

% TODO

\item The meaning of the keywords static and const

% TODO

\end{enumerate}

\end{examquestion}

\begin{examquestion}{2012}{3}{3}

In this question, where appropriate, you may use a short fragment of code to
complement your explanation.

\begin{enumerate}[label=(alph*)]

\item

\begin{enumerate}[label=(\roman*)]

\item What is the difference between a local and global variable in C?
(Consider variable scope, storage and initialisation)

% TODO

\item What are the properties of a static member variable in a C++ class?

% TODO

\end{enumerate}

\item

\begin{enumerate}[label=(\roman*)]

\item Briefly explain pointer arithmetic in C. Give an example code snippet
involving pointers in which it would be \textit{inappropriate} to use
pointer arithmetic, and explain why.

% TODO

\item Explain how in some respects pointers are equivalent to arrays, and
give one respect in which they differ.

% TODO

\end{enumerate}

\item Explain why a function might be declared virtual in a C++ superclass.

% TODO

\item

\begin{enumerate}[label=(\roman*)]

\item How does the use of void * pointer in C allow a form of polymorphism?
Give an example function declaration using the void * pointer.

% TODO

\item What is the main problem with the use of void *, and how does C++
improve on this? Give the improved function declaration in C++ for your
example function in part (d)(i).

% TODO

\end{enumerate}

\item

\begin{enumerate}[label=(\roman*)]

\item Why might it be useful to define a copy constructor for a C++ class?
Give an example of a copy constructor for a simple class.

% TODO

\item Why might it be useful to explicitly define the assignment operator
(=) for a C++ class? Give an example definition of the assignment operator
for a simple class.

% TODO

\end{enumerate}

\end{enumerate}

\end{examquestion}

\end{document}
