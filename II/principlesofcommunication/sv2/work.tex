\newcommand{\svcourse}{CST Part IA: Software Engineering and Security}
\newcommand{\svnumber}{1}
\newcommand{\svvenue}{Microsoft Teams}
\newcommand{\svdate}{2022-05-11}
\newcommand{\svtime}{15:00}
\newcommand{\svuploadkey}{CBd13xmL7PC1zqhNIoLdTiYUBnxZhzRAtJxv/ytRdM1r7qIfwMsxeVwM/pPcIo8l}

\newcommand{\svrname}{Dr Sam Ainsworth}
\newcommand{\jkfside}{oneside}
\newcommand{\jkfhanded}{yes}

\newcommand{\studentname}{Harry Langford}
\newcommand{\studentemail}{hjel2@cam.ac.uk}


\documentclass[10pt,\jkfside,a4paper]{article}

% DO NOT add \usepackage commands here.  Place any custom commands
% into your SV work files.  Anything in the template directory is
% likely to be overwritten!

\usepackage{fancyhdr}

\usepackage{lastpage}       % ``n of m'' page numbering
\usepackage{lscape}         % Makes landscape easier

\usepackage{verbatim}       % Verbatim blocks
\usepackage{listings}       % Source code listings
\usepackage{graphicx}
\usepackage{float}
\usepackage{epsfig}         % Embed encapsulated postscript
\usepackage{array}          % Array environment
\usepackage{qrcode}         % QR codes
\usepackage{enumitem}       % Required by Tom Johnson's exam question header

\usepackage{hhline}         % Horizontal lines in tables
\usepackage{siunitx}        % Correct spacing of units
\usepackage{amsmath}        % American Mathematical Society
\usepackage{amssymb}        % Maths symbols
\usepackage{amsthm}         % Theorems

\usepackage{ifthen}         % Conditional processing in tex

\usepackage[top=3cm,
            bottom=3cm,
            inner=2cm,
            outer=5cm]{geometry}

% PDF metadata + URL formatting
\usepackage[
            pdfauthor={\studentname},
            pdftitle={\svcourse, SV \svnumber},
            pdfsubject={},
            pdfkeywords={9d2547b00aba40b58fa0378774f72ee6},
            pdfproducer={},
            pdfcreator={},
            hidelinks]{hyperref}

\renewcommand{\headrulewidth}{0.4pt}
\renewcommand{\footrulewidth}{0.4pt}
\fancyheadoffset[LO,LE,RO,RE]{0pt}
\fancyfootoffset[LO,LE,RO,RE]{0pt}
\pagestyle{fancy}
\fancyhead{}
\fancyhead[LO,RE]{{\bfseries \studentname}\\\studentemail}
\fancyhead[RO,LE]{{\bfseries \svcourse, SV~\svnumber}\\\svdate\ \svtime, \svvenue}
\fancyfoot{}
\fancyfoot[LO,RE]{For: \svrname}
\fancyfoot[RO,LE]{\today\hspace{1cm}\thepage\ / \pageref{LastPage}}
\fancyfoot[C]{\qrcode[height=0.8cm]{\svuploadkey}}
\setlength{\headheight}{22.55pt}


\ifthenelse{\equal{\jkfside}{oneside}}{

 \ifthenelse{\equal{\jkfhanded}{left}}{
  % 1. Left-handed marker, one-sided printing or e-marking, use oneside and...
  \evensidemargin=\oddsidemargin
  \oddsidemargin=73pt
  \setlength{\marginparwidth}{111pt}
  \setlength{\marginparsep}{-\marginparsep}
  \addtolength{\marginparsep}{-\textwidth}
  \addtolength{\marginparsep}{-\marginparwidth}
 }{
  % 2. Right-handed marker, one-sided printing or e-marking, use oneside.
  \setlength{\marginparwidth}{111pt}
 }

}{
 % 3. Alternating margins, two-sided printing, use twoside.
}


\setlength{\parindent}{0em}
\addtolength{\parskip}{1ex}

% Exam question headings, labels and sensible layout (courtesy of Tom Johnson)
\setlist{parsep=\parskip, listparindent=\parindent}
\newcommand{\examhead}[3]{\section{#1 Paper #2 Question #3}}
\newenvironment{examquestion}[3]{
\examhead{#1}{#2}{#3}\setlist[enumerate, 1]{label=(\alph*)}\setlist[enumerate, 2]{label=(\roman*)}
\marginpar{\href{https://www.cl.cam.ac.uk/teaching/exams/pastpapers/y#1p#2q#3.pdf}{\qrcode{https://www.cl.cam.ac.uk/teaching/exams/pastpapers/y#1p#2q#3.pdf}}}
\marginpar{\footnotesize \href{https://www.cl.cam.ac.uk/teaching/exams/pastpapers/y#1p#2q#3.pdf}{https://www.cl.cam.ac.uk/\\teaching/exams/pastpapers/\\y#1p#2q#3.pdf}}
}{}


\begin{document}

\begin{enumerate}

    \item An ISP decides to limit the rate of P2P traffic to a maximum of $r\%$ in any given link. If this is exceeded, the ISP sends TCP reset packets to the P2P connection endpoints.

    \begin{enumerate}

        \item Explain how this reduces the rate of P2P traffic?

        TCP reset will close the TCP connection. If the peers re-establish the TCP connection, their TCP window sizes will be reset and thus the rate of communication will be decreased.

        \item Discuss how you might decide, in response to changing network traffic conditions at runtime, how to decide what proportion of TCL connections carrying P2P traffic should be ``reset''.
        (Hint: using a P.I.D\onedot response, based on control theory, is one option. Any others?)

        Let $k$ be the percentage of P2P channels which are reset each second.

        \begin{itemize}

            \item Make $k$ a function of the rate of change of queuing delay for the average queuing delay of the more important packets. i.e if the average loss increases by $1$ microsecond then we could reset
            $1\%$ of the P2P connections.

            \item Make $k$ a function of the queuing delay for more important packets. For example, a simple function could have $k = d$
            where $d$ is the average packet delay of an important packet (given in microseconds) and clamped at $100\%$.

            \item Run a sawtooth-like windowing protocol where we allow a percentage $r\%$ of the traffic to be P2P. Each second, if there is no packet loss due to congestion then increase the percentage by some
            small constant $\epsilon$ each second. However, if there is packet loss due to congestion, then halve the window size.

        \end{itemize}

        \item What is hop-by-hop flow control?

        How-by-hop flow control is where every node on a network runs flow control on every link. The result is a flow which is highly responsive to changes; and the flow which the endhosts actually see is the
        flow on the worst link (\emph{i.e}.\ the link with the least capacity). This is comparatively easy to implement: but it adds a lot of state and work for each intermediate node. Furthermore, it requires
        every node and link to implement it for to work properly. You cannot rely on this for internet-based systems and thus hop-by-hop flow control is not used in IP networks.

        \item Why are different flow control techniques required at Layer 2 and Layer 3?

        Flow control in Layer 2 is implemented via a windowing protocol such that a fast sender does not overwhelm a slow receiver. The receiver advertises a flow control window, which represents the maximum
        number of packets that they can receive at once. The sender then has at most that many packets in transit at once.

        I'm not aware of flow control being employed in the Internet Layer.

        \item Why do we need multiple user/network utility metrics?

        Differentiated Services. There are many different types of traffic which are sent over the network. They may have different requirements. For example, a user may be playing a game and require that low
        latency; while another user may be doing bulk data transfer and may care only about the bandwidth. Unifying these into a single utility metric will leave many users with suboptimal performance.
        by using many different bandwidth requirements, we aggregate many

    \end{enumerate}

\end{enumerate}

\end{document}
