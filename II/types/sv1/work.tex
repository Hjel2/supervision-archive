\newcommand{\svcourse}{CST Part IA: Introduction to Probability}
\newcommand{\svnumber}{1}
\newcommand{\svvenue}{Churchill, Room TBD}
\newcommand{\svdate}{2022-05-14}
\newcommand{\svtime}{11:00}
\newcommand{\svuploadkey}{PO5ogKIM8KQA22FZS8IAf8gxA8XKi19jxIBVHIfFZ+3GCBXuNUXS9lVN6bNYjxM/}

\newcommand{\svrname}{Mr Matthew Ireland}
\newcommand{\jkfside}{twoside}
\newcommand{\jkfhanded}{right}

\newcommand{\studentname}{Harry Langford}
\newcommand{\studentemail}{hjel2@cam.ac.uk}


\documentclass[10pt,\jkfside,a4paper]{article}

% DO NOT add \usepackage commands here.  Place any custom commands
% into your SV work files.  Anything in the template directory is
% likely to be overwritten!

\usepackage{fancyhdr}

\usepackage{lastpage}       % ``n of m'' page numbering
\usepackage{lscape}         % Makes landscape easier

\usepackage{verbatim}       % Verbatim blocks
\usepackage{epsfig}         % Embed encapsulated postscript
\usepackage{array}          % Array environment
\usepackage[nolinks]{qrcode}         % QR codes
\usepackage{enumitem}       % Required by Tom Johnson's exam question header

\usepackage{hhline}         % Horizontal lines in tables
\usepackage{siunitx}        % Correct spacing of units
\usepackage{amsmath}        % American Mathematical Society
\usepackage{amssymb}        % Maths symbols
\usepackage{amsthm}         % Theorems

\usepackage{ifthen}         % Conditional processing in tex

\usepackage[top=3cm,
            bottom=3cm,
            inner=2cm,
            outer=5cm]{geometry}

% PDF metadata + URL formatting
\usepackage[
            pdfauthor={\studentname},
            pdftitle={\svcourse, SV \svnumber},
            pdfsubject={},
            pdfkeywords={9d2547b00aba40b58fa0378774f72ee6},
            pdfproducer={},
            pdfcreator={},
            hidelinks]{hyperref}

\renewcommand{\headrulewidth}{0.4pt}
\renewcommand{\footrulewidth}{0.4pt}
\fancyheadoffset[LO,LE,RO,RE]{0pt}
\fancyfootoffset[LO,LE,RO,RE]{0pt}
\pagestyle{fancy}
\fancyhead{}
\fancyhead[LO,RE]{{\bfseries \studentname}\\\studentemail}
\fancyhead[RO,LE]{{\bfseries \svcourse, SV~\svnumber}\\\svdate\ \svtime, \svvenue}
\fancyfoot{}
\fancyfoot[LO,RE]{For: \svrname}
\fancyfoot[RO,LE]{\today\hspace{1cm}\thepage\ / \pageref{LastPage}}
\fancyfoot[C]{\qrcode[height=0.8cm]{\svuploadkey}}
\setlength{\headheight}{22.55pt}

\ifthenelse{\equal{\jkfside}{oneside}}{

 \ifthenelse{\equal{\jkfhanded}{left}}{
  % 1. Left-handed marker, one-sided printing or e-marking, use oneside and...
  \evensidemargin=\oddsidemargin
  \oddsidemargin=73pt
  \setlength{\marginparwidth}{111pt}
  \setlength{\marginparsep}{-\marginparsep}
  \addtolength{\marginparsep}{-\textwidth}
  \addtolength{\marginparsep}{-\marginparwidth}
 }{
  % 2. Right-handed marker, one-sided printing or e-marking, use oneside.
  \setlength{\marginparwidth}{111pt}
 }

}{
 % 3. Alternating margins, two-sided printing, use twoside.
}

\setlength{\parindent}{0em}
\addtolength{\parskip}{1ex}

% Exam question headings, labels and sensible layout (courtesy of Tom Johnson)
\setlist{parsep=\parskip, listparindent=\parindent}
\newcommand{\examhead}[3]{\section{#1 Paper #2 Question #3}}
\newenvironment{examquestion}[3]{
    \examhead{#1}{#2}{#3}\setlist[enumerate, 1]{label=(\alph*)}\setlist[enumerate, 2]{label=(\roman*)}
    \marginpar{\qrcode{https://www.cl.cam.ac.uk/teaching/exams/pastpapers/y#1p#2q#3.pdf}}
    \marginpar{\footnotesize \url{https://www.cl.cam.ac.uk/teaching/exams/pastpapers/y#1p#2q#3.pdf}}
}{}



\title{Types Supervision 1}

\author{
    Harry Langford \\
    hjel2@cam.ac.uk
}

\usepackage{rotating}
\usepackage{ebproof}

\begin{document}

\section{Warmup Questions}

\begin{enumerate}

    \item for a strongly-typed programming language, what are the advantages of a type system?

    \begin{itemize}

        \item abstraction

        \item optimization

        \item safety

    \end{itemize}

    \item Describe the concepts of weakening, exchange and substitution.

    \begin{itemize}

        \item Weakening

        Generally: if a term $T$ has property $P$ under context $\Theta, \Theta'$ then $T$ will still have property $P$ in the context $\Theta, \alpha, \Theta'$.

        Concretely, weakening for Simply Typed Lambda Calculus (STLC) is defined as:
        \[
            \Theta, \Theta' \vdash e: A \implies \Theta, x: X, \Theta' \vdash e: A
        \]

        \item Exchange

        Generally: if a term $T$ has property $P$ under context $\Theta, \alpha, \beta, \Theta'$ then $T$ will still have property $P$ in the context $\Theta, \beta, \alpha, \Theta'$.
        Intuitively, exchange means that the order in which objects occur in the context is not important.

        Concretely, exchange for the STLC is defined as:
        \[
            \Theta, x: X, y: X, \Theta' \vdash e: A \implies \Theta, y: Y, x: X, \Theta' \vdash e: A
        \]

        \item Substitution

        Concretely, substitution for the STLC is defined as:
        \[
            \Theta, x: X \vdash e: A \land \Theta \vdash e': X \implies \Theta \vdash [e'/x]e : A
        \]

    \end{itemize}

    \item Define the two properties of type safety.

    The two properties of Type Safety are Progress and Type Preservation.

    \begin{definition}[Progress]

        If an expression $e$ is well-typed then it is either a value, or there exists some expression $e'$ such that $e \rightsquigarrow e'$.

    \end{definition}

    \begin{definition}[Type Preservation]

        If an expression $e$ has the type $T$ and $e \rightsquigarrow e'$ then $e'$ also has they type $T$.

    \end{definition}

    \item What is the Curry-Howard Correspondence?

    The Curry Howard Correspondence is a correspondence between types systems an proof systems. The exact correspondences are in~Table~\ref{tab:curryhoward}.

    \begin{table}[H]

        \centering

        \begin{tabular}{c|c}

            Type Systems & Proof Systems \\

            \hline

            unit type & truth \\

            null type & falsehood \\

            functions & implication \\

            sum type & logical or \\

            product type & logical and \\

            reduction & normalisation \\

            value & cut-free proof \\

            reduction order & normalisation order \\

        \end{tabular}

        \caption{Correspondences in the Curry Howard Correspendence}

        \label{tab:curryhoward}

    \end{table}

    \item Give the typing rules and operational semantics of TLC for functions, pairs and sums.

    \textbf{Typing Rules:}

    \begin{gather*}
        \dfrac{
            \Gamma, x: A \vdash e: B
        }{
            \Gamma \vdash \lambda x: A \ldotp e: A \to B
        }
        \qquad
        \dfrac{
            \Gamma \vdash e: A \to B \qquad \Gamma \vdash e': A
        }{
            \Gamma \vdash e\, e': B
        }
        \\[2em]
        \dfrac{
            \Gamma \vdash e_1: A \qquad \Gamma \vdash e_2: B
        }{
            \Gamma \vdash (e_1, e_2): A \times B
        }
        \qquad
        \dfrac{
            \Gamma \vdash e: A \times B
        }{
            \Gamma \vdash \mathrm{fst}\, e: A
        }
        \qquad
        \dfrac{
            \Gamma \vdash e: A \times B
        }{
            \Gamma \vdash \mathrm{snd}\, e: B
        }
        \\[2em]
        \dfrac{
            \Gamma \vdash e_1: A
        }{
            \Gamma \vdash L\, e_1: A + B
        }
        \qquad
        \dfrac{
            \Gamma \vdash e_2: B
        }{
            \Gamma \vdash R\, e_1 + e_2: A + B
        }
        \qquad
        \dfrac{
            \Gamma \vdash e: A + B
            \qquad
            \Gamma, x: A \vdash e_1: C
            \qquad
            \Gamma, y: B \vdash e_2: C
        }{
            \Gamma \vdash \mathrm{case}(e, L\, x \to e_1, R\, y \to e_2) : C
        }
    \end{gather*}

    \textbf{Operational Semantics:}

    \begin{gather*}
        \dfrac{
            e \rightsquigarrow e'
        }{
            e\, e'' \rightsquigarrow e'\, e''
        }
        \qquad
        \dfrac{
            e' \rightsquigarrow e''
        }{
            v\, e' \rightsquigarrow v\, e''
        }
        \qquad
        \dfrac{
        }{
            (\lambda x \ldotp e)\, v \rightsquigarrow [v/x]e
        }
        \\[2em]
        \dfrac{
            e_1 \rightsquigarrow e_1'
        }{
            (e_1, e_2) \rightsquigarrow (e_1', e_2)
        }
        \qquad
        \dfrac{
            e_2 \rightsquigarrow e_2'
        }{
            (v_1, e_2) \rightsquigarrow (v_1, e_2')
        }
        \qquad
        \dfrac{
            e \rightsquigarrow e'
        }{
            \mathrm{fst}\, e \rightsquigarrow e'
        }
        \qquad
        \dfrac{
            e \rightsquigarrow e'
        }{
            \mathrm{snd}\, e \rightsquigarrow e'
        }
        \\[2em]
        \dfrac{
            e \rightsquigarrow e'
        }{
            L\, e \rightsquigarrow L\, e'
        }
        \qquad
        \dfrac{
            e \rightsquigarrow e'
        }{
            R\, e \rightsquigarrow L\, e'
        }
        \qquad
        \dfrac{
            e \rightsquigarrow e'
        }{
            \mathrm{case}(e', L\, x \to e_1, R\, y \to e_2)
        }
    \end{gather*}

\end{enumerate}

\section{Regular Questions}

\begin{enumerate}

    \item Extend the type preservation and progress proof to cover AND\@.

    \[
        \Phi(e, \tau) \triangleq e: \tau \implies e \text{ is a value} \vee \exists e'. e \rightsquigarrow e'
    \]

    Prove Progress by structural induction on the derivation of $e: \tau$. Let $\Phi(e, \tau)$ be the induction hypothesis:

    \begin{align*}
        \intertext{Case (AND):}
        & \dfrac{e_1: \mathrm{bool} \qquad e_2: \mathrm{bool}}{e_1 \land e_2: \mathrm{bool}} & & \text{by assumption} \\
        \intertext{Case $e_1$ is not a value}
        & e_1 \rightsquigarrow e_1' & & \text{by induction hypothesis} \\
        & e_1 \land e_2 \rightsquigarrow e_1' \land e_2 & & \text{by the reduction rules} \\
        \intertext{Case $e_1$ is a value}
        \intertext{Case $e_2$ is not a value}
        & e_2 \rightsquigarrow e_2' & & \text{by induction hypothesis} \\
        & e_1 \land e_2 \rightsquigarrow e_1 \land e_2' & & \text{by the reduction rules} \\
        \intertext{Case $e_2$ is a value}
        & e_1 \land e_2 & & \text{is a value}
    \end{align*}

    \[
        \Phi(e, e', \tau) \triangleq e: \tau \land e \rightsquigarrow e' \implies e': \tau
    \]

    Prove Type Preservation by structural induction on the derivation $e: \tau$. Let $\Phi(e, e', \tau)$ be the induction hypothesis:

    \begin{align*}
        \intertext{Case (AND):}
        & e \rightsquigarrow e' & & \text{by assumption} \\
        & \dfrac{e_1: \mathrm{bool} \qquad e_2: \mathrm{bool}}{e_1 \land e_2: \mathrm{bool}} & & \text{by assumption} \\
        & e = e_1 \land e_2 & & \text{by assumption} \\
        \intertext{Case $e_1$ is not a value}
        & e_1 \rightsquigarrow e_1' & & \text{by Progress} \\
        & e_1': \mathrm{bool} & & \text{by induction hypothesis} \\
        & e_1 \land e_2 \rightsquigarrow e_1' \land e_2 & & \text{by the reduction rules} \\
        & e_1' \land e_2 = e' & & \text{by determinacy} \\
        & e': \mathrm{bool} & & \text{by (AND)}
        \intertext{Case $e_1$ is a value}
        \intertext{Case $e_2$ is not a value}
        & e_2 \rightsquigarrow e_2' & & \text{by Progress} \\
        & e_2': \mathrm{bool} & & \text{by induction hypothesis} \\
        & e_1 \land e_2 \rightsquigarrow e_1 \land e_2' & & \text{by the reduction rules} \\
        & e_1 \land e_2 = e' & & \text{by determinacy} \\
        & e': \mathrm{bool} & & \text{by (AND)} \\
        \intertext{Case $e_1$ is a value}
        \intertext{$e_1 \land e_2$ is a value. Thus $e$ is a value and $\nexists e'. e \rightsquigarrow e'$: the premise was false in this case so $\Phi$ holds trivially.}
    \end{align*}

    \item Prove type safety for the unit and function cases of TLC\@.

    \begin{itemize}

        \item Progress

        \[
            \Phi(\Gamma, e, \tau) \triangleq \Gamma \vdash e: \tau \implies \text{ $e$ is a value } \vee \exists e'\ldotp e \rightsquigarrow e'
        \]

        I will prove this by structural induction on the derivation $\Gamma \vdash e: \tau$ with the induction hypothesis that for all subexpressions $e$, $\Phi(\Gamma, e, \tau)$. I will prove only the unit and
        function cases.

        \begin{itemize}

            \item \textbf{Case unit}
            \begin{align*}
                \intertext{By assumption, we have:}
                \dfrac{}{\Gamma \vdash \langle \rangle: 1}
                \intertext{We have that $\langle \rangle$ is a value. Thus we have $\Phi$ in this case.}
            \end{align*}
            \item \textbf{Case abstr}
            \begin{align*}
                \intertext{By assumption, we have:}
                \dfrac{\Gamma, x: A \vdash e: B}{\Gamma \vdash \lambda x: A\ldotp e: A \to B}
                \intertext{$\lambda x\ldotp e$ is a value. Thus we have $\Phi$ in this case.}
            \end{align*}
            \item \textbf{Case app}
            \begin{align*}
                \intertext{By assumption, we have:}
                & \dfrac{\Gamma\vdash e_1: A \to B \qquad \Gamma \vdash e_2: A}{\Gamma \vdash e_1\, e_2: B} \\
                & \Gamma\vdash e_1: A \to B && \text{subderivation} \\
                & \Gamma \vdash e_2: A && \text{subderivation} \\
                \intertext{By the induction hypothesis, we have that $e_1$ is a value or $\exists e_1'\ldotp e_1 \rightsquigarrow e_1'$}
                \intertext{Case $\exists e_1'\ldotp e_1 \rightsquigarrow e_1'$}
                & \dfrac{e_1 \rightsquigarrow e_1'}{e_1\, e_2 \rightsquigarrow e_1'\, e_2}
                \intertext{Thus we have that $e \rightsquigarrow e_1'\, e_2$}
                \intertext{Case $e_1$ is a value}
                \intertext{By the induction hypothesis, we have that $e_2$ is a value or $\exists e_2'\ldotp e_2 \rightsquigarrow e_2'$}
                \intertext{Case $\exists e_2'\ldotp e_2 \rightsquigarrow e_2'$}
                & \dfrac{e_2 \rightsquigarrow e_2'}{e_1\, e_2 \rightsquigarrow e_1\, e_2'}
                \intertext{Thus we have that $e \rightsquigarrow e_1\, e_2'$}
                \intertext{Case $e_2$ is a value $v_2$}
                \intertext{For $e_1$ to be a value and have type $A \to B$, we have that it must be of the form $\lambda x: A\ldotp e_3$ for some $e_3$}
                & \dfrac{}{(\lambda x: A \ldotp e_3)\, v_2 \rightsquigarrow [v_2/x]e_3}
                \intertext{Thus we have that $e \rightsquigarrow [v_2/x]e_3$}
            \end{align*}
        \end{itemize}

        \item Type Preservation
        \[
            \Phi(\Gamma, e, \tau) \triangleq \Gamma \vdash e: \tau \land e \rightsquigarrow e' \implies \Gamma \vdash e': \tau
        \]
        I will prove this by structural induction on the derivation of $e \rightsquigarrow e'$ with the induction hypothesis that for all subexpressions $e$, $\Phi(\Gamma, e, \tau)$. I will prove only the
        unit and function cases.
        \begin{itemize}

            \item \textbf{Case Unit}

            There is no reduction rule for any expression of type unit. Therefore, the premise of $\Phi$ is false and so it holds trivially.

            \item \textbf{Case app1}
            \begin{align*}
                & \dfrac{e_1 \rightsquigarrow e_1'}{e_1\, e_2 \rightsquigarrow e_1'\, e_2} && \text{by assumption} \\
                & \Gamma \vdash e_1\, e_2: \tau && \text{by assumption} \\
                & \dfrac{\Gamma \vdash e_1: \tau' \to \tau \qquad \Gamma \vdash e_2: \tau}{\Gamma \vdash e_1\, e_2: \tau} && \text{by inversion} \\
                & \Gamma \vdash e_1: \tau' \to \tau && \text{subderivation} \\
                & \Gamma \vdash e_2: \tau && \text{subderivation} \\
                & \Gamma \vdash e_1': \tau' \to \tau && \text{by induction hypothesis} \\
                & \dfrac{\Gamma \vdash e_1': \tau' \to \tau \qquad \Gamma \vdash e_2: \tau}{\Gamma \vdash e_1'\, e_2: \tau}
                \intertext{Thus, we have proved $\Phi$ in this case!}
            \end{align*}
            \item \textbf{Case app2}
            \begin{align*}
                & \dfrac{e_2 \rightsquigarrow e_2'}{v\, e_2 \rightsquigarrow v\, e_2'} && \text{by assumption} \\
                & \Gamma \vdash v\, e_2: \tau && \text{by assumption} \\
                & \dfrac{\Gamma \vdash v: \tau' \to \tau \qquad \Gamma \vdash e_2: \tau}{\Gamma \vdash v\, e_2: \tau} && \text{by inversion} \\
                & \Gamma \vdash v: \tau' \to \tau && \text{subderivation} \\
                & \Gamma \vdash e_2: \tau && \text{subderivation} \\
                & \Gamma \vdash e_2': \tau' && \text{by induction hypothesis} \\
                & \dfrac{\Gamma \vdash e_1: \tau' \to \tau \qquad \Gamma \vdash e_2': \tau}{\Gamma \vdash e_1\, e_2': \tau}
                \intertext{Thus, we have proved $\Phi$ in this case!}
            \end{align*}
            \item \textbf{Case sub}
            \begin{align*}
                & \dfrac{}{(\lambda x: \tau' \ldotp e)\, v \rightsquigarrow [v/x]e} && \text{by assumption} \\
                & \Gamma \vdash (\lambda x: \tau' \ldotp e)\, v: \tau && \text{by assumption} \\
                & \dfrac{\Gamma \vdash \lambda x: \tau' \ldotp e: \tau' \to \tau \qquad \Gamma \vdash e_2: \tau}{\Gamma \vdash v\, e_2: \tau} && \text{by inversion} \\
                & \Gamma \vdash \lambda x: \tau' \ldotp e: \tau' \to \tau && \text{subderivation} \\
                & \Gamma \vdash e_2: \tau && \text{subderivation} \\
                & \dfrac{\Gamma, x: \tau' \vdash e: \tau}{\Gamma \vdash \lambda x\ldotp e: \tau' \to \tau} && \text{by inversion} \\
                & \Gamma, x: \tau' \vdash e: \tau && \text{subderivation} \\
                & \Gamma \vdash [v/x]e: \tau && \text{substitution} \\
                \intertext{Thus, we have proved $\Phi$ in this case!}
            \end{align*}

        \end{itemize}

    \end{itemize}

    \item In the simply typed $\lambda$-calculus, is there any context $\Gamma$ and type $Y$ for which $\Gamma \vdash x\, x: Y$ (where $x$ is a free variable defined in $\Gamma$)? If so, give $\Gamma$, $Y$ and
    show a typing derivation, otherwise prove that no such context and type exists.

    No such type exists! Consider the following proof by contradiction.

    Assume that there exists a context $\Gamma$ and type $Y$ such that $\Gamma \vdash x\, x: Y$. Thus the following proof tree must hold.

    \begin{figure}[H]

        \centering

        \begin{prooftree}
            \Infer0{\Gamma \vdash x: X \to Y}
            \Infer0{\Gamma \vdash x: X}
            \Infer2{\Gamma \vdash x\, x: Y}
        \end{prooftree}

    \end{figure}

    However, this implies that both $x: X \to Y \in \Gamma$ and $x: X \in \Gamma$. Since these are not equal types, we have that $\Gamma$ is an ill-formed typing context. This is a contradiction! Thus there
    exists no context $\Gamma$ and type $Y$ such that $\Gamma \vdash x\, x: Y$.

    \item Under the Curry-Howard Correspondence (CHC), a proposition $P$ is represented by a type $\pi$. Proving $P$ corresponds to providing a term $t$ such that $\cdot \vdash p: \pi$.

    \begin{enumerate}

        \item We represent the negation $\neg P$ by a type $\pi \to 0$. why is this a reasonable representation?

        By the CHC, we have that $\pi \to 0$ corresponds to the logical expression $P \supset 0$.

        \[
            P \subset 0 \equiv \neg P \vee 0 \equiv \neg P
        \]

        Thus, by the CHC, we have that the proposition represented by this type is logically equivalent to the proposition $\neg P$.

        \item Try to prove De Morgan's laws by converting each propositional formula to a type and providing a term of that type in the empty context.

        Let $\pi$ be the type corresponding to the propositional formula $P$, and let $\rho$ be the type corresponding to the propositional formula $Q$.

        \begin{enumerate}

            \item $\neg P \land \neg Q \supset \neg P \vee Q$

            By the CHC we have that $\neg P \land \neg Q \supset \neg P \vee Q$ corresponds to the type $(\pi \to 0) \times (\rho \to 0) \to (\pi + \rho) \to 0$. I will prove that this proposition holds by
            showing that there exists a term which has that type for arbitrary $\pi$, $\rho$.

            Let $\Gamma = x: (\pi \to 0) \times (\rho \to 0), y: \pi + \rho$

            \begin{sidewaysfigure}

                \centering

                \begin{prooftree}
                    \Infer0{\Gamma \vdash: \pi + \rho}
                    \Infer0{\Gamma, a: \pi \vdash x: (\mathrm{\pi} \to 0) \times (\rho \to 0)}
                    \Infer1{\Gamma, a: \pi \vdash \mathrm{fst}\, x: \mathrm{\pi} \to 0}
                    \Infer0{\Gamma, a: \pi \vdash a: \pi}
                    \Infer2{\Gamma, a: \pi \vdash (\mathrm{fst}\, x)\, a: 0}
                    \Infer0{\Gamma, a: \pi \vdash x: (\mathrm{\pi} \to 0) \times (\rho \to 0)}
                    \Infer1{\Gamma, a: \pi \vdash \mathrm{snd}\, x: \mathrm{\rho} \to 0}
                    \Infer0{\Gamma, b: \pi \vdash b: \rho}
                    \Infer2{\Gamma, a: \pi \vdash (\mathrm{fst}\, x)\, a: 0}
                    \Infer3{\Gamma \vdash \mathrm{case}(y, L\, a \to (\mathrm{fst}\, x)\, a, R\, b \to (\mathrm{snd}\, x)\, b): 0}
                    \Infer1{x: (\pi \to 0) \times (\rho \to 0) \vdash \lambda y: \pi + \rho \ldotp \mathrm{case}(y, L\, a \to (\mathrm{fst}\, x)\, a, R\, b \to (\mathrm{snd}\, x)\, b): (\pi + \rho) \to 0}
                    \Infer1{\cdot \vdash \lambda x: (\pi \to 0) \times (\rho \to 0) \ldotp \lambda y: \pi + \rho \ldotp \mathrm{case}(y, L\, a \to (\mathrm{fst}\, x)\, a, R\, b \to (\mathrm{snd}\, x)\, b):
                    (\pi \to 0) \times (\rho \to 0) \to (\pi + \rho) \to 0}
                \end{prooftree}

                \caption{Typing derivation for a term of type $(\pi \to 0) \times (\rho \to 0) \to (\pi + \rho) \to 0$}

            \end{sidewaysfigure}

            \item $\neg (P \vee Q) \supset \neg P \land \neg Q$

            By the CHC we have that $\neg P \land \neg Q \supset \neg P \vee Q$ corresponds to the type $((\pi + \rho) \to 0) \supset (\pi \to 0) \times (\rho \to 0)$. I will prove that this proposition
            holds by showing that there exists a term which has that type for arbitrary $\pi$, $\rho$.

            Let $\Gamma = x: (\pi + \rho) \to 0$.

            \begin{sidewaysfigure}

                \centering

                \begin{prooftree}
                    \Infer0{\Gamma, y: \pi \vdash x: (\pi + \rho) \to 0}
                    \Infer0{\Gamma, y: \pi \vdash y: \pi}
                    \Infer1{\Gamma, y: \pi \vdash L\, y: (\pi + \rho)}
                    \Infer2{\Gamma, y: \pi \vdash x\, (L\, y): 0}
                    \Infer1{\Gamma \vdash \lambda y: \pi \ldotp x\, (L\, y): (\pi \to 0)}
                    \Infer0{\Gamma, y: \rho \vdash x: (\pi + \rho) \to 0}
                    \Infer0{\Gamma, y: \rho \vdash y: \rho}
                    \Infer1{\Gamma, y: \rho \vdash R\, y: (\pi + \rho)}
                    \Infer2{\Gamma, y: \rho \vdash x\, (R\, y): 0}
                    \Infer1{\Gamma \vdash \lambda y: \rho \ldotp \, (R\, y): (\rho \to 0)}
                    \Infer2{\Gamma \vdash (\lambda y: \pi \ldotp x\, (L\, y), \lambda y: \rho \ldotp x\, (R\, y): (\pi \to 0) \times (\rho \to 0)}
                    \Infer1{\cdot \vdash \lambda x: (\pi + \rho) \to 0 \ldotp (\lambda y: \pi \ldotp x\, (L\, y), \lambda y: \rho \ldotp x\, (R\, y): ((\pi + \rho) \to 0) \to (\pi \to 0) \times (\rho \to 0)}
                \end{prooftree}

                \caption{Typing derivation for a term of type $((\pi + \rho) \to 0) \supset (\pi \to 0) \times (\rho \to 0)$}

            \end{sidewaysfigure}

            \item $\neg P \vee \neg Q \supset \neg (P \land Q)$

            By the CHC, we have that the type this proposition corresponds to is $((\pi \to 0) + (\rho \to 0)) \to (\pi \times \rho \to 0)$.

            Let $\Gamma = x: (\pi \to 0) + (\rho \to 0), y: \pi \times \rho$

            \begin{sidewaysfigure}

                \centering

                \begin{prooftree}
                    \Infer0{\Gamma \vdash x: (\pi \to 0) + (\rho \to 0)}
                    \Infer0{\Gamma, a: \pi \to 0 \vdash a: \pi \to 0}
                    \Infer0{\Gamma, a: \pi \to 0 \vdash y: \pi \times \rho}
                    \Infer1{\Gamma, a: \pi \to 0 \vdash \mathrm{fst}\, y: \pi}
                    \Infer2{\Gamma, a: \pi \to 0 \vdash a\, (\mathrm{fst}\, y): 0}
                    \Infer0{\Gamma, b: \rho \to 0 \vdash b: \rho \to 0}
                    \Infer0{\Gamma, b: \rho \to 0 \vdash y: \pi \times \rho}
                    \Infer1{\Gamma, b: \rho \to 0 \vdash \mathrm{snd}\, y: \rho}
                    \Infer2{\Gamma, b: \rho \to 0 \vdash b\, (\mathrm{snd}\, y): 0}
                    \Infer3{\Gamma \vdash \mathrm{case}(x, L\, a \to a\, (\mathrm{fst}\, y), b \to b\, (\mathrm{snd}\, y)): 0}
                    \Infer1{x: (\pi \to 0) + (\rho \to 0) \vdash \lambda y: \pi \times \rho \ldotp \mathrm{case}(x, L\, a \to a\, (\mathrm{fst}\, y), b \to b\, (\mathrm{snd}\, y)): \pi \times \rho \to 0}
                    \Infer1{
                        \cdot \vdash \lambda x: (\pi \to 0) + (\rho \to 0) \ldotp \lambda y: \pi \times \rho \ldotp \mathrm{case}(x, L\, a \to a\, (\mathrm{fst}\, y), b \to b\, (\mathrm{snd}\, y)):
                        ((\pi \to 0) + (\rho \to 0)) \to (\pi \times \rho \to 0)
                    }
                \end{prooftree}

                \caption{Typing derivation for a term of type $((\pi \to 0) + (\rho \to 0)) \to (\pi \times \rho \to 0)$}

            \end{sidewaysfigure}

            \item $\neg (P \land Q) \supset \neg P \vee \neg Q$

            This law does not hold! The type it corresponds to under the CHC is $(\pi \times \rho \to 0) \to (\pi \to 0) \times (\rho \to 0)$. No term can have this type in the empty context. Such a term
            would take a function $f: (\pi \times \rho) \to 0$ and return functions $g_1: \pi \to 0$, $g_2: \rho \to 0$. Either function individually only takes a term of type $\pi$ or $\rho$ and so cannot call
            $f$. Thus the only way such a term could exist is if it could make terms of type $0$ -- which would imply the language was unsound.

        \end{enumerate}

        \label{itm:b}

        \item De Morgan's laws all hold in propositional logic. What does~\ref{itm:b} tell you about the STLC in relation to propositional logic?

        The STLC does not correspond to propositional logic.

    \end{enumerate}

    \item Extend logical relation to support products and sums.

    We define a logical relation $\mathrm{Halt}_T$ for all types $T$:
    \begin{align*}
        & \mathrm{Halt}_0 = \emptyset \\
        & \mathrm{Halt}_1 = \{e \mid e\ \mathrm{halts}\} \\
        & \mathrm{Halt}_{X \to Y} = \{e \mid e\ \mathrm{halts} \land \forall e' \in \mathrm{Halt}_X\ldotp e\ e' \in \mathrm{Halt}_X\} \\
        & \mathrm{Halt}_{X \times Y} = \{e \mid e\ \mathrm{halts}\} \\
        & \mathrm{Halt}_{X + Y} = \{L\, e \mid e \in \mathrm{Halt}_X\} \cup \{R\, e \mid e \in \mathrm{Halt}_Y\}
    \end{align*}

    \item

    \begin{examquestion}{2023}{8}{13}

        Consider G\"odel's T, the Simply-Typed Lambda Calculus with function and natural number types, with zero, successor and iterator term formers for teh natural number type.

        \begin{enumerate}[label=$(\alph*)$]

            \item Define a logical relation suitable for establishing the termination of closed programs in this language.

            I define an intermediate logical relation and use this to define a logical relation which is suitable for establishing termination in the language.

            The definition of the intermediate logical relation is as follows:
            \begin{align*}
                \mathrm{Halt}_1 =& \{e \mid e\ \mathrm{halts}\} \\
                \mathrm{Halt}_{\N_{\le 0}} =& \left\{e \mid e \rightsquigarrow^* z\right\} \\
                \forall i \in \N \ldotp \mathrm{Halt}_{\N_\le i+1} =& \left\{e \mid e \rightsquigarrow^* s(e') \land e' \in \mathrm{Halt}_{\N \le i}\right\} \\
                \mathrm{Halt}_{\N} =& \bigcup_{i \in \N} \mathrm{Halt}_{\N_\le i} \\
                \mathrm{Halt}_{X \to Y} =& \left\{e \mid e\ \mathrm{halts} \land \forall e' \in \mathrm{Halt}_X \ldotp e\, e' \in \mathrm{Halt}_Y\right\} \\
                &\cup \left\{\mathrm{iter}(e, z \to e_1, s(x) \to e_2) \mid e \in \mathrm{Halt}_\N \land e_1 \in \mathrm{Halt}_{X \to Y} \right. \\
                &\qquad \left.\land \lambda x: X \to Y \ldotp e_2 \in \mathrm{Halt}_{(X \to Y) \to (X \to Y)}\right\}\\
            \end{align*}

            \item State the fundamental lemma for this language.

            \[
                x_1: X_1, x_2: X_2, \ldots, x_n: X_n \vdash e: T \land \forall i \in \{1, \ldots n\}\ldotp \cdot \vdash v_i: X_i \implies [v_1/x_1, \ldots, v_n / x_n]e \in \mathrm{Halt}_T
            \]

            \item State formally what it means for a set of terms $X$ to he ``closed under reduction''.

            The set of terms $X$ is ``closed under reduction'' if, and only if:
            \[
                \forall e \in X. e \rightsquigarrow e' \implies e' \in X
            \]

            \item Prove the fundamental lemma holds for the iterator case.

            Prove the fundamental lemma for G\"odel's T by structural induction on $x_1: X_1, \ldots, x_n: X_n \vdash e: Z$.

            For brevity, let $\Gamma = x_1: X_1, x_2: X_2, \ldots, x_n: X_n$ and $\gamma = [v_1/x_1, \ldots, v_n / x_n]$

            Case (iter):
            \begin{align*}
                & \Gamma \vdash e: Z && \text{by assumption} \\
                & e = \mathrm{iter}(e_0, z \to e_1, s(x) \to e_2) && \text{by assumption} \\
                & \dfrac{
                    \Gamma \vdash e_0: \N \qquad \Gamma: Z \qquad \Gamma, x: Z \vdash e_2 : X
                }{
                    \Gamma \vdash \mathrm{iter}(e_0, z \to e_1, s(x) \to e_2): X
                } && \text{by assumption} \\
                & \Gamma \vdash e_0: \N && \text{subderivation} \\
                & \Gamma \vdash e_1: Z && \text{subderivation} \\
                & \Gamma, x: Z \vdash e_2: Z && \text{subderivation} \\
                & \forall i \in \{1, \ldots, n\}\ldotp v_i: X_i && \text{by assumption} \\
                & \gamma e \\
                =& \gamma\mathrm{iter}(e_0, z \to e_1, s(x) \to e_2) && \text{by assumption} \\
                =& \mathrm{iter}(\gamma e_0, z \to \gamma e_1, s(x) \to \gamma e_2) && \text{by definition of substitution} \\
                & \gamma e_0 \in \mathrm{Halt}_{\N} && \text{by the induction hypothesis} \\
                & \gamma e_1 \in \mathrm{Halt}_Z && \text{by the induction hypothesis} \\
            \end{align*}

            Notice that since we have $\gamma e_0 \in \mathrm{Halt}_{\N}$, we have $\gamma e_0 \rightsquigarrow^* n$. I will perform structural induction on $n$ for the remainder of the proof (\ie to prove
            $\gamma \mathrm{iter}(e_0, z \to e_1, s(x) \to e_2) \in \mathrm{Halt}_Z$)

            \begin{align*}
                \intertext{Case ($z$)}
                & \quad \gamma e_0 \rightsquigarrow^* z && \text{by assumption} \\
                & \quad \gamma \mathrm{iter}(e_0, z \to e_1, s(x) \to e_2) \rightsquigarrow \gamma \mathrm{iter}(z, z \to e_1, s(x) \to e_2) && \text{by the reduction
                rules}\\
                & \quad \gamma \mathrm{iter}(z, z \to e_1, s(x) \to e_2) \rightsquigarrow \gamma e_1 && \text{by the reduction rules} \\
                & \quad \gamma e_1 \rightsquigarrow^* v && \text{by assumption, since $\gamma e_1 \in \mathrm{Halt}_Z$} \\
                & \gamma \mathrm{iter}(z, z \to e_1, s(x) \to e_2) \rightsquigarrow^* v && \text{by transivitity of reduction} \\
                \intertext{Case ($s(v')$)}
                & \quad \gamma e_0 \rightsquigarrow^* s(x) && \text{by assumption} \\
                & \quad \gamma \mathrm{iter}(e_0, z \to e_1, s(x) \to e_2) \rightsquigarrow \gamma \mathrm{iter}(s(v'), z \to e_1, s(x) \to e_2) && \text{by the reduction
                rules}\\
                & \quad \gamma \mathrm{iter}(s(v'), z \to e_1, s(x) \to e_2) \rightsquigarrow \gamma [v'/x]e_2 && \text{by the reduction rules} \\
                & \quad \gamma [v'/x]e_1 \rightsquigarrow^* v && \text{by assumption, since $\gamma [v'/x]e_1 \in \mathrm{Halt}_Z$} \\
                & \gamma \mathrm{iter}(s(v'), z \to e_1, s(x) \to e_2) \rightsquigarrow^* v && \text{by transivitity of reduction} \\
            \end{align*}

            We have now proved the fundamental lemma in the iter case!

        \end{enumerate}

    \end{examquestion}

    \begin{examquestion}{2020}{8}{15}

        \begin{enumerate}[label=(\textit{\alph*})]

            \setcounter{enumii}{1}

            \item

            \begin{enumerate}[label=(\roman*)]

                \item Give the typing rules for Peano natural numbers and their eliminator.

                Peano natural numbers can be represented either as $z$ (for zero) or $s(x)$ for the successor of some other natural number $x$. Their eliminator is the \texttt{iter} function -- which can be
                interpreted as bounded recursion.

                \[
                    \dfrac{}{\Gamma \vdash z: \N} \qquad \dfrac{\Gamma \vdash x: \N}{\Gamma \vdash s(x): \N} \\
                    \dfrac{\Gamma \vdash e: \N \qquad \Gamma \vdash e_1: X \qquad \Gamma, x: X \vdash e_2: X \to X}{\mathrm{iter}(e, z \to e_1, s(x) \to e_2): X}
                \]

                \item Using the rules given above, define the addition function.

                \[
                    \mathrm{add} \triangleq \lambda m\ldotp \lambda n\ldotp \mathrm{iter}(m, z \to n, s(x) \to s(x))
                \]

                \item Let a binary tree be either a leaf \texttt{Leaf} or a node \texttt{Node(l, x, r)} where \texttt{l} and \texttt{r} are subtrees, and \texttt{x} is a natural number. Give typing rules for trees
                corresponding to this prose description, including an eliminator.

                I will consider the eliminator $\mathrm{fold}$ which aggregates the nodes in a tree with a user-defined ternary operation.
                For example $\mathrm{Fold}(e, \mathrm{Leaf} \to z, \mathrm{Node}(l, x, r) \to \mathrm{add}(l, \mathrm{add}(x, r)))$ would sum all the nodes in the tree.

                \begin{gather*}
                    \dfrac{}{\Gamma \vdash \mathrm{Leaf}: \mathrm{tree}} \qquad
                    \dfrac{\Gamma \vdash e_1: \mathrm{tree} \qquad \Gamma \vdash e_2: \mathrm{tree}}{\Gamma \vdash \mathrm{Node}(e_1, x, e_2): \mathrm{tree}} \\[2em]
                    \dfrac{\Gamma \vdash e: \mathrm{tree} \qquad \Gamma \vdash e_1: X \qquad \Gamma, l: X, x: \N, r: X \vdash e_2: X}{\Gamma \vdash \mathrm{fold}(e, \mathrm{Leaf} \to e_1, \mathrm{Node} \to e_2): X}
                \end{gather*}

                \item Using the rules given above, define a function \texttt{size} which takes a binary tree and returns the total number of nodes in the tree.

                \[
                    \mathrm{size} = \lambda x: \mathrm{tree}\ldotp \mathrm{fold}(x, \mathrm{Leaf} \to z, \mathrm{Node}(l, x, r) \to s(\mathrm{add}(l, r))
                \]

            \end{enumerate}

        \end{enumerate}

    \end{examquestion}

\end{enumerate}

\section{Extension Questions}

\begin{enumerate}

    \item Find a context $\Gamma$ under which the term $f\, x\, y$ has type $\mathrm{Bool}$. \ie $\Gamma \vdash f\, x\, y: \mathrm{Bool}$.

    In the STLC the representation of $\mathrm{Bool}$ is given by $1 + 1$. I will work in this langauge.

    \[
        \Gamma = f: 1 \to 1 \to 1 + 1, x: 1, y: 1
    \]

    \item Can you give a simple description of the set of all such contexts?

    \[
        G = \{(f: X \to Y \to 1 + 1, x: X, y: Y) \mid X, Y \text{ are valid types}\}
    \]

    \item OCaml has a polymorphic equals-function. That is, $\mathrm{equal}: a \to a \to \mathrm{bool}$ is defined for all types. What are the implications on safety of the type system?

    From computation theory; we have that for the lambda calculus, the existence of an equals function implies that there are expressions which loop infinitely. Applying similar logic here means that the
    existence of a polymorphic equals function means there is an expression in OCaml which loops infinitely.

\end{enumerate}

\end{document}
