\newcommand{\svcourse}{CST Part IA: Software Engineering and Security}
\newcommand{\svnumber}{1}
\newcommand{\svvenue}{Microsoft Teams}
\newcommand{\svdate}{2022-05-11}
\newcommand{\svtime}{15:00}
\newcommand{\svuploadkey}{CBd13xmL7PC1zqhNIoLdTiYUBnxZhzRAtJxv/ytRdM1r7qIfwMsxeVwM/pPcIo8l}

\newcommand{\svrname}{Dr Sam Ainsworth}
\newcommand{\jkfside}{oneside}
\newcommand{\jkfhanded}{yes}

\newcommand{\studentname}{Harry Langford}
\newcommand{\studentemail}{hjel2@cam.ac.uk}


\documentclass[10pt,\jkfside,a4paper]{article}
\usepackage{mathtools}
\usepackage{ebproof}
\usepackage{stmaryrd}

% DO NOT add \usepackage commands here.  Place any custom commands
% into your SV work files.  Anything in the template directory is
% likely to be overwritten!

\usepackage{fancyhdr}

\usepackage{lastpage}       % ``n of m'' page numbering
\usepackage{lscape}         % Makes landscape easier

\usepackage{verbatim}       % Verbatim blocks
\usepackage{listings}       % Source code listings
\usepackage{graphicx}
\usepackage{float}
\usepackage{epsfig}         % Embed encapsulated postscript
\usepackage{array}          % Array environment
\usepackage{qrcode}         % QR codes
\usepackage{enumitem}       % Required by Tom Johnson's exam question header

\usepackage{hhline}         % Horizontal lines in tables
\usepackage{siunitx}        % Correct spacing of units
\usepackage{amsmath}        % American Mathematical Society
\usepackage{amssymb}        % Maths symbols
\usepackage{amsthm}         % Theorems

\usepackage{ifthen}         % Conditional processing in tex

\usepackage[top=3cm,
            bottom=3cm,
            inner=2cm,
            outer=5cm]{geometry}

% PDF metadata + URL formatting
\usepackage[
            pdfauthor={\studentname},
            pdftitle={\svcourse, SV \svnumber},
            pdfsubject={},
            pdfkeywords={9d2547b00aba40b58fa0378774f72ee6},
            pdfproducer={},
            pdfcreator={},
            hidelinks]{hyperref}

\renewcommand{\headrulewidth}{0.4pt}
\renewcommand{\footrulewidth}{0.4pt}
\fancyheadoffset[LO,LE,RO,RE]{0pt}
\fancyfootoffset[LO,LE,RO,RE]{0pt}
\pagestyle{fancy}
\fancyhead{}
\fancyhead[LO,RE]{{\bfseries \studentname}\\\studentemail}
\fancyhead[RO,LE]{{\bfseries \svcourse, SV~\svnumber}\\\svdate\ \svtime, \svvenue}
\fancyfoot{}
\fancyfoot[LO,RE]{For: \svrname}
\fancyfoot[RO,LE]{\today\hspace{1cm}\thepage\ / \pageref{LastPage}}
\fancyfoot[C]{\qrcode[height=0.8cm]{\svuploadkey}}
\setlength{\headheight}{22.55pt}


\ifthenelse{\equal{\jkfside}{oneside}}{

 \ifthenelse{\equal{\jkfhanded}{left}}{
  % 1. Left-handed marker, one-sided printing or e-marking, use oneside and...
  \evensidemargin=\oddsidemargin
  \oddsidemargin=73pt
  \setlength{\marginparwidth}{111pt}
  \setlength{\marginparsep}{-\marginparsep}
  \addtolength{\marginparsep}{-\textwidth}
  \addtolength{\marginparsep}{-\marginparwidth}
 }{
  % 2. Right-handed marker, one-sided printing or e-marking, use oneside.
  \setlength{\marginparwidth}{111pt}
 }

}{
 % 3. Alternating margins, two-sided printing, use twoside.
}


\setlength{\parindent}{0em}
\addtolength{\parskip}{1ex}

% Exam question headings, labels and sensible layout (courtesy of Tom Johnson)
\setlist{parsep=\parskip, listparindent=\parindent}
\newcommand{\examhead}[3]{\section{#1 Paper #2 Question #3}}
\newenvironment{examquestion}[3]{
\examhead{#1}{#2}{#3}\setlist[enumerate, 1]{label=(\alph*)}\setlist[enumerate, 2]{label=(\roman*)}
\marginpar{\href{https://www.cl.cam.ac.uk/teaching/exams/pastpapers/y#1p#2q#3.pdf}{\qrcode{https://www.cl.cam.ac.uk/teaching/exams/pastpapers/y#1p#2q#3.pdf}}}
\marginpar{\footnotesize \href{https://www.cl.cam.ac.uk/teaching/exams/pastpapers/y#1p#2q#3.pdf}{https://www.cl.cam.ac.uk/\\teaching/exams/pastpapers/\\y#1p#2q#3.pdf}}
}{}


\newcommand{\pack}{\ensuremath{\mathit{pack}}}
\newcommand{\letpack}{\ensuremath{\text{let}\ \mathit{pack}}}
\newcommand{\type}[1]{\ensuremath{#1\ \text{type}}}
\newcommand{\ctx}[1]{\ensuremath{#1\ \text{ctx}}}
\newcommand{\abr}[2]{\ensuremath{\langle #1, #2 \rangle}}
\newcommand{\brk}[1]{\ensuremath{\llbracket #1 \rrbracket}}
\newcommand{\sub}{\ensuremath{\mathsf{sub}}}

\begin{document}

\section{Warmup Questions}

\begin{enumerate}

    \item Write down the rules for introduction and elimination of abstract data types in System F (i.e.\@ \pack\ and let \pack). Explicitly specify the ``client code'' that doesn't get to see the abstracted type.

    \begin{gather*}
        \dfrac{
            \Theta \vdash \type{A} \qquad \Theta \vdash \ctx{\Gamma} \qquad \Theta; \Gamma, x: \alpha \vdash e: B
        }{
            \Theta; \Gamma \vdash \pack_{\alpha.B}(A, e): \exists \alpha\ldotp B
        }(\pack) \\\\
        \dfrac{
            \Theta; \Gamma \vdash \pack_{\alpha.B}(A, e): \exists \alpha\ldotp B \qquad \Theta; \Gamma, x: B \vdash e': C
        }{
            \Theta; \Gamma \vdash \letpack(\alpha, x) = \pack_{\alpha.B}(A, e)\ \text{in}\ e': C
        }(\mathit{letpack})
    \end{gather*}

    \item What do $\sigma$ and $\Sigma$ represent?

    $\sigma$ represents the store. While $\Sigma$ represents the store type.

    \item Write down the typing rules and operational semantics for the imperative additions to STLC\@.

    \textbf{Typing Rules:}

    We add specific rules for interacting with the store. We then add typing rules for typing the store. We then add a rule for using both of these to type a context. Note that we must also modify all existing
    typing rules to use a store typing context as well.

    Interacting with the store:

    \begin{gather*}
        \dfrac{
            \Gamma; \Sigma \vdash e: \mathrm{ref}\ T
        }{
            \Gamma; \Sigma \vdash !e: T
        }
        \qquad
        \dfrac{
            \Gamma; \Sigma \vdash e: T
        }{
            \Gamma; \Sigma \vdash \mathrm{new}\ e: \mathrm{ref}\ T
        }
        \\\\
        \dfrac{
            \Gamma; \Sigma \vdash e: T \qquad \Gamma; \Sigma \vdash \ell: \mathrm{ref}\ T
        }{
            \Gamma; \Sigma \vdash \ell \coloneqq e: \mathrm{unit}
        }
        \qquad
        \dfrac{
            \ell: T \in \Sigma
        }{
            \Gamma; \Sigma \vdash \ell: \mathrm{ref}\ T
        }
    \end{gather*}

    Modifying existing typing rules:
    \begin{gather*}
        \dfrac{
        \Gamma, x: A; \Sigma \vdash e: B
        }{
        \Gamma; \Sigma \vdash \lambda x: A\ldotp e: A \to B
        }
        \qquad
        \dfrac{
        \Gamma; \Sigma \vdash e: A \to B
        \qquad
        \Gamma; \Sigma \vdash e': A
        }{
        \Gamma; \Sigma \vdash e\ e': B
        }
        \\\\
        \ldots
    \end{gather*}

    Typing the Store:

    \begin{gather*}
        \dfrac{
        }{
            \cdot \vdash \cdot: \cdot
        }
        \qquad
        \dfrac{
            \Sigma \vdash \sigma: \Sigma' \qquad \Sigma \vdash v: T
        }{
            \Sigma \vdash (\sigma, \ell: v): (\Sigma', \ell: T)
        }
    \end{gather*}

    Typing Configurations:

    \begin{gather*}
        \dfrac{
            \Sigma \vdash \sigma: \Sigma \qquad \cdot; \Sigma \vdash e: T
        }{
            \abr{\sigma}{e}: (\Sigma, T)
        }
    \end{gather*}

    \textbf{Reduction Rules:}

    We start by making all reduction rules operate on configurations of the form $\abr{\sigma}{e}$. This requires modifying all existing rules to make them carry a $\sigma$ in all premises and
    conclusions. Next, we add specific rules for interacting with the store.

    Interacting with the Store:

    \begin{gather*}
        \dfrac{
            (\ell, v) \in \sigma
        }{
            \abr{\sigma}{!\ell} \rightsquigarrow \abr{\sigma}{v}
        }
        \qquad
        \dfrac{
            \abr{\sigma}{e} \rightsquigarrow \abr{\sigma}{e'}
        }{
            \abr{\sigma}{!e} \rightsquigarrow \abr{\sigma}{!e'}
        }
        \qquad
        \\\\
        \dfrac{
        }{
            \abr{\sigma}{\ell \coloneqq v} \rightsquigarrow \abr{(\sigma, \ell: v)}{()}
        }
        \qquad
        \dfrac{
            \abr{\sigma}{e_1} \rightsquigarrow \abr{\sigma}{e_1'}
        }{
            \abr{\sigma}{e_1 \coloneqq e_2} \rightsquigarrow \abr{\sigma}{e_1' \coloneqq e_2}
        }
        \qquad
        \dfrac{
            \abr{\sigma}{e_2} \rightsquigarrow \abr{\sigma}{e_2'}
        }{
            \abr{\sigma}{\ell \coloneqq e_2} \rightsquigarrow \abr{\sigma}{\ell \coloneqq e_2'}
        }
        \\\\
        \dfrac{
            \ell \notin \sigma
        }{
            \abr{\sigma}{\mathrm{new}\ v} \rightsquigarrow \abr{(\sigma, \ell: v)}{()}
        }
        \qquad
        \dfrac{
            \abr{\sigma}{e} \rightsquigarrow \abr{\sigma}{e'}
        }{
            \abr{\sigma}{\mathrm{new}\ e} \rightsquigarrow \abr{\sigma}{\mathrm{new}\ e'}
        }
    \end{gather*}

    Threading $\sigma$ through all other reduction rules:
    \[
        \dfrac{
        \abr{\sigma}{e_1} \rightsquigarrow \abr{\sigma}{e_1'}
        }{
        \abr{\sigma}{e_1\ e_2} \rightsquigarrow \abr{\sigma}{e_1'\ e_2}
        }
        \qquad
        \dfrac{
        }{
        \abr{\sigma}{(\lambda x\ldotp e)\ v} \rightsquigarrow \abr{\sigma}{[v/x]e}
        }
        \\\\
        \ldots
    \]

    \item Explain the reason to separate typing logic for pure and impure terms when using monads to track effects and explain how they are linked together.

    We separate terms based on whether they have side-effects or not. Those terms which do have side effects (impure) have monadic types. Those terms which don't (pure); do not have monadic types. This allows
    the type system to determine whether evaluating a particular expression will have any side effects. This is useful when considering compiler optimisations as we can determine what can be done in parallel or
    lazily; and what has to be done in-order.

\end{enumerate}

\section{Regular Questions}

\begin{enumerate}

    \item For each of the following PLC (System F) typing judgements, is there a PLC type $A_i$ that makes the judgement provable? In each case, give a type $A_i$ and a typing derivation for, or explain why a
    typing derivation cannot exist.

    \begin{enumerate}

        \item $\cdot; \cdot \vdash \lambda x: (\forall \alpha\ldotp \alpha)\ldotp (\Lambda \beta\ldotp x\ \beta): A_1$

        \label{enm:question1a}

        $A_1 = (\forall \alpha\ldotp \alpha) \to \forall \beta \ldotp \beta$ makes the judgement provable.

        \begin{figure}[H]

            \centering

            \begin{prooftree}
                \infer0{\beta; x: \forall \alpha\ldotp \alpha \ldotp x: \forall \alpha\ldotp \alpha}
                \infer0{\beta; x: \forall \alpha\ldotp \alpha \ldotp \beta: \beta}
                \infer2{\beta; x: \forall \alpha\ldotp \alpha \ldotp x\ \beta: \beta}
                \infer1{\cdot; x: \forall \alpha\ldotp \alpha \vdash \Lambda \beta\ldotp x\ \beta: \forall \beta\ldotp \beta}
                \infer1{\cdot; \cdot \vdash \lambda x: (\forall \alpha\ldotp \alpha)\ldotp (\Lambda \beta\ldotp x\ \beta): (\forall \alpha\ldotp \alpha) \to \forall \beta \ldotp \beta}
            \end{prooftree}

            \caption{Typing derivation for \ref{enm:question1a}}

        \end{figure}

        \item $\cdot; \cdot \vdash \Lambda \alpha\ldotp \lambda x: \alpha\ldotp \Lambda \beta \ldotp x\ \beta: A_2$

        There is no $A_2$ such that the typing derivation is provable. The reason for this is that we are applying $\beta$ to a term of type $\alpha$. But $\alpha$ does not have to take any type argument. So if
        this has any type then we allow the application of types to terms which do not take types (such as values).

        \item $\cdot; \cdot \vdash \lambda x: A_3\ldotp \Lambda \alpha\ldotp (x\ (\alpha \to \alpha)\ (x\ \alpha)): A_3 \to \forall \beta\ldotp \beta$

        There is no $A_3$ such that the the typing derivation is provable.

        By repeated inversion on the LHS of the function application, we can see that $A_3 = \forall \alpha\ldotp X \to \alpha$ and $\forall \alpha\ldotp X$ for some $X$. However, notice that $A_3$ is bound
        before $\alpha$. So $\alpha$ is \textit{not visible} to $A_3$ when it is bound -- and it is only passed $\alpha \to \alpha$. Therefore such a type is impossible!

        \item $\cdot; \cdot \vdash \lambda x: A_4\ldotp \Lambda \alpha\ldotp (x\ (\alpha \to \alpha)\ (x\ \alpha)): A_4 \to \forall \alpha\ldotp (\alpha \to \alpha)$

        $A_4 = \forall \beta \ldotp \beta \to \beta$ makes the judgement provable.

        \label{enm:question1d}

        \begin{figure}[H]

            \begin{prooftree}
                \infer0{\alpha; x: \forall \beta \ldotp \beta \to \beta \vdash x: \forall \beta\ldotp \beta \to \beta}
                \infer0{\alpha \vdash \type{\alpha \to \alpha}}
                \infer2{\alpha; x: \forall \beta \ldotp \beta \to \beta \vdash x\ (\alpha \to \alpha): (\alpha \to \alpha) \to (\alpha \to \alpha)}
                \infer0{\alpha; x: \forall \beta \ldotp \beta \to \beta \vdash x: \forall \beta\ldotp \beta \to \beta}
                \infer0{\alpha \vdash \type{\alpha}}
                \infer2{\alpha; x: \forall \beta \ldotp \beta \to \beta \vdash (x\ \alpha): \alpha \to \alpha}
                \infer2{\alpha; x: \forall \beta \ldotp \beta \to \beta \vdash x\ (\alpha \to \alpha)\ (x\ \alpha): \alpha \to \alpha}
                \infer1{\cdot; x: \forall \beta \ldotp \beta \to \beta\vdash \lambda \alpha\ldotp (x\ (\alpha \to \alpha)\ (x\ \alpha)): \forall \alpha\ldotp (\alpha \to \alpha)}
                \infer1{\cdot; \cdot \vdash \lambda x: \forall \beta \ldotp \beta \to \beta\ldotp \lambda \alpha\ldotp (x\ (\alpha \to \alpha)\ (x\ \alpha)): (\forall \beta \ldotp \beta \to \beta) \to \forall \alpha\ldotp (\alpha \to \alpha)}
            \end{prooftree}

            \caption{Typing derivation for \ref{enm:question1d}}

        \end{figure}

        \item $\cdot; \cdot \vdash \Lambda \alpha\ldotp \lambda x: A_5\ldotp (x\ (\alpha \to \alpha)\ (x\ \alpha)): \forall \alpha\ldotp (\alpha \to \alpha)$

        There is no type $A_5$ which makes the judgement provable. Consider the type $\forall \alpha\ldotp \alpha \to \alpha$. Since the term first take $x: A_5$ as an argument, we are constrained that
        if there is a type $A_5$ then we must have $A_5 = \alpha$. But we then pass the types $(\alpha \to \alpha)$ to $x$. This places additional constraints on its type meaning that we have $A_5$ cannot
        possibly be the type $\alpha$! So we have a contradiction and therefore there is no type $A_5$ which makes the judgment provable

    \end{enumerate}

    \item Complete Exercises 2, 3 and 4 from Lecture 5:

    \textbf{Exercises 2 and 3 confused me!}

    \begin{enumerate}[label=(\arabic*)]

        \setcounter{enumii}{1}

        \item Define a Church encoding for the unit type.

        \[
            \mathsf{unit} = 1
        \]

        \item Define a church encoding for the empty type.

        \[
            \mathsf{empty} = 0
        \]

        \item Define a Church encoding for binary trees, corresponding to the ML datatype

        \texttt{type tree = Leaf | Node of tree * X * tree}

        Let $T$ be the type of a tree, $N$ be the type of a node and $L$ be the type of a leaf:

        $L$ is the unit type:
        \[
            L = 1
        \]

        $N$ is a triple:
        \[
            N = \forall \alpha\ldotp (X \to Y \to Z \to \alpha) \to \alpha
        \]

        $T$ is a sum of $L$ or $N$:
        \[
            T = L + N
        \]
        Where $L + N$ can be defined using the definition of sum types given in the notes.

    \end{enumerate}

    \item The \textit{Church numerals} are defined as follows:
    \begin{align*}
        c_0 &= \Lambda \alpha\ldotp \lambda z: \alpha \ldotp \lambda s: \alpha \to \alpha\ldotp z \\
        c_1 &= \Lambda \alpha\ldotp \lambda z: \alpha \ldotp \lambda s: \alpha \to \alpha\ldotp s\ z \\
        c_2 &= \Lambda \alpha\ldotp \lambda z: \alpha \ldotp \lambda s: \alpha \to \alpha\ldotp s(s\ z) \\
        c_3 &= \Lambda \alpha\ldotp \lambda z: \alpha \ldotp \lambda s: \alpha \to \alpha\ldotp s(s(s\ z))
    \end{align*}

    The successor function that takes $c_n$ to $c_{n + 1}$ is defined in the lectures as
    \[
        \mathit{suc}(n) \stackrel{\mathit{def}}{=} \Lambda \alpha\ldotp \lambda z: \alpha\ldotp \lambda s: \alpha \to \alpha\ldotp s(n\ \alpha\ z\ s)
    \]
    Using just the variables $\alpha$, $n$, $s$ and $z$, find another way to define the successor function, $\mathit{suc}$ on Church numerals.
    \[
        \mathit{suc}_2(n) \stackrel{\mathit{def}}{=} \Lambda \alpha\ldotp \lambda z: \alpha\ldotp \lambda s: \alpha \to \alpha\ldotp n\ \alpha\ (s\ z)\ s
    \]

    \item Complete Exercises 1, 2 and 3 from Lecture 6.

    \begin{enumerate}[label=(\arabic*)]

        \item Prove the other direction of the closure property for the $\Theta \vdash \forall \alpha\ldotp A$ type case.

        \textbf{Closure (for $\forall \alpha\ldotp A$):} if $\theta$ is an interpretation of $\Theta$ then $\brk{\Theta \vdash \type{\forall \alpha\ldotp A}}\theta$ is a semantic type.

        To prove this, we have to prove that $\brk{\Theta \vdash \type{\forall \alpha\ldotp A}}\theta$ is closed and all terms in it halt. The closure property of a Semantic Type $X$ is
        $e \rightsquigarrow e' \implies e' \in X \iff e \in X$. We have to prove the direction $\impliedby$.
        \begin{align}
            & e \rightsquigarrow e' && \text{Assumption} \label{closure:1} \\
            & e \in \brk{\Theta \vdash \type{\forall \alpha \ldotp A}}\theta && \text{Assumption} \label{closure:2} \\
            & \forall (C, X)\ldotp e\ C \in \brk{\Theta, \alpha \vdash \type{A}}(\theta, X/\alpha) && \text{Definition of \ref{closure:2}} \label{closure:3}\\
            \intertext{Fix $(C, X)$ for arbitrary $(C, X)$}
            & \qquad e\ C \in \brk{\Theta, \alpha \vdash \type{A}}(\theta, X/\alpha) && \text{by \ref{closure:3}} \label{closure:4} \\
            & \qquad e\ C \rightsquigarrow e'\ C && \text{\textsc{CongForall} on \ref{closure:1}} \label{closure:5} \\
            & \qquad e'\ C \rightsquigarrow e'\ C && \text{Induction on \ref{closure:4} and \ref{closure:5}} \label{closure:6} \\
            & \forall (C, X)\ldotp e'\ C \in \brk{\Theta, \alpha \vdash \type{A}}(\theta, X/\alpha) && \text{by \ref{closure:6} since $(C, X)$ was arbitrary} \label{closure:7} \\
            & e' \in \brk{\Theta \vdash \type{A}}\theta && \text{by \ref{closure:7}}
        \end{align}

        \item Prove the other direction of the substitution property for the $\Theta \vdash \forall \alpha\ldotp A$ type case.

        \textbf{Substitution (for $\forall \alpha\ldotp A$):}
        $\brk{\Theta, \alpha \vdash \type{\forall \beta\ldotp B}}(\theta, \brk{\Theta \vdash \type{A}}) = \brk{\Theta \vdash \type{[A/\alpha](\forall \beta\ldotp B)}}\theta$.
        \begin{align}
            & e \in \brk{\Theta \vdash \type{[A/\alpha](\forall \beta\ldotp B)}}\theta && \text{Assumption} \\
            & e \in \brk{\Theta \vdash \type{\forall \beta\ldotp [A/\alpha]B}}\theta && \text{Definition of substitution} \\
            & \forall (C, X)\ldotp e'\ C \in \brk{\Theta, \beta \vdash \type{[A/\alpha]B}}(\theta, X/\beta) && \text{Definition} \label{sub:3}\\
            \intertext{Fix $(C, X)$ for arbitrary $(C, X)$}
            & \qquad e\ C \in \brk{\Theta, \beta \vdash \type{[A/\alpha]B}}(\theta, X/\beta) && \text{by \ref{sub:3}} \\
            & \qquad e\ C \in \brk{\Theta, \alpha, \beta \vdash \type{B}}(\theta, \brk{\Theta \vdash \type{A}}\theta, X/\beta) && \text{Induction} \label{sub:5} \\
            & \forall (C, X)\ldotp e\ C \in \brk{\Theta, \alpha, \beta \vdash \type{B}}(\theta, \brk{\Theta \vdash \type{A}}\theta, X/\beta) && \text{by \ref{sub:5}} \\
            & e \in \brk{\Theta, \alpha \vdash \forall \beta\ldotp \type{B}}(\theta, \brk{\Theta \vdash \type{A}}\theta) && \text{by definition}
        \end{align}

        \item Prove the fundamental lemma for the forall-introduction case $\Theta; \Gamma \vdash \Lambda \alpha\ldotp e: \forall \alpha\ldotp A$.

        The Fundamental Lemma states that $\Theta = \alpha_1, \ldots, \alpha_n$, $\Gamma = x_1: A_1, \ldots, x_n: A_n$ and $\Theta; \Gamma \vdash e: B$, $\Theta \vdash \ctx{\Gamma}$, $\theta$ interprets $\Theta$ and
        for each $x_i: A_i \in \Gamma$, we have $e_1 \in \brk{\Theta \vdash \type{A_i}}\theta$ then we have $[C_1/\alpha_1, \ldots C_k / \alpha_k][e_1/x_1\ldots e_n/x_n]e \in \brk{\Theta \vdash \type{B}}\theta$.

        We must prove this for the forall-introduction case $\Theta; \Gamma \vdash \Lambda \alpha\ldotp e: \forall \alpha\ldotp A$:
        \begin{align}
            & \Theta; \Gamma \vdash \Lambda \alpha\ldotp e: \forall \alpha\ldotp A && \text{Assumption} \\
            & \Theta \vdash \ctx{\Gamma} && \text{Assumption} \\
            & \theta \text{ interprets } \Theta && \text{Assumption} \label{fun:3} \\
            & \forall x_i: A_i \in \Gamma\ldotp e_1 \in \brk{\Theta \vdash \type{A_i}}\theta && \text{Assumption} \\
            & \Theta; \Gamma \vdash \Lambda \alpha\ldotp e: \forall \alpha\ldotp A && \text{Assumption} \\
            & \Theta, \alpha; \Gamma \vdash e: A && \text{Definition} \\
            \intertext{Consider arbitrary $(C, X)$}
            & \qquad (\theta, X/\alpha) \text{ interprets } (\Theta, \alpha) && \text{Definition on \ref{fun:3}} \\
            & \qquad [C_1/\alpha_1, \ldots, C_k/\alpha_k, C/\alpha][e_1/x_1, \ldots, e_n/x_n]e \in \brk{\Theta, \alpha \vdash \type{B}}(\theta, X/\alpha) && \text{Induction} \\
            & \qquad [C_1/\alpha_1, \ldots, C_k/\alpha_k][e_1/x_1, \ldots, e_n/x_n](\Lambda \alpha\ldotp e)\ C \in \brk{\Theta, \alpha \vdash \type{B}}(\theta, X/\alpha) && \text{Inversion} \label{fun:9} \\
            & \forall (C, X)\ldotp [C_1/\alpha_1, \ldots, C_k/\alpha_k][e_1/x_1, \ldots, e_n/x_n](\Lambda \alpha\ldotp e)\ C \in \brk{\Theta, \alpha \vdash \type{B}}(\theta, X/\alpha) && \text{by \ref{fun:9}} \\
            & [C_1/\alpha_1, \ldots, C_k/\alpha_k][e_1/x_1, \ldots, e_n/x_n](\Lambda \alpha\ldotp e) \in \brk{\Theta \vdash \type{\forall \alpha\ldotp B}}\theta && \text{Definition of $\brk{-}$}
        \end{align}

        \textbf{I found this diffucult and lack intuition for which operations are permitted i.e.\@ \ref{fun:9}}

    \end{enumerate}

    \item Complete Exercise 1 from Lecture 7. Show how Landin's knot works by unrolling $\mathrm{fib}(3)$.

    \begin{lstlisting}[language=Caml]
let fib = knot (fun f n -> if n <= 1 then 1 else n * f (n - 1))
    \end{lstlisting}

    Landin's knot is defined as
    \begin{lstlisting}[language=Caml]
let knot =
    fun f ->
        let r = ref (fun n -> 0) in
        let recur = fun n -> !r n in
        let () = r := fun n -> f recur n in
        recur
    \end{lstlisting}

    By substituting the definition of \texttt{fib}, we have:
    \begin{lstlisting}[language=Caml]
let fib =
    let r = ref (fun n -> 0) in
    let recur = fun n -> !r n in
    let () = r := fun n -> (fun f n -> if n<=1 then 1 else n * f (n-1)) recur n in
    recur
    \end{lstlisting}

    By substituting the definition of \texttt{fib(3)}, we have:
    \begin{lstlisting}[language=Caml]
let r = ref (fun n -> 0) in
let recur = fun n -> !r n in
let () = r := fun n -> if n <= 1 then 1 else n * recur (n - 1) in
recur 3
    \end{lstlisting}

    By expanding the definitions a few more levels, we have
    \begin{lstlisting}[language=Caml]
let r = ref (fun n -> 0) in
let recur = fun n -> !r n in
let () = r := fun n -> if n <= 1 then 1 else n * recur (n - 1) in
3 * recur 2
    \end{lstlisting}

    Expanding the definitions a few more levels, we have
    \begin{lstlisting}[language=Caml]
let r = ref (fun n -> 0) in
let recur = fun n -> !r n in
let () = r := fun n -> if n <= 1 then 1 else n * recur (n - 1) in
3 * 2 * 1 * recur 0
    \end{lstlisting}

    We see that \texttt{recur} is the recursive call. By making a cycle of references using \texttt{r}, we get \texttt{recur} to indirectly call itself.

    \item This question concerns the monadic language given in Lecture 8.

    \begin{enumerate}

        \item Given types $X$ and $Y$ define a term:
        \[
            \cdot; \cdot \vdash \mathrm{fmap}: (X \to Y) \to (T\ X \to T\ Y)
        \]
        such that for all terms $f: X \to Y$ and values $v: X$ and $v': Y$ where $f v \rightsquigarrow^* v'$ we have
        \[
            \abr{\sigma}{\mathrm{let}\ y = \mathrm{fmap}\ f\ \{\mathrm{return}\ v\}; \mathrm{return}\ y} \rightsquigarrow^* \abr{\sigma}{\mathrm{return}\ v'}
        \]

        Define $\mathrm{fmap}$ as follows:
        \[
            \mathrm{fmap} = \lambda f: X \to Y\ldotp \lambda x: X\ldotp \mathrm{let}\ z = x; \mathrm{return}\ (f\ z)
        \]

        \item For every type $X$, define terms:
        \begin{align*}
            \cdot; \cdot &\vdash \eta_X: X \to T\ X \\
            \cdot; \cdot &\vdash \mu_X: T\ (T\ X) \to T\ X \\
        \end{align*}

        such that for all values $v: T\ X$ and $v': X$, where $\abr{\sigma}{\mathrm{let}\ x = v; \mathrm{return}\ x} \rightsquigarrow^* \abr{\sigma'}{\mathrm{return}\ \sigma'}$ we have
        \[
            \abr{\sigma}{\mathrm{let}\ y = \mu_X(\eta_{(TX)}\ v); \mathrm{return}\ y} \rightsquigarrow^* \abr{\sigma'}{\mathrm{return}\ v'}
        \]
        \begin{align*}
            \eta_X &= \lambda x: X\ldotp \mathrm{return}\ x \\
            \mu_X &= \lambda x: T\ (T\ X)\ldotp \mathrm{let}\ z = x; z \\
        \end{align*}

    \end{enumerate}

    \item

    \begin{examquestion}{2019}{9}{14}

        \newcommand{\nat}{\ensuremath{\mathsf{Nat}}}
        \newcommand{\zero}{\ensuremath{\mathsf{Zero}}}
        \newcommand{\suc}{\ensuremath{\mathsf{Succ}}}
        \newcommand{\iter}{\ensuremath{\mathsf{Iter}}}
        \newcommand{\option}{\ensuremath{\mathsf{Option}_A}}
        \newcommand{\none}{\ensuremath{\mathsf{None}}}
        \newcommand{\some}{\ensuremath{\mathsf{Some}}}
        \newcommand{\case}{\ensuremath{\mathsf{Case}}}
        \newcommand{\pred}{\ensuremath{\mathsf{Pred}}}

        \begin{enumerate}[label=(\alph*)]

            \setcounter{enumii}{1}

            \item In System F, given (i) a Church encoding \nat\ for the natural numbers, (ii) a Church encoding for the $\zero: \nat$ and $\suc: \nat \to \nat$ constructors, and (iii) a type and definition for
            the iteration operator \iter\ for natural numbers.
            \begin{align*}
                \nat &= \forall \alpha \ldotp \alpha \to (\alpha \to \alpha) \to \alpha \\
                \zero &= \Lambda \alpha \ldotp \lambda z\ldotp \lambda s\ldotp z \\
                \suc &= \Lambda \alpha \ldotp \lambda z\ldotp \lambda s\ldotp f\ z \\
                \mathrm{typeof}(\iter) &= \forall \alpha\ldotp \nat \to \alpha \to (\alpha \to \alpha) \to \alpha \\
                \iter &= \Lambda \alpha\ldotp \lambda n: \nat\ldotp \lambda e_1: \alpha\ldotp \lambda e_2: \alpha \to \alpha\ldotp n\ \alpha\ e_1\ e_2
            \end{align*}

        \end{enumerate}

        \item

        \begin{enumerate}[label=(\roman*)]

            \item In System F, give a Church encoding for (i) an \option\ type, (ii) the definitions of the $\none: \option$ and $\some: A \to \option$ operations, and (iii) the type and definition of the case
            operation on options.
            \begin{align*}
                \option &= 1 + A = \forall \alpha\ldotp (1 \to \alpha) \to (A \to \alpha) \to \alpha\\
                \none &= \Lambda \alpha\ldotp \lambda n: \alpha\ldotp \lambda s: A \to \alpha\ldotp n\\
                \some &= \lambda x: A \to \Lambda \alpha\ldotp \lambda n: \alpha\ldotp \lambda s: A \to \alpha\ldotp g\ s(x)\\
                \mathrm{typeof}(\case[X]) &= X \to (A \to X) \to (1 + A) \to X \\
                \case[X] &= \lambda n: X\ldotp \lambda s: A \to X\ldotp \lambda x: 1 + A\ldotp x\ X\ n\ s
            \end{align*}

            \item Assume that $n: B$ and $s: A \to B$, and then

            \begin{enumerate}[label=(\Roman*)]

                \item Prove that $\case[B]\ n\ s\ \none = n$
                \begin{align*}
                     & \case[B]\ n\ s\ \none \\
                    =& (\lambda n': B\ldotp \lambda s': A \to X\ldotp \lambda x: 1 + A\ldotp x\ B\ n'\ s')\ n\ s\ \none && \text{Definition of $\case[B]$} \\
                    =& \none\ X\ n\ s && \text{Substitution} \\
                    =& (\Lambda \alpha\ldotp \lambda n: \alpha\ldotp \lambda s: A \to \alpha\ldotp n)\ X\ n\ s && \text{Definition of $\none$} \\
                    =& n && \text{Substitution}
                \end{align*}

                \item Prove that $\case[B]\ n\ s\ (\some\ x) = s\ x$
                \begin{align*}
                     & \case[B]\ n\ s\ (\some\ x) \\
                    =& (\lambda n': B\ldotp \lambda s': A \to X\ldotp \lambda x: 1 + A\ldotp x\ B\ n'\ s')\ n\ s\ (\some\ x) && \text{Definition of $\case[B]$} \\
                    =& \some\ x\ X\ n\ s && \text{Substitution} \\
                    =& (\lambda x: A \to \Lambda \alpha\ldotp \lambda n: \alpha\ldotp \lambda s: A \to \alpha\ldotp g\ s(x))\ x\ X\ n\ s && \text{Definition of $\some$} \\
                    =& s\ x && \text{Substitution}
                \end{align*}

            \end{enumerate}

            \item In System F, define a predecessor operation $\pred: \nat \to \nat$, which returns \zero\ if given \zero\ as an argument, and returns $n$ if given $\suc\ n$ as an argument.

            \[
                \pred = \lambda n: \nat\ldotp \case[\nat]\ (n\ \none\ (\case[1 + \nat]\ (\some\ \zero)\ (\lambda m\ldotp \some\ (\suc\ n))))\ 0\ (\lambda n\ldotp n)
            \]

            \item In System F, define a subtraction operator $\sub: \nat \to \nat \to \nat$ which is defined to be \textit{saturating}. That is, $\sub\ m\ n$ returns the difference if $m \ge n$ and returns $0$
            otherwise.
            \[
                \sub = \lambda m: \nat\ldotp \lambda n: \nat\ldotp \iter(n, z \to m, s(x) \to \pred\ x)
            \]

        \end{enumerate}

    \end{examquestion}

    \item

    \begin{examquestion}{2021}{9}{15}

        \newcommand{\every}{\ensuremath{\mathsf{every}}}
        \newcommand{\bool}{\ensuremath{\mathsf{Bool}}}
        \newcommand{\true}{\ensuremath{\mathsf{true}}}
        \newcommand{\false}{\ensuremath{\mathsf{false}}}
        \newcommand{\List}[1]{\ensuremath{\mathsf{List}\ #1}}
        \newcommand{\xs}{\ensuremath{\mathit{xs}}}
        \newcommand{\fold}{\ensuremath{\mathsf{fold}}}

        \begin{enumerate}[label=(\alph*)]

            \item In a simply-typed lambda calculus augmented with first-class continuations, booleans, a list type and its iterator (i.e., fold, but not full recursion), write a function
            \[
                \every: (X \to \bool) \to \List{X} \to \bool
            \]
            such that $\every\ p\ \xs$ returns \true\ if every element of \xs\ satisfies $p$, and \false\ otherwise. This function should also stop iterating over the list as soon as it finds a false element. You
            may use SML or OCaml-style notation if desired, but explain any notation beyond the basic lambda-calculus.

            \textbf{I was confused by ``stop iterating over the list as soon as it finds a false element'': as I understand it, this seems impossible without recursion (as\fold\ is atomic).}

            The interpretation which (by elimination) I settled on was that the \textit{predicate} would not be unnecessarily evaluated on the elements of the list once the result of computation was known, but
            that fold itself could continue.
            \[
                \every = \lambda p: \bool \to X\ldotp \lambda \xs: \List{X}\ldotp \fold\ \true\ (\lambda \mathit{acc}\ldotp \lambda x: X\ldotp \mathrm{if}\ \mathit{acc}\ \mathit{then}\ p\ x\ \mathit{else}\
                \mathit{acc})
            \]

            \item In the monadic lambda calculus with stat, suppose we change the typing rule for reading locations to not cause a monadic effect: if we suggest changing the monadic lambda calculus to permit
            treating reads as pure:
            \[
                \dfrac{\ell: X \in \Sigma}{\Sigma; \Gamma \vdash \ell: X}
            \]
            \begin{enumerate}

                \item Is this rule still typesafe? Informally but carefully justify your answer.

                No. Consider Landin's knot. We can ``tie'' Landins knot and the return value is an expression which only reads from locations. This expression will loop infinitely and so is not typesafe.
                However, if reads are pure then it \textit{will} be considered as typesafe. So this rule is \textbf{not} typesafe.

                \item Is the following \textit{common subexpression elimination} transformation sound? Either give an argument why it is, or supply a counterexample and explain why it shows it is not.

                \begin{verbatim}
let x = return e1;              let x = return e1;
let y = e2;           =====>    let y = e2;
let z = return e1;              [z/x]e3
e3
                \end{verbatim}

                I presume the original expression was typo'd and meant to read \texttt{[x/z]e3} -- else the second expression can be badly-scoped by any term $e_3$ which uses either $z$ or $x$.

                This transformation is \textbf{not} safe. Consider the situation that $e_1$ is an impure computation (with non-idempotent side effects) and $e_3$ is some expression whose result depends on the
                state that is changed by $e_1$. We would have that the left-hand expression would run $e_1$ twice while the one on the right would only run it once. This means that they would start executing
                $e_3$ with different stores!

                Example: $e_1 = \ell \coloneqq !\ell + 1$, $e_2 = ()$, $e_3 = !\ell$ with the LHS expression denoted by $\mathit{LHS}$ and the RHS expression denoted by $\mathit{RHS}$.
                $\abr{\ell: 0}{\mathit{LHS}} \rightsquigarrow \abr{\ell: 2}{2}$ while $\abr{\ell: 0}{\mathit{RHS}} \rightsquigarrow \abr{\ell: 1}{1}$. Clearly, we can see that these expressions are
                \textbf{not} the same.

            \end{enumerate}

        \end{enumerate}

    \end{examquestion}

    \item

    \begin{examquestion}{2020}{9}{15}

        \newcommand{\fail}{\ensuremath{\mathsf{fail}}}
        \newcommand{\try}[2]{\ensuremath{\mathsf{try}\ #1\ \mathsf{except}\ #2}}
        \newcommand{\exn}[1]{\ensuremath{\mathsf{Exn}\ #1}}

        \begin{enumerate}[label=(\alph*)]

            \setcounter{enumii}{1}

            \item Suppose we extend the simply-typed lambda calculus with the ability to raise exceptions with the \fail\ construct, and the ability to catch exceptions with the \try{e_1}{e_2}\ construct.
            Suppose also that we track exceptions monadically, with the type \exn{A} representing possibly-failing computations of $A$.

            \begin{enumerate}[label=(\roman*)]

                \item Give a typing rule for signalling an error with \fail.
                \[
                    \dfrac{
                    }{
                    \Gamma \vdash \fail: \exn{A}
                    }
                \]

                \item Give a typing rule for trapping an error with \try{e_0}{e_1}. Does your type for this term have an effect? Justify your design.
                \[
                    \dfrac{
                    \Gamma \vdash e_0: \exn{A} \qquad \Gamma \vdash e_1: A
                    }{
                        \Gamma \vdash \try{e_0}{e_1} : A
                    }
                \]
                This term can have an effect: consider the case $A = \exn{B}$ for some $B$. In this case, we have that the return value from this term is \textit{still} itself an exception. This means that it
                affects control flow of other terms and thus has an effect!

            \end{enumerate}

        \end{enumerate}

    \end{examquestion}

\end{enumerate}

\section{Extension Questions}

\begin{enumerate}

    \item Write a term \texttt{and} that takes two Church booleans and returns their conjunction. (How do you change this for \texttt{or}?)
    \[
        \texttt{and} =
        \Lambda \beta\ldotp
        \lambda b_1: \forall \alpha\ldotp \alpha \to \alpha \to \alpha\ldotp
        \lambda b_2: \forall \alpha\ldotp \alpha \to \alpha \to \alpha\ldotp
        \lambda t: \beta\ldotp
        \lambda f: \beta\ldotp
        b_1\ \alpha\ (b_1\ \alpha\ t\ f)\ f
    \]
    To change for \texttt{or}, we shift the second boolean to be the result of the false condition.
    \[
        \texttt{or} =
        \Lambda \beta\ldotp
        \lambda b_1: \forall \alpha\ldotp \alpha \to \alpha \to \alpha\ldotp
        \lambda b_2: \forall \alpha\ldotp \alpha \to \alpha \to \alpha\ldotp
        \lambda t: \beta\ldotp
        \lambda f: \beta\ldotp
        b_1\ \alpha\ t\ (b_1\ \alpha\ t\ f)
    \]

    \item Write a function \texttt{equal} that takes two Church numerals and returns a church boolean.

    Using the $\mathsf{sub}$ and $\mathsf{and}$ functions defined earlier,
    \[
        \texttt{equal} = \lambda n: \mathsf{nat}\ldotp \lambda m: \mathsf{nat}\ldotp \mathsf{and}(\mathsf{iter}(\mathsf{sub}(m, n), z \to \mathsf{true}, s(x) \to \mathsf{false}),
        \mathsf{iter}(\mathsf{sub}(n, m), z \to \mathsf{true}, s(x) \to \mathsf{false}))
    \]

    \item
    \begin{examquestion}{2014}{9}{13}

        \begin{enumerate}[label=(\alph*)]

            \setcounter{enumii}{1}

            \item Find, with justification, a PLC type $\tau$ for which the following typings are both provable:
            \begin{gather*}
                \{\} \vdash
                \Lambda \alpha\ldotp \Lambda \beta\ldotp \lambda x: \alpha\ldotp \lambda y: \beta\ldotp \Lambda \gamma\ldotp \lambda z: \tau\ldotp z\ x\ y
                : \forall \alpha\ldotp \forall \beta\ldotp \alpha \to \beta \to \forall \gamma\ldotp (\tau \to \gamma)
                \\\\
                \{\} \vdash
                \Lambda \alpha\ldotp \Lambda \beta \ldotp \lambda z: \forall \gamma\ldotp \tau \to \gamma\ldotp z\ \alpha\ (\lambda x: \alpha\ldotp \lambda y: \beta\ldotp x)
                : \forall \alpha\ldotp \forall \beta \ldotp (\forall \gamma\ldotp \tau \to \gamma) \to \alpha
            \end{gather*}

            \[
                \tau = \alpha \to \beta \to \gamma
            \]
            In the first expression, we have $z: \tau$. We see that $z$ is passed $x: \alpha$ and $y: \beta$ as arguments and returns $\gamma$. Therefore in the first expression, $\tau = \alpha \to \beta \to \gamma$.

            In the second expression, we have $z: \forall \gamma\ldotp \tau \to \gamma$. We then see $z\ \alpha\ (\lambda x: \alpha\ldotp \lambda y: \beta\ldotp x): \alpha$.
            So we have $\tau = \alpha \to \beta \to \alpha$. But, since $\alpha$ was passed to $z$ as argument, $\alpha$ and $\gamma$ were unified; so we can replace any occurrence of an $\alpha$ in the type of
            $z$ with a $\gamma$ and the type will still be correct. So $\tau = \alpha \to \beta \to \gamma$ is a possible type. This is the only such which is consistent with the first equation.

            \item Give infinitely may different closed PLC expressions in beta-normal form of type $\forall \alpha\ldotp (\alpha \to \alpha) \to (\alpha \to \alpha)$.

            Consider expressions of the form:
            \[
                \Lambda \alpha \ldotp \lambda f: \alpha \to \alpha \ldotp \lambda x: \alpha \ldotp f^n(x)
            \]
            where $f^n(x) = \underbrace{f(\ldots(f}_{\times n}(x)\ldots)$.

            \item Ues your answer to part (a) to show that there is no closed PLC expression $Y$ of type $\forall \alpha \ldotp (\alpha \to \alpha) \to \alpha$ for which the beta-reduction
            $Y\ \alpha\ f \to f(Y\ \alpha\ f)$ \textbf{holds}.

            Assume there is such a $Y$. By definition, we have $Y\ \mathsf{nat}\ \mathsf{succ} \rightsquigarrow \mathsf{succ}(Y\ \mathsf{nat}\ \mathsf{succ})$. Since all terms in PLC terminate, we also have
            $Y\ \mathsf{nat}\ \mathsf{succ} \rightsquigarrow^* n$ for some $n$. Thus, we have $n = n + 1$. This is clearly false and therefore the original assumption that some $Y$ existed which had this
            condition must be false. Therefore, no such $Y$ can exist.

        \end{enumerate}

    \end{examquestion}

\end{enumerate}

\end{document}
